% -*- root: These.tex -*-

% PARTI B1

\Anotecontent{artificial_societies}{ \hl{Un terme aujourd'hui désuet si on en crois chattoe brown ... }}

\Anotecontent{gilbert_confidence}{Le fait que \textcite{Gilbert2000a} fasse cette confidence dans un article intitulé \foreignquote{english}{Modelling Sociality : The View from Europe} dans l'ouvrage \foreignquote{english}{Dynamics in Human and Primate Societies} de Kohler et Gumerman \autocite{Kohler2000} n'est probablement pas anodin, et révèle l'existence passé d'une forme de cloisonement entre les deux foyers Européen et Américain sur ce sujet. Car face à l'ambition affiché de l'ouvrage d'Epstein et Axtell \foreignquote{english}{Growing artificial societies} \autocite{Epstein1996}, il est fort dommageable de voir que les projets européens, riche pourtant à cette époque d'un certain avancement sur le sujet, ne sont qu'à peine évoqués en introduction, la plupart des références n'allant pas en deça des années 1990. Le modèle Anasazi étant initialement inspiré des travaux sur Sugarscape \autocite{Rauch2002}, la participation des européens à cet ouvrage pourrait en définitive être interprété comme le signe d'une collaboration retrouvée.}

\Anotecontent{histoire_sugarscape}{Le journaliste \textcite{Rauch2002} a interviewé Joshua Epstein à ce sujet en 2002 : \foreignquote{english}{One day in the early 1990s, when he was giving a talk about his model of arms races, he met Axtell, who was graduate student. He wound up bringing Axtell to Brookings, in 1992. Not long after, Epstein attended a conference at the Santa Fe Institue [...] At Santa Fe juste then a big subject was artificial life, often called A-Life. \enquote{All of the work was about coral reefs, ecology, growing things that look like trees, growing things that look like flocks of birds, schools of fish, coral, and so on,} Epstein told me. \enquote{And I thought, jeez, why don't we try to use these techniques to grow societies?} Fired up, he returned to Brookings and discussed the idea with Axtell. There followed the inevitable napkin moment, when the two of them sat in the cafeteria and sketched out a simple artificial world in which little hunter-gatherer creatures would move around a landscape finding, storing, and consuming the only resource, sugar.} Le modèle est écrit sur sa propre base logicielle \textit{Object Pascal}, c'est à dire découplé d'une plateforme agent existante. Le code source n'est pas disponible en libre accès, et la plupart des versions que l'on trouve aujourd'hui sur les plateformes sont donc des réimplémentations basé sur la description faites des auteurs du modèle.}

\Anotecontent{varela_modele_ca}{Varela, Hersini, Bourgine \autocite{Bourgine1992} organise à Paris en 1991 la première conférence européenne ECAL sur la vie artificielle, mais il faut savoir que Varela développe aussi un modèle d'automate cellulaire appuyant sa théorie dès 1974 en Fortran, décrit par \textcite{McMullin1997b}. Sa \enquote{redécouverte} a permis de le sauvegarder sous une autre forme, dans un projet mené par \textcite{McMullin1997}.}

\Anotecontent{smith_bio}{Voici ce que Alvy Ray Smith, un informaticien pionnier dans le domaine de l'infographie, titulaire d'une thèse sur les automates cellulaires en 1970 et également connu pour ses travaux sur la représentation graphique des L-Systems, écrit sur son site internet vis à vis d'un état de l'art qu'il publie en \autocite{SmithIII1976} suite à ce qu'il apelle une des toutes premières conférences sur la VA : \foreignquote{english}{This is an extensive survey and bibliography of the field of CA (I was calling them \enquote{polyautomata} at the time) up to 1975. I added the words \enquote{Cellular Automata and} to the original title when I transcribed the original article into online form. It was originally written as the introduction to the German edition of Theory of Self-Reproducing Automata, by John von Neumann, edited (posthumously) by Arthur W Burks, University of Illinois Press, Urbana, 1968. Von Neumann performed the work, completed by Burks, in 1952-53. The German publishers never published the German edition, but gave me permission to publish my survey in the book listed above, the proceedings of a conference held in Noordwijkerhout, The Netherlands, Apr 1975. I like to call this conference ALife0 - for Artificial Life 0 conference - since it was the first attempt I know to cross-fertilization between biologists and computer scientists. Many of the players at this conference were present for ALife1, ALife2, etc - cf Simple Nontrivial Self-Reproducing Machines . Other participants at ALife0 were Karel Culik, Pauline Hogeweg, John Holland, Aristid Lindenmayer, and Stanislaw Ulam.}}

\Anotecontent{helmreich_IA}{L'Anthropologue \textcite[10]{Helmreich1998} ayant fait du SFI son terrain d'étude est tout à fait lucide sur cette question, et met le lecteur en garde au debut de sont ouvrage : \foreignquote{english}{Not all Artificial Life scientists are happy with how the recent history of the field is told, with how this shapes the terrain of inquiry, or with how the Santa Fe Institute is privileged in popular accounts [...] Many Europe-based researchers argue that Artificial Life was not created ex nihilo in New Mexico but has descended from tangled international lineages of cybernetics, systems theory, Artificial Intelligence, self-organization theory, origins of life research, and theoretical biology. The Chilean-born and Paris-based biologist Fransisco Varela [...] has argued that the materialization of Artificial Life in New Mexico has focused attention on overly computational views of life and that the naming of Artificial Life on the analogy to Artificial Intelligence has only made this more intense. U.S narrations of Artificial Life history are notorious in the international community for their erasure or marginalization of European and Latin American precedents and scientists, of transnational collaborations, and of the social, intellectual, and economic contexts that produced the science in some places and not others. [...] Indeed, my own decision to do field-work on the community at Santa Fe was powerfully guided by readings of popular science, and this book runs the risk of reinforcing the mainstream myth that \enquote{Artificial Life was born out of Zeus' head in Santa Fe, New Mexico, in the 1980's} -- as one Europe-based scientist sardonically summarized it to me.}}

\Anotecontent{echelle_optimization}{\foreignquote{english}{Evolutionary computation, the field of simulating evolution on a computer, provides the basis for moving toward a new philosophy of machine intelligence. Evolution is categorized by several levels of hierarchy: the gene, the chromosome, the individual, the species, and the ecosystem. Thus there is an inevitable choice that must be made when constructing a simulation of evolution. Inevitably, attention must be focused at a particular level in the hierarchy, and the remainder of the simulation is to some extent determined by that perspective. Ultimately, the question that must be answered is, \enquote{What exactly is being evolved?}} \autocite{Fogel1998}}

\Anotecontent{livret_CREA}{Le lecteur intéressé trouvera sur cette généalogie complexe de la notion des éléments de reflexion dans un livret édité par le CREA en 1985. Celui-ci relate un travail de trois ans de recherches sur la \enquote{Genealogie de l'auto-organisation} basé sur le dépouillements des archives du BCL. Ce livret traite des recherches réalisé par les chercheurs Isabelle Stengers, Pierre Livet, Pierre Lévy (lecture commenté des archives du BCL); et contient également des interviews menés par Isabelle Stengers de Fogelman-Soulié, Atlan, Von Foerster, Weisbuch, Varela. \autocite{CREA1985}}

\Anotecontent{pouvreau_livre1949}{\enquote{Ce dernier ouvrage est la synthèse des réflexions de Bertalanffy au cours des deux décennies passées, une systématisation des thèmes qu’il a jusqu’alors développés dans ses divers écrits. [...] L’originalité de ce livre par rapport à ses écrits antérieurs tient au fait que Bertalanffy, dans l’esprit du projet de \enquote{ systémologie générale } qui constitue désormais le cœur de ses préoccupations intellectuelles, ne se limite pas à y argumenter la nécessité d’un point de vue \enquote{ organismique } dans l’ensemble des domaines de la biologie – en particulier, à démontrer la pertinence et la fécondité des concepts de \enquote{ système ouvert } et d’\enquote{ ordre hiérarchique }. Il s’efforce aussi d’y établir l’évolution convergente de l’ensemble des sciences naturelles, sociales et humaines, ainsi que de la philosophie, vers une épistémologie centrée sur les concepts de système et d’organisation dynamique. La logique de Das biologische Weltbild est l’incarnation de l’idée fondamentale de Bertalanffy, et l’aboutissement d’un chemin initié dès sa thèse doctorale de 1926 : le dépassement de l’organicisme en direction du systémisme. Son livre se concluant en conséquence par un exposé des grandes lignes de sa \enquote{ systémologie générale }.}\autocite[46]{Pouvreau2006}}

\Anotecontent{taylor_openended}{\textcite{Taylor1999} parle pour ce cas de \textit{Open-Ended Evolution} : \foreignquote{english}{This term refers to a system in which components continue to evolve new forms continuously, rather than grinding to a halt when some sort of `optimal' or stable position is reached[...]Note that open-ended evolution does not necessarily imply any sort of evolutionary progress.[...]Also, by using the term `open-ended' I wish to imply that an indefinite variety of phenotypes are attainable through the evolutionary process, rather than continuous change being achieved by, for example, cycling through a finite set of possible forms.}}

\Anotecontent{taylor_reproduction}{Les deux termes réplication et reproduction sont souvent utilisés de façon synonyme mais renvoient en réalité à des études différentes, la première évoquant la capacité à reproduire une copie conforme, alors que la seconde renvoie au processus d'évolution naturel par la mise en oeuvre d'opération et de matériel génétique \autocites{Sipper1998,Taylor1999}}

\Anotecontent{renfrew_futur_archeology}{\foreignquote{english}{
There are the several elements that may come together to form this new morphogenetic paradigm in archaeology. The first is the concept of \enquote{system trajectory,} seen not merely in traditional system-theory terms, but in the dynamical sense facilitated by differential topology, including catastrophe theory. The second is the whole approach to self-organizing systems, pionereed by the \enquote{Brussels School,} which overlaps in some respects with the foregoing. The third is preoccupation with information flow, stressed by van der Leeuw, and cogently set out by Johnson in his chapter in this volume and in earlier publications. The fourth element is computer simulation, if it can be developed to cope with the complexity that we are dealing with in such a way as to escape the inflexibility of so many algorithms: The enthousiasm of Doran gives hope that it can.} \autocite[463]{Renfrew1982b}}

\Anotecontent{doran_86_DAI}{\foreignquote{english}{This paper reports initial experiments with a computer program which embodies an abstract model of a sociocultural system. The model displays a form of spontaneous collapse. Central to the model is the adoption and discard of mutually beneficial and cumulative contracts between the component actors of the system.[...] Allen(1982) has argued the relevance of \enquote{dissipatlve structures} and multiactor system concepts to the emergence of modern urban structure including global and local fluctuations. The CONTRACT model I describe here has a number of aspects in common with Allen's work. The CONTRACT model is based on three main assumptions. The first is simply that a sociocultural system may usefully be modelled in abstract computational terms. The second assumption is that a sociocultural system may be regarded as a distributed problem-solver, that is, it is solving the problem of how best to manipulate its environment in order to maximise its own \enquote{wellbeing}. The system is distributed in that there arc multiple loci of decision, actors, each of which has only partial knowledge, and in that the criterion of success, \enquote{wellbelng}, is itself distributed over the decision making loci and locally defined. The third assumption is that the knowledge which the problem-solver necessarily uses to solve its problem is to be identified with cumulative technological knowledge cooperatively deployed.} \autocite{Doran1986b}}

\Anotecontent{doran_82_DAI}{ \textcite{Doran1982} présente un modèle générique pour étudier les comportements d'un système socio-culturel. Pour simplifier, les agents sont amenés à se structurer pour exploiter au mieux les ressources d'un environnement; structure dont l'émergence doit être le reflet des interactions (contrat) et des capacités de cognitions (représentations, mémoire, objectif) propre à chacun des acteurs décidant de participer à cette économie. Voici le résumé qu'il donne à un des schéma qu'il présente \foreignquote{english}{A set of concurrent actors, the multiactor system, is structured by a pattern of contracts that effects exploitation of the environment. Each actor has its own simplified and typically distorted representation(\enquote{cognized model}) of the multiactor system and environment, and this representation determines its individual contract participation.} Doran fait références plusieurs fois à la possible adéquation  entre les problématiques rencontrés dans de telle systèmes sociaux et les progrès fait par l'intelligence artificielle dans la résolution de problèmes en environnement distribué : \foreignquote{english}{We need the concept of a set of processes that run concurrently, which in some suitable way exchange information(\enquote{pass messages}) and which thus collectively effect some required computation [...] Discovering ways in which a system of concurrent communicating processes can engage in heuristic human-like problem-solving is an important current research topic (for example Smith1979). This work is closely relevant to the study of the capabilities of sociocultural systems [...]}}

\Anotecontent{doran_85_DAI}{ \foreignquote{english}{In this paper I shall suggest that important problems of natural language and of individual and cultural knowledge mays usefully be approached by a computational route. Central to my argument will be the concept of a multi-actor system (sometimes called a \enquote{multi-agent system} in the research litterature). In artificial intelligence work, discussions of multi-actor systems typically envisage a collection of semi-autonomous computer controlled devices [...] which cooperate to perform some task in their common real world environment. However, an alternative is a single computer program which \textbf{simulates} actors in a modelled environment. In this case the aim is to use the study of a modelled multi-actor system to further understanding of real system -- both those that might be constructed and those human systems that are in existence around us.}\autocite[160]{Doran1985}} 

\Anotecontent{doran1982_reclamation}{ 
\foreignquote{english}{Several years ago \autocite{Doran1982}, I suggested that multiple agent systems (MAS) theory could form a basis of models of socio-cultural dynamics including the growth of social complexity. Since then MAS theory and distributed artificial intelligence (DAI) generally have developed substantially (\autocite{Bond1988} Gasser and Huhns 1989; Demazeau and Muller 1990 ) and now the idea of studying \enquote{societies} on computers is becoming not just tenable but fashionable - altought the emphasis is as yet largely on studying the properties of systems of abstract rather than realistic agents. In spite of this limitation, it now looks possible to develop my original suggestion in a more serious way, and briefly to compare it with the more prominent alternatives.} \autocite{Doran1997} 

Preuve de sa connaissance dans le domaine de l'intelligence artificielle et l'archéologie, ses articles précédents dans les années 1970, il se réfère très tôt et de multiple fois \autocites{Doran1992, Doran1994a} à l'article très connu sur les DAI de Alan Bond et Les Gasser en 1988 \autocite{Bond1988}. EOS est donc comme il le dit lui même dans \autocite{Doran1994a} en réalité un double projet qui lui permet de développer des questions de recherche au croisement de ces travaux en intelligence artificielle distribué et de l'archéologie, une trajectoire de recherche qu'il cultive depuis longtemps comme en témoigne déjà ces travaux (projet CONTRACT, EXCHANGE \autocite{Doran1986b}) au contact des nouveautés systémiques qui touche l'archéologie courant des années 1980. C'est donc dans la continuité de ces travaux que le projet EOS se met en place au début des années 1990, lui permettant d'activer cette triple synergie, entre un modèle archéologique de sociétés \autocite{Mellars1985}, des questionnements plus théoriques sociologiques, et le développement d'un \textit{testbeds} agent spécialisé (MCS/IPEM dévelopé en Prolog) au coeur de l'université d'ESSEX \autocite{Doran1992}}

\Anotecontent{note_bond_liens}{A noter sur ce point qu'il existe quand même des liens, et que déjà certains auteurs tel que \textcite{Bond1988} pointent déjà dans les années 1980 l'absence et la nécessité de la mise en place d'une boucle d'échange fructeuse entre disciplines autour des DAI : \foreignquote{english}{ Moroever, others have suggested that DAI may draw from and contribute to others disciplines, both absorbing and providing theorical and methodological fondations [ Chandrasekn81, Lesser83, Wesson81]} et d'ajouter plus loin la citation de Wesson en 1981: \foreignquote{english}{Fields of study heretofore ignored by AI : organization theory, sociology and economics , to name a few - can contribute to the study of DAI. Probably DAI advance these fields as well by providing a modelling technology suitable for precise specification and implementation of theories of organizational behavior } [Wesson81, p18] }

\Anotecontent{gilbert_date_clef}{Voici quelques jalons relevés au détour ( éditorial JASSS, page wikipédia, \autocite{Gilbert1999a} ) de sa prolifique bibliographie que lui même considère comme intéressant d'un point de vue historique pour la discipline.  

 \begin{itemize}
  \item Avril 1992, à Guilford (UK) s'ouvre le premier workshop nommé \foreignquote{english}{Simulating Societies} qui donnera lieu à un tout premier ouvrage \autocite{gilbert1994}. S'ensuivront plusieurs autres workshop un peu partout en Europe, comme celui de Sienne en Italie l'année d'après en juillet 1993, qui donnera lieu à la publication d'un deuxième ouvrage important en 1995 \autocite{Gilbert1995a}.
  \item En 1995, une conférence sur cette thématique est donné à Schoß Dagstuhl in Germany
  \item En 1997 le « first international conference on Computer Simulation and the Social Science» a lieu a Cortona en Italie. Celui çi est reconduit une deuxième fois en 1999 à Paris. 
  \item Au printemps 1998, Nigel Gilbert annonce le lancement de JASSS, premier journal éléctronique ayant pour thème la simulation en science sociale. Celui ci est ouvert à une publication largement inter-disciplinaire, et va s'imposer rapidement comme une référence dans ce microcosme qu'est encore la simulation en science sociale. La liste de diffusion \href{www.jiscmail.ac.uk/cgi-bin//webadmin?A0=simsoc}{@SIMSOC} voit également le jour cette année là.
  \item En 1999, Nigel Gilbert et Klaus G. Troitzsch publie le premier manuel  pour enseigner l'usage de la simulation à un plus large public. Depuis celui ci à été republié en 2005 \autocite{Gilbert2005}
 \end{itemize}
}

\Anotecontent{premier_ouvrage_gilbert}{Premier ouvrage publié sur ce thème par Nigel Gilbert \autocite{gilbert1994}, « Simulating Societies » réalise une première, sinon la première tentative de publication réunissant d'emblée autant d'acteurs usant du multi-agent dans leur disciplines. Reprenant la plupart des interventions réalisés àla conférence de Guilford en 1992, il réunit pour la première fois une multitude de point de vue en abordant le thème de la modélisation agent à la fois sur des aspects méthodologique \autocite{drogoul1994multi} et thématique avec la présentation de cas d'application dans divers domaines centrés autour de la simulation de modèles de «sociétés humaines» : archéologie, développement pour l'écologie, sociologie, etc. Si le livre n'est pas spécifiquement centré autour des ABM car d'autres types de modèles sont également présentés, il est clair dès la présentation des chapitres que c'est ce nouveau formalisme qui suscite le plus d'intérêt et de discussion. Si le formalisme agent est connu et reconnu par la suite pour sa capacité intégratrive dans des publications ultérieures, force est de constater dans cet ouvrage qu'il se présente déjà en 1994 comme un outil d'expression priviligié et immédiatement inter-disciplinaire. L'archéologie et l'anthropologie sont ici majoritaire, conséquence directe selon les auteurs de la capacité des chercheurs opérants dans ces disciplines a se positionner plus facilement comme observateur de notre société, facilitant ainsi l'extraction des éléments clés à intégrer dans les modèles. Cette remarque n'est pas anodine, et témoigne d'une observation faite par la suite en lisant ce premier chapitre introductif. Celui çi s'appuie sur un exemple de modèle archéologique (effondrement Maya) que les auteurs utilisent comme un fil conducteur pour desambiguiser un certain nombre de termes et de concepts propre au process de simulation , et table en conclusion sur un espoir non dissimulé d'arriver à formuler au travers  de ce travail un début de cadre de modélisation commun qui réunit les différentes approches existantes. Les réflexion sur un certain nombre de concepts montre un recul étonnant pour un premier ouvrage de réflexion sur la question. Ainsi, la partie \emph{key concepts in modelling and simulation} aborde de façon succinte les points suivants : définition d'un modèle, de simulation, explicitation de la différence entre modèle spécifié et modèle implementé, précisions sur la notion d'équifinalité des modèles, introduction à la validation.}

\Anotecontent{gilbert_EOS}{\foreignquote{english}{We can now examine an example of a simulation based on DAI principles to see whether it fits neatly into any of these theoretical perspectives on the relationship between macro and micro. I have been associated with the EOS (Emergence of Organised Society) project since its inception, although Jim Doran and Mike Palmer are the people who have done all the work (Doran et al. 1994, Doran \& Palmer, Chapter 6 in this volume).} \autocite[128]{Gilbert1995a}}

\Anotecontent{description_imagine_simulation}{\foreignquote{english}{What the computer offers is the possibility of writing a computer program which embodies, at some level of abstraction, precise specifications of all the relevant factors and of their interactions. These specifications need not embody any general laws, nor need they be mathematical in form, but each must be operationally complete so that together they enable the machine to generate a possible « history » of the island for the period. In addition, the program will generate an estimate of what the consequential archaeological record would be. Given such a program, the task of the archaeologist would be to vary the factor specifications, using his own experience and insight, until the events and deposits predicted by the machine best matched the actual excavation evidence. It would, in fact, be very much a case of « reconstructing the events at the scene of the crime» with the machine doing the tedious task of moving the « actors» and « scenery». For our specific example, the simulation might have as its main components:
\begin{itemize}
\item (a) a fixed « map» of the island including information about climate, vegetation and fauna, together with
\item (b) a specification of the type of settlement characteristic of each population, including information about its size, material products and demand upon the natural environment, and 
\item (c) rules specifying the dynamics of the system - the rules which determine where and when settlements are founded, when a settlement is abandoned, what forms of trade and conflict there are between settlements, and in what ways the material cultures of the populations evolve.
\end{itemize}
The machine would simulate the passage of time by repeatedly updating the map and the settlements « attached» to it by reference to the rules and specifications given - and the « history» so generated might well be both surprising and illuminating.}}

\Anotecontent{foerster_interview}{ Dans l'interview relate dans le cahier 8 du CREA \autocite[257-258]{CREA1985}, Foerster rapelle les motivations premières à l'origine du projet BCL, dont le projet initial ne visait pas forcément le rattachement au projet cybernétique, comme il a eu lieu par la suite : \enquote{ [...] Il y a eu trois conférences sur l'auto-organisation. Les deux premières organisées par Yovits et Cameron. La troisième année, c'est moi qui ai organisé la conférence sur \enquote{Principles of Self-Organizing Systems}, et cette fois Ross Ashby a participé à la rencontre. Pas avant. Ross n'était pas encore au BCL à l'époque. En fait, mon laboratoire, à l'origine, n'était pas consacré à la cybernétique. J'étais associé au groupe \enquote{cybernétique} de Macy, d'accord, mais j'ai apellé notre laboratoire non pas Cybernetic Computer Laboratory, mais Biological Computer Laboratory. Notre intérêt était indépendant des notions principales de la cybernétique qui, je dois le dire, m'apparaissaient à l'époque [...] assez peu fascinantes. Ce qui me fascinait c'était la causalité circulaire.[...] Ce qui m'intéressait vraiment, c'était les principes de computation des organismes vivants. [...] Nous n'étions pas des cybernéticiens. Chacun venait avec sa compétence. Ma compétence était en physique, et donc je jouais le rôle de l'avocat de la physique. Je prenais garde à ce que des lois de la physique ne soient pas violés. Il y avait donc un effort coopératif. C'est seulement plus tard que la cybernétique a pris un grand intérêt, [...] cinq ou six ans après le début du BCL, peut-être plus.}}

\Anotecontent{connexionisme_symbolisme}{\hl{A détailler avec Crevier, et ce que j'ai lu dans Restnick, plus citation du papier de Minsky sur le perceptron. La notion de connexionisme est en un certain sens intéressante, car mis à l'écart pendant des années, elle revient avec la notion forte de décentralisation en IA. L'analyse de Restnick sur ce point...}}

\Anotecontent{liaison_prigogine_foerster}{\hl{Voir ce que dit Foerster dans son interview au CREA...}}

%\Anotecontent{nature_ccs}{The program therefore developed its own core curriculum, establishing courses in automata theory, information and probability theory, analog and digital computers; and in natural language, psychology, and biology treated from an information-processing point of view. Students were also required to take a course in modern algebra and, when it became available, an advanced course in programming. \href{http://um2017.org/2017_Website/History_of_Computer_%26_Communications_Sciences.html}{@History of CCS}}

\Anotecontent{influence_turing}{bien peu de psychologues étaient disposés à s’intéresser à ce modèle qui s’opposait en fin de compte à toutes les écoles (behaviorisme, psychologie génétique, psychologie de la Gestalt, phénoménologie ou psychanalyse) 2 . C’est chez les chercheurs concernés par la construction d’automates de calcul que ce modèle suscita un vif intérêt. McCullough lui-même fut mis sur sa voie à la suite de la démonstration en 1936 par le logicien anglais Alan Turing de l’« isomorphisme » entre toute machine capable de réaliser un calcul fondé sur une procédure algorithmique et une « machine universelle » abstraite dotée d’un « programme » où figurent des instructions et des données que la machine lit sur un ruban de longueur infinie et où elle inscrit ses résultats\autocites[777-778]{Pouvreau2013}{Husbands2012}
}


\Anotecontent{mcculloch_ratioClub}{Une initiative que l'on peut rattacher à la relation complexe qu'il a tissé avec les nombreux membres du \textit{Ratio Club} fondé en 1949 par John Bates, et parmis lesquels vont figurer plusieurs scientifiques aux travaux notoires, dont plusieurs seront amenés à participer par invitation de McCulloch à des séjours aux Etats Unis : 

Des relations qui commence d'ailleur bien avant, car les travaux de Turing ont déjà traversé l'atlantique et marque d'une influence - peut être réciproque - les travaux de ce dernier avec ceux de  McCulloch, Pitts et Von Neummann. \Anote{influence_turing} Les travaux sur les neurones trouvant un écho positif dans les réflexions menés par Turing sur la  expose déjà une large correspondance avec certains des membres de ce groupe, membres qui par ailleurs n'ont pas attendu l'attribution officielle de Wierner pour esquisser des idées similaires à ce qui va devenir par la suite la \enquote{pensée cybernétique} 
, comme .  un groupe de cybernéticien anglais parmis lesquel figure Ashby (qui connait par ailleurs les écrits de McCulloch avant 1946 si on en croit la lettre de Bateson à Ashby daté de décembre 1946) entre 1949 et 1958, et dont l'influence scientifique de ses membres est aujourd'hui largement reconnu.

Spécifié que McCulloch entretient des rapports intéressants avec les différents membres, dont Ashby ne fut qu'un des membres parmis d'autres invités, preuve aussi de l'influence des penseurs anglais dans la structuration du mouvement cybernétique américain. Sur ce point on pourra se référer aux travaux répétés et très intéressant menés par \autocite{Husbands2012}, dont la plupart de ces réflexions sont tirés.}

\Anotecontent{sous_discipline_biologie}{ \enquote{ Surtout dans les années 1920 et 1930, les sciences biologiques furent en effet le lieu d’un mouvement dont la vocation était de réhabiliter la légitimité d’approches holistiques et néanmoins non métaphysiquement vitalistes de ce que Bertalanffy appelait les \enquote{ problèmes de la vie }, que ce soit d’une manière générale, à partir de considérations épistémologiques, ou sur un mode spécifique et empiriquement fondé, relatif à des problèmes bien circonscrits. Sept domaines furent plus spécifiquement concernés : l’embryologie, la théorie de l’évolution phylogénétique, la morphologie, la théorie de l’hérédité, l’étude du comportement de l’organisme individuel dans son environnement (soit selon la perspective du système formé par l’organisme et son environnement, soit selon la perspective de l’autonomie acquise par l’organisme par rapport à cet environnement) et, enfin, celle des relations entre espèces biologiques (biocénologie). } \autocite[153]{Pouvreau2013}}

\Anotecontent{ordre_desordre}{\enquote{En réalité, et Ashby fut explicite sur ce point en 1962, toute sa cybernétique justifiait l’idée de l’impossibilité d’une \enquote{auto-organisation} dans un système n’interagissant pas avec son environnement, les changements organisationnels devant tirer leur source de l’extérieur du système – il critiqua d’ailleurs la pertinence même du concept d’\enquote{ auto-organisation }, en toute rigueur \enquote{ auto-contradictoire } : le système improprement dit \enquote{ auto-organisé } détecte au moyen de ses échanges avec son environnement et sous la forme de perturbations affectant ses \enquote{ variables essentielles } la \enquote{ variété } de cet environnement, et ne peut gagner lui-même de \enquote{ variété } qu’en collectant de l’information sur cet environnement ou en tentant de contrôler les échanges de matière et d’énergie qu’il entretient avec lui. La théorie de la \enquote{ variété } apportait en fin de compte un fondement logico-mathématique à l’idée dont l’origine se trouve chez Fechner et que nous avons vue opposée par Bertalanffy à Schrödinger dès 1949, selon laquelle l’ordre \enquote{ organismique } ne doit pas être pensé comme \enquote{ issu de l’ordre }, mais comme émergeant \enquote{ épigénétiquement } du chaos selon des principes inhérents aux systèmes dynamiques : Ashby donna une impulsion significative à ce qui allait devenir, notamment par l’intermédiaire de Prigogine et Atlan, le fameux principe d’\enquote{ ordre à partir du bruit }} \autocite[800]{Pouvreau2013}}

\Anotecontent{conrad_model}{\foreignquote{english}{Largely because of the limits of memory, computational power, and available runtime (at Stanford University in the late 1960s there was only one IBM 360 mainframe for the entire campus), Michael’s first computer model of an evolving ecosystem was highly simplified and could represent only three hierarchical levels, the genetic, the organismic, and the population in a discrete, finite one-dimensional workld. Populations were limited to a few hundred individuals, and runs were limited to a few hundred generations. In keeping with the ineffability of fitness in real ecosystems, Michael was careful not to explicitly define any fitness function in his program, but allowed the reproductive success or failure of the biota to depend on the ecosystem interactions at every level so that reproduction, competition, and niche selection were subject to variation as the ecosystem itself evolved. For this reason, the behavior of the model was difficult to analyze in any detailed causal terms. }\autocite{Pattee2002}}

\Anotecontent{def_biomathematique}{\enquote{Afin de lever d’emblée toute ambiguïté quant à ce dont il sera question dans ce chapitre, une
définition s’impose au préalable : j’appelle \enquote{ biologie mathématique } (ou \enquote{ biomathématique }) ce qui se veut le parfait analogue pour la biologie de la physique mathématique : une entreprise de construction mathématique de concepts et de lois biologiques, où les mathématiques sont vouées à entretenir avec la biologie un rapport \enquote{ non plus descriptif, mais formateur } (Bachelard), ou mieux encore, \enquote{ constituant } (Lévy-Leblond). En ce sens, on ne peut parler de biologie mathématique ni dans le cas où les mathématiques n’ont pour fonction que de résumer statistiquement ou d’ajuster des données empiriques, ni dans celui où elles ne s’introduisent que par le biais des principes et lois physico-chimiques tels qu’on peut les mettre en œuvre dans l’analyse des objets biologiques.} \autocite[515]{Pouvreau2013}}

\Anotecontent{piaget_cloture}{\enquote{L’œuvre de Jean Piaget représente une étape cruciale dans l’histoire des modèles de la circularité biologique. En allant au-delà du contexte de l’embryologie, Piaget élabore une approche conceptuelle générale qui vise explicitement à cerner les spécificités de l’autodétermination biologique, par la jonction théorique entre circularité, autodétermination et dimension thermodynamique. En particulier, il développe un concept théorique fondamental dans cette tradition, celui de \enquote{ clôture organisationnelle } [...] , entendu comme complémentaire à celui d’ouverture thermodynamique de von Bertalanffy. L’objectif de Piaget est de rendre compatible l’idée d’un flux constant de matière et énergie entre l’organisme et l’environnement avec celle d’un ordre circulaire constitutif, qui maintient le système au cours du temps. Le concept de clôture de Piaget décrit la dynamique propre du vivant comme une forme d’autodétermination, dans le sens fondamental d’une connexion entre \textit{activité} et \textit{existence}, réalisée par un réseau circulaire de relations entre les composants de l’organisme, dont dépendent son unité et individuation.La distinction entre clôture organisationnelle et ouverture thermodynamique est le pivot théorique sur lequel les élaborations plus récentes de l’autodétermination biologique s’appuieront, plus ou moins explicitement. Cela vaut non seulement par rapport à la relation profonde entre stabilité de l’organisation et variations des processus sous-jacents, mais également en relation avec les interactions adaptatives de l’organisme avec l’environnement. Sur ce point, Piaget réinterprète et généralise les concepts d’assimilation et adaptation – initialement formulés par l’embryologie de Waddington – et décrit les interactions d’un organisme avec l’environnement en termes d’adaptation, conçue comme une assimilation des perturbations qui induit une \enquote{ autorégulation } interne (accommodation). Ainsi, l’organisation adapte le réseau circulaire de relations en fonction des perturbations, tout en maintenant la clôture qui réalise l’autodétermination} \autocite{Mossio2014}}

\Anotecontent{etude_pouvreau_mossio}{ A la suite d'un échange privé avec David Pouvreau daté du premier octobre 2014, il est apparu qu'une nette préfiguration de la notion de \enquote{cloture organisationelle} apparaisse dans les travaux de Bertalanffy au travers de sa théorie \enquote{bionomogénétique} de l'évolution.  Sur ce point, des travaux sont en cours, notamment avec le philosophe Matteo Mossio, pour tenter de reconstruire un historique de la notion de cloture qui tiennent compte à juste titre des travaux préalable, notamment ceux de Bertalanffy.
\textcite[619]{Pouvreau2013} définit et replace ce terme dans le contexte des débats sur l'évolution dans les années 1930 : \enquote{s’exerceraient des \enquote{ lois internes de forme }, ou \enquote {de structures }, qui s’expriment chez les organismes par des \enquote{ caractéristiques d’organisation } autonomes n’ayant \enquote{ rien à voir avec l’adaptation } ; des \enquote{ lois de la morphogenèse immanentes } qui codétermineraient l’évolution phylogénétique en opérant une sélection parmi les variations contingentes admissibles au titre de cette évolution, et conféreraient à celle-ci une \enquote{ directivité intérieure }. Typiquement \enquote{ organismique } par l’expression du principe d’\enquote{ activité primaire } qu’elle manifestait, il s’agissait d’une conception dynamique interprétant tout processus morphogénétique comme une \enquote{ réaction entre les ‘puissances’ inhérentes à l’organisme et les conditions extérieures }. Tout en demeurant sur le \enquote{ sol ferme } de la science, elle semblait à Bertalanffy fournir le cadre adéquat pour répondre à l’ensemble des critiques adressées aux théories \enquote{ sélectionniste } et \enquote{ mutationniste } }. 
L'\enquote{activité primaire} étant entendu selon \textcite[60]{Pouvreau2013} comme \enquote{Un schème anti-mécaniciste qu’il opposa tout au long de sa carrière à celui de la \enquote{ réactivité primaire }, selon lui à l’oeuvre aussi bien en biologie (par exemple dans la théorie des tropismes de Jacques Loeb ) qu’en psychologie (avec le behaviorisme) et en théorie de la connaissance (empirismes) – avec cette nuance importante que la combinaison de ce schème avec celui du \enquote{ système ouvert } faisait diverger Bertalanffy de la monadologie leibnizienne : \enquote{L’organisme, même sous des conditions extérieures constantes, donc en l’absence de stimulation extérieure, ne représente pas un système au repos, mais un système actif, mû intérieurement [innerlich gewegt] [...] Il faut considérer comme primaire l’activité autonome, et non la réactivité (le réflexe) }. C’est ce schème, en particulier, qui se devine en amont de sa conception épigénétique de la morphogenèse organique, lorsqu’il postula en 1928 afin d’expliquer ce phénomène un \enquote{ principe de formation immanent à l’organisation de la matière } } Ces définitions dont la complexité est apparente demande pour être mieux comprise de se plonger pleinement dans les écrits de \textcites[451-453, 158-161]{Pouvreau2013} se rapportant spécifiquement à la \enquote{bionomogénétique}.}


\Anotecontent{deux_principes_autoorganisation}{La théorie organismique de Bertalanffy fait état de deux grands principes, qui reprennent en unifiant les développements passés et multiples de nombreuses influences exposés en détail dans les publications de Pouvreau et Drack \autocites{Pouvreau2013, Pouvreau2007}. Le premier principe repris et travaillé dans la théorie de Bertalanffy est celui bien connu de système ouvert en équilibre de flux, éloigné de l'équilibre;  le deuxième moins connu est le principe de hierarchisation. Le couplage de ces deux principes ... on a une expression du vivant qui découle d'un principe d'auto-organisation, dont Pouvreau montre qu'il s'accord assez bien avec celui évoqué par Ashby dans son homéostat.}

\Anotecontent{Pouvreau_secondprincipe}{Le « second principe » de Bertalanffy était le schéma suivant de développement épigénétique d’un système, dont on peut remarquer qu’il correspond bien à ce que Chauvet a récemment posé comme une « caractéristique de la vie dans la matière ». Dans une étape « primaire », le système est « unitaire » : il forme une « totalité équipotentielle » ayant des capacités maximales de régulation. Aucune de ses parties n’y est encore investie d’une fonction spécifique. Dans un second temps survient un processus de « ségrégation » au cours duquel le système se « scinde » en sous-systèmes dont le développement spécifique ultérieur se prédétermine. Un processus de « différenciation progressive » engage alors chaque sous-système dans la voie de développement qui lui a été ainsi assignée. Il se caractérise par l’attribution de fonctions déterminées à ces sous-systèmes et la constitution de structures plus ou moins rigides. C’est un processus d’autonomisation relative et de spécialisation des parties et des processus, qui implique pour le système dans son ensemble une « perte de régulabilité » (ou de « plasticité ») et que Bertalanffy appela à partir de 1937 la « mécanisation progressive ». \autocite[476-477]{Pouvreau2013}}

\Anotecontent{piaget_mossio}{\enquote{L’oeuvre de Jean Piaget représente une étape cruciale dans l’histoire des modèles de la circularité biologique. En allant au-delà du contexte de l’embryologie, Piaget élabore une approche conceptuelle générale qui vise explicitement à cerner les spécificités de l’autodétermination biologique, par la jonction théorique entre circularité, autodétermination et dimension thermodynamique. [...] L’objectif de Piaget est de rendre compatible l’idée d’un flux constant de matière et énergie entre l’organisme et l’environnement avec celle d’un ordre circulaire constitutif, qui maintient le système au cours du temps. Le concept de clôture de Piaget décrit la dynamique propre du vivant comme une forme d’autodétermination, dans le sens fondamental d’une connexion entre activité et existence, réalisée par un réseau circulaire de relations entre les composants de l’organisme, dont dépendent son unité et individuation. La distinction entre clôture organisationnelle et ouverture thermodynamique est le pivot théorique sur lequel les élaborations plus récentes de l’autodétermination biologique s’appuieront, plus ou moins explicitement. Cela vaut non seulement par rapport à la relation profonde entre stabilité de l’organisation et variations des processus sous-jacents, mais également en relation avec les interactions adaptatives de l’organisme avec l’environnement. Sur ce point, Piaget réinterprète et généralise les concepts d’assimilation et adaptation – initialement formulés par l’embryologie de Waddington – et décrit les interactions d’un organisme avec l’environnement en termes d’adaptation, conçue comme une assimilation des perturbations qui induit une « autorégulation » interne (accommodation). Ainsi, l’organisation adapte le réseau circulaire de relations en fonction des perturbations, tout en maintenant la clôture qui réalise l’autodétermination} \autocite[12]{Mossio2014}}

\Anotecontent{terme_bioinformatique}{Paulien Hogeweg définit son objet de recherche \foreignquote{english}{on multilevel evolution aims to understand how different levels of selection, complex genotype phenotype mapping, and interactions over different timescales contribute to the evolution of complexity at the level of individuals and/or ecosystems.} On comprend pourquoi ces travaux ont eu autant d'impact du coté de la biologie, de l'écologie \autocite{Hogeweg1988}, que de l'informatique \autocites{Hogeweg1979, Hogeweg1983, Drogoul1993}. Dans cet extrait d'article \autocite{Hogeweg2011} elle témoigne de l'origine du mot \enquote{bioinformatique} début 1970. Un terme qui, au contraire de son usage actuel \autocite{Giavitto2002}, se rapproche initialement en bien des aspects de ce que \textcite{Giavitto2002} décrit comme ce processus d'aller retour entre les deux approches de \textit{Biological computing} et \textit{Computational Biology} : \foreignquote{english}{In the beginning of the 1970s, Ben Hesper and I started to use the term \enquote{bioinformatics} for the research we wanted to do, defining it as \enquote{the study of informatic processes in biotic systems}. [...] It seemed to us that one of the defining properties of life was information processing in its various forms [...] At a minimum, we felt that that information processing could serve as a useful metaphor for understanding living systems. We therefore thought that in addition to biophysics and biochemistry, it was useful to distinguish bioinformatics as a research field (or what we termed a “work concept”).[...] 
While evolutionary models mainly dealt with invasion of mutants and changing allele frequencies, the question of how evolution leads to complex organisms was not addressed. [...] To meet the challenge of a \enquote{constructive evolutionary biology} became another long-term goal of bioinformatics as we envisioned it. Research in artificial intelligence at this time was exploring new representations of information processing systems, often inspired by biological systems, [...] demonstrating the power of an individual self-centered approach to generating and/or understanding more global structures. We felt that the re-introduction of biologically inspired computational ideas back into biology was needed in order to begin to understand biological systems as information processing systems. In particular, a focus on local interaction leading to emergent phenomena at multiple scales seemed to be missing in most biological models.[...] In short, under the heading of bioinformatics we wanted to combine pattern analysis and dynamic modeling and apply them to the challenge of unraveling pattern generation and informatic processes in biotic systems at multiple scales.} \autocite{Hogeweg2011}}

\Anotecontent{conrad_explanation}{\foreignquote{english}{\textcite{Conrad1970} also studied general properties of evolution systems but from a different perspective that involved the simulation of a hierarchic ecosystem. A population of cell-like individual organisms placed in an array of environmental cells was subjected to a strict materials conservation law that induced competition for survival. The organisms were capable of mutual cooperation, as well as executing biological strategies that included genetic recombination and the modification of the expression of their genome. No fitness criteria were introduced explicitly as part of the program. Instead, the simulation was viewed as an ecosystem in which genetic, individual, and populational interactions would occur and behavior patterns would emerge.}\autocite[62]{Fogel2006b}}

\Anotecontent{patte_deception}{\foreignquote{english}{The hope of \enquote{strong} artificial life was stated by Langton : \enquote{We would like to build models that are so lifelike that they would cease to be models of life and become examples of life themselves.} Very little has been said at this workshop about how we would distinguish computer simulations from realizations of life, and virtually nothing has been said about how these relate to theories of life, that is, how the living can be distinguished from the non-living. The aim of this paper is to begin such a discussion.} \autocite[63]{Pattee1988}}