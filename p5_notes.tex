% -*- root: These.tex -*-

\Anotecontent{doran_note}{Comme il me le confirme dans un échange par email daté de septembre 2014, James E. Doran rencontre Nigel Gilbert en 1984 lors de la conférence \textit{Artificial Intelligence and Sociology} \autocite{Gilbert1985} organisé par Gilbert et Heath : \foreignquote{english}{Yes I think I first met Nigel Gilbert at the meeting he organised with Heath.}}


\Anotecontent{mcculloch_ratioClub}{Une initiative que l'on peut rattacher à cette relation complexe qu'il a su tisser avec les nombreux membres du \textit{Ratio Club} fondé en 1949 par John Bates, et parmi lesquels vont figurer plusieurs scientifiques aux travaux notoires. Membres qui par ailleurs n'ont pas attendu l'attribution officielle de Wierner pour esquisser des idées similaires à ce qui va devenir par la suite la \enquote{pensée cybernétique}. McCulloch entretient des rapports avec les différents membres, dont Ashby ne fut qu'un des membres parmis d'autres invités, preuve aussi de l'influence des penseurs anglais dans la structuration du mouvement cybernétique américain. Des relations qui commencent d'ailleurs bien avant, car les travaux de Turing ont déjà traversés l'atlantique et marquent d'une influence - peut être réciproque - les travaux de ce dernier avec ceux de McCulloch, Pitts et Von Neummann. Plusieurs de ces membres seront amenés à participer à des séjours aux Etats Unis par invitation de McCulloch. Ashby, dont le premier contact écrit avec McCulloch daterait d'avant 1946 (si on en croit la lettre de Bateson à Ashby daté de décembre 1946) va ainsi être amené à effectuer plusieurs voyages aux Etats-Unis entre 1949 et 1958. Sur ces différents points on pourra se référer aux travaux passionants menés par \textcite{Husbands2012}, dont la plupart de ces réflexions sont tirés.}

\Anotecontent{description_imagine_simulation}{\foreignquote{english}{What the computer offers is the possibility of writing a computer program which embodies, at some level of abstraction, precise specifications of all the relevant factors and of their interactions. These specifications need not embody any general laws, nor need they be mathematical in form, but each must be operationally complete so that together they enable the machine to generate a possible « history » of the island for the period. In addition, the program will generate an estimate of what the consequential archaeological record would be. Given such a program, the task of the archaeologist would be to vary the factor specifications, using his own experience and insight, until the events and deposits predicted by the machine best matched the actual excavation evidence. It would, in fact, be very much a case of « reconstructing the events at the scene of the crime» with the machine doing the tedious task of moving the « actors» and « scenery». For our specific example, the simulation might have as its main components:
\begin{itemize}
\item (a) a fixed « map» of the island including information about climate, vegetation and fauna, together with
\item (b) a specification of the type of settlement characteristic of each population, including information about its size, material products and demand upon the natural environment, and
\item (c) rules specifying the dynamics of the system - the rules which determine where and when settlements are founded, when a settlement is abandoned, what forms of trade and conflict there are between settlements, and in what ways the material cultures of the populations evolve.
\end{itemize}
The machine would simulate the passage of time by repeatedly updating the map and the settlements « attached» to it by reference to the rules and specifications given - and the « history» so generated might well be both surprising and illuminating.}}

\Anotecontent{doran_archeologie}{\foreignquote{english}{When I went to Oxford University in 1959 (to study Mathematics) I joined the university archaeological society -- my interest in archaeology initially from programmes on TV. Then later on an Oxford archaeologist named Dennis Britten told me that Roy Hodson at the Institute of Archaeology in London was using maths/computing and I got in touch with Roy and worked with him -- see our 1966 Nature publication}(échange par email avec James E. Doran daté de septembre 2014)}

\Anotecontent{rencontre_renfrew}{Sa rencontre avec Renfrew aurait eu lieu avant la conférence de 1980 \autocite{Renfrew1982}, à l'Université de Sheffield, grâce à Roy Hodson : \foreignquote{english}{ Earlier, when he was at the University of Sheffield I think, very probably via Roy Hodson}(échange par email avec James E. Doran daté de septembre 2014)}

\Anotecontent{doran_DAI}{\foreignquote{english}{DAI started in the early 80s (eg \enquote{Contract Net}), and I would have been aware immediately because I was teaching a course on, in effect, \enquote{Latest in AI}. Some of the leaders of DAI produced DAI software testbeds eg MACE and even talked about doing social simulations. (See the volume of papers \enquote{Readings in DAI}. ) The link to archaeology, which I had from 1966 been publishing in, would have been pretty obvious I think. I just by chance knew about both subjects.} (échange par email avec James E. Doran daté de septembre 2014)}

\Anotecontent{renfrew_futur_archeology}{\foreignquote{english}{
There are the several elements that may come together to form this new morphogenetic paradigm in archaeology. The first is the concept of \enquote{system trajectory,} seen not merely in traditional system-theory terms, but in the dynamical sense facilitated by differential topology, including catastrophe theory. The second is the whole approach to self-organizing systems, pionereed by the \enquote{Brussels School,} which overlaps in some respects with the foregoing. The third is preoccupation with information flow, stressed by van der Leeuw, and cogently set out by Johnson in his chapter in this volume and in earlier publications. The fourth element is computer simulation, if it can be developed to cope with the complexity that we are dealing with in such a way as to escape the inflexibility of so many algorithms: The enthousiasm of Doran gives hope that it can.} \autocite[463]{Renfrew1982b}}



\Anotecontent{gardin_doran}{Gardin organised and chaired that meeting. See the proceedings which include record of discussions etc. I will check my copy if I can find it! However, I never worked with or talked directly to Gardin much at any time -- we had very different ideas. When Gardin and colleagues later worked on \enquote{simulation} he always, I think, meant simulating the archaeologist (typically, and rather strangely, with an Expert System!). By \enquote{simulation} I always meant. from 1970 onwards, what is now called agent-based modelling of eg a dynamic population of households}

\Anotecontent{gilbert_EOS}{\foreignquote{english}{We can now examine an example of a simulation based on DAI principles to see whether it fits neatly into any of these theoretical perspectives on the relationship between macro and micro. I have been associated with the EOS (Emergence of Organised Society) project since its inception, although Jim Doran and Mike Palmer are the people who have done all the work (Doran et al. 1994, Doran \& Palmer, Chapter 6 in this volume).} \autocite[128]{Gilbert1995a}}

\Anotecontent{note_bond_liens}{Dans un de ses articles \textit{Emergence in Social Simulation} \textcite{Gilbert1995} s'appuie sur le peu de questionnements réels dans la littérature reliant DAI et Sociologie. Il faut noter toutefois que ce n'est pas la première fois qu'une telle remarque est faite, et \textcite{Bond1988} par exemple pointe déjà dans les années 1980 l'absence et la nécessité de la mise en place d'une boucle d'échange fructeuse entre disciplines autour des DAI : \foreignquote{english}{ Moroever, others have suggested that DAI may draw from and contribute to others disciplines, both absorbing and providing theorical and methodological fondations [ Chandrasekn81, Lesser83, Wesson81]} et d'ajouter plus loin la citation de Wesson en 1981: \foreignquote{english}{Fields of study heretofore ignored by AI : organization theory, sociology and economics, to name a few - can contribute to the study of DAI. Probably DAI advance these fields as well by providing a modelling technology suitable for precise specification and implementation of theories of organizational behavior } [Wesson81, p18] }

\Anotecontent{gilbert_date_clef}{Voici quelques jalons de ce mouvement relevés dans différentes sources :

 \begin{itemize}
  \item Avril 1992, à Guilford (UK) s'ouvre le premier workshop nommé \foreignquote{english}{Simulating Societies} qui donnera lieu à un tout premier ouvrage \autocites{Doran1994,Gilbert1994b}. S'ensuivront plusieurs autres workshop un peu partout en Europe, comme celui de Sienne en Italie l'année d'après en juillet 1993, qui donnera lieu à la publication d'un deuxième ouvrage important en 1995 \autocite{Gilbert1995a}.
  \item En 1995, une conférence sur cette thématique est donnée à Schoß Dagstuhl en Allemagne.
  \item En 1997 le \textit{first international conference on Computer Simulation and the Social Science} a lieu a Cortona en Italie. Celui çi est reconduit une deuxième fois en 1999 à Paris.
  \item Au printemps 1998, Nigel Gilbert annonce le lancement de JASSS, premier journal éléctronique ayant pour thème la simulation en sciences humaines et sociales. Celui ci est ouvert à une publication largement interdisciplinaire, et va s'imposer rapidement comme une référence dans ce microcosme qu'est encore la simulation en science sociale. La liste de diffusion \href{www.jiscmail.ac.uk/cgi-bin//webadmin?A0=simsoc}{@SIMSOC} voit également le jour cette année là.
  \item En 1999, Nigel Gilbert et Klaus G. Troitzsch publie le premier manuel  pour enseigner l'usage de la simulation à un plus large public. Depuis celui ci à été republié en 2005 \autocite{Gilbert2005}
 \end{itemize}
}

\Anotecontent{doran1982_reclamation}{
\foreignquote{english}{Several years ago \autocite{Doran1982}, I suggested that multiple agent systems (MAS) theory could form a basis of models of socio-cultural dynamics including the growth of social complexity. Since then MAS theory and distributed artificial intelligence (DAI) generally have developed substantially (\autocite{Bond1988} Gasser and Huhns 1989; Demazeau and Muller 1990 ) and now the idea of studying \enquote{societies} on computers is becoming not just tenable but fashionable - altought the emphasis is as yet largely on studying the properties of systems of abstract rather than realistic agents. In spite of this limitation, it now looks possible to develop my original suggestion in a more serious way, and briefly to compare it with the more prominent alternatives.} \autocite{Doran1997}

Preuve de sa connaissance dans le domaine de l'intelligence artificielle  il se réfère très tôt et de multiple fois \autocites{Doran1992, Doran1994a} à l'article très connu sur les DAI de Alan Bond et Les Gasser en 1988 \autocite{Bond1988}. EOS est donc, comme il le dit lui même dans \autocite{Doran1994a}, un double projet qui lui permet de développer des questions de recherche au croisement de ses travaux en intelligence artificielle distribuée et de l'archéologie. Une trajectoire de recherche qu'il cultive depuis longtemps comme en témoigne déjà ses travaux des années 1980 (projet CONTRACT, EXCHANGE \autocite{Doran1986b}). Le projet EOS se place en continuité de ses précédents travaux, et lui permet d'activer dès les années 1990, cette triple synergie entre un modèle archéologique de sociétés \autocite{Mellars1985}, des questionnements plus théoriques sociologiques, et le développement d'un \textit{testbeds} agent spécialisé (MCS/IPEM dévelopé en Prolog) au coeur de l'université d'ESSEX \autocites{Doran1992,Doran1991}}


\Anotecontent{inspiration_ferber}{ \enquote{Mais quelques travaux ont voulu rester dans les idées initiales que prônaient Hewitt et qu’il confirma avec ses notions de “sémantique des systèmes ouverts” (Hewitt 1991; Hewitt 1985). P. Carle (Carle 1992), S. Giroux (Giroux et Senteni 1992) et J. Ferber (Ferber 1987), tout en estimant que les langages d’acteurs sont effectivement de trés bons outils pour l’implémentation de calculs parallèles, considèrent néanmoins qu’ils présentent des caractéristiques tellement originales qu’ils modifient par leur présence la notion même d’architecture multi-agent en envisageant les agents et les systèmes multi-agents comme des extensions naturelles de la notion d'acteur.} \autocite[145]{Ferber1995} Par Systèmes Ouverts d'Information (\textit{Open Information Systems}) il faut comprendre un système \enquote{[...] dans lequel la connaissance n'est pas la somme des connaissances de tous les agents, mais la résultante de l'interaction de plusieurs micro-théories, c'est-à-dire de savoirs et savoir-faire associés à des agents.} \autocite[238]{Ferber1995}}

\Anotecontent{inspiration_wooldridge}{\foreignquote{english}{Throughout the 1970s, several other researchers developed prototypical multiagent systems. The first was Carl Hewitt, who proposed the Actor model of computation \autocite{Hewitt1973}}. \autocite[399]{Wooldridge2009}}

\Anotecontent{doran_85_DAI}{ \foreignquote{english}{In this paper I shall suggest that important problems of natural language and of individual and cultural knowledge mays usefully be approached by a computational route. Central to my argument will be the concept of a multi-actor system (sometimes called a \enquote{multi-agent system} in the research litterature). In artificial intelligence work, discussions of multi-actor systems typically envisage a collection of semi-autonomous computer controlled devices [...] which cooperate to perform some task in their common real world environment. However, an alternative is a single computer program which \textbf{simulates} actors in a modelled environment. In this case the aim is to use the study of a modelled multi-actor system to further understanding of real system -- both those that might be constructed and those human systems that are in existence around us.}\autocite[160]{Doran1985}}

\Anotecontent{doran_86_DAI}{\foreignquote{english}{This paper reports initial experiments with a computer program which embodies an abstract model of a sociocultural system. The model displays a form of spontaneous collapse. Central to the model is the adoption and discard of mutually beneficial and cumulative contracts between the component actors of the system.[...] Allen(1982) has argued the relevance of \enquote{dissipatlve structures} and multiactor system concepts to the emergence of modern urban structure including global and local fluctuations. The CONTRACT model I describe here has a number of aspects in common with Allen's work. The CONTRACT model is based on three main assumptions. The first is simply that a sociocultural system may usefully be modelled in abstract computational terms. The second assumption is that a sociocultural system may be regarded as a distributed problem-solver, that is, it is solving the problem of how best to manipulate its environment in order to maximise its own \enquote{wellbeing}. The system is distributed in that there arc multiple loci of decision, actors, each of which has only partial knowledge, and in that the criterion of success, \enquote{wellbelng}, is itself distributed over the decision making loci and locally defined. The third assumption is that the knowledge which the problem-solver necessarily uses to solve its problem is to be identified with cumulative technological knowledge cooperatively deployed.} \autocite{Doran1986b}}

\Anotecontent{doran_82_DAI}{ \textcite{Doran1982} présente un modèle générique pour étudier les comportements d'un système socio-culturel. Pour simplifier, les agents sont amenés à se structurer pour exploiter au mieux les ressources d'un environnement; structure dont l'émergence doit être le reflet des interactions (contrat) et des capacités de cognitions (représentations, mémoire, objectif) propres à chacun des acteurs décidant de participer à cette économie. Voici le résumé qu'il donne à un des schémas qu'il présente \foreignquote{english}{A set of concurrent actors, the multiactor system, is structured by a pattern of contracts that effects exploitation of the environment. Each actor has its own simplified and typically distorted representation(\enquote{cognized model}) of the multiactor system and environment, and this representation determines its individual contract participation.} Doran fait références plusieurs fois à la possible adéquation  entre les problématiques rencontrées dans de tels systèmes sociaux et les progrès faits par l'intelligence artificielle dans la résolution de problèmes en environnement distribué : \foreignquote{english}{We need the concept of a set of processes that run concurrently, which in some suitable way exchange information(\enquote{pass messages}) and which thus collectively effect some required computation [...] Discovering ways in which a system of concurrent communicating processes can engage in heuristic human-like problem-solving is an important current research topic (for example Smith1979). This work is closely relevant to the study of the capabilities of sociocultural systems [...]}}


\Anotecontent{mace_systeme}{\enquote{Autre \enquote{monstre sacré} de l’IAD, le système Mace développé par L. Gasser eut un impact considérable sur l’ensemble des recherches ultérieures en IAD \autocite{Gasser1987}. [...] L. Gasser, en reliant ses travaux à ceux de Hewitt sur les acteurs, montrait non seulement qu’il était possible de réaliser un SMA à partir de la notion d’envoi de message, mais aussi que cela n’était pas suffisant, une organisation sociale ne pouvant se ramener à un simple mécanisme de communication. Il faut en plus introduire des notions telles que les représentations d’autrui et faire en sorte qu’un agent puisse raisonner sur ses compétences et ses croyances. En outre, on doit distinguer, comme le faisait Mace, la compétence effective, le \enquote{savoir-faire} directement applicable, de la connaissance qu’un agent peut avoir de sa propre compétence. On peut dire que, peu ou prou, toutes les plates-formes actuelles de développement de SMA sont des descendants directs ou indirects de MACE.} \autocite[30]{Ferber1995}}

\Anotecontent{gilbert_confidence}{Le fait que \textcite{Gilbert2000a} fasse cette confidence dans un article intitulé \textit{Modelling Sociality : The View from Europe} dans l'ouvrage \textit{Dynamics in Human and Primate Societies} de \textcite{Kohler2000} n'est probablement pas anodin, et révèle l'existence passée d'un cloisonnement (au moins sur la forme) entre les deux foyers européen et américain sur ce sujet. Car face à l'ambition affichée de l'ouvrage d'Epstein et Axtell \foreignquote{english}{Growing artificial societies} \autocite{Epstein1996}, il est fort dommageable de voir que les projets européens, riches pourtant à cette époque d'un certain avancement sur le sujet, ne sont qu'à peine évoqués en introduction, la plupart des références n'allant pas en deçà des années 1990. Le modèle Anasazi étant initialement inspiré des travaux sur Sugarscape \autocite{Rauch2002}, la participation des européens à cet ouvrage pourrait en définitive être interprétée comme le signe d'une collaboration retrouvée.}

\Anotecontent{neumann_biologie}{ Depuis quelques années déjà il y a une redécouverte des écrits de Von Neumann, dont les réflexions/implications biologiques semblent aller au-delà de la seule \textit{Artificial Life} :   \foreigntextquote{english}[McMullin2000]{I claim that this reveals the true depth of von Neumann’s achievement and influence on the subsequent deveopment of this field; and, further, that it generates a whole family of new consequent problems which can still serve to inform—if not actually define—the field of Artificial Life for many years to come. [...]  Where I differ from these, and indeed, most other, commentators, is that I think it is a mistake to view von Neumann’s problem as having been wholly, or even largely, concerned with self-reproduction! Of course, this is not to deny that von Neumann did, indeed, present a design for a self-reproducing automaton. I do not dispute that at all. Rather, my claim is that this self-reproducing capability, far from being the object of the design, is actually an incidental—indeed, trivial, though highly serendipitous—corollary of von Neumann’s having solved at least some aspects of a far deeper problem. This deeper problem is what I call the evolutionary growth of complexity. More specifically, the problem of how, in a general and open-ended way, machines can manage to construct other machines more “complex” that themselves. For if our best theories of biological evolution are correct, and assuming that biological organisms are, in some sense, “machines”, then we must hold that such a constructive increase in complexity has happened not just once, but innumerable times in the course of phylogenetic evolution.}}


\Anotecontent{ancien_survey}{On trouves des informations intéressantes sur les premières et très diverses applications des automates cellulaires dans de multiples états de l'art, comme par exemple celui de \textcites{Smith1976,Hiebeler1990, Ganguly2003} et probablement bien d'autres.}

\Anotecontent{ordre_desordre}{\enquote{En réalité, et Ashby fut explicite sur ce point en 1962, toute sa cybernétique justifiait l’idée de l’impossibilité d’une \enquote{auto-organisation} dans un système n’interagissant pas avec son environnement, les changements organisationnels devant tirer leur source de l’extérieur du système – il critiqua d’ailleurs la pertinence même du concept d’\enquote{ auto-organisation }, en toute rigueur \enquote{ auto-contradictoire } : le système improprement dit \enquote{ auto-organisé } détecte au moyen de ses échanges avec son environnement et sous la forme de perturbations affectant ses \enquote{ variables essentielles } la \enquote{ variété } de cet environnement, et ne peut gagner lui-même de \enquote{ variété } qu’en collectant de l’information sur cet environnement ou en tentant de contrôler les échanges de matière et d’énergie qu’il entretient avec lui. La théorie de la \enquote{ variété } apportait en fin de compte un fondement logico-mathématique à l’idée dont l’origine se trouve chez Fechner et que nous avons vue opposée par Bertalanffy à Schrödinger dès 1949, selon laquelle l’ordre \enquote{ organismique } ne doit pas être pensé comme \enquote{ issu de l’ordre }, mais comme émergeant \enquote{ épigénétiquement } du chaos selon des principes inhérents aux systèmes dynamiques : Ashby donna une impulsion significative à ce qui allait devenir, notamment par l’intermédiaire de Prigogine et Atlan, le fameux principe d’\enquote{ ordre à partir du bruit }} \autocite[800]{Pouvreau2013}}



\Anotecontent{piaget_mossio}{\enquote{L’oeuvre de Jean Piaget représente une étape cruciale dans l’histoire des modèles de la circularité biologique. En allant au-delà du contexte de l’embryologie, Piaget élabore une approche conceptuelle générale qui vise explicitement à cerner les spécificités de l’autodétermination biologique, par la jonction théorique entre circularité, autodétermination et dimension thermodynamique. [...] L’objectif de Piaget est de rendre compatible l’idée d’un flux constant de matière et énergie entre l’organisme et l’environnement avec celle d’un ordre circulaire constitutif, qui maintient le système au cours du temps. Le concept de clôture de Piaget décrit la dynamique propre du vivant comme une forme d’autodétermination, dans le sens fondamental d’une connexion entre activité et existence, réalisée par un réseau circulaire de relations entre les composants de l’organisme, dont dépendent son unité et individuation. La distinction entre clôture organisationnelle et ouverture thermodynamique est le pivot théorique sur lequel les élaborations plus récentes de l’autodétermination biologique s’appuieront, plus ou moins explicitement. Cela vaut non seulement par rapport à la relation profonde entre stabilité de l’organisation et variations des processus sous-jacents, mais également en relation avec les interactions adaptatives de l’organisme avec l’environnement. Sur ce point, Piaget réinterprète et généralise les concepts d’assimilation et adaptation – initialement formulés par l’embryologie de Waddington – et décrit les interactions d’un organisme avec l’environnement en termes d’adaptation, conçue comme une assimilation des perturbations qui induit une \enquote{ autorégulation } interne (accommodation). Ainsi, l’organisation adapte le réseau circulaire de relations en fonction des perturbations, tout en maintenant la clôture qui réalise l’autodétermination} \autocite[12]{Mossio2014}}

\Anotecontent{biologie_pattee_ca}{\foreigntextquote{english}[\cite{Pattee2001}]{I tried several self-organizing schemes using automata models for generating and replicating simulated copolymer sequences (Pattee, 1961, 1965), but it became clear that the evolutionary potential of all these models was very limited. I eventually recognized a fundamental problem in all such rule-based self-organizing schemes, namely, that in so far as the organizing depends on internal fixed rules, the generated structures will have limited potential complexity, and in so far as any novel organizing arises from the outside environment, the novel structures have no possibility of reliable replication without a symbolic memory that could reconstruct the novel organization. The first computer simulation I felt had some interesting evolutionary potential was developed by Michael Conrad (1969) in which genetic, cellular, population, and ecological levels were all represented. However, other than abstract conservation principles, this was a physics- free model that did not address Pearson's question or the nature of symbols in measurement and control processes. (Conrad and Pattee, 1970).}}

\Anotecontent{vonneuman_openended}{\foreignquote{english}{One of the major achievements of von Neumann’s work was to clarify the logical relation-ship between description (the instruction tape, or genotype), and construction (the execution of the instructions to eventually build a new individual, or phenotype) in self-replicating systems. However, as already mentioned and as emphasised recently by McMullin (1992), his work was always within the context of self-replicating systems which would also possess great evolutionary potential.}}

\Anotecontent{taylor_reproduction}{Les deux termes réplication et reproduction sont souvent utilisés de façon synonyme mais renvoient en réalité à des études différentes, la première évoquant la capacité à reproduire une copie conforme, alors que la seconde renvoie au processus d'évolution naturel par la mise en oeuvre d'opérations et de matériels génétique \autocites{Sipper1998,Taylor1999}}

\Anotecontent{taylor_openended}{\textcite{Taylor1999} parle pour ce cas de \textit{Open-Ended Evolution} : \foreignquote{english}{This term refers to a system in which components continue to evolve new forms continuously, rather than grinding to a halt when some sort of `optimal' or stable position is reached[...] Note that open-ended evolution does not necessarily imply any sort of evolutionary progress.[...] Also, by using the term `open-ended' I wish to imply that an indefinite variety of phenotypes are attainable through the evolutionary process, rather than continuous change being achieved by, for example, cycling through a finite set of possible forms.}}

\Anotecontent{liaison_prigogine_foerster}{\enquote{ \textbf{Isabelle Stengers} : Les mathématiques dont parlent les physiciens aujourd'hui impliquent des notions comme points critiques, bifurcations ... Est-ce que çà a été nouveau pour vous, ou aviez vous déjà pensé à ce type de mathématiques ? \\
\textbf{Heinz Von Foerster} : Non. Cela a été nouveau. Le problème des bifurcations avait fait partie de ma formation en mécanique non linéaire. Je connaissais l'instrument mathématique. Mais ni moi ni mes collègues n'ont pensé à appliquer cette mathématique au problème de l'auto-organisation. Lorsque c'est arrivé, j'ai trouvé que c'était une manière très élégante de parler de cela parce que vous avez ces sauts non linéaires, qui sont très élégants. \\ \textbf{Isabelle stengers} : Est-ce que cela aurait pu vous apporter quelque chose ? \\ \textbf{Heinz Von Foerster} : Pas à l'époque des premiers papiers sur l'auto-organisation. Mais plus tard, nous avons été de plus en plus intéressés par le problème cognitif, qui associe organisation et cognition. [...]} \autocite[255-256]{CREA1985}}

\Anotecontent{histoire_sugarscape}{Le journaliste \textcite{Rauch2002} a interviewé Joshua Epstein à ce sujet en 2002 : \foreignquote{english}{One day in the early 1990s, when he was giving a talk about his model of arms races, he met Axtell, who was graduate student. He wound up bringing Axtell to Brookings, in 1992. Not long after, Epstein attended a conference at the Santa Fe Institue [...] At Santa Fe juste then a big subject was artificial life, often called A-Life. \enquote{All of the work was about coral reefs, ecology, growing things that look like trees, growing things that look like flocks of birds, schools of fish, coral, and so on,} Epstein told me. \enquote{And I thought, jeez, why don't we try to use these techniques to grow societies?} Fired up, he returned to Brookings and discussed the idea with Axtell. There followed the inevitable napkin moment, when the two of them sat in the cafeteria and sketched out a simple artificial world in which little hunter-gatherer creatures would move around a landscape finding, storing, and consuming the only resource, sugar.} Le modèle est écrit sur sa propre base logicielle \textit{Object Pascal}, c'est à dire découplé d'une plateforme multi-agents existante. Le code source n'est pas disponible en libre accès, et la plupart des versions que l'on trouve aujourd'hui sur les plateformes sont donc des réimplémentations basées sur la description faite les auteurs du modèle.}

\Anotecontent{futur_histoire_acteur}{Il faut comprendre que le formalisme \enquote{Acteur} est un terrain de recherche théorique riche de sa propre histoire et de ses propres influences dans le domaine de l'intelligence artificielle distribuée, dont on trouve récit dans les publications récentes des auteurs \autocite{Hewitt2014}. Dans cette dernière, Hewitt définit le \enquote{modèle Acteur} comme \foreignquote{english}{[...] a mathematical theory that treats \enquote{Actors} as the universal primitives of digital computation. The model has been used both as a framework for a theoretical understanding of concurrency, and as the theoretical basis for several practical implementations of concurrent systems. [...] An Actor is a computational entity that, in response to as message it receives, can concurrently: send messages to addresses of Actors that it has; create new Actors; designate how to handle the next message it receives.[...] The Actor model can be used as a framework for modelling, understanding, and reasoning about, a wide range of concurrent systems.}}

\Anotecontent{ton_difference}{Voir par exemple la différence de ton qui existe entre le \href{http://www.eoht.info/page/Information+theory}{@site}, mais aussi les notes de bas de pages de \autocite[277]{Lemoigne1977}}

