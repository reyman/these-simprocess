% -*- root: These.tex -*-

Un grand merci à Denise Pumain pour un nombre incalculable de choses ! Pour votre humanité, votre patience, votre confiance, votre support scientifique, moral et financier. Merci aussi pour toute cette énergie que vous avez su insuffler dans la création et l'animation de cette si belle équipe de modélisation, et cela pendant toute la durée de l'ERC GeoDiverCity. Ce fût un grand honneur et une formidable expérience de travailler sur des projets aussi innovants avec des gens aussi compétents pendant ces 5 dernières années.   

\medskip

Merci à Arnaud Banos, modélisateur et scientifique de choc (malgré son addiction pour tout ce qui clignote) pour nous avoir aussi bien conseillé à tout point de vue. Merci aussi pour toutes les opportunités scientifiques que tu nous as mis sous le nez. Tu nous as propulsé sur la grille avec L’ISC-PIF bien sûr, mais tu nous as également fait connaître ta famille de joyeux drilles du réseau MAPS, et la joie de se retrouver chaque année pour faire de la science tous ensemble. 

\medskip
Merci à Lena Sanders qui m'a permis d’intégrer cette formidable aventure humaine et interdisciplinaire qu’est l’ANR TransMonDyn. Tant de modèles de simulation à réaliser (et à évaluer !), tant de nouvelles problématiques à découvrir et de nouveaux terrains à parcourir avec nos amis les archéologues, linguistes et historiens.

\medskip
Un grand grand merci aussi à Thomas Louail et Florent Le Néchet, mais aussi Hélène Mathian et Mathieu Delage pour m’avoir permis d’intégrer ce super laboratoire qu’est Géographie-cités. Ce fut un honneur d’apprendre la modélisation et de faire mes toutes premières armes dans le multi-agents à vos côtés. Sans vous cette thèse n’existerait tout simplement pas.

\medskip
Merci à Hélène Mathian, une scientifique “tip-top” à la bonne humeur constante et communicative, toujours prête à aider et conseiller les doctorants dans leurs galères scientifiques, mais pas seulement :) 

\medskip
Merci à toute l’équipe du modèle épidémiologique MicMac. Je ne pensais pas qu’un jour on m’autoriserais à découper des gens pour les mettre ensuite dans des avions infectés ! C’était sans compter sur le pouvoir des mathématiques et de Nathalie Corson. Il me tarde de pouvoir rediscuter avec vous tous (Benoit Gaudou, Nathalie Corson, Arnaud Banos et Vincent Laperrière) des évolutions de ce super modèle Netlogo.

\medskip
Merci au jury d’avoir accepté de lire, de rapporter et d’examiner ce travail : Alexis Drogoul, Didier Josselin, Anne Ruas, et Frédéric Amblard. Merci également à Jean-Louis Giavitto pour sa participation à mon comité de thèse, et ses conseils précieux.

\medskip
Merci à tous les scientifiques qui ont acceptés de me confier une petite partie de cette aventure qu’a du être la découverte de l’informatique et des centres de calcul dans les années 1970. Je pense en particulier à Colette Cauvin, Philippe Cibois, Jean-Philippe Genet, et Alexandre Kyche.

\medskip
Je remercie également tous les géographes pionniers quantitativistes anglo-saxons pour toutes les anecdotes qu’ils ont bien voulu me confier sur les pratiques de la simulation dans les années 1950-1970 : Brian J. L. Berry, Duane F. Marble, Waldo R. Tobler, Michael Batty.

\medskip
Je tient également à remercier les autres scientifiques de disciplines diverses ayant accepter de répondre à mes questions : James E. Doran (archéologie et ABM) , David Hiebeler et Nelson Minar (Swarm et Artificial Life), Stéphane Bura et Eric Espenel (Simpop1).

\medskip
Merci à Franck Varenne d’avoir répondu le plus patiemment du monde à mes questions philosophiques naïves, et à mes demandes d’éclaircissements épistémologiques pas toujours très bien formulées. Merci également à David Pouvreau pour ses écrits passionants en biologie et pour son aide dans la compréhension des concepts au coeur du projet systémique de Bertalanffy.

\medskip
Merci à Saber Marrouchi évidemment ! Pour sa bonne humeur, sa disponibilité, sa gentillesse qui ont accompagnés toutes ces innombrables pauses cafés ayant rythmé cette aventure. Merci également à Véronique Degout et Martine Laborde dans la résolution (efficace et heureuse !) des problématiques administratives que rencontre toute vie de doctorant bien remplie.

\medskip
Merci à Clara Schmitt pour ces quatre années de binôme mémorables tant sur le plan scientifique que sur le plan humain. Merci aussi à Romain Reuillon et Mathieu Leclaire, dont le logiciel, les compétences et la bonne humeur ont rendu possible et agréable cette aventure des géographes en \textit{terra incognita} dans le monde du HPC. Merci à Paul, mon binôme du quatrième étage de la rue du four, roi du \textit{blind test} et de la métaphore, scientifique hors pair, mais aussi engagé dans une entreprise de soutien moral particulièrement appréciée dans ces derniers mois de thèse difficiles. 

\medskip
Merci infiniment aux relecteurs des derniers instants, en particulier Clémentine Cottineau et Robin Cura, mais aussi Paul Chapron et Hadrien Commenges. Je vous suis éternellement redevable tant vos commentaires et votre aide m’a été précieuse dans l’écriture et la relecture de ce manuscrit.

\medskip
Merci à tous les \textbf{lamibo} pour cette ambiance exceptionelle qui règne au 5ème étage du laboratoire. De façon quasi-aléatoire, je remercie donc tout autant : Solène, Clémentine, Olivier, Robin, Clara, Julie FC., Marion, Hadri, Paul, Stavros, Lucie, Julie G., Zoé, Romain, Brenda, Thomas, Juste, Odile, Florent, Elfie, Mathieu, Sylvestre, Antonin, Dorian, Pierre, Sylvain, Marion, Delphine, Dimitra, Charlène,  Antonio, Paola, etc.

\medskip
Je remercie également mes parents, ma soeur, grand Seb, et toute la famille ! A ce titre, et parce qu’ils font aussi partie de la famille, je voudrais également saluer tous mes amis du sud-ouest, pour tous les bons moments qu’ils ont su habilement placer sur un parcours de rédaction de plus en plus serré ces dernières années : Clément Céline (et Patapon !), Damien et Céline, Thibault et Florence, Charlotte et Hervé, Jonathan, Raphaël, Thomas. 

\medskip
Un grand merci à Emilie, bien sûr, qui m'a donné la force et le courage d’aller jusqu’au bout de cette aventure (presque toujours avec le sourire!) 
