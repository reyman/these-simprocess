% -*- root: These.tex -*-

\Anotecontent{doran_note}{Comme il me le confirme dans un échange par email daté de septembre 2014, James E. Doran rencontre Nigel Gilbert en 1984 lors de la conférence \textit{Artificial Intelligence and Sociology} \autocite{Gilbert1985} organisé par Gilbert et Heath : \foreignquote{english}{Yes I think I first met Nigel Gilbert at the meeting he organised with Heath.}}


\Anotecontent{description_imagine_simulation}{\foreignquote{english}{What the computer offers is the possibility of writing a computer program which embodies, at some level of abstraction, precise specifications of all the relevant factors and of their interactions. These specifications need not embody any general laws, nor need they be mathematical in form, but each must be operationally complete so that together they enable the machine to generate a possible « history » of the island for the period. In addition, the program will generate an estimate of what the consequential archaeological record would be. Given such a program, the task of the archaeologist would be to vary the factor specifications, using his own experience and insight, until the events and deposits predicted by the machine best matched the actual excavation evidence. It would, in fact, be very much a case of « reconstructing the events at the scene of the crime» with the machine doing the tedious task of moving the « actors» and « scenery». For our specific example, the simulation might have as its main components:
\begin{itemize}
\item (a) a fixed « map» of the island including information about climate, vegetation and fauna, together with
\item (b) a specification of the type of settlement characteristic of each population, including information about its size, material products and demand upon the natural environment, and
\item (c) rules specifying the dynamics of the system - the rules which determine where and when settlements are founded, when a settlement is abandoned, what forms of trade and conflict there are between settlements, and in what ways the material cultures of the populations evolve.
\end{itemize}
The machine would simulate the passage of time by repeatedly updating the map and the settlements « attached» to it by reference to the rules and specifications given - and the « history» so generated might well be both surprising and illuminating.}}

\Anotecontent{doran_archeologie}{\foreignquote{english}{When I went to Oxford University in 1959 (to study Mathematics) I joined the university archaeological society -- my interest in archaeology initially from programmes on TV. Then later on an Oxford archaeologist named Dennis Britten told me that Roy Hodson at the Institute of Archaeology in London was using maths/computing and I got in touch with Roy and worked with him -- see our 1966 Nature publication}(échange par email avec James E. Doran daté de septembre 2014)}

\Anotecontent{rencontre_renfrew}{Sa rencontre avec Renfrew aurait eu lieu avant la conférence de 1980 \autocite{Renfrew1982}, à l'Université de Sheffield, grâce à Roy Hodson : \foreignquote{english}{ Earlier, when he was at the University of Sheffield I think, very probably via Roy Hodson}(échange par email avec James E. Doran daté de septembre 2014)}

\Anotecontent{doran_Dai}{\foreignquote{english}{DAI started in the early 80s (eg \enquote{Contract Net}), and I would have been aware immediately because I was teaching a course on, in effect, \enquote{Latest in AI}. Some of the leaders of DAI produced DAI software testbeds eg MACE and even talked about doing social simulations. (See the volume of papers \enquote{Readings in DAI}. ) The link to archaeology, which I had from 1966 been publishing in, would have been pretty obvious I think. I just by chance knew about both subjects.} (échange par email avec James E. Doran daté de septembre 2014)}

\Anotecontent{renfrew_futur_archeology}{\foreignquote{english}{
There are the several elements that may come together to form this new morphogenetic paradigm in archaeology. The first is the concept of \enquote{system trajectory,} seen not merely in traditional system-theory terms, but in the dynamical sense facilitated by differential topology, including catastrophe theory. The second is the whole approach to self-organizing systems, pionereed by the \enquote{Brussels School,} which overlaps in some respects with the foregoing. The third is preoccupation with information flow, stressed by van der Leeuw, and cogently set out by Johnson in his chapter in this volume and in earlier publications. The fourth element is computer simulation, if it can be developed to cope with the complexity that we are dealing with in such a way as to escape the inflexibility of so many algorithms: The enthousiasm of Doran gives hope that it can.} \autocite[463]{Renfrew1982b}}



\Anotecontent{gardin_doran}{Gardin organised and chaired that meeting. See the proceedings which include record of discussions etc. I will check my copy if I can find it! However, I never worked with or talked directly to Gardin much at any time -- we had very different ideas. When Gardin and colleagues later worked on \enquote{simulation} he always, I think, meant simulating the archaeologist (typically, and rather strangely, with an Expert System!). By \enquote{simulation} I always meant. from 1970 onwards, what is now called agent-based modelling of eg a dynamic population of households}

\Anotecontent{gilbert_EOS}{\foreignquote{english}{We can now examine an example of a simulation based on DAI principles to see whether it fits neatly into any of these theoretical perspectives on the relationship between macro and micro. I have been associated with the EOS (Emergence of Organised Society) project since its inception, although Jim Doran and Mike Palmer are the people who have done all the work (Doran et al. 1994, Doran \& Palmer, Chapter 6 in this volume).} \autocite[128]{Gilbert1995a}}

\Anotecontent{note_bond_liens}{Dans un de ses articles \textit{Emergence in Social Simulation} \textcite{Gilbert1995} s'appuie sur le peu de questionnements réels dans la littérature reliant DAI et Sociologie. Il faut noter toutefois que ce n'est pas la première fois qu'une telle remarque est faite, et \textcite{Bond1988} par exemple pointe déjà dans les années 1980 l'absence et la nécessité de la mise en place d'une boucle d'échange fructeuse entre disciplines autour des DAI : \foreignquote{english}{ Moroever, others have suggested that DAI may draw from and contribute to others disciplines, both absorbing and providing theorical and methodological fondations [ Chandrasekn81, Lesser83, Wesson81]} et d'ajouter plus loin la citation de Wesson en 1981: \foreignquote{english}{Fields of study heretofore ignored by AI : organization theory, sociology and economics, to name a few - can contribute to the study of DAI. Probably DAI advance these fields as well by providing a modelling technology suitable for precise specification and implementation of theories of organizational behavior } [Wesson81, p18] }

\Anotecontent{gilbert_date_clef}{Voici quelques jalons de ce mouvement relevés dans différentes sources :

 \begin{itemize}
  \item Avril 1992, à Guilford (UK) s'ouvre le premier workshop nommé \foreignquote{english}{Simulating Societies} qui donnera lieu à un tout premier ouvrage \autocites{Doran1994,Gilbert1994b}. S'ensuivront plusieurs autres workshop un peu partout en Europe, comme celui de Sienne en Italie l'année d'après en juillet 1993, qui donnera lieu à la publication d'un deuxième ouvrage important en 1995 \autocite{Gilbert1995a}.
  \item En 1995, une conférence sur cette thématique est donnée à Schoß Dagstuhl en Allemagne.
  \item En 1997 le \textit{first international conference on Computer Simulation and the Social Science} a lieu a Cortona en Italie. Celui çi est reconduit une deuxième fois en 1999 à Paris.
  \item Au printemps 1998, Nigel Gilbert annonce le lancement de JASSS, premier journal éléctronique ayant pour thème la simulation en sciences humaines et sociales. Celui ci est ouvert à une publication largement interdisciplinaire, et va s'imposer rapidement comme une référence dans ce microcosme qu'est encore la simulation en science sociale. La liste de diffusion \href{www.jiscmail.ac.uk/cgi-bin//webadmin?A0=simsoc}{@SIMSOC} voit également le jour cette année là.
  \item En 1999, Nigel Gilbert et Klaus G. Troitzsch publie le premier manuel  pour enseigner l'usage de la simulation à un plus large public. Depuis celui ci à été republié en 2005 \autocite{Gilbert2005}
 \end{itemize}
}

\Anotecontent{doran1982_reclamation}{
\foreignquote{english}{Several years ago \autocite{Doran1982}, I suggested that multiple agent systems (MAS) theory could form a basis of models of socio-cultural dynamics including the growth of social complexity. Since then MAS theory and distributed artificial intelligence (DAI) generally have developed substantially (\autocite{Bond1988} Gasser and Huhns 1989; Demazeau and Muller 1990 ) and now the idea of studying \enquote{societies} on computers is becoming not just tenable but fashionable - altought the emphasis is as yet largely on studying the properties of systems of abstract rather than realistic agents. In spite of this limitation, it now looks possible to develop my original suggestion in a more serious way, and briefly to compare it with the more prominent alternatives.} \autocite{Doran1997}

Preuve de sa connaissance dans le domaine de l'intelligence artificielle et l'archéologie, ses articles précédents dans les années 1970, il se réfère très tôt et de multiple fois \autocites{Doran1992, Doran1994a} à l'article très connu sur les DAI de Alan Bond et Les Gasser en 1988 \autocite{Bond1988}. EOS est donc comme il le dit lui même dans \autocite{Doran1994a} en réalité un double projet qui lui permet de développer des questions de recherche au croisement de ces travaux en intelligence artificielle distribué et de l'archéologie, une trajectoire de recherche qu'il cultive depuis longtemps comme en témoigne déjà ces travaux (projet CONTRACT, EXCHANGE \autocite{Doran1986b}) au contact des nouveautés systémiques qui touche l'archéologie courant des années 1980. C'est donc dans la continuité de ces travaux que le projet EOS se met en place au début des années 1990, lui permettant d'activer cette triple synergie, entre un modèle archéologique de sociétés \autocite{Mellars1985}, des questionnements plus théoriques sociologiques, et le développement d'un \textit{testbeds} agent spécialisé (MCS/IPEM dévelopé en Prolog) au coeur de l'université d'ESSEX \autocite{Doran1992}}

\Anotecontent{doran_85_DAI}{ \foreignquote{english}{In this paper I shall suggest that important problems of natural language and of individual and cultural knowledge mays usefully be approached by a computational route. Central to my argument will be the concept of a multi-actor system (sometimes called a \enquote{multi-agent system} in the research litterature). In artificial intelligence work, discussions of multi-actor systems typically envisage a collection of semi-autonomous computer controlled devices [...] which cooperate to perform some task in their common real world environment. However, an alternative is a single computer program which \textbf{simulates} actors in a modelled environment. In this case the aim is to use the study of a modelled multi-actor system to further understanding of real system -- both those that might be constructed and those human systems that are in existence around us.}\autocite[160]{Doran1985}}

\Anotecontent{doran_86_DAI}{\foreignquote{english}{This paper reports initial experiments with a computer program which embodies an abstract model of a sociocultural system. The model displays a form of spontaneous collapse. Central to the model is the adoption and discard of mutually beneficial and cumulative contracts between the component actors of the system.[...] Allen(1982) has argued the relevance of \enquote{dissipatlve structures} and multiactor system concepts to the emergence of modern urban structure including global and local fluctuations. The CONTRACT model I describe here has a number of aspects in common with Allen's work. The CONTRACT model is based on three main assumptions. The first is simply that a sociocultural system may usefully be modelled in abstract computational terms. The second assumption is that a sociocultural system may be regarded as a distributed problem-solver, that is, it is solving the problem of how best to manipulate its environment in order to maximise its own \enquote{wellbeing}. The system is distributed in that there arc multiple loci of decision, actors, each of which has only partial knowledge, and in that the criterion of success, \enquote{wellbelng}, is itself distributed over the decision making loci and locally defined. The third assumption is that the knowledge which the problem-solver necessarily uses to solve its problem is to be identified with cumulative technological knowledge cooperatively deployed.} \autocite{Doran1986b}}

\Anotecontent{doran_82_DAI}{ \textcite{Doran1982} présente un modèle générique pour étudier les comportements d'un système socio-culturel. Pour simplifier, les agents sont amenés à se structurer pour exploiter au mieux les ressources d'un environnement; structure dont l'émergence doit être le reflet des interactions (contrat) et des capacités de cognitions (représentations, mémoire, objectif) propre à chacun des acteurs décidant de participer à cette économie. Voici le résumé qu'il donne à un des schéma qu'il présente \foreignquote{english}{A set of concurrent actors, the multiactor system, is structured by a pattern of contracts that effects exploitation of the environment. Each actor has its own simplified and typically distorted representation(\enquote{cognized model}) of the multiactor system and environment, and this representation determines its individual contract participation.} Doran fait références plusieurs fois à la possible adéquation  entre les problématiques rencontrés dans de telle systèmes sociaux et les progrès fait par l'intelligence artificielle dans la résolution de problèmes en environnement distribué : \foreignquote{english}{We need the concept of a set of processes that run concurrently, which in some suitable way exchange information(\enquote{pass messages}) and which thus collectively effect some required computation [...] Discovering ways in which a system of concurrent communicating processes can engage in heuristic human-like problem-solving is an important current research topic (for example Smith1979). This work is closely relevant to the study of the capabilities of sociocultural systems [...]}}