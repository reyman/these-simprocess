



article Pumain D., Saint-Julien T., Sanders L., 1986, Urban dynamics of some French cities. European Journal of Operational Research, 25, 3-10. Ainsi que Pumain D., Saint-Julien T., Sanders L., 1987, Application of a dynamic urban model. Geographical Analysis, n°1-2, 152-168, dans la thèse de Lena Sanders (Sanders1984), et dans un ouvrage collectif en 1989 “Ville et auto-organisation” (Pumain1989 p 112).


Replication supports model validation because validation is a process that determines a correspondence between the outputs from an implemented model and real-world measures. If the replicated model produces different outputs than the original model then that raises questions as to which outputs correspond more to real world data. If the replicated model's outputs are closer to the real world data that lends support to the validity of the replicated model as compared with the original model. More importantly, model replication raises questions about the details of the original modeling decisions and how they correspond to the real world. These questions help clarify whether there is sufficient correspondence between the original model and the real world. Replication forces the model replicater to examine the face validity of the original model by re-evaluating the original mapping between the real world and the conceptual model, since the replicater must re-implement those same concepts.


--

Guermond va décider d'embaucher directement dans leur équipes des mathématiciens/informaticiens comme Patrice Langlois. Une décision qui permet de maintenir un certain niveau technique dans le laboratoire, d’introduire de nouvelles compétences comme par exemple la manipulation d'Automates Cellulaires courant des années 1980, tout en assurant une certaine continuité dans l’enseignement de la programmation et des statistiques aux Géographes dans ces universités.


%Il semble bien donc que cela soit par la construction d’une coopération entre géographes et disciplines plus rompu à l’informatique, comme les statistiques, mathématiques ou la physique que se construisent majoritairement en France les nouveaux modèles de simulations.

%Un autre marqueur intéressant peut être soulevé, révélateur d'une époque ou le programme informatique n'a pas encore acquis de valeur patrimoniale, est celui d'une absence totale de stratégie pour la sauvegarde des modèles de simulations ainsi réalisés. Pour un modèle tel que Simpop 1 réalisé au début des années 1990, il est étonnant de retrouver les compte rendu de réunion semaine par semaine parfaitement conservé sous leur formes papiers et éléctroniques, mais aucune version éléctronique archivé, la gestion de cet aspect étant délégué de façon implicite au travail des informaticiens. Comme si la seule vrai valeur du modèle résidait plus dans sa fonction explicative, formalisatrice, permise par sa construction, plutôt que dans sa capacité à produire et reproduire un résultat.


Il n’y a pas je crois de travail de synthèse existant permettant d'acter en géographie la “possible” expression de ce désengagement des géographes dans la formation en “programmation”, et la façon dont elle pourrait se traduire à la fois dans les enseignements et son impact sur les projets dans les laboratoires de recherche plus spécialisé dans la modélisation. Car cette démocratisation de l’outil, si elle permet le passage  plutot \enquote{heureux} de certaine de ces techniques dans le vocabulaire courant du géographe (AFC; ACP; SIG; etc.) \autocite{Pumain2002} elle dessert également une autre vision de l'informatique, celle de l'informatique \enquote{boite-noire}. En effet qu’advient t il de la programmation comme activité “créative” dès lors que son autre fonction principale définissant l’apprentissage de celle-ci comme un \enquote{passage obligé vers d’autres applications} disparait au profit de logiciel plus simple à utiliser ? \textcite[4]{LeBerre1987} cite ainsi à propos de sa formation à l’informatique opéré dans un contrat entre la DGRST et le groupe Dupont, \enquote{il m’a fait refuser l’utilisation de l’informatique presse bouton, dangereuse pour le travail scientifique, et qui malheureusement se répand avec la diffusion des micro-ordinateurs}

Ce mouvement est peut être en train de s'inverser avec les efforts de la génération de modélisateurs précédentes, et l'avénement de support plus accessible pour la modélisation. Si ce combat n'est pas encore gagné, un autre nous attend déjà, il s'agit maintenant de ré-engager les géographes modélisateurs à utiliser la puissance mise à disposition par le HPC.




----

Un point déjà évoqué au début de notre exposé, les géographes modélisateurs ayant déjà expérimenté certains de ces challenges guidés par les évolutions technologiques. 

Dirigées vers l'usage du HPC pour l'exploration de modèle de simulation, des tentatives de calibration du modèle de Peter Allen se font dès les années 1980, guidées de façon automatique par un algorithme de minimisation basé sur une fonction objectif minimisant les écarts entre données simulées et données empiriques. Preuve que dans l'activité de modélisation, l'usage du HPC a aussi pu être à un moment donné motivé par un autre besoin plus méthodologique, que cette seule absence effective de micro-ordinateur pour exécuter les modèles.

Ce qui nous amène à un deuxième argument pouvant également expliquer la durabilité des relations avec certains centres de calculs. En effet, dans certains cas, les challenges scientifique d'hier sont aussi restés pour certains des challenges scientifiques d'aujourd'hui. La nécessité d'accès à une ressource informatique de puissance supérieure à un micro, reste comme dans le cas d'exploration des modèles de simulation, toujours une réalité.



En ce qui concerne la seconde question questionnant les usages HPC en SHS et en géographie vis à vis de leur \enquote{apparente} absence sur ce terrain, il faut bien noter que cet état de fait n'a pas toujours été vrai. Se posera donc ensuite la question suivante, que s'est il donc passé pour que l'on constate aujourd'hui une telle absence des géographes sur ce terrain ?

Avant de revenir plus en détail sur l'évolution des pratiques vers le HPC à Géographie-Cités, il me paraissait important de questionner cette absence d'intérét général pour le HPC, en les mettant en perspective d'une certaine actualité questionnant les pratiques l'enseignement de l'informatique.

Car si ce n'est pas vraiment l'absence de challenges, ou de géographes trop aventureux qui a pu faire oublier à la majeure partie des géographes cette existence de ressources informatiques inespérées au-delà du simple micro-ordinateur, alors quelles pistes de réflexion nous reste-t-il  ?

% transition à refaire.
Cette première tentative sera aussi l'occasion pour nous de faire le lien (section \ref{ssec:hist_pratiques}) entre cette première période d'accès à l'informatique, l'expression d'un premier besoin latent, la transformation du paysage dans le HPC, et l'installation puis la transformation des pratiques dans ce laboratoire Géographie-Cités, marquant un retour, espérons cette fois-ci définitif, vers l'usage du HPC pour l'exploration de nos modèles de simulation.

