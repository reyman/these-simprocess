




Replication supports model validation because validation is a process that determines a correspondence between the outputs from an implemented model and real-world measures. If the replicated model produces different outputs than the original model then that raises questions as to which outputs correspond more to real world data. If the replicated model's outputs are closer to the real world data that lends support to the validity of the replicated model as compared with the original model. More importantly, model replication raises questions about the details of the original modeling decisions and how they correspond to the real world. These questions help clarify whether there is sufficient correspondence between the original model and the real world. Replication forces the model replicater to examine the face validity of the original model by re-evaluating the original mapping between the real world and the conceptual model, since the replicater must re-implement those same concepts.


--

EN COURS PROBLEME DORDONANCEMENT.

Si on revient à l'étude des supports incitant les géographes à la programmation Fortran/Dynamo pour la modélisation, quelqu'uns sont bien publiés, contenant pour certains des codes sources ou algorithmes détaillés, comme celui de \textcite{Guermond1984}, \textcite{Dauphine1987}. Toutefois on est en droit de se questionner sur la mise en pratique réelle de ces enseignements informatiques pour la création de nouvelle simulation par les géographes. En effet en dehors de l'application de modèle existants, peu de \textbf{nouveaux} modèles de simulation, c'est à dire conçu et programmé pour aborder de façon spécifique une nouvelle question géographique, semble émerger dans cette période 80-90 en dehors des équipes et des modèles déjà cités.

En comparaison, alors qu'en France sort le livre d'André Dauphiné introduisant Dynamo/Stella aux géographes \autocite{Dauphine1987}, si on regarde ce qui se fait de plus innovant pour la simulation en grande bretagne, par exemple sous la plume d'\textcites{Openshaw1983, Openshaw1988, Openshaw2000}, c'est un fossé immense que l'on apercoit entre ces deux pratiques de l'informatique pour la simulation.

Guermond va décider d'embaucher directement dans leur équipes des mathématiciens/informaticiens comme Patrice Langlois. Une décision qui permet de maintenir un certain niveau technique dans le laboratoire, d’introduire de nouvelles compétences comme par exemple la manipulation d'Automates Cellulaires courant des années 1980, tout en assurant une certaine continuité dans l’enseignement de la programmation et des statistiques aux Géographes dans ces universités.


%Il semble bien donc que cela soit par la construction d’une coopération entre géographes et disciplines plus rompu à l’informatique, comme les statistiques, mathématiques ou la physique que se construisent majoritairement en France les nouveaux modèles de simulations.

La découverte en 1990 de tout nouveaux paradigme technologique comme la programmation orienté objet, ou les langages acteurs, supports principaux de la modélisation agents se fait donc soit par le biais d’informaticiens devenus géographes, ou par la mise en oeuvre de collaborations. Celle-ci, riche d’enseignement, ont permis de façon conjointe avec la démocratisation des outils de modélisations de faire évoluer progressivement ce schéma initial finalement assez rigide de collaboration avec les informaticiens, jusqu’à la constitution d’une équipe inter-disciplinaire qualifié non seulement pour raccrocher les tout derniers dévelopement informatiques, mais également capable de proposer ces outils dans un référentiel technique susceptible je pense d'intéresser une partie des géographes modélisateurs.

%Un autre marqueur intéressant peut être soulevé, révélateur d'une époque ou le programme informatique n'a pas encore acquis de valeur patrimoniale, est celui d'une absence totale de stratégie pour la sauvegarde des modèles de simulations ainsi réalisés. Pour un modèle tel que Simpop 1 réalisé au début des années 1990, il est étonnant de retrouver les compte rendu de réunion semaine par semaine parfaitement conservé sous leur formes papiers et éléctroniques, mais aucune version éléctronique archivé, la gestion de cet aspect étant délégué de façon implicite au travail des informaticiens. Comme si la seule vrai valeur du modèle résidait plus dans sa fonction explicative, formalisatrice, permise par sa construction, plutôt que dans sa capacité à produire et reproduire un résultat.


Ce que ne dit pas Openshaw dans sa métaphore, c’est que ce transfert de compétence, s’accompagne aussi d’un mouvement de démocratisation des techniques chez les géographes. Si certains géographes anglo-saxons des années 1970 ont pour certains accès à cette double compétence informatique dans leur formations (Batty par exemple programme depuis ses débuts, les différents pionniers américains Marble, Pitts, etc.), ce n’est par exemple pas le cas en France en Géographie, ou une phase d’acquisition des techniques informatiques, mathématiques et statistiques constitue un préalable à une plus large diffusion de la toute nouvelle géographie quantitative. 

Il n’y a pas je crois de travail de synthèse existant permettant d'acter en géographie la “possible” expression de ce désengagement des géographes dans la formation en “programmation”, et la façon dont elle pourrait se traduire à la fois dans les enseignements et son impact sur les projets dans les laboratoires de recherche plus spécialisé dans la modélisation. Car cette démocratisation de l’outil, si elle permet le passage  plutot \enquote{heureux} de certaine de ces techniques dans le vocabulaire courant du géographe (AFC; ACP; SIG; etc.) \autocite{Pumain2002} elle dessert également une autre vision de l'informatique, celle de l'informatique \enquote{boite-noire}. En effet qu’advient t il de la programmation comme activité “créative” dès lors que son autre fonction principale définissant l’apprentissage de celle-ci comme un \enquote{passage obligé vers d’autres applications} disparait au profit de logiciel plus simple à utiliser ? \textcite[4]{LeBerre1987} cite ainsi à propos de sa formation à l’informatique opéré dans un contrat entre la DGRST et le groupe Dupont, \enquote{il m’a fait refuser l’utilisation de l’informatique presse bouton, dangereuse pour le travail scientifique, et qui malheureusement se répand avec la diffusion des micro-ordinateurs}

Ce mouvement est peut être en train de s'inverser avec les efforts de la génération de modélisateurs précédentes, et l'avénement de support plus accessible pour la modélisation. Si ce combat n'est pas encore gagné, un autre nous attend déjà, il s'agit maintenant de ré-engager les géographes modélisateurs à utiliser la puissance mise à disposition par le HPC.

\begin{table}[!htbp]
\begin{sidecaption}[fortoc]{Classement basé sur la brochure pour les 20 ans du centre, ainsi que les dernières informations disponible sur le site de l'IDRIS}
	[tab:prankingIDRIS]
	\centering
	\begin{tabular}{@{}llllll@{}}
\toprule
\multicolumn{6}{c}{A l'IDRIS (1993 - maintenant)} \\ \midrule
Durée & Identifiant machine & Nom & Type & Processeurs & Puissance \\ \midrule
1993-1999 & Vectoriel Cray C98 & Atlas & vectoriel & 8 & 7,6 GFlops \\
1994-1999 & Vectoriel Cray C94 & Axis & vectoriel & 4 & 3,8 GFlops \\
1995-1997 & Cray T3D & Kaos & processeurs & 128 & 19,2 GFlops \\
1996-2002 & Cray T3E & Aleph & processeurs & 256 & 153,6 GFlops \\
1999-2006 & NEC & Uqbar & vectoriel & 38 & 304 GFlops \\
2002-2008 & IBM power 4 & Zahir & processeurs & 1024 & 6,5 TFlops \\
2008-2012 & IBM Blue Gene & Babel & processeurs & 40960 & 139 TFlops \\
2013 & IBM x3750 & Ada & coeurs & 10624 & 230 TFlops \\
2013 & IBM Blue Gene & Turing & coeurs & 98304 & 1258 TFlops \\ \bottomrule
\end{tabular}
\end{sidecaption}
\end{table}

Dans ce contexte d'abandon progressif des usages des centre de calculs dans la fin des années 1980, l'équipe de modélisateurs de géographie cités, bien consciente des problématiques de calibration décrites chez les anglo-saxon \autocite{Batty1976} s'est penché sur les techniques et surtout l'expertise disponible en France. Il y a donc eu des tentatives d'utilisation des centres de calculs pour la calibration de modèles, par exemple celui de Peter Allen, tentatives qui n'ont malheureusement pas abouti à l'époque, probablement du fait d'un argument qu'à également énoncé Openshaw en 88 : La puissance de calcul et les méthodes associés (les techniques multi-objectif progressent par exemple surtout à partir des années 1990 ! voir section \ref{sssec:historique_EA}) alors disponible n'était vraiment pas suffisante pour ce type d'application.

Cela dit, trente ans plus tard, pourquoi le modèle de Peter Allen n'a toujours pas été réévalué à la lumière de nouvelles techniques d'analyse dont l'évolution s'est calé sur une puissance informatique disponible presque mille fois plus importante qu'au début des années 1990 ?

Pour donner un ordre d'idée sur l'évolution des moyens de calcul disponible au CNRS de 1990 à 2015 on peut regarder l'évolution des supercalculateurs de l'IDRIS (ex-CIRCE fondé en 1993) dans le tableau \ref{tab:prankingIDRIS} : entre les deux calculateurs vectoriels Cray disponible en 93/94, le premier bond réalisé en 1996 par le passage à un autre type d'architecture parallèle, et la puissance aujourd'hui disponible sur les calculateurs Ada et Turing, on est passé du GFlops au PFlops (1 PFlops = 1000000 GFlops). On verra que ce n'est pas la seule révolution des années 1990, et que la présence de cette puissance à disposition, même si elle devrait être un motif suffisant pour amener les géographes à imaginer de possibles applications, doit encore être complété pour sa démocratisation par un mode d'accès plus adapté aux usages réels des modélisateurs.

