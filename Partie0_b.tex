% -*- root: These.tex -*-

\section{La validation des modèles de simulation}
\label{sec:constante_problematique}

Les termes \foreignquote{english}{Validation \& Verification} tels que définis par les institutions de normalisation sont conçus comme génériques et valables pour des disciplines autres que l'ingénierie logicielle (section \ref{ssec:triple_lecture}). Dans ce sous-ensemble de pratiques, la simulation dispose de sa propre branche historique, dans laquelle des spécialistes raffinent et organisent depuis les années 1960 ces notions en mettant en oeuvre des typologies d'outils et des méthodologies de conception et d'évaluation standardisées \autocite{Nance2002}. Ces définitions sont parfois reprises pour encadrer des travaux en sciences humaines et sociales, qui côtoient aussi une utilisation de ces termes en philosophie des sciences. L'ambiguité et le mélange des termes dans les publications semblent aussi courant que les débats sans fin \autocites{David2009,Augusiak2014}, ce qui nous obligent à regarder de plus près comment ces termes sont employés dans différentes disciplines (section \ref{ssec:triple_lecture}). Trois branches utilisant ces termes seront abordées : le courant historique de la \textit{V\&V} (section \ref{sssec:def_generique_validation}), la philosophie des sciences (section \ref{sssec:philo_sciences}) , et la communauté des proches modélisateurs (section \ref{sssec:validation_modelisateurs}).

% ssec:transition_annee70
% sssec:realite_neopositiviste
% sssec:progressive_systemique
% sssec:forrester_impact

La généricité et le manque d'incarnation géographique du point de vue de la \textit{V\&V}, l'approche philosophique très éloignée des pratiques ou la tendance à remarquer cette problématique de la Validation comme liée à une technologie particulière, sont des arguments qui nous poussent à reposer cette question de l'explication par la modélisation en prenant en compte son inscription historique.

Comme déjà entrevu à la fin du chapitre 1 (section \ref{ssec:crise_mutation}), les années 1970 sont considérées comme des années charnières. Il sera intéressant de mettre en perspective les arguments d'une géographie radicale critique des approches modélisatrices (néo-positivisme, fétichisme spatial\Anote{fetichisme_spatial}, etc.) avec la réalité des transformations touchant une branche quantitative en pleine évolution.

La section \ref{sssec:realite_neopositiviste} propose de déconstruire avec les arguments disponibles ce point de vue qui voit dans l'application pratique de la méthodologie néo-positiviste un support crédible à l'explication dans la construction de modèles en géographie (section \ref{sssec:realite_neopositiviste}). On peut s'intéresser à la diffusion des prémisses systémiques \autocites{Chorley1962, Berry1964a, Haggett1965,Harvey1969} semées par les géographes des années 1960 \ref{sssec:progressive_systemique} et soulever ainsi la réification en géographie d'un paradigme explicatif plus pertinent pour analyser les objets géographiques que les cadres logiques jusqu'alors empruntés aux influents Viennois Hempel ou Popper \autocite{Besse2000}. Des modèles explicatifs souvent cités, mais en réalité rarement appliqués ou même appliquables dans les faits, y compris dans d'autres sciences que géographiques \autocite{Bechet2013}. Cette mise en difficulté du modèle logique néo-positiviste laisse la place à d'autres modèles en philosophie des sciences, mais surtout donne à voir de nouveaux concepts dont certains trouvent écho de façon directe ou indirecte dans des problématiques plus opérationelles. C'est le cas par exemple des différentes expressions dérivées du problème de \enquote{sous-détermination} de Quine\Anote{varenne_quine} \autocite{Varenne2014}, ou de \enquote{l'équifinalité} d'abord biologique puis étendue à tous les systèmes-ouverts par Bertalanffy \autocite{Pouvreau2013}. La qualité des explications avancées par les modélisateurs doivent donc s'adapter à ce nouvel horizon, et se réinventer dans des discours, des méthodologies tenant compte de ces concepts.

% ssec:evaluation_construction
% sssec:hermann_contexte
% ssec:confrontation_sysmodelise_sysobserve
% sssec:equifinalite

Pour \textcites{Batty2001, Batty2005b} c'est le modèle \textit{Urban Dynamics} de \textcite{Forrester1969} qui cristallise le mieux ce changement de point de vue chez les modélisateurs de l'urbain (section \ref{sssec:forrester_impact}). Une transformation dans la façon de penser la construction des modèles qui s'accompagne aussi d'un certain regain d'intérêt \autocite{Batty1976} pour des branches de développement ayant toujours abordé la modélisation sous un angle \textit{bottom-up} et spatio-temporel en géographie \autocites{Hagerstrand1952, Hagerstrand1967a, Morrill1965, Morrill1965b, Marble1972, Ward1973}\Anote{marble_decline} ou dans des disciplines connexes \autocite{Orcutt1957}.

Il est alors intéressant de confronter le point de vue assez neuf de Forrester vis-à-vis de la construction des modèles et de la Validation, avec ceux des géographes et des courants de pensée de l'époque sur cette question, dont on a vu au chapitre 1 qu'elle arrive dans les débats sur la simulation dès la fin des années 1960 \autocites{Naylor1967, Hermann1967, Dutton1971, Guetzkow1962, Guetzkow1972}

De Naylor à Hermann, on observera dès les années 1970 une grande différence dans la façon de traiter la validation des modèles (section \ref{ssec:evaluation_construction}). C'est en partant ensuite des propositions très actuelles (section \ref{sssec:hermann_contexte}) posées par \textcite{Hermann1967} que l'on introduira les débats les plus récents sur cette question de la validation. L'objectif étant de déconstruire cette notion (section \ref{sssec:confrontation_sysmodelise_sysobserve}) jusqu'à développer un cadre explicatif plus compatible avec la construction et l'évaluation des modèles de simulation dans notre discipline (section \ref{sssec:equifinalite}), la géographie.

%Il n'est pas  ici de relater en détail cette construction d'une géographie radicale, humaniste ou comportementale, on retiendra seulement que ces courants se forment principalement à la convergence de problématiques politiques (crises économique nationales et internationales, guerres), de revendications théoriques (rejet des méthodes quantitatives et accusation de\Anote{fetichisme_spatial}) et/ou méthodologiques (retour de l’herméneutique).


%Les acteurs prônant une démarche scientifique teinté de néo-positivisme largement inspiré des sciences physiques sont alors la cible idéale de ces nouveaux acteurs, et vont subir un large front de critique.

%Gregory, dont on mobilise le point de vue pour critiquer la vision néo-positiviste / positiviste en géographie, utilise ce dernier argument de façon conjointe avec la pensée d'Habermas pour charger les dérives entraînées par les méthodes quantitatives, et proposer un autre style de pensée axé sur la réconciliation d'un point de vue structuraliste, phénoménologique et critique pour entre autre éviter l'écueil du \enquote{fétichisme spatial}\Anote{fétichisme spatial}. A la lecture d'ouvrage comme ceux de Gregory, dont la démarche de dépassement n'est pas sans levée des critiques pertinentes, il nous semble a posteriori que sa vision du mouvement quantitatif est en partie biaisé, d'une part parce que la réalité des pratiques peut tout à fait s'éloigner des discours tenus par quelques leaders d'opinion, tel qu'Harvey ou Bunge, et d'autres part parce que les critiques externes au mouvement, comme Gregory font mine d'ignorer une partie des transformations qui opère depuis le début des années 1970 en interne dans les pratiques visés.

%Ainsi, afin de montrer que la discipline géographique n'a pas attendu l'émergence de tels discours parfois extrémistes, nous avons aperçu dans la section \ref{ssec:crise_mutation} que les modèles de simulation économiques spatialisés, ont adopté au vu de leurs maigres résultats une démarche plus explicative permise entre autre par l'évolution des moyens de simulations, et que cette confrontation avec la problématique de validation a été formulée comme centrale par les modélisateurs pionniers et cela de façon explicite dans des ouvrages collectifs abordant cette question \autocite{Marble1972}. Si sur le fond il n'y a rien de critiquable à vouloir développer un autre style de pensée en opposition de certains excès constatés relatifs aux usages de ces nouvelles méthodes quantitatives, sur la forme il en résulte chez certains géographes l'émergence d'un amalgame malheureux qui associe un peu trop rapidement méthode quantitative positiviste, et modèle d'inspiration économique néo-libéraliste . Cette dualité opposant géographe (et géographie) qualitativiste/quantitativiste n'est plus considéré comme constructive \autocite{Sheppard2001}.

%Outre le fait que cette ouverture s'accompagne d'innovations méthodologiques permettant l'opérationalisation des concepts, s'ouvrent en parallèle avec la chute du néo-positivisme de nouveaux débats autour de l'explication \autocite{Hedstrom2010} à la fois chez les praticiens (les \enquote{mécanismes générateurs} de Boudon, les \foreignquote{english}{causal-mechanisms} plus récents des biologistes, les \foreignquote{english}{generative mechanisms} d'Epstein) mais également chez les philosophes des sciences en biologie (Salmon, Machamer, etc.) où les thèses de Popper-Hempel, bien que souvent citées, sont en réalité rarement appliquées ou même appliquables dans les faits \autocite{Bechet2013}.

%Un retour sur la démarche de construction des modèles en géographie s'avère nécessaire pour comprendre les éléments qui nous ont échappé dans la continuité de cette problématique qu'est la validation des modèles. En s'appuyant sur les témoignage de \autocite{Batty2001, Pumain2003} on parvient très bien à décrire ce basculement opéré à la charnière des années 1970, alors même que les géographes accèdent peu à peu aux concepts opérant dans le paradigme systémique \autocite{Harvey1969}, et que l'insuffisance des démarches de construction de modèles devient prégnante.

%L'enjeu ici est d'autant plus important qu'il se double d'une réalité opérationelle, faisant des problématiques de sous-détermination (Quine) ou d'équifinalité (Bertalanffy) des concepts tout à fait tangibles, dont la manipulation déborde du cercle des philosophes des sciences pour venir parasiter les débats des modélisateurs en SHS, dont la qualité des explications avancées doit s'adapter à cet horizon, et se réinventer dans des discours, des méthodologies plus spécifiques.



% -*- root: These.tex -*-

\subsection{Une lecture pluri-disciplinaire des problématiques liés à la validation}
\label{ssec:triple_lecture}

%Si le modélisateur est au courant des simplifications opérés dans les hypothèses censés représenter ,  la dynamique de construction introduit dans l'activité de construction des modèles une incertitude supplémentaire qui nous oblige à repenser l'activité de validation.


Notre tentative pour pointer quelque une des transformations touchant l'activité modélisatrice dans le giron universitaire entre 1950 et 1970 recoupe bien évidemment la description de vagues d'innovations déjà identifiées par ailleurs, et nous aurons l'ocasion de revenir plusieurs fois sur cette période charnière que sont les années 1970 par la suite. 

L'originalité du travail réside plutôt dans l'identification par une communauté de modélisateurs déjà très hétérogènes d'un ensemble de facteurs limitant le développement et la diffusion de l'activité de modélisation appuyé par l'usage de l'ordinateur. 

Fait quelque peu destabilisant pour un modélisateur cotoyant encore cette expression dans les publications en 2015, le \enquote{problème de la validation} apparait en effet très tôt parmis ces différents facteurs. 

Une question nous vient alors très rapidement à l'esprit, et mérite d'être posé, même si elle on verra par la suite qu'elle est un peu naïve. Comment se fait il que ce problème apparu il a presque 50 ans de façon quasi conjointe avec l'invention et l'adoption de la simulation par différentes communautées de pionniers modélisateurs soit encore aussi présente aujourd'hui, par exemple dans le cadre des publications de communautées fortement inter-disciplinaire comme JASSS ? 

En réalité, s'agit il vraiment du même problème ? Car entre la révolution quantitative partisanne d'une révolution plus globale qui voie les principaux verrou informatique s'effacer ou se déplacer, que faut il encore entendre d'une telle expression ? 

La difficulté d'analyser ce terme à la fois dans ses évolutions temporelles,  et dans sa diversité d'inscription disciplinaire 


\textbf{en interne}

par une communauté hétérogènes de modélisateurs comme dommageable pour la diffusion de cette activité de modélisation sur ordinateur. 


	détour d'une analyse rapide des mutations qu'a subit la modélisation en géographie au détour des années 1970, il est d'ores et déjà visible que la progression informatique n'apparait plus comme la seule contrainte 

Proposer une étude exhaustive de ce terme est délicat

\subsubsection{Les définitions de la validation en V\&V}
\label{sssec:def_generique_validation}

Les termes \foreignquote{english}{Validation \& Verification} ou \textit{V\&V} proviennent à l'origine de l'ingénierie des systèmes, et peuvent être rattachés au concept de \enquote{qualité} tel qu'il est défini par la famille de règles ISO établies par l'organisation mondiale de normalisation.

Décomposable en plusieurs branches cette discipline à part possède une branche dédiée à l'expertise logicielle. De ce fait, il n'existe pas réellement de définition ni de théories ou méthodologies officiellement acceptables, l'acceptation des termes pouvant varier fortement selon les branches d'application.

On trouve toutefois quelques références dans des livres dédiés à la terminologie standard pour la \enquote{gestion de projet} dans un large panel de disciplines, telle que le PMBOK (\textit{A guide to the project Management Body of Knowledge}) \autocite{PMBOK2013}. Résultats d'un travail certifié par des associations ou des organismes étatiques tels que IEEE et ANSI, ce dernier propose une définition générale de ces termes pour l'ingénierie logicielle :

\foreignquote{english}{Verification and validation (V\&V) processes are used to determine whether the development products of a given activity conform to the requirements of that activity and whether the product satisfies its intended use and user needs.}

et revient ensuite plus spécifiquement sur les termes :

\begin{itemize}
\item \textbf{Validation} \foreignquote{english}{The assurance that a product, service, or system meets the needs of the customer and other identified stakeholders. It often involves acceptance and suitability with external customers. Contrast with verification.}
\item \textbf{Verification} \foreignquote{english}{The evaluation of whether or not a product, service, or system complies with a regulation, requirement, specification, or imposed condition. It is often an internal process. Contrast with validation.}
\end{itemize}

Les termes tels qu'ils sont définis sont finalement bien trop généraux pour envisager de les appliquer tels quels dans notre domaine de compétence. Dérivé de la branche de l'\textit{Operational Research (OR)}, les auteurs de la communauté restreinte des \textit{systems analysis or modelling and Simulation (M\&S) } engagent dès les années 1960-70 des efforts pour standardiser ces définitions pour la simulation.

\Anotecontent{first_time_validation}{La citation de Churchman par \textcite{Naylor1966} est tiré de \autocite[165]{Nance2002} : \foreignquote{english}{\foreignquote{english}{X simulates Y} is true if, and only if, (a) X and Y are formal systems, (b) Y is taken to be the real system, (c) X is taken to be an approximation to the real system and (d) the rules of validity in X are non-error-free.} \autocite{Nance2002} }

Parmi les différents auteurs participant de ce mouvement ( Naylor, Finger, Oren, Hermann, Zeigler, Nance, Banks, Gass, Balci, Sargent, etc.), \textcite{Naylor1966} est considéré avec West Churchman (1963) comme un des tout premier à avoir attiré et cristalisé \Anote{first_time_validation} dans de multiples publications l'attention sur cette problématique importante de la V\&V.

Cet économiste formé à l'informatique dans la branche des \foreignquote{english}{management sciences} \autocite{Stricklin1985} est un des premiers en 1967 \autocite{Naylor1967} à publier dans un article nommé \foreignquote{english}{Verification of Computer simulation models} une méthode abordant spécifiquement la question de la crédibilité des connaissances qui peuvent être apportées par un modèle de simulation. Une méthode qu'il va mettre spontanément en tension avec les débats qui agitent la communauté des philosophes à cette même période.

Malgré ces efforts et sa volonté de porter le débat loin dans la communauté inter-disciplinaire (voir les premiers ouvrage collectifs sur l'usage de la simulation dans les \enquote{behavior science} \autocite{Dutton1971, Guetzkow1972} \hl{A verifier}), la démarcation entre les deux termes est encore peu claire \autocites[165]{Nance2002}[3]{Balci1986}. \footnote{\foreignquote{english}{Thomas Naylor, a coauthor of the book cited above, deserves credit for drawing major attention to the validation issue in the 1960s: Is the model actually representing the truthful behavior of the referent system? His work, above and in later publications (Naylor 1971, Naylor and Finger 1967), exerted a major influence in framing validation within different philosophical perspectives. Numerous techniques that can be used were identified or developed. While the issues of both verification and validation were of concern from the early days of simulation, often no clear distinction was made between the two terms.} \autocite[165]{Nance2002}}

\Anotecontent{balci_standard}{\foreignquote{english}{A uniform, standard terminology is yet nonexistent. A recent literature review \autocite{Balci1984} indicated the usage of 16 terms [...] Except some early papers which appearead between 1966 and 1972, model verification and model validation have been most of the time consistently defined reflecting the following differentiation : \textbf{model verification} refers to building the model right; and \textbf{model validation} refers to building the right model. \autocite{Balci1986}}}

Il faudra attendre le début des années 1980 pour qu'un standard émerge, grâce à des financements étatiques \autocite{Balci1986}, mais également du fait des efforts fournis par des auteurs comme Sargent et Balci \autocite{Nance2002}, qui collectent et organisent dans une typologie cohérente l'existant statistique et méthodologique, une activité qu'ils poursuivent encore aujourd'hui \autocite{Balci1998}.\Anote{balci_standard}

Pour \autocite[22]{Oberkampf2010} \foreignquote{english}{A Key milestone in the early work by the OR community was the publication of the first definitions of V\&V by the Society of Computer Simulation (SCS) in 1979 \autocite{Schlesinger1979}}, un des instituts avec la U.S GAO (U.S General Accounting Office) à fournir des spécifications en 1979 \autocite{Balci1986}

\begin{itemize}
\item \textbf{Model Verification} \foreignquote{english}{substantiation that a computerized model represents a conceptual model within specified limit of accuracy.}
\item \textbf{Model Validation} \foreignquote{english}{substantiation that a computerized model within its domain of applicability possesses a satisfactory range of accuracy consistent with the intended application of the model.}
\end{itemize}

\begin{figure}[h]
\begin{sidecaption}[fortoc]{Un des tout premiers schémas issus de la publication de la SCS \autocite{Oberkampf2010,Schlesinger1979}}[fig:S_VV]
  \centering
 \includegraphics[width=.7\linewidth]{schelinger_schema1979.png}
  \end{sidecaption}
\end{figure}

Même si elles sont plus anciennes et de portée moins générale, ces définitions de la \textit{V\&V} semblent plus pertinentes, car évoquées plus régulièrement par les chercheurs en sciences sociales; les travaux les plus cités étant ceux de \textcite{Kleijnen1995}, ou \textcite{Sargent2010} qui placent leurs travaux dans la continuité de ces définitions. L'avancée de leurs travaux peut être suivie en feuilletant les \textit{Proceedings of the Winter Simulation Conference} où la problématique de la \textit{V\&V} est réévaluée régulièrement au regard des nouvelles connaissances. Ce schéma \ref{fig:S_VV} est devenu un classique repris et régulièrement amendé \autocite{Sargent2010}. Voici la lecture qu'en fournit \autocite{Oberkampf2010}

\foreignquote{english}{The \textbf{conceptual model} comprises all relevant information, modelling assumptions, and mathematical equations that describe the physical process or process of interest. [...] The SCS defined \textbf{qualification} as \enquote{Determination of adequacy of the conceptual model to provide an acceptable level of agreement for the domain of intended application}. The \textbf{computerized model} is an operational computer program that implements a conceptual model using computer programming. Modern terminology typically refers to the computerized model as the computer model or code.}

Ce schéma a la particularité suivante, il \foreignquote{english}{ [...] emphasizes that \textbf{verification} deals with the relationship between the conceptual model and computerized model and that \textbf{validation} deals with the relationship between the computerized model and reality. These relationships are not always recognized in other definitions of V\&V [...]}

\Anotecontent{Kleijnen_def}{\foreignquote{english}{This paper uses the definitions of V \& V given in the classic simulation textbook by Law and Kelton (1991, p.299): \enquote{Verification\textbf{Verification} is determining that a simulation computer program performs as intended, i.e., debugging the computer program .... \textbf{Validation} is concerned with determining whether the conceptual simulation model (as opposed to the computer program) is an accurate representation of the system under study}. Therefore this paper assumes that verification aims at a \enquote{perfect} computer program, in the sense that the computer code has no programming errors left (it may be made more efficient and more user friendly). Validation, however, can not be assumed to result in a perfect model, since the perfect model would be the real system itself (by definition, any model is a simplification of reality). The model should be \enquote{good enough}, which depends on the goal of the model.}}

\Anotecontent{Sargent_def}{\foreignquote{english}{\textbf{Model verification} is often defined as \enquote{ensuring that the computer program of the computerized model and its implementation are correct} and is the definition adopted here. \textbf{Model validation} is usually defined to mean \enquote{substantiation that a computerized model within its domain of applicability possesses a satisfactory range of accuracy consistent with the intended application of the model} \autocite{Schlesinger1979} and is the definition used here. A model sometimes becomes accredited through model accreditation. Model accreditation determines if a model satisfies specified model accreditation criteria according to a specified process. A related topic is model credibility. Model credibility is concerned with developing in (potential) users the confidence they require in order to use a model and in the information derived from that model. A model should be developed for a specific purpose (or application) and its validity determined with respect to that purpose [...]A model is considered valid for a set of experimental conditions if the model’s accuracy is within its acceptable range, which is the amount of accuracy required for the model’s intended purpose.}}

Autrement dit, \foreignquote{english}{The OR community clearly recognized, as it still does today, that V\&V are tools for assessing the accuracy of the conceptual and computerized models.} Un avis partagé par \textcite{Kleijnen1995} \Anote{Kleijnen_def} , \textcite{Balci1998}, et \textcite{Sargent2010} \Anote{Sargent_def} mais également des auteurs de références sur le sujet dans les sciences humaines et sociales \autocite{Amblard2006} \hl{Prend le bout de texte la dessus}.

Seulement, cette forme de relâchement sur la correspondance entre réalité et modèle, et ce positionnement plus relativiste de la validation n'a pas toujours été une évidence; les premières définitions de Naylor par exemple, sont toujours usitées, et continuent si on en croit des auteurs comme \textcite{Kleindorfer1998} à semer le trouble dans certaines disciplines.

\Anotecontent{VV_philout}{ \foreignquote{english}{During the last two decades a workable and constructive approach to the concepts, terminology, and methodology of V\&V has been developped, but it was based on pratical realities in business and government, \textbf{not} the issue of obsolute thruth in the philosophy of nature} \autocite{Oberkampf2010}
\foreignquote{english}{A very old philosophical question is: do humans have accurate knowledge of reality or do they have only flickering images of reality, as Plato stated? In this paper, however, we take the view that managers act as if their knowledge of reality were sufficient. Also see Barlas and Carpenter (1990), Landry and Oral (1993), and Naylor, Balintfy, Burdick and Chu (1966, pp.310-320).} \autocite{Kleijnen1995}
\foreignquote{english}{With the strong interest in verification from the software engineering community, this contrasting but complementary explanation of the term was quite important. The effort to place valida- tion in a cost-risk framework moved the concept from a philosophical explanation in earlier works to a form more useable for simulation practitioners.} \autocite[165-166]{Nance2002}}

Mais en excluant ainsi de son analyse la partie subjective et philosophique de la \enquote{Validation}\Anote{VV_philout} pour se concentrer sur la seule partie opérationnelle, ces méthodologies restent pour le modélisateur une coquille vide décevante, qui demande encore à être incarnée thématiquement. Autrement dit, ces méthodes si elles prennent bien en compte la dimension dynamique et incrémentale nécessaire à la construction d'un modèle de simulation qui tendrait vers une réalité en accord avec la question posée, l'organisation des connaissances nécessaires pour guider ce processus reste à la lecture de ces typologies une opération quelque peu énigmatique pour les modélisateurs géographes. On retombe sur une des critiques soulevées précédemment dans la section \ref{sec:critiques_simulation} sur l'absence constatée dans les publications de méthodologie standard pour la validation qui prendrait en compte les problématiques spécifiques d'une discipline. \footnote{Aujourd'hui des disciplines comme l'écologie proposent des méthodologies plus spécifiques, comme la méthode POM proposé par Grimm sur lequel nous reviendront par la suite \hl{mettre une ref et un appel à la section}}

Une position compréhensible pour ces auteurs oeuvrant pour la standardisation, alors même que ces termes sont toujours d'usages toujours assez variables. Une des conséquences visibles tient dans ces incompréhensions et ces débats terminologiques sans fin \autocite{David2009} que l'on observe parfois en marge des discussions inter-disciplinaires. Cette gamme d'acceptions différentes tient souvent au transfert hasardeux des terminologies entre l'ingénierie des M\&S, la philosophie des sciences, et la thématique d'un chercheur en sciences sociales qui se retrouve en position intermédiaire de ces deux derniers. Un exercice d'équilibriste périlleux, car comme le fait remarquer \textcite{Kleijnen1995} en citant astucieusement une note de bas de page de \textcite{Barlas1990}, en philosophie il est tout à fait possible de voir la signification des deux termes inversée! \hl{Expliquez mieux que verification pourrait se traduire en philosophie pour certains par representation de la vérité, du “reel”, alors que le fait même de modéliser implique qu’on en soit loin}

\subsubsection{La philosophie des sciences}
\label{sssec:philo_sciences}

Il ne s'agit pas de se lancer ici dans un exposé historique des courants et débats s'étant succédés dans cette discipline, mais d'amener de façon illustrative et avec quelques références récentes l'émergence ces 20 dernières années d'une \enquote{épistémologie de la simulation} reprenant (en parasitant parfois le débat comme on l'a cité au dessus) de son point de vue certains débats évoqués chez les praticiens de la simulation; la question de validation étant comme on l'a vu dans le chapitre 1 un sujet de longue date chez les praticiens de la simulation, mais aussi chez les premiers acteurs fondateurs de la V\&V.

\hl{redite : L'objectif n'est donc pas tant de développer une argumentation critique exposant l'ensemble de ces points de vues, car ce n'est pas l'objet de cette thèse, que de tenter de s'insérer (et non de s'enfermer) dans ces réflexions en spécifiant en quoi celle ci diffère, néglige ou font peu écho à nos pratiques et réflexion historique en sciences sociales.}

Le premier obstacle avec laquelle les acteurs supportant cette nouvelle épistémologie doivent cohabités est évidemment la contre-argumentation questionnant cette même necessité d'opérer une nouvelle sous-division épistémologique. Car existe-t-il réellement des spécificité à la connaissance dérivé de l'étude de l'objet simulation, et si oui quelles sont elles réellement ? Autrement dit, existe t il une différence fondamentale entre les questionnements déjà posés dans le cadre d'une épistémologie des modèles et ceux évoqués dans le cadre d'une épistémologie de la simulation ?

\Anotecontent{frilosite_philoScience}{\foreignquote{english}{As computer simulation methods have made their way into novel disciplines, the issue of their trustworthiness for generating new knowledge has often loomed large, especially when they have competed for attention with experiments or analytically tractable modeling methods. The relevant question is always whether or not the results of a particular computer simulation are accurate enough for their intended purpose.[...] Given our long-standing preoccupation with issues of confirmation, it might seem obvious that philosophers of science would have the resources to easily approach these questions.} \autocite{Winsberg2013}}

parmi les auteurs ouvertement favorable à la création d'une nouvelle épistémologie, on citera entre autre les efforts de \autocites{Winsberg2001, Winsberg2009, Winsberg2013} qui pousse dans chacune de ses publications les \enquote{philosophes des sciences} à sortir de la seule étude de la \enquote{théorie de la confirmation} pour aller vers un terrain un peu plus aventureux \Anote{frilosite_philoScience}, celui de l'étude de la crédibilité des explications et des hypothèses dans leur dépendance au contexte.

Il propose de résumer l'originalité d'une telle épistémologie en évoquant l'inférence spécifique que produisent l'étude simultanée de trois point sur la simulation. \foreignquote{english}{ \textcite{Winsberg2001} argued that, unlike the epistemological issues that take center stage in traditional confirmation theory, an adequate EOCS must meet three conditions. 
downward, motley, and autonomous.[...] These three features were meant to be offered as conditions of adequacy; for which any adequate epistemology of simulation must account. Against the background of the growing use of simulation in the sciences, an adequate epistemology for the philosophy of science needs to explain the fact that simulation results and computational models are often taken to be reliable despite these three features. Winsberg (2001) argues that simulation requires a new epistemology precisely because traditional stories in philosophy of science about how knowledge claims get credentialed cannot explain them.}

Cette typologie a soulevé un certain nombre de critiques chez les philosophes des sciences, dont la plus longue et la plus argumenté est surement celle de \textcite{Frigg2009} dont on trouve le résumé des points saillants dans les publications de \textcites{Winsberg2009, Winsberg2013} mais également de bien d'autres auteurs qui se réfèrent à ce débat pour se positionner \textcites{Yanoff2010, Eckhart2010}.

Le deuxième point de débat intéressant réside dans le qualificatif souvent donné à la simulation de \enquote{laboratoire virtuel pour l'expérimentation}. Si les philosophes des sciences ne peuvent que s'incliner face au constat d'une telle banalisation du terme, dont nous avons donné nous même un aperçu de son ancienneté d'usage dans les sciences sociales dans le chapitre 1; il existe quand même chez les philosophes la volonté de mettre à l'épreuve les fondements et les conséquences pour la connaissance extraite d'une telle analogie.

\Anotecontent{HackingCartwright}{\enquote{Nos deux livres ont plus d'un point commun. L'un et l'autre accordent peu d'importance à la vérité des théories et avouent un faible pour certaines entités théoriques. Cartwright soutient que seules les lois phénoménologiques de la physique parviennent à la vérité tandis que, dans la partie B de ce livre, je fais remarquer que la science expérimentale est plus indépendante de la théorie que ce que l'on veut bien généralement admettre. Nous ne partons pas des mêmes postulats anti-théoriques car elle considère les modèles et les approximations alors que c'est surtout l'expérience qui m'intéresse, mais nos conceptions convergent.}\autocite{Hacking1983}}

\Anotecontent{Phan_Varenne_theorie}{\foreignquote{english}{Consequently, in the first neo-positivist epistemology, models were viewed not as autonomous objects, but as theoretically driven derivative instruments. Following the modelistic turn in mathematical logic, the semantic epistemological conception of scientific models persisted to emphasize on theory.} \autocite{Phan2010}}

Un débat d'autant plus actif qu'on assiste depuis ces 20 dernières années à un véritable renouveau des questionnements dans le cadre d'une \enquote{épistémologie de l'expérimentation} jusqu'alors relativement peu considéré par la majorité des philosophes des sciences \Anote{Phan_Varenne_theorie}. \textcites{Phan2008, Phan2010} citent ainsi les contributions importantes d'auteurs comme Fischer(1996), Galison (1987, 1997), Franklin (1986, 1996), Morrisson(1993, 1999), mais également les efforts de Hacking (1983) et Cartwright.

\Anotecontent{def_cartwright}{\enquote{Disons qu'il y a des théories, des modèles et des phénomènes. Il serait normal de penser que les modèles sont doublement des modèles. Ils sont modèles pour les phénomènes et modèles pour la théorie. [...] Le réalisme scientifique est ici tout particulièrement concerné. Cartwright est pour l'essentiel anti-réaliste à propos des théories. Pour cela, elle s'appuie en partie sur les modèles. Elle fait remarquer que non seulement les modèles ne peuvent être déduits de la théorie qui les englobe, mais plus encore que les physiciens utilisent à leur gré divers modèles qui, sans pourtant se recouper, cohabitent tous au sein de la même théorie. Et cependant,ces modèles sont les seules représentations formelles disponibles des lois phénoménologiques que nous tenons pour vraies. Elle affirme que seules ces lois phénoménologiques nous permettent d'avancer. Toutes les modélisations de ces lois ne peuvent être vraies ensemble puisqu'elles ne sont pas compatibles. Et rien ne permet de penser qu'un modèle est supérieur à un autre. Aucun n'est vraiment justifié par la théorie qui le porte. Plus encore, les modèles ont tendance à résister aux changements de théorie, c'est-à-dire que le modèle est conservé même si la théorie s'avère inadéquate. Il y a plus de vérité locale dans les modèles incompatibles que dans les théories, pourtant plus sophistiquées.[...] L'idéal de la science n'est pas l'unité mais dans une abondance et diversité de plus en plus grandes.} \autocite[350]{Hacking1983}}

\Anotecontent{def_hacking}{\enquote{Le \textit{réaliste à propos des entités} affirme que bon nombre d'entités théoriques existent vraiment. L'anti-réaliste s'oppose à ces entités qui ne sont pour lui que fictions, constructions logiques ou éléments d'un processus intellectuel d'appréhension du monde. Un anti-réaliste moins dogmatique dirait que nous n'avons pas, et ne pouvons avoir, de raison de supposer que ces entités ne sont pas des fictions. Peut-être existent-elles,mais le présupposer n'est pas nécessaire à notre compréhension du monde.

Le \textit{réaliste à propos des théories} dit que les théories
sont soit vraies, soit fausses et ce indépendamment de ce que nous percevons : la science, elle au moins, vise à obtenir la vérité et la vérité est le monde tel qu'il est. L'anti-réaliste dit des théories qu'elles sont au mieux
prouvées, adéquates, opératoires, acceptables - quoi-que incroyables, entre autres qualificatifs possibles.} \autocite[59]{Hacking1983}}

On retiendra principalement pour notre argumentaire cette propriété d'indépendance retrouvé de l'expérimentation par rapport à la théorie \Anote{def_cartwright}, dont on peut trouver un très bon manifeste dans les écrits de \textcite{Hacking1983} et Cartwright \Anote{def_hacking}, ces derniers se positionnant comme des antiréalistes des théories, tout en étant des réalistes des entités théoriques. Un point de vue très bien résumé à la fois dans \textcite{Hacking1983} et \textit{Théorie, Réalité, Modèle} de \textcite[226-231]{Varenne2012}

Sur la notion de modèle dans sa relation à l'expérimentation, il semblerait qu'un consensus se dégage chez les philosophes \autocites{Morgan2009, Varenne2013} autour du modèle perçu comme un \enquote{médiateur autonome} articulant théorie, pratiques et données dans un contexte spécifique d'une question et d'un cadre technico-social. \autocite[2]{Phan2010}

Il y a probablement un point intéressant à développer entre cette argument du modèle autonome, et les récents travaux en sciences sociales pour qualifier au travers d'une grille de lecture \autocites{Banos2013a, Sanders2013} le positionnement \autocites{Banos2013, Schmitt2013} et le déplacement des modèles de simulation au travers d'une part de leur construction \autocite{Cottineau2014b}, mais également de leur réutilisation \autocite{Schmitt2014}. Une autre façon de démontrer en quoi cette capacité à cumuler de façon flou différentes fonctions épistémiques donné dans la spécification minimale de Varenne pour la simulation \autocite{Varenne2013} est intéressante dès lors qu'il s'agit de tracer la trajectoire disciplino-temporelle de certains modèles : daisyWorld \autocite{Dutreuil2013}, Schelling \autocite {Bulle2005}, SugarScape, etc.)

\Anotecontent{winsberg_exper_simu_link}{Another unique feature of the epistemology of simulation is the ease with which it can draw inspiration from the epistemology of experiment.}

Les acteurs pronant comme Winsberg une épistémologie de la simulation n'hésite alors pas à débattre pour ce qui est des différents parallèle que l'on peut tracer avec les réflexions de cette communauté. \Anote{winsberg_exper_simu_link}.

Pour ne pas se perdre dans les différents points de vues sur le sujet et bénéficier d'une vue plus large incluant les réflexions des praticiens, on pourra se référer au travail opéré par \textcite{Varenne2001} dans son article \textit{What does a computer simulation prove?}, qui propose une lecture du débat au travers de d'une typologie soulevant trois grandes thèses : I - La simulation est elle un outils commes les autres \textit{A simulation is only a tool} ? II - ou bien l'équivalent fusionnel d'une expérimentation classique (\textit{A simulation is an experiment}) ? III - ou se positionne-t-elle comme médiateur entre la théorie et expérimentation ? (\textit{A computer simulation is an intermediate between theory and experiment})? 

%L'expérimentation mène sa vie propre et entretient diverses relations avec la spéculation, le calcul, la construction de modèles, l'invention et la technologie. Mais alors que le calculateur, le spéculateur et le constructeur e modèles peuvent être anti-réalistes, l'expérimentateur, lui, doit être réaliste. p18 

On trouve donc un grand nombres de travaux, toutes disciplines confondues (les philosophes des sciences ne sont pas les seuls à se poser ce type de question, comme nous verrons par la suite), qui tentent d'établir par le biais de différentes grilles de lecture l'appartenance de ce \enquote{nouveau?} mode d'expérimentation à une des catégories de cette grille. \textit{Pourquoi ? Au delà du jeu d'esprit, quel est l'enjeu motivant une telle comparaison ?}

\Anotecontent{moto_hacking}{Une remarque qui renvoie d'ailleur explicitement à sa lecture du moto d'Hacking \foreignquote{english}{experiments have a life of their own} et à la notion d'autonomie (\textit{autonomous}) de sa synthèse précédemment, qui marque le fait que dans certains cas (impossibilité d'observation, manque de données), la simulation doit faire la preuve des connaissances (\textit{background knowledge}) apportés sur appel de ses propres ressources.}

\Anotecontent{experimental_warranting_belief}{\foreignquote{english}{The central idea of this thread is that experiments are the canonical entities that play a central role in warranting our belief in scientific hypotheses, and that therefore the degree to which we ought to think that simulations can also play a role in warranting such beliefs depends on the extent to which they can be identified as a kind of experiment} \autocite{Winsberg2009}}

Partant du fait que l'expérimentation joue un grand rôle dans l'établissement d'une crédibilité pour les hypothèses avancés, il s'agit de mesurer à quel point la simulation serait susceptible d'apporter les mêmes garanties dès lors qu'on accepte de la voir comme une sorte d'expérimentation.\Anote{experimental_warranting_belief}

On s'appuiera dans la suite de cette argumentation sur la lecture de Winsberg, un philosophe des sciences que l'on estime plutôt partisan de la III thèse dans la classification ci dessus. Ce dernier s'appuie largement sur les travaux d'Hacking, mais aussi Galison pour construire sa réflexion, par exemple en arguant\foreignquote{english}{ [...] that some of the techniques that simulationists use to construct their models get credentialed in much the same way that Hacking says that instruments and experimental procedures and methods do; the credentials develop over an extended period of time and become deeply tradition-bound.} \autocites{Winsberg2003, Winsberg2013} \Anote{moto_hacking}

Winsberg résume ce débat en deux thèses opposés : \foreignquote{english}{Identity Thesis} qui consiste à dire que la simulation est littéralement une expérimentation, et \foreignquote{english}{Epistemology Identity Thesis} qui consiste à penser qu'il existe une dépendance entre les garanties de crédibilité qui pourront être accordé par les résultats de la simulation et leur capacité à être plus ou moins définie en tant qu'expérience. Si la première thèse semble assez bien correspondre au point I de la classification de Varenne, la deuxième semble être une sous-variation du point I.

La plupart des auteurs cités par la suite dans ce débat sont des philosophes des sciences spécialisé en économie (Guala , Morgan, Maki, Simon ) qui rejettent comme Winsberg (plus spécialisé en physique) assez naturellement ces deux thèse \autocite{Winsberg2009}, mais avec des arguments assez différents, qu'il convient d'évoquer pour bien comprendre la complexité de ce débat, assez théorique. 

\Anotecontent{maki_phan}{\foreignquote{english}{For Mäki, abstractions in models are similar to abstractions in experiments as they both can be interpreted as a kind of isolation [...] This analogy between models and experiments is called \enquote{isolative analogy} by Guala (2008). From Mäki’s standpoint, a model can be said to be experimented in its explanatory dimension: the finality of such a model is to explore the explanatory power of some causal mechanism taken in isolation.} \autocite{Phan2008}}

parmi les différents point de vue existant, on citera par exemple le sous-débat de l'\foreignquote{english}{isolative analogy} relaté ici au travers des publications de \textcite{Phan2008, Phan2010} apellant les points de vue de Morgan et Guala contre Maki (2005). Ce dernier voit dans la similitudes entre isolement théorique du modèle comme expérience de pensée et isolement expérimental \Anote{maki_phan} la possibilité de rejoindre une des deux thèses évoqués par Winsberg, établissant d'une façon ou d'une autre que \textit{les modèles sont des expériences, et les expériences des modèles}. Mais ce type d'argument, et on le suppose tout ceux qui se rapportent à l'évocation d'analogies pour justifier d'une équivalence de puissance épistémique se heurterai, comme on va le voir, à une différence fondamentale.

\Anotecontent{guala_phan_winsberg}{Winsberg résume le point de vue de Guala(2002) ainsi \foreignquote{english}{Guala argues that simulation differ fundamentally from experiments in that the object of manipulation in an experiment bears a material similarity to the target of interest, but in a simulation, the similarity between object and target are merely formal.}, mais on peut trouver une version réactualisé en 2008 dans l'article de \textcite[4.2]{Phan2010} \foreignquote{english}{In a simulation, one reproduces the behavior of a certain entity or system by means of a mechanism and/or material that is radically different in kind from that of a simulated entity (...) In this sense, \enquote{models simulate} whereas \enquote{ experimental systems} do not. Theoretical models are conceptual entities, whereas experiments are made of the same \enquote{stuff} as the target entity they are exploring and aiming at understanding}\autocite[14]{Guala2008}}

\textcite{Phan2010} et \textcite{Winsberg2013} cite le point de vue de Guala (2002, 2008), partagé par Morgan(2002, 2005) et se référant aux travaux de Simon (1969). Ceux-ci s'appuient sur une différence de relation qui existe entre système à étudier et système cible dans chacun des deux cas. En effet, dans le cas des expérience, la comparaison s'appuie avant tout sur une similarité matérielle, alors que dans le cas de la simulation la comparaison est limité à une comparaison formelle entre les objets.\Anote{guala_phan_winsberg}

\Anotecontent{Winsberg_critique_morvan}{\foreignquote{english}{Interestingly, while Morgan accepts this argument against the identity thesis, she seems to hold to a version of the epistemological dependency thesis. She argues, in other words, that the difference between experiments and simulations identified by Guala implies that simulations are epistemologically inferior to real experiments - that they have intrinsically less power to warrant belief in hypotheses about the real world.} \autocite[841]{Winsberg2013}}

Morgan(2002, 2005) accepte le point de vue Guala et Simon, mais s'en sert pour réduire indirectement le pouvoir épistémique de la simulation. Un argument bien résumé par \textcite{Phan2008} \enquote{Pour Morgan (2005) modèles et expériences partagent des fonctions de médiateurs et peuvent fonctionner \textit{sur un mode expérimental}, mais les expériences \textit{réelles} offrent un \textit{pouvoir épistémique} d'investigation de la réalité empirique plus fort.} Ce qui fait dire à Winsberg que Morgan serait indirectement plutot partisan de sa deuxième thèse.\Anote{Winsberg_critique_morvan}
\Anotecontent{winsberg_mereformal}{\hl{A compléter avec ce que dit Winsberg2013}}

Pour \textcite{Winsberg2009} le flou des arguments avancé par Morgan et Guala  (\textit{material similarity}, \textit{mere formal similarity}) ne permet pas d'exclure complétement et définitivement la première thèse.\Anote{winsberg_mereformal} Celui-ci se range malgré tout du coté de Guala, et préfère là aussi rejetter cette thèse, mais à la faveur de sa propre argumentation; ce qui lui permet de rejetter à la fois l'argument Morgan pointant l'infériorité épistémique de la simulation, et la deuxième thèse. Il argue que les simulations et l'expérience diffère principalement par la nature du \textit{background knownledge}, c'est à dire protocoles et les connaissances mobilisés.

Des modélisateurs et épistémologues en sciences sociales beaucoup plus proche de nos pratique comme Phan et Varenne trouve un argument convaincant dans ce dernier point, car \foreignquote{english}{Aujourd'hui, comme le souligne Winsberg, la crédibilité des modèles de simulation repose largement sur la \textit{confiance} que nous pouvons avoir dans les compétences des modélisateurs, informaticiens, expérimentateurs et observateurs, ainsi que dans les composants ou plateformes qui supportent les expériences de simulation.} \textcite{Phan2008}

\Anotecontent{gilbert_critique}{\foreignquote{english}{\enquote{[t]he major difference is that while in an experiment, one is controlling the actual object of interest (for example, in a chemistry experiment, the chemicals under investigation), in a simulation one is experimenting with a model rather than the phenomenon itself.} \autocite[14]{Gilbert2005}. But this doesn't seem right. [...] It is false that real experiments always manipulate exactly their targets of interest. In fact, in both real experiments and simulations, there is a complex relationship between what is manipulated in the investigation on the one hand, and the real-world systems that are the targets of the investigation on the other. In cases of both experiment and simulation, therefore, it takes an argument of some substance to establish the ‘external validity’ of the investigation – to establish that what is learned about the system being manipulated is applicable to the system of interest. Mendel, for example, manipulated pea plants, but he was interested in learning about the phenomenon of heritability generally \autocite{Winsberg2013}}}

\Anotecontent{guala_morgan_reality_experiments}{\foreignquote{english}{The identity thesis itself has drawn criticism from Guala (2002) and Morgan(2002). Guala begins by dismissing what he takes to be a poor argument against it. The poor argument goes something like this : simulations are not at all like real experiments because real experiments manipulate the real-world systems that are the very target of the investigation, while simulation merely manipulate \enquote{models} of the target system. What both Guala and Morgan correclty point out is that it is, quite generally speaking, false.}}

Autre sous-débat évoqués par \textcite{Winsberg2013}, on suppose en partie en réponse à sur son article précédent et très similaire \autocite{Winsberg2009}, la critique de l'\textit{identity thesis} comme évoqué par Gilbert et Troitzsch (1999), dont il pense \Anote{gilbert_critique}, en accord avec Guala (2002) \autocite{Winsberg2009} mais également Morgan et Parker \autocite{Winsberg2013} qu'elle est un argument trop faible pour rejeter l'\textit{identity thesis}. \Anote{guala_morgan_reality_experiments} 

Si les arguments de Winsberg semblent convaincant, \textcites{Peschard2010b, Peschard2013} tente dans une analyse critique d'en montrer les biais, et apporte dans son article des objections tout à fait crédible issue de son domaine d'expertise. Pour ne citer qu'un de ces argument, si il existe bien un intermédiaire de mesure issue d'un modèle, comme l'indique Winsberg, il existe également un sous système en prise directe avec la réalité physique de ce monde. En conclusion, elle estime que si il y a bien une certaine forme de similarité entre cibles épistémiques de la simulation et de l'expérience, pour elle ces activités ne peuvent pas être épistémiquement équivalentes, ce qui n'empeche en rien selon elle la coopération fructeuse des deux approches. \hl{Ajouter une footnote avec explication}

\textcite{Winsberg2013} résume le point de vue de \autocite{Peschard2010} ainsi, \textit{Thus, simulation is distinct from experiment, according to her, in that its epistemic target (as opposed to merely its epistemic motivation) is distinct from the object being manipulated.} Autrement dit, même si la motivation menant à l'expérience est bien eloigné (la motivation), l'objet manipulé dans une expérience est bien celui du monde physique, alors que dans le cas de la simulation c'est l'ordinateur. Or autant la motivation peut apprendre de l'objet manipulé dans le monde physique, autant il n'est pas ici dans notre intérêt d'apprendre sur l'ordinateur en tant qu'objet. Dans ce cas là on pointe une différence, mais on peut également appeler selon \textcite{Winsberg2013} et Morrisson (2009) l'argument inverse pointant au contraire une similarité. L'objet expérimenté étant le plus souvent choisi en tenant compte justement de sa capacité de \textit{surrogate} rapport à la question que l'on se pose effectivement, un point commun entre la construction de simulation et d'expérimentation. 

Winsberg conclu en ajoutant que l'expérimentation, contrairement à ce que l'on pourrait penser, n'est pas forcément et immédiatement plus crédible si on ne lui ajoute pas un bagage de connaissance : \textit{Experiments are not automatically more reliable than simulations, despite their differences. [...] It would seem that there are identifiable differences between ordinary experiments and simulations, but there is nothing about these differences that makes one or the other intrinsically more epistemically powerful.}  \autocites{Winsberg2009, Winsberg2013}

\textcite{Varenne2001} avance alors un autre argument intéressant : \foreignquote{english}{Indeed, when you read (Von Neumann 1951), you see that analog models are inferior to digital models because of the accuracy control limitations in the first ones. Following this argument, if you consider a prototype, or a real experiment in natural sciences, is it anything else than an analog model of itself? The test on the prototype is a real experiment. But is it something different and better than the handling of an analog model? So the possibilities to make sophisticated and accurate measures on this model - i.e. to make sophisticated real experiment - rapidly are decreasing, while your knowledge is increasing. These considerations are troublesome because it sounds as if nature was not a good model of itself and had to be replaced and simulated to be properly questioned and tested! It looks as if it was not possible any more to end a paper on simulation by reassuringly using the traditional word: \enquote{Simulation will never replace real experiments”.} }

Ces derniers paragraphes montrent que le débat est loin d'être fixé, et il semblerait là encore que ce soit la définition du contexte d'application qui détermine le mieux la capacité explicative de la simulation, car comme le dit Winsberg \enquote{l'impossibilité d'expérimenter} existe dans bien des disciplines, comme les sciences sociales, mais également la biologie ou la physique, ou les tentatives de reconstitution simulé d'univers ou d'étoiles dans des super calculateur de plus en plus puissant montre qu'il existe un interet explicatif à cette pratique. On pensera notamment aux projets d'expérimentation récents extremement complexe et couteux en physique (laser megajoule de bordeaux, projet ITER pour la fusion).

Et c'est sur ce point que l'argumentation de la plupart des philosophes des sciences est tout à la fois aussi intéressant que problématique. Pour continuer sur Winsberg, celui ci traite de ces problématiques en se positionnant uniquement du point de vue des sciences physiques. Un fait dont il reconnait prudement les conséquences que peuvent avoir l'inclusion d'un contexte différent sur sa synthèse : \foreignquote{english}{Parker (forthcoming) has made the point that the usefulness of these conditions is somewhat compromised by the fact that it is overly focused on simulation in the physical sciences, and other disciplines where simulation is theory-driven and equation-based. This seems correct. In the social and behavioral sciences, and other disciplines where agent-based simulation (see 2.2) are more the norm, and where models are built in the absence of established and quantitative theories, EOCS probably ought to be characterized in other terms.

For instance, some social scientists who use agent-based simulation pursue a methodology in which social phenomena (for example an observed pattern like segregation) are explained, or accounted for, by generating similar looking phenomena in their simulations (Epstein and Axtell 1996; Epstein 1999). But this raises its own sorts of epistemological questions. What exactly has been
accomplished, what kind of knowledge has been acquired, when an observed
social phenomenon is more or less reproduced by an agent-based simulation?
Does this count as an explanation of the phenomenon? A possible explanation?
(see e.g., Grüne-Yanoff 2007).

It is also fair to say, as Parker does (forthcoming), that the conditions outlined above pay insufficient attention to the various and differing purposes for which simulations are used (as discussed in 2.4). [...] Indeed, it is also fair to say that much more work could be done in classifying the kinds of purposes to which computer simulations are put and the constraints those purposes place on the structure of their epistemology.}

Des philosophes des sciences ont donc saisi cette opportunité de critiquer l'approche de Winsberg pour soulever dans des tentatives de typologies parfois intéressantes \autocite{Eckhart2010} les points de divergences que soulève l'utilisation d'une philosophies des sciences naturelle inadapté à la simulation en sciences sociales. Malheureusement, au cours de ces mêmes lectures, on constate que cette critique se retourne vers les modélisateurs et praticiens des sciences sociales, et mène cette fois ci dans une analyse incomplète du contexte historique au mieux à des interprétations erronés (voir le débat animé entre \autocite{Yanoff2008}  \autocites{Elsenbroich2012, Chattoe2011}), et au pire à des approximations et  conseils de mise en oeuvre totalement déplacé \autocite{Eckhart2010} vis à vis de disciplines qui disposent comme on l'a vu d'une véritable histoire autour de l'usage des méthodes computationelles.

Car bien que la recherche des points communs et des différences entre réalité de l'expérimentation physique et virtuelle apparaisse comme un débat intéressant, il faut bien avouer que celui ci ne peut que difficilement s'adapter à la quasi absence d'expérimentation au sens classique dans les sciences sociales. Ainsi, même si la simulation partage certaines des propriétés de l'expérimentation classique, il y a quand même quelque chose de paradoxal à vouloir absolument analyser le rapport de la simulation à l'expérimentation alors même que c'est cette absence qui justement motive son utilisation dans notre discipline, hormis peut etre pour mettre plus en avant cette incapacité à formuler un unique cadre fédérateur par un tel débat. Comme le dit très justement \textcite{Phan2008} {[...] les sciences économiques et sociales sont plus volontiers concernées par l’opposition entre \enquote{le modèle et l’enquête}  (Gérard-Varet et Passeron, 1995) que par celle entre \enquote{l’expérience et le modèle} (Legay, 1997)}

La notion de modèle vue comme médiateur autonome entre théorie et modèle doit elle aussi être repensé pour les sciences humaine, et la géographie; car les théories si elles peuvent exceptionnelement servir à dériver des modèles, celle ci ne peuvent qu'être difficilement rapporté à leur équivalent en science physique \autocite{Pumain1997}.

D'un coté les sciences physiques semble encore viser l'etablissement d'un cadre fédérateur alors qu'il semble que les théories et les modèles en sciences sociales - hormis peut être le cas particulier de l'économie - soit au contraire pourvoyeur de richesse dans leur capacité à apporter un nouvel éclairage sur un phénomène observé. \hl{a préciser peut etre}

A cela il faut ajouter que le modèle en géographie opère dans un cadre épistémique particulier qui n'est pas forcément celui de toute les sciences humaines. Ainsi, bien que les notions et le rapport entre les notions d'observation du \textit{singulier} et du \textit{général} soient théoriquement à la portée de toute disciplines \autocite{Dastes1992}, il semblerait que la géographie trouve un intérét particulier pour la constitution de sa démarche explicative à articuler des éléments de connaissance pris dans les grandes familles explicatives historique, écologique, et spatiale; justifiant ainsi de niveaux d'explication plus ou moins en interaction mobilisant chacun des déterminants de nature différentes. Avec la possibilité d'intégrer à tout moment dans l'explication les résidus qui tiennent d'un dialogue entre méchanismes généraux et singularité historique, écologique ou spatiale. Car quelque soit le registre explicatif choisi il reste dans les deux cas \textit{ [...] une part d'explication relevant de ce que l'on peut qualifier de singularités locales, non prédictibles à partir de mécanismes généraux, mais nécessitant d'appréhender l'histoire spécifique du lieu}. La conséquence étant une diversité de modèles support de l'explanan (l’explication que l’on propose du phénomène auquel on s’intéresse) dont l'évolution sur la forme et le fond n'a eu de cesse d'éclairer l'explanandum sous un jour différent. \autocite{Dastes1992, Sanders2000, Sanders2013} \hl{Mais il y a aussi le multi-échelles.}

\hl{A mieux introduire ici} En 1986, lors d'un débat sur les apports de l'AI en géographie, avec les apports de la branche discrete de la simulation, Couclelis montre que les considération philosophiques abordés jusqu'ici sont en réalité très vite abordés par les géographes \foreignquote{english}{A further insight to emerge from discrete model theory, which has some interesting philo- sophical implications, deserves a few comments. The widespread belief that proposing a model corresponds to the assertion that the real phenomena must have some similar structure is contradicted by the sharp distinction between structure and behavior drawn in discrete model theory. Although structure governs behavior, the converse is not true, so that obtaining a model that reproduces some behavior well does not entitle one to make any inferences about the \enquote{real} structure of the phenomenon represented. In fact, it is doubtful whether we may talk about the structure of real phenomena in other than a metaphorical sense. A real system may be no more than the universe of potentially acquirable data.} \autocite{Couclelis1986}

L'éclairage sur la méthodologie sous-jacente à la construction des modèles, pourtant un élément au coeur du raisonnement dans la discipline géographique depuis la révolution quantitative, a encore moins de chance d'être évoqué dans ces publications philosophiques, au détriment d'une réflexion statique plus axé sur la nature de l'objet simulation, et de sa relation au monde.

Or l'évolution des réflexions touchant l'activité de modélisation se construit il me semble à une échelle de reflexion tout à fait différente, celle contextualisé de pratiques guidés par une activité de résolution de questions spécifique à l'analyse spatiale, dont la mise en oeuvre s'appuie sur une chaine de traitements flexibles utilisant à bon escient et de façon cumulative l'arrivée historique de nouveaux outils et avec eux leur capacité à renouveller les questionnements : les statistiques, les modèles, les simulation.

\Anotecontent{ce qui n'est pas sans nous rapeller les difficultés évoqués dans le chapitre 1 sur l'inadéquation et le danger que représente les modes de transmissions actuels.}

\Anotecontent{remarque_Varenne_2001}{\foreignquote{english}{The second thesis of this article is that none of the three categories of arguments could be applied to contemporary sciences in general, whatever their objects, their methods and the moment of their history we consider. None of these three categories could be considered as the only true one. We cannot have a general point of view on the value of computer simulations, because of the different implications and meanings of mathematics in the different fields of science, and because of the various philosophies of nature at stake. This fact remains true for a given field throughout its own history, because the role of mathematics and the definition of the studied object evolve: You cannot find a unique and stable value that would be given to its simulation uses once for all. Again and hopefully, this thesis illustrates the fact that it does not belong to the historian to decide on the value of computer simulation in a given field but to the scientists themselves. These preliminary reflections prove the importance to investigate the intellectual history of contemporary sciences and not only their sociological construction nor their philosophical general insights.}\autocite{Varenne2001}}

Ces rapides remarques nous éclaire sur la latence qui existe entre la réflexion récentes d'épistémologues comme Winsberg, Grüne-Yanoff et la réalité théorique et pratique en géographie. \textcite{Varenne2001} avait déjà bien cerné dans la synthèse faite en 2001 qu'il n'y avait pas dans sa classification une position meilleure ou plus convaincante qu'une autre, la réponse se trouvant comme pour la notion de modèle avant tout dans l'étude du contexte, et donc de l'histoire des disciplines face à cet objet simulation. \Anote{remarque_Varenne_2001} Cela ne veut pas dire que les débats évoqués précédemment en sont automatiquement invalidés, seulement qu'il faut être probablement plus regardant vis à vis des remarques générales et des conclusions beaucoup trop hative qui peuvent parfois en découler. 

Les travaux croisés de praticiens (Pumain, Sanders, Banos), d'épistémologues ou historien des sciences propre à la géographie (Orain, Besse, Robic, Cuyala) ou s'en approchant (Varenne, Phan) permettent d'une part d'apposer un premier filtre sur ces réflexions génériques pour s'y référer prudemment, et d'autre part d'innover en questionnant nos démarches dans ce qu'elles ont d'originales, cette fois ci appuyé sur une lecture des pratiques certe pas toujours parfaite mais pouvant au moins être qualifié de \textit{bottom-up} 

%\hl{Un travail conséquent à la croisée de différentes approches, les travaux d'historiens et épistémologues des sciences somme Orain, Besse, Cuyala, et la lecture plus spécifique de l'évolution des méthodes numériques puis computationelles et de leur apports d'un point de vue pratique et théorique dont on trouve source à la fois dans les nombreux travaux des praticiens, mais également dans des travaux de plus long cours comme celui qu'est en train de réaliser Varenne dans son HDR.  == REDITE}

%\hl{ont su voir rapidement l'intérét de développer plus en avant les spécificités attachés à la simulation en science sociale, déjà riche de réflexion sur les apports successifs et cumulatifs de techniques de simulations \autocites{Banos2013, Varenne2008}, en s'intégrant au débat d'une communauté inter-disciplinaire structuré autour de la modélisation agent, qui émerge dans les sciences sociales au début des années 1990. Ce débat par contre ne fait semble-t-il que commencer dans le courant plus \textit{mainstream} des philosophe des sciences.}

%tel que celle des géographes pratiquant la simulation depuis les années 1950, tels que celle qui a émergé autour de la modélisation agent dans les années 1990. 

%Ainsi comme on a pu le voir dans le chapitre 1, le terme laboratoire virtuel pour l'expérimentation apparait très tot dans les sciences sociales, et des auteurs ont pour l'époque déjà donnés de très bonne raisons pour l'emploi de ce terme; les aspects dynamiques de la simulation en faisait partie.

\subsubsection{La Validation vue par une communauté de modélisateurs}
\label{sssec:communautes_jasss}

Depuis le début des années 1990 et la diffusion progressive du méta-formalisme agent \autocite{Treuil2008} dans les sciences humaines et sociales, les modélisateurs géographes peuvent, en plus des pratiques internes à la géographie, se tourner vers les discussions opérés dans une communauté d'acteurs internationaux et inter-disciplinaire. On trouvera sur ce sujet une tentative d'exploration des fondement historiques de ce mouvement dans l'annexe \hl{ref annexe}. Sorti des ouvrages fondateurs, c'est principalement autour du \textit{Journal of Artificial Societies and Social Simulation} (JASSS) fondé en 1998 que gravitent la plupart des discutants pertinents sur la problématique de la Validation. 

\paragraph{Persistance des guides méthodologiques et des évocations du problème de la validation}

Cette étape de validation, l'écueil le plus important surement, est pourtant souvent évoqué comme une étape cruciale dans bon nombres de guides méthodologiques pour \enquote{la bonne construction des modèles}, qu'il soit ancien \autocite[195]{Beshers1965} \autocite{Naylor1966, Naylor1967}, récent \autocite{Amblard2006, Gilbert2008}.

En prenant les écrits de \textcite[301]{Doran1975}, qui fait quasiment figure de co-créateur du mouvement de part son introduction des DAI à Gilbert \autocite{Gilbert1985} (voir également l'annexe déjà cité), on retrouve une filiation directe entre ces écrits de 1975 et 2000 sur cette question.

Ainsi on peut dire que \textcite[300-301]{Doran1975} dans \textit{Mathematical Models and Computer Simulations} est déjà au courant des travaux sur la question dans les sciences sociales, car il cite \autocite{Guetzkow1972}, et reprend à son compte un protocole de construction de modèle dont la description n'a finalement que peu bougé ces dernières années \foreignquote{english}{Any serious simulation study involves major effort at a number of stages : advanced planning; collection and organisation of suitable data; detailed specification of simulation; writing and initial \enquote{debugging} of the computer program; preliminary testing and validation of the program; it's use in the sequence of experiments designed to achieved specified objectives; and finally the study and interpretation of the results obtained. It is easy to underestimate the magnitude of the total effort required.It is all the more important to have a clear idea of what the simulation study is intended to achieve; either a broad investigation of the behaviour of simuland or, more likely, a determined attempt to examine the behaviour of certain variables of interest (cost? death rate? output? public approval?) and to discover to what extent they can be controlled.} 

Si la validation est décrite dans des termes classiques comme une comparaison avec les données, \textcite[301]{Doran1975} soulève par la suite quelles difficultés de l'expérimentation pouvant en définitive grever cette précédente tâche. Car comment valider corectement une simulation compte tenu de tout ces problèmes posés par l'expérimentation ? \foreignquote{english}{In any simulation, \textit{validation} is a matter of great importance. How can it be ensured that the model is indeed a reliable guide to reality ? [...] Once simulation has been validated it can be put to useful work. At this point a major problem appears. [...] any stochastic simulation must be run many times, and effectively one is sampling the behaviour of the variable of interest. This makes for much book keeping and for many complications.[...] Simulations poses two unexpected experimental problems. First, the number of variables and parameters in the simulation is liable to be very large, [...] Second, it may prove disconcertingly difficult to comprehend what is going on within the simulation, just as it is often difficult to comprehend a complex part of the real outside world. [...] These problems are of more than purely technical interest. They arise from the use of tool of sufficient complexity that it's details can extend human comprehension to the limit.}

En 2000, même si les termes se sont raffinés, et que de nouveau problèmes semblent avoir fait leur apparitions du fait des spécificités de l'outil (aspect cognitif par exemple), le lecteur n'est pas dépaysé sur le fond par la description que fait \textcite{Doran2000} des \foreignquote{english}{Hard problems in the use of agent-based modelling} : \foreignquote{english}{Skills and Time requierement, Which type of Model? , What level of Abstraction ?, Searching a massive parameter space ? The problem of validation ?}

Pour Openshaw, 

, une preuve qui vient s'ajouter à celle déjà évoqué dans le chapitre 1 pour justifier de l'enracinnement de cette problématique dans l'histoire de la simulation.


Si on observe bien une constance dans le développement des guides méthodologiques pour la construction de modèles de simulation \footnote{Rajouter comparaison entre deux suites de points}, l'étape ayant attrait à la validation ne reste que très rarement développé ou mis en oeuvre dans des exemples concrets \Anote{grimqualite}. Un exemple encore récent est la publication dans la revue JASSS (dont on rapelle qu'elle a été initié par Gilbert) à la critique cinglante de \autocite{Manzo2007a} vis à vis du guide méthologique établit récemment par \autocite{Gilbert2008}, très concis sur ces questions (7 pages au format A5). Or, si même les auteurs aussi cités et reconnus que Gilbert, qui ont la chance de publier pour la première fois cette méthode dans une collection aussi reconnu (\textit{Sage Quantitative Applications in the Social Sciences}), ne donne pas l'exemple, que doit on attendre des plus jeunes recrues s'essayant à cette technique ?

\hl{Transition}

Toutefois là ou naïvement, avec les évolutions de l'informatique, on aurait pu s'attendre à voir émerger dans les publications des applications concretes permettant d'aborder, même de façon incomplète, cette problématique, ce n'est pourtant pas du tout ce que l'on constate de façon générale.

Dans l'étude mené par \textcite{Heath2009} entre 1998 et 2008 sur 279 publications, l'auteur considère que seul 35 \% des modèles sont validés conceptuellement et informatiquement, même si une amélioration est à noter entre 2005 et 2008, ou ce chiffre monte à 43\% environ. Toutefois, si dans l'ensemble des modèles agents récupérés, on ne garde que les modèles agents classés en science sociale, ce chiffre chute fortement, avec 28\% de modèle validés, et quasiment 37\% de modèle qui n'aborde même pas ce problème \Anote{survey_heath}.

Si on regarde plus du coté des techniques, comme par exemple les analyse de sensibilités, cité régulièrement pour leur utilité dans la \enquote{validation interne} des modèles \autocite{Amblard2006}, le résultat n'est guère plus encourageant. Du moins si on en croit l'étude de \textcite{Thiele2014} entre 2009-2010 \Anote{survey_thiele}, mais également celle plus restreinte de \textcite{Cottineau2015} sur le volume JASSS de Mars 2014 \Anote{survey_cottineau}.

Finalement, sans même faire intervenir les critiques issue de débats plus épistémologiques (comme ceux que l'on a vu précédemment), cette seule insuffisance dans l'utilisation des moyens existants pour évaluer nos modèles suffit largement à prolonger un cercle vicieux où l'absence d'évaluation nourrit une perpétuelle remise en question de cet outil et de sa scientificité \Anote{serpent_mer}. Cette seule connaissances des dynamiques à l'oeuvre dans les modèles ne fait pas tout, mais elle offre déjà une base de discussion marqué par l'honneteté de cette démarche.

A l'image de certain auteurs, on pourra reprocher l'absence d'un protocole plus standard encadrant cette problématique. 

Cette discussion questionne aussi l'absence d'un protocole, d'un standard pour l'évaluation des modèles. %est cité comme un des nombreux écueils avec lequel se bat toujours la discipline, comme en témoigne les discussions réccurentes de nombreux auteurs sur un sujet dont la complexité touche à une dimension technique, que méthodologique ou philosophique. 

Autant d'auteurs \autocite{Richiardi2006} \autocite[198]{Fagiolo2007} \autocite{Moss2008} \autocite{Windrum2007} \autocite{Barlas1996} \autocite{Amblard2003} \autocite{OSullivan2004} \autocite{Doran2000} \autocite{Crooks2012} \autocite{Rouchier2013} dont il faudra par la suite développer les discussions.  


validation des modèles de simulation agents, on se rend compte qu'un certain nombre de problématiques persistent et limitent toujours la diffusion des modèles en dehors du cercle bienveillant de l'inter-disciplinarité \autocite{Richiardi2006}. Ce problème, loin d'être un isolat touchant uniquement les sciences humaines et sociales, existe également dans d'autres disciplines, comme en écologie, où même lorsque les modèles sont publiés, l'absence de protocole pour répliquer, évaluer le modèle est courant \autocite{Grimm1999}. \hl {ref à vérifier}

Un problème qui met un peu plus en danger des communautés de chercheurs dont on sais déjà leur isolement dans certain pays : archéologie, sociologie \autocite{Manzo2007}, et économie \autocites{Lehtinen2007, Richiardi2006} pour n'en citer que quelqu'un. 

cette critique récurrente de l'outil sur le plan de la scientificité, une faiblesse qui constitue toujours un danger pour la pérénité des pratiques inter-disciplinaire autour de la simulation agents \autocite[220]{Squazzoni2010}, a tel point que la communauté se dote de guides de survie pour se protéger des sceptiques. \autocite{Waldherr2013} Est ce là le seul fait d'une mauvaise communication autour de notre discipline comme le laisserait penser la lecture de ces travaux ? 


A : Les approches appliqués les plus sérieuse nous paraissent celle de Grimm, et celle de Behavior search, elles seront commenté en conclusion.

= Article de référence est clairement Amblard 2006

L'article de référence pour la validation de modèle agent en géographie est clairement celui d'Amblard2006, dans le livre dédié à l'agent. Celui-ci ne comportement pas malheureusement de volet applicatif.

= Les géographes interviennent dans cette communautée, exprime forme de détachement

A vouloir mettre en évidence un protocole de validation générique axé autour du seul modèle de simulation, on applanit sans le vouloir une démarche de construction des connaissances mobilisant un ensemble de modèle et de méthode. Il est important de rapeller sans cesse le poids de cet héritage disciplinaire à l'interface de questionnement plus génériques, comme le font régulièrement dans notre discipline Hélène Mathian, Léna Sanders, et plus récemment Cottineau2014b.

Les géographes modélisateurs, même si ils interviennent également au contact de cette communauté, continuent pour certains à développer une approche ou le SMA n'est qu'un outil parmis d'autres hérités d'une longue histoire de la géographie avec la simulation. La question de la validation s'inscrit dans une démarche beaucoup plus globale ou interviennent des multiplicité de modèles et de méthodes.

= Geographes \textbf{centré sur la démarche} et non sur l'outil

Pour raccrocher cette remarque avec les dernières analyses portant sur l'usage des techniques de simulation en géographie, les observateurs (Varenne) tout autant que les acteurs \autocite{Sanders2013}\Anote{sanders_couplage_spirale} de ces pratiques tendent à mettre en avant une tendance croissante à la pluri-formalisation croissante des modèles agents, preuve que cette multiplicité des formalismes et des techniques est de plus en plus considérée comme une richesse dans l'approche de systèmes complexes.

Dans son analyse sur la place de l'explication en analyse spatiale, \textcite{Sanders2000} voit plus dans le rapport de l'outil à la démarche adoptée (exploratoire ou hypothético-deductif) une question d'interprétation, ce qui contextualise encore un peu plus la place de l'outil \enquote{Ce sont en effet la manière dont l'outil est inséré dans une chaîne de réflexion et de traitement et l'interprétation des informations figurant en entrée et en sortie qui permettent de donner le statut descriptif ou explicatif d'une démarche.} Avant de conclure un peu plus loin à la fin de son analyse \enquote{[...]il n'y a pas de relation simple et fixe, ni entre les niveaux d'observation et la nature des explications, ni entre les outils de l'analyse spatiale et l'explication. La variété des approches permet de diversifier les éclairages sur les phénomènes que l'on cherche à expliquer. Nos démarches en analyse spatiale consistent ainsi davantage à \enquote{éclairer} qu'à \enquote{démontrer} et identifier des jeux de causalités bien stricts.}

Pour Maryvonne le Berre encore, \enquote{Il n’y a donc pas d’adaptation de la géographie à la technique mais recherche d’une technique adaptée à chaque objet d’étude géographique}. \textcite{Sanders2013} expose une idée assez similaire de pré-existance des concepts sur l'outils, car pour elle \enquote{La comparaison entre famille de modèles montre qu'il existe des décalages temporels entre la construction conceptuelle d'un champ et la mise au point d'outils appropriés pour la tester. Dans une certaine mesure on peut avancer que les outils \enquote{ont rattrapé} les concepts dans ce champ de recherche.}

= Reste qu'il faut quand meme un support à ce developpement


\subsubsection{synthèse}

% Faire la synthèse des trois approches, notamment en redescendant la partie déjà ecrite à la fin de la partie philo, c'est mieux de noter la spécificité de la géographie pour justifier le fait d'une introspection dans les habitudes de construction en géographie.

Si la communauté \textit{Models\&Simulations} propose aujourd'hui un cadre d'analyse cohérent avec la dynamique attendue chez les géographes pour la construction des modèles, il lui manque toutefois une incarnation géographique qu'il va falloir extraire de nos propres exigences de construction.

Pour comprendre comment la notion de validation se construit en marge de ces deux discours, il faut revenir sur ce qui fait sens dans l'explication pour les géographes. En repartant des transformations que subit la géographie quantitative dans les années 1970 au contact du paradigme systémique, prise dans une nouvelle réflexion des objets géographiques , dont la percolation chez les géographes s'observe dans la nouveauté le champs lexical, les méthodes, mais également les techniques.

Au delà des outils il y a un fond commun, à la construction de modèle en géographie, et qui n'est que très rarement traité dans ces lectures, celui de la mise en œuvre de la construction. Nous verrons qu'il y a des raisons à cela, la dépendance au contexte en est une, et cette activité hautement flexible, bien qu'influencé par la qualité du substrat technique, pose dans la mobilisation des hypothèses des questions génériques à ce dernier : l'hypothèse que j'ai choisi apporte elle un éclairage sur la question posé qui guide la construction du modèle ?

- Oui, et je peux le prouver 

- Non, mais cela ne prouve pas que cette hypothèse n'est en soit pas valable 

Il ya une forme de permanence dans les questions posés par la construction d'un modèle ou d'un modèle de simulation qui apelle à la construction d'une vision de la validation plus appliqué en géographie => ce qui veut dire qu'il lui faudra un support, et cela viendra par la suite

\hl{--------------- Pas fini, tu peux sauter à la partie d'après :) ------}




% -*- root: These.tex -*-

\subsection{La difficile validation des modèles de simulation explicatif en géographie}
\label{ssec:validation_compatible_shs}

\subsubsection{La lecture multiple des problématiques liés à la validation}
\label{sssec:triple_lecture}

%Si le modélisateur est au courant des simplifications opérés dans les hypothèses censés représenter ,  la dynamique de construction introduit dans l'activité de construction des modèles une incertitude supplémentaire qui nous oblige à repenser l'activité de validation.

\paragraph{Les définitions de la validation en V\&V}
\label{ssec:def_generique_validation}

Les termes \foreignquote{english}{Validation \& Verification} ou \textit{V\&V} proviennent à l'origine de l'ingénierie des systèmes, et peuvent être rattachés au concept de \enquote{qualité} tel qu'il est défini par la famille de règles ISO établies par l'organisation mondiale de normalisation.

Décomposable en plusieurs branches cette discipline à part possède une branche dédiée à l'expertise logicielle. De ce fait, il n'existe pas réellement de définition ni de théories ou méthodologies officiellement acceptables, l'acceptation des termes pouvant varier fortement selon les branches d'application.

On trouve toutefois quelques références dans des livres dédiés à la terminologie standard pour la \enquote{gestion de projet} dans un large panel de disciplines, telle que le PMBOK (\textit{A guide to the project Management Body of Knowledge}) \autocite{PMBOK2013}. Résultats d'un travail certifié par des associations ou des organismes étatiques tels que IEEE et ANSI, ce dernier propose une définition générale de ces termes pour l'ingénierie logicielle :

\foreignquote{english}{Verification and validation (V\&V) processes are used to determine whether the development products of a given activity conform to the requirements of that activity and whether the product satisfies its intended use and user needs.}

et revient ensuite plus spécifiquement sur les termes :

\begin{itemize}
\item \textbf{Validation} \foreignquote{english}{The assurance that a product, service, or system meets the needs of the customer and other identified stakeholders. It often involves acceptance and suitability with external customers. Contrast with verification.}
\item \textbf{Verification} \foreignquote{english}{The evaluation of whether or not a product, service, or system complies with a regulation, requirement, specification, or imposed condition. It is often an internal process. Contrast with validation.}
\end{itemize}

Les termes tels qu'ils sont définis sont finalement bien trop généraux pour envisager de les appliquer tels quels dans notre domaine de compétence. Dérivé de la branche de l'\textit{Operational Research (OR)}, les auteurs de la communauté restreinte des \textit{systems analysis or modelling and Simulation (M\&S) } engagent dès les années 1960-70 des efforts pour standardiser ces définitions pour la simulation.

\Anotecontent{first_time_validation}{La citation de Churchman par \textcite{Naylor1966} est tiré de \autocite[165]{Nance2002} : \foreignquote{english}{\foreignquote{english}{X simulates Y} is true if, and only if, (a) X and Y are formal systems, (b) Y is taken to be the real system, (c) X is taken to be an approximation to the real system and (d) the rules of validity in X are non-error-free.} \autocite{Nance2002} }

Parmi les différents auteurs participant de ce mouvement ( Naylor, Finger, Oren, Hermann, Zeigler, Nance, Banks, Gass, Balci, Sargent, etc.), \textcite{Naylor1966} est considéré avec West Churchman (1963) comme un des tout premier à avoir attiré et cristalisé \Anote{first_time_validation} dans de multiples publications l'attention sur cette problématique importante de la V\&V.

Cet économiste formé à l'informatique dans la branche des \foreignquote{english}{management sciences} \autocite{Stricklin1985} est un des premiers en 1967 \autocite{Naylor1967} à publier dans un article nommé \foreignquote{english}{Verification of Computer simulation models} une méthode abordant spécifiquement la question de la crédibilité des connaissances qui peuvent être apportées par un modèle de simulation. Une méthode qu'il va mettre spontanément en tension avec les débats qui agitent la communauté des philosophes à cette même période.

Malgré ces efforts et sa volonté de porter le débat loin dans la communauté inter-disciplinaire (voir les premiers ouvrage collectifs sur l'usage de la simulation dans les \enquote{behavior science} \autocite{Dutton1971, Guetzkow1972} \hl{A verifier}), la démarcation entre les deux termes est encore peu claire \autocites[165]{Nance2002}[3]{Balci1986}. \footnote{\foreignquote{english}{Thomas Naylor, a coauthor of the book cited above, deserves credit for drawing major attention to the validation issue in the 1960s: Is the model actually representing the truthful behavior of the referent system? His work, above and in later publications (Naylor 1971, Naylor and Finger 1967), exerted a major influence in framing validation within different philosophical perspectives. Numerous techniques that can be used were identified or developed. While the issues of both verification and validation were of concern from the early days of simulation, often no clear distinction was made between the two terms.} \autocite[165]{Nance2002}}

\Anotecontent{balci_standard}{\foreignquote{english}{A uniform, standard terminology is yet nonexistent. A recent literature review \autocite{Balci1984} indicated the usage of 16 terms [...] Except some early papers which appearead between 1966 and 1972, model verification and model validation have been most of the time consistently defined reflecting the following differentiation : \textbf{model verification} refers to building the model right; and \textbf{model validation} refers to building the right model. \autocite{Balci1986}}}

Il faudra attendre le début des années 1980 pour qu'un standard émerge, grâce à des financements étatiques \autocite{Balci1986}, mais également du fait des efforts fournis par des auteurs comme Sargent et Balci \autocite{Nance2002}, qui collectent et organisent dans une typologie cohérente l'existant statistique et méthodologique, une activité qu'ils poursuivent encore aujourd'hui \autocite{Balci1998}.\Anote{balci_standard}

Pour \autocite[22]{Oberkampf2010} \foreignquote{english}{A Key milestone in the early work by the OR community was the publication of the first definitions of V\&V by the Society of Computer Simulation (SCS) in 1979 \autocite{Schlesinger1979}}, un des instituts avec la U.S GAO (U.S General Accounting Office) à fournir des spécifications en 1979 \autocite{Balci1986}

\begin{itemize}
\item \textbf{Model Verification} \foreignquote{english}{substantiation that a computerized model represents a conceptual model within specified limit of accuracy.}
\item \textbf{Model Validation} \foreignquote{english}{substantiation that a computerized model within its domain of applicability possesses a satisfactory range of accuracy consistent with the intended application of the model.}
\end{itemize}

\begin{figure}[h]
\begin{sidecaption}[fortoc]{Un des tout premiers schémas issus de la publication de la SCS \autocite{Oberkampf2010,Schlesinger1979}}[fig:S_VV]
  \centering
 \includegraphics[width=.7\linewidth]{schelinger_schema1979.png}
  \end{sidecaption}
\end{figure}

Même si elles sont plus anciennes et de portée moins générale, ces définitions de la \textit{V\&V} semblent plus pertinentes, car évoquées plus régulièrement par les chercheurs en sciences sociales; les travaux les plus cités étant ceux de \textcite{Kleijnen1995}, ou \textcite{Sargent2010} qui placent leurs travaux dans la continuité de ces définitions. L'avancée de leurs travaux peut être suivie en feuilletant les \textit{Proceedings of the Winter Simulation Conference} où la problématique de la \textit{V\&V} est réévaluée régulièrement au regard des nouvelles connaissances. Ce schéma \ref{fig:S_VV} est devenu un classique repris et régulièrement amendé \autocite{Sargent2010}. Voici la lecture qu'en fournit \autocite{Oberkampf2010}

\foreignquote{english}{The \textbf{conceptual model} comprises all relevant information, modelling assumptions, and mathematical equations that describe the physical process or process of interest. [...] The SCS defined \textbf{qualification} as \enquote{Determination of adequacy of the conceptual model to provide an acceptable level of agreement for the domain of intended application}. The \textbf{computerized model} is an operational computer program that implements a conceptual model using computer programming. Modern terminology typically refers to the computerized model as the computer model or code.}

Ce schéma a la particularité suivante, il \foreignquote{english}{ [...] emphasizes that \textbf{verification} deals with the relationship between the conceptual model and computerized model and that \textbf{validation} deals with the relationship between the computerized model and reality. These relationships are not always recognized in other definitions of V\&V [...]}

\Anotecontent{Kleijnen_def}{\foreignquote{english}{This paper uses the definitions of V \& V given in the classic simulation textbook by Law and Kelton (1991, p.299): \enquote{Verification\textbf{Verification} is determining that a simulation computer program performs as intended, i.e., debugging the computer program .... \textbf{Validation} is concerned with determining whether the conceptual simulation model (as opposed to the computer program) is an accurate representation of the system under study}. Therefore this paper assumes that verification aims at a \enquote{perfect} computer program, in the sense that the computer code has no programming errors left (it may be made more efficient and more user friendly). Validation, however, can not be assumed to result in a perfect model, since the perfect model would be the real system itself (by definition, any model is a simplification of reality). The model should be \enquote{good enough}, which depends on the goal of the model.}}

\Anotecontent{Sargent_def}{\foreignquote{english}{\textbf{Model verification} is often defined as \enquote{ensuring that the computer program of the computerized model and its implementation are correct} and is the definition adopted here. \textbf{Model validation} is usually defined to mean \enquote{substantiation that a computerized model within its domain of applicability possesses a satisfactory range of accuracy consistent with the intended application of the model} \autocite{Schlesinger1979} and is the definition used here. A model sometimes becomes accredited through model accreditation. Model accreditation determines if a model satisfies specified model accreditation criteria according to a specified process. A related topic is model credibility. Model credibility is concerned with developing in (potential) users the confidence they require in order to use a model and in the information derived from that model. A model should be developed for a specific purpose (or application) and its validity determined with respect to that purpose [...]A model is considered valid for a set of experimental conditions if the model’s accuracy is within its acceptable range, which is the amount of accuracy required for the model’s intended purpose.}}

Autrement dit, \foreignquote{english}{The OR community clearly recognized, as it still does today, that V\&V are tools for assessing the accuracy of the conceptual and computerized models.} Un avis partagé par \textcite{Kleijnen1995} \Anote{Kleijnen_def} , \textcite{Balci1998}, et \textcite{Sargent2010} \Anote{Sargent_def} mais également des auteurs de références sur le sujet dans les sciences humaines et sociales \autocite{Amblard2006} \hl{Prend le bout de texte la dessus}.

Seulement, cette forme de relâchement sur la correspondance entre réalité et modèle, et ce positionnement plus relativiste de la validation n'a pas toujours été une évidence; les premières définitions de Naylor par exemple, sont toujours usitées, et continuent si on en croit des auteurs comme \textcite{Kleindorfer1998} à semer le trouble dans certaines disciplines.

\Anotecontent{VV_philout}{ \foreignquote{english}{During the last two decades a workable and constructive approach to the concepts, terminology, and methodology of V\&V has been developped, but it was based on pratical realities in business and government, \textbf{not} the issue of obsolute thruth in the philosophy of nature} \autocite{Oberkampf2010}
\foreignquote{english}{A very old philosophical question is: do humans have accurate knowledge of reality or do they have only flickering images of reality, as Plato stated? In this paper, however, we take the view that managers act as if their knowledge of reality were sufficient. Also see Barlas and Carpenter (1990), Landry and Oral (1993), and Naylor, Balintfy, Burdick and Chu (1966, pp.310-320).} \autocite{Kleijnen1995}
\foreignquote{english}{With the strong interest in verification from the software engineering community, this contrasting but complementary explanation of the term was quite important. The effort to place valida- tion in a cost-risk framework moved the concept from a philosophical explanation in earlier works to a form more useable for simulation practitioners.} \autocite[165-166]{Nance2002}}

Mais en excluant ainsi de son analyse la partie subjective et philosophique de la \enquote{Validation}\Anote{VV_philout} pour se concentrer sur la seule partie opérationnelle, ces méthodologies restent pour le modélisateur une coquille vide décevante, qui demande encore à être incarnée thématiquement. Autrement dit, ces méthodes si elles prennent bien en compte la dimension dynamique et incrémentale nécessaire à la construction d'un modèle de simulation qui tendrait vers une réalité en accord avec la question posée, l'organisation des connaissances nécessaires pour guider ce processus reste à la lecture de ces typologies une opération quelque peu énigmatique pour les modélisateurs géographes. On retombe sur une des critiques soulevées précédemment dans la section \ref{sec:critiques_simulation} sur l'absence constatée dans les publications de méthodologie standard pour la validation qui prendrait en compte les problématiques spécifiques d'une discipline. \footnote{Aujourd'hui des disciplines comme l'écologie proposent des méthodologies plus spécifiques, comme la méthode POM proposé par Grimm sur lequel nous reviendront par la suite \hl{mettre une ref et un appel à la section}}

Une position compréhensible pour ces auteurs oeuvrant pour la standardisation, alors même que ces termes sont toujours d'usages toujours assez variables. Une des conséquences visibles tient dans ces incompréhensions et ces débats terminologiques sans fin \autocite{David2009} que l'on observe parfois en marge des discussions inter-disciplinaires. Cette gamme d'acceptions différentes tient souvent au transfert hasardeux des terminologies entre l'ingénierie des M\&S, la philosophie des sciences, et la thématique d'un chercheur en sciences sociales qui se retrouve en position intermédiaire de ces deux derniers. Un exercice d'équilibriste périlleux, car comme le fait remarquer \textcite{Kleijnen1995} en citant astucieusement une note de bas de page de \textcite{Barlas1990}, en philosophie il est tout à fait possible de voir la signification des deux termes inversée! \hl{Expliquez mieux que verification pourrait se traduire en philosophie pour certains par representation de la vérité, du “reel”, alors que le fait même de modéliser implique qu’on en soit loin}

\paragraph{La philosophie des sciences}

Il ne s'agit pas de se lancer ici dans un exposé historique des courants et débats s'étant succédés dans cette discipline, mais d'amener de façon illustrative et avec quelques références récentes l'émergence ces 20 dernières années d'une \enquote{épistémologie de la simulation} reprenant (en parasitant parfois le débat comme on l'a cité au dessus) de son point de vue certains débats évoqués chez les praticiens de la simulation; la question de validation étant comme on l'a vu dans le chapitre 1 un sujet de longue date chez les praticiens de la simulation, mais aussi chez les premiers acteurs fondateurs de la V\&V.

\hl{redite : L'objectif n'est donc pas tant de développer une argumentation critique exposant l'ensemble de ces points de vues, car ce n'est pas l'objet de cette thèse, que de tenter de s'insérer (et non de s'enfermer) dans ces réflexions en spécifiant en quoi celle ci diffère, néglige ou font peu écho à nos pratiques et réflexion historique en sciences sociales.}

Le premier obstacle avec laquelle les acteurs supportant cette nouvelle épistémologie doivent cohabités est évidemment la contre-argumentation questionnant cette même necessité d'opérer une nouvelle sous-division épistémologique. Car existe-t-il réellement des spécificité à la connaissance dérivé de l'étude de l'objet simulation, et si oui quelles sont elles réellement ? Autrement dit, existe t il une différence fondamentale entre les questionnements déjà posés dans le cadre d'une épistémologie des modèles et ceux évoqués dans le cadre d'une épistémologie de la simulation ?

\Anotecontent{frilosite_philoScience}{\foreignquote{english}{As computer simulation methods have made their way into novel disciplines, the issue of their trustworthiness for generating new knowledge has often loomed large, especially when they have competed for attention with experiments or analytically tractable modeling methods. The relevant question is always whether or not the results of a particular computer simulation are accurate enough for their intended purpose.[...] Given our long-standing preoccupation with issues of confirmation, it might seem obvious that philosophers of science would have the resources to easily approach these questions.} \autocite{Winsberg2013}}

Parmis les auteurs ouvertement favorable à la création d'une nouvelle épistémologie, on citera entre autre les efforts de \autocites{Winsberg2001, Winsberg2009, Winsberg2013} qui pousse dans chacune de ses publications les \enquote{philosophes des sciences} à sortir de la seule étude de la \enquote{théorie de la confirmation} pour aller vers un terrain un peu plus aventureux \Anote{frilosite_philoScience}, celui de l'étude de la crédibilité des explications et des hypothèses dans leur dépendance au contexte.

Il propose de résumer l'originalité d'une telle épistémologie en évoquant l'inférence spécifique que produisent l'étude simultanée de trois point sur la simulation. \foreignquote{english}{ \textcite{Winsberg2001} argued that, unlike the epistemological issues that take center stage in traditional confirmation theory, an adequate EOCS must meet three conditions. 
downward, motley, and autonomous.[...] These three features were meant to be offered as conditions of adequacy; for which any adequate epistemology of simulation must account. Against the background of the growing use of simulation in the sciences, an adequate epistemology for the philosophy of science needs to explain the fact that simulation results and computational models are often taken to be reliable despite these three features. Winsberg (2001) argues that simulation requires a new epistemology precisely because traditional stories in philosophy of science about how knowledge claims get credentialed cannot explain them.}

Cette typologie a soulevé un certain nombre de critiques chez les philosophes des sciences, dont la plus longue et la plus argumenté est surement celle de \textcite{Frigg2009} dont on trouve le résumé des points saillants dans les publications de \textcites{Winsberg2009, Winsberg2013} mais également de bien d'autres auteurs qui se réfèrent à ce débat pour se positionner \textcites{Yanoff2010, Eckhart2010}.

Le deuxième point de débat intéressant réside dans le qualificatif souvent donné à la simulation de \enquote{laboratoire virtuel pour l'expérimentation}. Si les philosophes des sciences ne peuvent que s'incliner face au constat d'une telle banalisation du terme, dont nous avons donné nous même un aperçu de son ancienneté d'usage dans les sciences sociales dans le chapitre 1; il existe quand même chez les philosophes la volonté de mettre à l'épreuve les fondements et les conséquences pour la connaissance extraite d'une telle analogie.

\Anotecontent{HackingCartwright}{\enquote{Nos deux livres ont plus d'un point commun. L'un et l'autre accordent peu d'importance à la vérité des théories et avouent un faible pour certaines entités théoriques. Cartwright soutient que seules les lois phénoménologiques de la physique parviennent à la vérité tandis que, dans la partie B de ce livre, je fais remarquer que la science expérimentale est plus indépendante de la théorie que ce que l'on veut bien généralement admettre. Nous ne partons pas des mêmes postulats anti-théoriques car elle considère les modèles et les approximations alors que c'est surtout l'expérience qui m'intéresse, mais nos conceptions convergent.}\autocite{Hacking1983}}

\Anotecontent{Phan_Varenne_theorie}{\foreignquote{english}{Consequently, in the first neo-positivist epistemology, models were viewed not as autonomous objects, but as theoretically driven derivative instruments. Following the modelistic turn in mathematical logic, the semantic epistemological conception of scientific models persisted to emphasize on theory.} \autocite{Phan2010}}

Un débat d'autant plus actif qu'on assiste depuis ces 20 dernières années à un véritable renouveau des questionnements dans le cadre d'une \enquote{épistémologie de l'expérimentation} jusqu'alors relativement peu considéré par la majorité des philosophes des sciences \Anote{Phan_Varenne_theorie}. \textcites{Phan2008, Phan2010} citent ainsi les contributions importantes d'auteurs comme Fischer(1996), Galison (1987, 1997), Franklin (1986, 1996), Morrisson(1993, 1999), mais également les efforts de Hacking (1983) et Cartwright.

\Anotecontent{def_cartwright}{\enquote{Disons qu'il y a des théories, des modèles et des phénomènes. Il serait normal de penser que les modèles sont doublement des modèles. Ils sont modèles pour les phénomènes et modèles pour la théorie. [...] Le réalisme scientifique est ici tout particulièrement concerné. Cartwright est pour l'essentiel anti-réaliste à propos des théories. Pour cela, elle s'appuie en partie sur les modèles. Elle fait remarquer que non seulement les modèles ne peuvent être déduits de la théorie qui les englobe, mais plus encore que les physiciens utilisent à leur gré divers modèles qui, sans pourtant se recouper, cohabitent tous au sein de la même théorie. Et cependant,ces modèles sont les seules représentations formelles disponibles des lois phénoménologiques que nous tenons pour vraies. Elle affirme que seules ces lois phénoménologiques nous permettent d'avancer. Toutes les modélisations de ces lois ne peuvent être vraies ensemble puisqu'elles ne sont pas compatibles. Et rien ne permet de penser qu'un modèle est supérieur à un autre. Aucun n'est vraiment justifié par la théorie qui le porte. Plus encore, les modèles ont tendance à résister aux changements de théorie, c'est-à-dire que le modèle est conservé même si la théorie s'avère inadéquate. Il y a plus de vérité locale dans les modèles incompatibles que dans les théories, pourtant plus sophistiquées.[...] L'idéal de la science n'est pas l'unité mais dans une abondance et diversité de plus en plus grandes.} \autocite[350]{Hacking1983}}

\Anotecontent{def_hacking}{\enquote{Le \textit{réaliste à propos des entités} affirme que bon nombre d'entités théoriques existent vraiment. L'anti-réaliste s'oppose à ces entités qui ne sont pour lui que fictions, constructions logiques ou éléments d'un processus intellectuel d'appréhension du monde. Un anti-réaliste moins dogmatique dirait que nous n'avons pas, et ne pouvons avoir, de raison de supposer que ces entités ne sont pas des fictions. Peut-être existent-elles,mais le présupposer n'est pas nécessaire à notre compréhension du monde. 

Le \textit{réaliste à propos des théories} dit que les théories
sont soit vraies, soit fausses et ce indépendamment de ce que nous percevons : la science, elle au moins, vise à obtenir la vérité et la vérité est le monde tel qu'il est. L'anti-réaliste dit des théories qu'elles sont au mieux
prouvées, adéquates, opératoires, acceptables - quoi-que incroyables, entre autres qualificatifs possibles. } \autocite[59]{Hacking1983}}

On retiendra principalement pour notre argumentaire cette propriété d'indépendance retrouvé de l'expérimentation par rapport à la théorie \Anote{def_cartwright}, dont on peut trouver un très bon manifeste dans les écrits de \textcite{Hacking1983} et Cartwright \Anote{def_hacking}, ces derniers se positionnant comme des antiréalistes des théories, tout en étant des réalistes des entités théoriques. Un point de vue très bien résumé à la fois dans \textcite{Hacking1983} et \textit{Théorie, Réalité, Modèle} de \textcite[226-231]{Varenne2012}

Sur la notion de modèle dans sa relation à l'expérimentation, il semblerait qu'un consensus se dégage chez les philosophes \autocites{Morgan2009, Varenne2013} autour du modèle perçu comme un \enquote{médiateur autonome} articulant théorie, pratiques et données dans un contexte spécifique d'une question et d'un cadre technico-social. \autocite[2]{Phan2010}

Il y a probablement un point intéressant à développer entre cette argument du modèle autonome, et les récents travaux en sciences sociales pour qualifier au travers d'une grille de lecture \autocites{Banos2013a, Sanders2013} le positionnement \autocites{Banos2013, Schmitt2013} et le déplacement des modèles de simulation au travers d'une part de leur construction \autocite{Cottineau2014b}, mais également de leur réutilisation \autocite{Schmitt2014}. Une autre façon de démontrer en quoi cette capacité à cumuler de façon flou différentes fonctions épistémiques donné dans la spécification minimale de Varenne pour la simulation \autocite{Varenne2013} est intéressante dès lors qu'il s'agit de tracer la trajectoire disciplino-temporelle de certains modèles : daisyWorld \autocite{Dutreuil2013}, Schelling \autocite {Bulle2005}, SugarScape, etc.)

\Anotecontent{winsberg_exper_simu_link}{Another unique feature of the epistemology of simulation is the ease with which it can draw inspiration from the epistemology of experiment.}

Les acteurs pronant comme Winsberg une épistémologie de la simulation n'hésite alors pas à débattre pour ce qui est des différents parallèle que l'on peut tracer avec les réflexions de cette communauté. \Anote{winsberg_exper_simu_link}.

Pour ne pas se perdre dans les différents points de vues sur le sujet et bénéficier d'une vue plus large incluant les réflexions des praticiens, on pourra se référer au travail opéré par \textcite{Varenne2001} dans son article \textit{What does a computer simulation prove?}, qui propose une lecture du débat au travers de d'une typologie soulevant trois grandes thèses : I - La simulation est elle un outils commes les autres \textit{A simulation is only a tool} ? II - ou bien l'équivalent fusionnel d'une expérimentation classique (\textit{A simulation is an experiment}) ? III - ou se positionne-t-elle comme médiateur entre la théorie et expérimentation ? (\textit{A computer simulation is an intermediate between theory and experiment})? 

%L'expérimentation mène sa vie propre et entretient diverses relations avec la spéculation, le calcul, la construction de modèles, l'invention et la technologie. Mais alors que le calculateur, le spéculateur et le constructeur e modèles peuvent être anti-réalistes, l'expérimentateur, lui, doit être réaliste. p18 

On trouve donc un grand nombres de travaux, toutes disciplines confondues (les philosophes des sciences ne sont pas les seuls à se poser ce type de question, comme nous verrons par la suite), qui tentent d'établir par le biais de différentes grilles de lecture l'appartenance de ce \enquote{nouveau?} mode d'expérimentation à une des catégories de cette grille. \textit{Pourquoi ? Au delà du jeu d'esprit, quel est l'enjeu motivant une telle comparaison ?}

\Anotecontent{moto_hacking}{Une remarque qui renvoie d'ailleur explicitement à sa lecture du moto d'Hacking \foreignquote{english}{experiments have a life of their own} et à la notion d'autonomie (\textit{autonomous}) de sa synthèse précédemment, qui marque le fait que dans certains cas (impossibilité d'observation, manque de données), la simulation doit faire la preuve des connaissances (\textit{background knowledge}) apportés sur appel de ses propres ressources.}

\Anotecontent{experimental_warranting_belief}{\foreignquote{english}{The central idea of this thread is that experiments are the canonical entities that play a central role in warranting our belief in scientific hypotheses, and that therefore the degree to which we ought to think that simulations can also play a role in warranting such beliefs depends on the extent to which they can be identified as a kind of experiment} \autocite{Winsberg2009}}

Partant du fait que l'expérimentation joue un grand rôle dans l'établissement d'une crédibilité pour les hypothèses avancés, il s'agit de mesurer à quel point la simulation serait susceptible d'apporter les mêmes garanties dès lors qu'on accepte de la voir comme une sorte d'expérimentation.\Anote{experimental_warranting_belief}

On s'appuiera dans la suite de cette argumentation sur la lecture de Winsberg, un philosophe des sciences que l'on estime plutôt partisan de la III thèse dans la classification ci dessus. Ce dernier s'appuie largement sur les travaux d'Hacking, mais aussi Galison pour construire sa réflexion, par exemple en arguant\foreignquote{english}{ [...] that some of the techniques that simulationists use to construct their models get credentialed in much the same way that Hacking says that instruments and experimental procedures and methods do; the credentials develop over an extended period of time and become deeply tradition-bound.} \autocites{Winsberg2003, Winsberg2013} \Anote{moto_hacking}

Winsberg résume ce débat en deux thèses opposés : \foreignquote{english}{Identity Thesis} qui consiste à dire que la simulation est littéralement une expérimentation, et \foreignquote{english}{Epistemology Identity Thesis} qui consiste à penser qu'il existe une dépendance entre les garanties de crédibilité qui pourront être accordé par les résultats de la simulation et leur capacité à être plus ou moins définie en tant qu'expérience. Si la première thèse semble assez bien correspondre au point I de la classification de Varenne, la deuxième semble être une sous-variation du point I.

La plupart des auteurs cités par la suite dans ce débat sont des philosophes des sciences spécialisé en économie (Guala , Morgan, Maki, Simon ) qui rejettent comme Winsberg (plus spécialisé en physique) assez naturellement ces deux thèse \autocite{Winsberg2009}, mais avec des arguments assez différents, qu'il convient d'évoquer pour bien comprendre la complexité de ce débat, assez théorique. 

\Anotecontent{maki_phan}{\foreignquote{english}{For Mäki, abstractions in models are similar to abstractions in experiments as they both can be interpreted as a kind of isolation [...] This analogy between models and experiments is called \enquote{isolative analogy} by Guala (2008). From Mäki’s standpoint, a model can be said to be experimented in its explanatory dimension: the finality of such a model is to explore the explanatory power of some causal mechanism taken in isolation.} \autocite{Phan2008}}

Parmis les différents point de vue existant, on citera par exemple le sous-débat de l'\foreignquote{english}{isolative analogy} relaté ici au travers des publications de \textcite{Phan2008, Phan2010} apellant les points de vue de Morgan et Guala contre Maki (2005). Ce dernier voit dans la similitudes entre isolement théorique du modèle comme expérience de pensée et isolement expérimental \Anote{maki_phan} la possibilité de rejoindre une des deux thèses évoqués par Winsberg, établissant d'une façon ou d'une autre que \textit{les modèles sont des expériences, et les expériences des modèles}. Mais ce type d'argument, et on le suppose tout ceux qui se rapportent à l'évocation d'analogies pour justifier d'une équivalence de puissance épistémique se heurterai, comme on va le voir, à une différence fondamentale.

\Anotecontent{guala_phan_winsberg}{Winsberg résume le point de vue de Guala(2002) ainsi \foreignquote{english}{Guala argues that simulation differ fundamentally from experiments in that the object of manipulation in an experiment bears a material similarity to the target of interest, but in a simulation, the similarity between object and target are merely formal.}, mais on peut trouver une version réactualisé en 2008 dans l'article de \textcite[4.2]{Phan2010} \foreignquote{english}{In a simulation, one reproduces the behavior of a certain entity or system by means of a mechanism and/or material that is radically different in kind from that of a simulated entity (...) In this sense, \enquote{models simulate} whereas \enquote{ experimental systems} do not. Theoretical models are conceptual entities, whereas experiments are made of the same \enquote{stuff} as the target entity they are exploring and aiming at understanding}\autocite[14]{Guala2008}}

\textcite{Phan2010} et \textcite{Winsberg2013} cite le point de vue de Guala (2002, 2008), partagé par Morgan(2002, 2005) et se référant aux travaux de Simon (1969). Ceux-ci s'appuient sur une différence de relation qui existe entre système à étudier et système cible dans chacun des deux cas. En effet, dans le cas des expérience, la comparaison s'appuie avant tout sur une similarité matérielle, alors que dans le cas de la simulation la comparaison est limité à une comparaison formelle entre les objets.\Anote{guala_phan_winsberg}

\Anotecontent{Winsberg_critique_morvan}{\foreignquote{english}{Interestingly, while Morgan accepts this argument against the identity thesis, she seems to hold to a version of the epistemological dependency thesis. She argues, in other words, that the difference between experiments and simulations identified by Guala implies that simulations are epistemologically inferior to real experiments - that they have intrinsically less power to warrant belief in hypotheses about the real world.} \autocite[841]{Winsberg2013}}

Morgan(2002, 2005) accepte le point de vue Guala et Simon, mais s'en sert pour réduire indirectement le pouvoir épistémique de la simulation. Un argument bien résumé par \textcite{Phan2008} \enquote{Pour Morgan (2005) modèles et expériences partagent des fonctions de médiateurs et peuvent fonctionner \textit{sur un mode expérimental}, mais les expériences \textit{réelles} offrent un \textit{pouvoir épistémique} d'investigation de la réalité empirique plus fort.} Ce qui fait dire à Winsberg que Morgan serait indirectement plutot partisan de sa deuxième thèse.\Anote{Winsberg_critique_morvan}
\Anotecontent{winsberg_mereformal}{\hl{A compléter avec ce que dit Winsberg2013}}

Pour \textcite{Winsberg2009} le flou des arguments avancé par Morgan et Guala  (\textit{material similarity}, \textit{mere formal similarity}) ne permet pas d'exclure complétement et définitivement la première thèse.\Anote{winsberg_mereformal} Celui-ci se range malgré tout du coté de Guala, et préfère là aussi rejetter cette thèse, mais à la faveur de sa propre argumentation; ce qui lui permet de rejetter à la fois l'argument Morgan pointant l'infériorité épistémique de la simulation, et la deuxième thèse. Il argue que les simulations et l'expérience diffère principalement par la nature du \textit{background knownledge}, c'est à dire protocoles et les connaissances mobilisés.

Des modélisateurs et épistémologues en sciences sociales beaucoup plus proche de nos pratique comme Phan et Varenne trouve un argument convaincant dans ce dernier point, car \foreignquote{english}{Aujourd'hui, comme le souligne Winsberg, la crédibilité des modèles de simulation repose largement sur la \textit{confiance} que nous pouvons avoir dans les compétences des modélisateurs, informaticiens, expérimentateurs et observateurs, ainsi que dans les composants ou plateformes qui supportent les expériences de simulation.} \textcite{Phan2008}

\Anotecontent{gilbert_critique}{\foreignquote{english}{\enquote{[t]he major difference is that while in an experiment, one is controlling the actual object of interest (for example, in a chemistry experiment, the chemicals under investigation), in a simulation one is experimenting with a model rather than the phenomenon itself.} \autocite[14]{Gilbert2005}. But this doesn't seem right. [...] It is false that real experiments always manipulate exactly their targets of interest. In fact, in both real experiments and simulations, there is a complex relationship between what is manipulated in the investigation on the one hand, and the real-world systems that are the targets of the investigation on the other. In cases of both experiment and simulation, therefore, it takes an argument of some substance to establish the ‘external validity’ of the investigation – to establish that what is learned about the system being manipulated is applicable to the system of interest. Mendel, for example, manipulated pea plants, but he was interested in learning about the phenomenon of heritability generally \autocite{Winsberg2013}}}

\Anotecontent{guala_morgan_reality_experiments}{\foreignquote{english}{The identity thesis itself has drawn criticism from Guala (2002) and Morgan(2002). Guala begins by dismissing what he takes to be a poor argument against it. The poor argument goes something like this : simulations are not at all like real experiments because real experiments manipulate the real-world systems that are the very target of the investigation, while simulation merely manipulate \enquote{models} of the target system. What both Guala and Morgan correclty point out is that it is, quite generally speaking, false.}}

Autre sous-débat évoqués par \textcite{Winsberg2013}, on suppose en partie en réponse à sur son article précédent et très similaire \autocite{Winsberg2009}, la critique de l'\textit{identity thesis} comme évoqué par Gilbert et Troitzsch (1999), dont il pense \Anote{gilbert_critique}, en accord avec Guala (2002) \autocite{Winsberg2009} mais également Morgan et Parker \autocite{Winsberg2013} qu'elle est un argument trop faible pour rejeter l'\textit{identity thesis}. \Anote{guala_morgan_reality_experiments} 

Si les arguments de Winsberg semblent convaincant, \textcites{Peschard2010b, Peschard2013} tente dans une analyse critique d'en montrer les biais, et apporte dans son article des objections tout à fait crédible issue de son domaine d'expertise. Pour ne citer qu'un de ces argument, si il existe bien un intermédiaire de mesure issue d'un modèle, comme l'indique Winsberg, il existe également un sous système en prise directe avec la réalité physique de ce monde. En conclusion, elle estime que si il y a bien une certaine forme de similarité entre cibles épistémiques de la simulation et de l'expérience, pour elle ces activités ne peuvent pas être épistémiquement équivalentes, ce qui n'empeche en rien selon elle la coopération fructeuse des deux approches. \hl{Ajouter une footnote avec explication}

\textcite{Winsberg2013} résume le point de vue de \autocite{Peschard2010} ainsi, \textit{Thus, simulation is distinct from experiment, according to her, in that its epistemic target (as opposed to merely its epistemic motivation) is distinct from the object being manipulated.} Autrement dit, même si la motivation menant à l'expérience est bien eloigné (la motivation), l'objet manipulé dans une expérience est bien celui du monde physique, alors que dans le cas de la simulation c'est l'ordinateur. Or autant la motivation peut apprendre de l'objet manipulé dans le monde physique, autant il n'est pas ici dans notre intérêt d'apprendre sur l'ordinateur en tant qu'objet. Dans ce cas là on pointe une différence, mais on peut également appeler selon \textcite{Winsberg2013} et Morrisson (2009) l'argument inverse pointant au contraire une similarité. L'objet expérimenté étant le plus souvent choisi en tenant compte justement de sa capacité de \textit{surrogate} rapport à la question que l'on se pose effectivement, un point commun entre la construction de simulation et d'expérimentation. 

Winsberg conclu en ajoutant que l'expérimentation, contrairement à ce que l'on pourrait penser, n'est pas forcément et immédiatement plus crédible si on ne lui ajoute pas un bagage de connaissance : \textit{Experiments are not automatically more reliable than simulations, despite their differences. [...] It would seem that there are identifiable differences between ordinary experiments and simulations, but there is nothing about these differences that makes one or the other intrinsically more epistemically powerful.}  \autocites{Winsberg2009, Winsberg2013}

\textcite{Varenne2001} avance alors un autre argument intéressant : \foreignquote{english}{Indeed, when you read (Von Neumann 1951), you see that analog models are inferior to digital models because of the accuracy control limitations in the first ones. Following this argument, if you consider a prototype, or a real experiment in natural sciences, is it anything else than an analog model of itself? The test on the prototype is a real experiment. But is it something different and better than the handling of an analog model? So the possibilities to make sophisticated and accurate measures on this model - i.e. to make sophisticated real experiment - rapidly are decreasing, while your knowledge is increasing. These considerations are troublesome because it sounds as if nature was not a good model of itself and had to be replaced and simulated to be properly questioned and tested! It looks as if it was not possible any more to end a paper on simulation by reassuringly using the traditional word: \enquote{Simulation will never replace real experiments”.} }

Ces derniers paragraphes montrent que le débat est loin d'être fixé, et il semblerait là encore que ce soit la définition du contexte d'application qui détermine le mieux la capacité explicative de la simulation, car comme le dit Winsberg \enquote{l'impossibilité d'expérimenter} existe dans bien des disciplines, comme les sciences sociales, mais également la biologie ou la physique, ou les tentatives de reconstitution simulé d'univers ou d'étoiles dans des super calculateur de plus en plus puissant montre qu'il existe un interet explicatif à cette pratique. On pensera notamment aux projets d'expérimentation récents extremement complexe et couteux en physique (laser megajoule de bordeaux, projet ITER pour la fusion).

Et c'est sur ce point que l'argumentation de la plupart des philosophes des sciences est tout à la fois aussi intéressant que problématique. Pour continuer sur Winsberg, celui ci traite de ces problématiques en se positionnant uniquement du point de vue des sciences physiques. Un fait dont il reconnait prudement les conséquences que peuvent avoir l'inclusion d'un contexte différent sur sa synthèse : \foreignquote{english}{Parker (forthcoming) has made the point that the usefulness of these conditions is somewhat compromised by the fact that it is overly focused on simulation in the physical sciences, and other disciplines where simulation is theory-driven and equation-based. This seems correct. In the social and behavioral sciences, and other disciplines where agent-based simulation (see 2.2) are more the norm, and where models are built in the absence of established and quantitative theories, EOCS probably ought to be characterized in other terms.

For instance, some social scientists who use agent-based simulation pursue a methodology in which social phenomena (for example an observed pattern like segregation) are explained, or accounted for, by generating similar looking phenomena in their simulations (Epstein and Axtell 1996; Epstein 1999). But this raises its own sorts of epistemological questions. What exactly has been
accomplished, what kind of knowledge has been acquired, when an observed
social phenomenon is more or less reproduced by an agent-based simulation?
Does this count as an explanation of the phenomenon? A possible explanation?
(see e.g., Grüne-Yanoff 2007).

It is also fair to say, as Parker does (forthcoming), that the conditions outlined above pay insufficient attention to the various and differing purposes for which simulations are used (as discussed in 2.4). [...] Indeed, it is also fair to say that much more work could be done in classifying the kinds of purposes to which computer simulations are put and the constraints those purposes place on the structure of their epistemology.}

Des philosophes des sciences ont donc saisi cette opportunité de critiquer l'approche de Winsberg pour soulever dans des tentatives de typologies parfois intéressantes \autocite{Eckhart2010} les points de divergences que soulève l'utilisation d'une philosophies des sciences naturelle inadapté à la simulation en sciences sociales. Malheureusement, au cours de ces mêmes lectures, on constate que cette critique se retourne vers les modélisateurs et praticiens des sciences sociales, et mène cette fois ci dans une analyse incomplète du contexte historique au mieux à des interprétations erronés (voir le débat animé entre \autocite{Yanoff2008}  \autocites{Elsenbroich2012, Chattoe2011}), et au pire à des approximations et  conseils de mise en oeuvre totalement déplacé \autocite{Eckhart2010} vis à vis de disciplines qui disposent comme on l'a vu d'une véritable histoire autour de l'usage des méthodes computationelles.

Car bien que la recherche des points communs et des différences entre réalité de l'expérimentation physique et virtuelle apparaisse comme un débat intéressant, il faut bien avouer que celui ci ne peut que difficilement s'adapter à la quasi absence d'expérimentation au sens classique dans les sciences sociales. Ainsi, même si la simulation partage certaines des propriétés de l'expérimentation classique, il y a quand même quelque chose de paradoxal à vouloir absolument analyser le rapport de la simulation à l'expérimentation alors même que c'est cette absence qui justement motive son utilisation dans notre discipline, hormis peut etre pour mettre plus en avant cette incapacité à formuler un unique cadre fédérateur par un tel débat. Comme le dit très justement \textcite{Phan2008} {[...] les sciences économiques et sociales sont plus volontiers concernées par l’opposition entre \enquote{le modèle et l’enquête}  (Gérard-Varet et Passeron, 1995) que par celle entre \enquote{l’expérience et le modèle} (Legay, 1997)}

La notion de modèle vue comme médiateur autonome entre théorie et modèle doit elle aussi être repensé pour les sciences humaine, et la géographie; car les théories si elles peuvent exceptionnelement servir à dériver des modèles, celle ci ne peuvent qu'être difficilement rapporté à leur équivalent en science physique \autocite{Pumain1997}.

D'un coté les sciences physiques semble encore viser l'etablissement d'un cadre fédérateur alors qu'il semble que les théories et les modèles en sciences sociales - hormis peut être le cas particulier de l'économie - soit au contraire pourvoyeur de richesse dans leur capacité à apporter un nouvel éclairage sur un phénomène observé. \hl{a préciser peut etre}

A cela il faut ajouter que le modèle en géographie opère dans un cadre épistémique particulier qui n'est pas forcément celui de toute les sciences humaines. Ainsi, bien que les notions et le rapport entre les notions d'observation du \textit{singulier} et du \textit{général} soient théoriquement à la portée de toute disciplines \autocite{Dastes1992}, il semblerait que la géographie trouve un intérét particulier pour la constitution de sa démarche explicative à articuler des éléments de connaissance pris dans les grandes familles explicatives historique, écologique, et spatiale; justifiant ainsi de niveaux d'explication plus ou moins en interaction mobilisant chacun des déterminants de nature différentes. Avec la possibilité d'intégrer à tout moment dans l'explication les résidus qui tiennent d'un dialogue entre méchanismes généraux et singularité historique, écologique ou spatiale. Car quelque soit le registre explicatif choisi il reste dans les deux cas \textit{ [...] une part d'explication relevant de ce que l'on peut qualifier de singularités locales, non prédictibles à partir de mécanismes généraux, mais nécessitant d'appréhender l'histoire spécifique du lieu}. La conséquence étant une diversité de modèles support de l'explanan (l’explication que l’on propose du phénomène auquel on s’intéresse) dont l'évolution sur la forme et le fond n'a eu de cesse d'éclairer l'explanandum sous un jour différent. \autocite{Dastes1992, Sanders2000, Sanders2013} \hl{Mais il y a aussi le multi-échelles, etc.}

L'éclairage sur la méthodologie sous-jacente à la construction des modèles, pourtant un élément au coeur du raisonnement dans la discipline géographique depuis la révolution quantitative, a encore moins de chance d'être évoqué dans ces publications philosophiques, au détriment d'une réflexion statique plus axé sur la nature de l'objet simulation, et de sa relation au monde.

Or l'évolution des réflexions touchant l'activité de modélisation se construit il me semble à une échelle de reflexion tout à fait différente, celle contextualisé de pratiques guidés par une activité de résolution de questions spécifique à l'analyse spatiale, dont la mise en oeuvre s'appuie sur une chaine de traitements flexibles utilisant à bon escient et de façon cumulative l'arrivée historique de nouveaux outils et avec eux leur capacité à renouveller les questionnements : les statistiques, les modèles, les simulation.

\Anotecontent{ce qui n'est pas sans nous rapeller les difficultés évoqués dans le chapitre 1 sur l'inadéquation et le danger que représente les modes de transmissions actuels}

\Anotecontent{remarque_Varenne_2001}{\foreignquote{english}{The second thesis of this article is that none of the three categories of arguments could be applied to contemporary sciences in general, whatever their objects, their methods and the moment of their history we consider. None of these three categories could be considered as the only true one. We cannot have a general point of view on the value of computer simulations, because of the different implications and meanings of mathematics in the different fields of science, and because of the various philosophies of nature at stake. This fact remains true for a given field throughout its own history, because the role of mathematics and the definition of the studied object evolve: You cannot find a unique and stable value that would be given to its simulation uses once for all. Again and hopefully, this thesis illustrates the fact that it does not belong to the historian to decide on the value of computer simulation in a given field but to the scientists themselves. These preliminary reflections prove the importance to investigate the intellectual history of contemporary sciences and not only their sociological construction nor their philosophical general insights.}\autocite{Varenne2001}}

Ces rapides remarques nous éclaire sur la latence qui existe entre la réflexion récentes d'épistémologues comme Winsberg, Grüne-Yanoff et la réalité théorique et pratique en géographie. \textcite{Varenne2001} avait déjà bien cerné dans la synthèse faite en 2001 qu'il n'y avait pas dans sa classification une position meilleure ou plus convaincante qu'une autre, la réponse se trouvant comme pour la notion de modèle avant tout dans l'étude du contexte, et donc de l'histoire des disciplines face à cet objet simulation. \Anote{remarque_Varenne_2001} Cela ne veut pas dire que les débats évoqués précédemment en sont automatiquement invalidés, seulement qu'il faut être probablement plus regardant vis à vis des remarques générales et des conclusions beaucoup trop hative qui peuvent parfois en découler. 

Les travaux croisés de praticiens (Pumain, Sanders, Banos), d'épistémologues ou historien des sciences propre à la géographie (Orain, Besse, Robic, Cuyala) ou s'en approchant (Varenne, Phan) permettent d'une part d'apposer un premier filtre sur ces réflexions génériques pour s'y référer prudemment, et d'autre part d'innover en questionnant nos démarches dans ce qu'elles ont d'originales, cette fois ci appuyé sur une lecture des pratiques certe pas toujours parfaite mais pouvant au moins être qualifié de \textit{bottom-up} 


\hl{Communauté histoire JASSS ? }

%\hl{Un travail conséquent à la croisée de différentes approches, les travaux d'historiens et épistémologues des sciences somme Orain, Besse, Cuyala, et la lecture plus spécifique de l'évolution des méthodes numériques puis computationelles et de leur apports d'un point de vue pratique et théorique dont on trouve source à la fois dans les nombreux travaux des praticiens, mais également dans des travaux de plus long cours comme celui qu'est en train de réaliser Varenne dans son HDR.  == REDITE}

%\hl{ont su voir rapidement l'intérét de développer plus en avant les spécificités attachés à la simulation en science sociale, déjà riche de réflexion sur les apports successifs et cumulatifs de techniques de simulations \autocites{Banos2013, Varenne2008}, en s'intégrant au débat d'une communauté inter-disciplinaire structuré autour de la modélisation agent, qui émerge dans les sciences sociales au début des années 1990. Ce débat par contre ne fait semble-t-il que commencer dans le courant plus \textit{mainstream} des philosophe des sciences.}

%tel que celle des géographes pratiquant la simulation depuis les années 1950, tels que celle qui a émergé autour de la modélisation agent dans les années 1990. 

%Ainsi comme on a pu le voir dans le chapitre 1, le terme laboratoire virtuel pour l'expérimentation apparait très tot dans les sciences sociales, et des auteurs ont pour l'époque déjà donnés de très bonne raisons pour l'emploi de ce terme; les aspects dynamiques de la simulation en faisait partie.

\hl{=> Transition apport de Varenne par rapport à tout ce bazar}


C'est là que le travail de Varenne réalisé au cours des années 2000 \autocites{Varenne2008, Varenne2013} apparait assez audacieux, en proposant une typologie de fonctions épistémiques flexible et cumulable, il propose une grille de lecture permettant d'intégrer à la fois la diversité des approches dans les disciplines (inter) mais également l'évolution de ces même approches à l'intérieur d'une discipline (intra). Un découplage qui permet également une définition plus fine des rapports que peuvent entretenir les disciplines entre le modèle et la simulation.

\hl{ Varenne, la simulation comme expérience de second genre, la possibilité d'un rapport à l'empirie ... (a voir si je rentre la dedans maintenant ou si je garde ça pour plus tard dans la partie construction de modèle de simulation}



\hl{--------------- Pas fini, tu peux sauter à la partie d'après :) ------}


 %La typologie de Varenne est intéressante car elle sous entend une grande partie des sous débats ou raffinements qui peuvent exister sur ce thème, \autocite{Eckhart2010}

%Et c'est vrai que des propriétés intéressantes développés par Hacking comme l'autonomie des modèles et de ce fait l'autonomie des résultats, est un concept intéressant lorsqu'on le rattache à la vie des modèles de simulations tels que nous les construisons.


 %On pourra également arguer que c'est bien là le problème des sciences de la complexité, c'est qu'il est difficile sinon impossible de rendre compte du fonctionnement global d'un système en étudiant seulement les éléments qui le constitue, coupés de tout ou partie de leur interactions%, avec pour effet l'intrication des causes et des effets.


%++ Innovation en géographie des ABÙ, par rapport aux système dynamique outre la flexibilité exposé, c'est l'apport de la pluriformalisation et la possibilité de formuler (ou pas) un rapprochement entre entité virtuelle et réelle (dénotation interne / externe de Varenne); avec tout les dangers qu'un tel rapprochement suppose... cf les individu micro pour les sociologues, les villes pour les géographe, etc. Mais les modèles restent des modèles causaux, ou ce qui est dans le modèle compte plus pour l'explication que le modèle en lui meme en tant qu'instantané ++

%Mais j'aimerais revenir à présent sur l'apport historique d'Hermann à ces débats, un acteur important dans l'histoire de la V\&V, et dont il me semble on mesure encore l'actualité des questionnements qu'il souleve en 1967.



\subsubsection{synthèse}


Si la communauté \textit{Models\&Simulations} propose aujourd'hui un cadre d'analyse cohérent avec la dynamique attendue chez les géographes pour la construction des modèles, il lui manque toutefois une incarnation géographique qu'il va falloir extraire de nos propres exigences de construction.

Pour comprendre comment la notion de validation se construit en marge de ces deux discours, il faut revenir sur ce qui fait sens dans l'explication pour les géographes. En repartant des transformations que subit la géographie quantitative dans les années 1970 au contact du paradigme systémique, prise dans une nouvelle réflexion des objets géographiques , dont la percolation chez les géographes s'observe dans la nouveauté le champs lexical, les méthodes, mais également les techniques.

Au delà des outils il y a un fond commun, à la construction de modèle en géographie, et qui n'est que très rarement traité dans ces lectures, celui de la mise en oeuvre de la construction. Nous verrons qu'il y a des raisons à cela, la dépendance au contexte en est une, et cette activité hautement flexible, bien qu'influencé par la qualité du substrat technique, pose dans la mobilisation des hypothèses des questions génériques à ce dernier : l'hypothèse que j'ai choisi est elle représentative ?

Il ya une forme de permanence dans les questions posés par la construction d'un modèle ou d'un modèle de simulation qui apelle à la construction d'une vision de la validation plus appliqué en géographie.

(ce qui veut dire qu'il lui faudra un support, et cela viendra par la suite)

\hl{--------------- Pas fini, tu peux sauter à la partie d'après :) ------}




\subsection{De la validation à la construction des modèles de simulation par l'évaluation}
\label{ssec:evaluation_construction}

% Permanence des questions évoqués pour la construction d'un modèle de simulation, plus complexification de la validation liés à la pluriformalisation.

En montrant que la validation est dépendante au contexte, Hermann a permis de lever un certain nombre de questions remarquables par leur actualité dans le cadre de nos propre problématique de construction.  %La mise en avant d'une possibilité de validation dépendante à l'objectif nous oblige inévitablement à prendre en compte l'activité de construction comme activité validante.

\subsubsection{Des modalités de validation dépendante au contexte, l'apport d'Hermann à une première formalisation du problème}

\paragraph{Une vision de la validation différente chez les pionniers du mouvement S\&G}

Charles F. Hermann opère dans la branche des simulations appelées à l'époque par Shubik les \textit{Man-Machine Games} \autocite{Shubik1972}. Une catégorie de simulation qui intègre dans son exécution un couplage entre un ou plusieurs systèmes numériques et des humains, qui peuvent être amenés à interagir entre eux, ou avec les machines. Ce type de simulation de structure hétérogène est intéressante dans le sens où elle permet d'intégrer l'arbitraire humain dans une chaîne d'interaction complexe qui n'aurait pas pu être établie autrement, du fait de l'impossibilité de programmer des interactions et des réactions humaines face à des situations précises. Même si ce type de techniques est motivé par une multitude d'usages, ce n'est pas par hasard si elle se développe particulièrement au cours de la guerre froide aux Etats-unis, toujours sous la direction d'institutions militaires. Ce genre de techniques permettant par exemple de simuler et de reproduire des guerres au travers d'inter-relations diplomatiques et/ou économiques \autocite{Hermann1967b}, avec la possibilité de mesurer via des indicateurs adaptés l'importance et l'impact de différents scénarii sur le couple humain/machine.

Ce type de simulation est particulièrement représenté dans des publications qui traitent de la simulation au sens large, comme par exemple le journal \textit{Simulation and Gaming} ou \textit{S\&G} \autocite{Crookall2011}, dont l'activité remonte au début des années 1970. On retrouve parmi les auteurs ayant participé au développement de la discipline des personnalités importantes comme Guetzkow, Shubik, Coleman, etc. \autocite{Crookall2012}. Aujourd'hui, le terme à évolué vers ce que l'on pourrait probablement appeler des jeux sérieux, l'utilisation de l'ordinateur n'étant plus forcément un élément obligatoire dans ce type de simulation. Du côté des objectifs qui sont aujourd'hui susceptibles de motiver l'utilisation de ces techniques, \textcite{Shubik2009} définit une taxonomie en 6 objectifs : \textit{teaching, experimentation, entertainment, therapy and diagnosis, operations, training }

Cette présence d'une dimension humaine dans les simulations introduit une complexité qui touche forcément à plusieurs objets d'études des sciences humaines (psychologie, sociologie, etc.), et il n'est donc pas étonnant que l'on retrouve ce type de publication dès l'apparition des premiers ouvrages inter-disciplinaires sur la simulation, quand elle ne les pilote pas; Harold Guetzkow par exemple est un des personnages importants qui gravitent autour de Herbert Simon au GSIA (Graduate School of Industrial Administration) de Carnegie Tech dans les années 1950-56 \autocite{Guetzkow2004}, et qui a beaucoup oeuvré pour le développement de la simulation dans ces sciences politiques et psychologiques (\textit{Inter-Nation simulation laboratory}) \autocite{Janda2011, Druckman2010}. Celui ci s'inscrit exactement dans la même branche que Hermann, et apparaît deux fois comme premier éditeur dans des recueils de textes pluri-disciplinaires traitant de la simulation au sens large, preuve aussi de son implication dans le développement et la diffusion de ces techniques au delà de sa propre discipline \autocite{Guetzkow1962, Guetzkow1972}

\paragraph{L'apport du contexte dans l'évolution du sens attaché à l'activité de simulation}

Ce qui est intéressant dans ce type de simulations, c'est qu'elles forcent à penser la validation des modèles sous un angle qui doit nécessairement tenir compte de la variabilité inhérente aux comportements humains, par essence difficilement évaluables et réplicables. C'est de cette contrainte, et parce que \textcite{Hermann1967} s'intéresse aux modèles de simulation pour d'autres objectifs que la prédiction (\textit{teaching, training, theory-building}), que celui-ci développe à mon sens une vision de la validation beaucoup plus réaliste pour les sciences sociales que celle proposée à la même période par Naylor.

\foreignquote{english}{First, the validity of an operating system is affected by the purpose or use for which the game or simulation is constructed [...]}\autocite[217]{Hermann1967}

% Plus d'information à ajouter, soit sur la dite boucle (sachant que le conceptual correspond quand meme pas mal à ce que lon fait, voir Sargent2010), Si la boucle définit par les tenants de la \textit{V\&V} n'est pas inintéressante, et de façon générale résume bien le cycle de vie qui correspond à la construction d'une simulation, de nombreuses questions reste en suspens sur le choix et la mise en œuvre des techniques telles qu'elles sont décrites. La construction et la mise en oeuvre des critères en fait partie. Les objectifs sont cités dans la définitions mais on ne rentre pourtant pas dans le détail de la relation entre ces objectifs et la construction du modèle, qui est laissé à l'expertise de l'utilisateur, en cela Hermann ne propose pas mieux dans sa description d'une boucle modélisatrice que les dernières avancées portés par Sargent2010, toutefois sa réflexion est par son orientation, et par sa précocité de réflexion son intéressante il me semble à citer. les moyens technique de la mise en oeuvre par exemple ? 

%Dans l'explication sociologique, la réalité structurelle n'est pas forcément d'intérét pour la construction du modèle. (bulle)

%Cette observation amène Hermann à considérer que la validation des composantes de la structure mérite une attention tout aussi importante que la seule comparaison avec des données de sorties, notamment dans un cadre explicatif.  curl -k -o ~/backups/pinboard-backups/pinboard-$(date +\%y\%m\%d).json 'https://api.pinboard.in/v1/posts/all?&auth_token=username:APItokenhere&format=json'

En s'appuyant sur ce premier argument évoquant l'existence d'une dépendance liant processus de validation et objectif poursuivi par le modélisateur, Hermann semble \textit{de facto} mettre en défaut une définition de la simulation ayant comme première et unique vocation de représenter au mieux le système observé. Les modalités de la validation étant maintenant définies par rapport au contexte, la possibilité d'un critère unique pour juger de la validation de façon universelle paraît tout à fait improbable. Afin de montrer qu'il ne s'agit pas seulement d'une question de disponibilités des données, et pour amener par la suite sa proposition de méthode multi-critères, Hermann s'attaque donc en premier lieu à réduire la portée des confirmations apportées sur un système observé par l'emploi de la seule technique de validation basée sur la comparaison de données en sortie des modèles de simulation.

Pour montrer qu'il existe des limitations dans la confiance que l'on peut mettre dans la validation lorsqu'il s'agit de comparer des données historiques (dans le cas des simulations de reproduction de guerre, on parle ici plutôt de reproduire des séries d'événements historiques) -cela même si elles sont idéalement toute rendues disponible- aux données en sorties de simulation, \textcite{Hermann1967b} s'appuient sur les travaux de \textcite{Pool1965}.

\foreignquote{english}{This correspondence does not demonstrate that the simulation correctly represents the structure and processes that were operative in the historical occurence. We are speculating on the similarity between the historical and simulated inputs on the basis of the similarity of their outputs. Different relationships among various combination of properties in the simulation conceivably could produce outcomes like those in the historical situation.

A simulation of the 1960 national Presidential election predicted the percentage of the vote for each candidate - the outcome - with considerable success. The designers of that simulation observe, however, that \enquote{it may legitimaly be asked what in the equations accounted for this success, and whether there were parts of the equations in the simulation that contributed nothing or even did harm} Further analysis of the equations in the simulation revealed that the outcome was predicted despite the fact that at least one equation misrepresented aspects of voter turnout. Part of the structure was incorrect, but the simulated result still matched the actual outcome. Despite this difficulty, our confidence that the simulation has captured some aspects of the voting process is greater than it would have been if the simulation had failed to replicate the campaign outcome. Confidence in the simulation would increase further as the operating model demonstrated ability to produce outcomes that corresponded with various elections. In sum, the similarity between simulation and historical events can provide at best only indirect and partial evidence for the correctness of the simulated structures and processes that produced the outcome.}



Ce que nous dit Hermann ici, à la différence de Naylor, c'est que même dans le cas idéal ou toutes les données serait présente, ce mode classique de validation ne peut pas être suffisant, cela quelque soit l'objectif poursuivi par le modélisateur. Un constat que nous avions déjà acquis à la lecture des déboires des géographes avec les préceptes de validation néo-positivistes, associant dans une démarche de modélisation instrumentaliste prédiction et explication (section \ref{sssec:realite_neopositiviste}).

Ce constat reste encore valide aujourd'hui, car comme le rappelle très justement \textcite[32]{Bulle2005}, \enquote{ les problèmes posés en sciences humaines visent cependant, en général, la compréhension des phénomènes. Dans cette optique, l’objet premier de la modélisation n’est pas de faire \enquote{coïncider} les modèles construits avec la réalité qui est celle des effets. Le test par la prévision ne peut assurer des qualités explicatives des modèles.}

Un point de vue partagé par \textcite[106]{Amblard2006}, pour qui \enquote{[...] la recherche de similitudes avec les données, si elle peut être utile, ne peut absolument pas être un critère unique et définitif de validation}

Suivant ces conseil, si Forrester avait appliqué lors de la construction de son modèle \textit{Urban Dynamics} des analyses de sensibilités (voir le type de critère \textit{variable-parameter testing} de \autocite{Hermann1967}) tel que le propose Hermann, il aurait probablement conclu, comme ont pu faire ces détracteurs par la suite, à l'inutilité d'une bonne partie des hypothèses intégrés dans son modèle, qui s'avèrent en réalité très peu influente sur la dynamique observé en sortie des simulations.

Autre point important, l'existence de multiples objectifs de modélisation permet à Hermann certe de révéler la diversité et l'attachement de la validation à un contexte, mais surtout de noter d'une part comment la variation de ce dernier affecte les modalités de cette comparaison entre système simulé et système observé, et d'autre part comment cela affecte la perception du résultat engendré par cette comparaison.

\foreignquote{english}{The first comment is that the validation of an operating model cannot be separated from the purpose for which it is designed and used. [...] The second observation somewhat mediates the first. For the most part the various purposes for conducting games and simulations do not negate the need for criteria we can use to estimate the degree of fidelity with which one system (the operating model) reproduces aspects of another (the reference system). Given some purposes for using games and simulations (such as exploring nonexistent universes), finding appropriate criteria in the referent system is quite difficult. With other objectives, the value of the operating model may remain even if the fit between the model and various criteria representing the observable universe is poor (as in theory building).} \autocite[219]{Hermann1967}

Indirectement, on observe ici le transfert d'une définition de la simulation comme simple \enquote{type de modèle} vers la définition plus générale d'une simulation \enquote{ caractérisée non pas tant par l’unité d’une fonction cognitive qu’elle assurerait toujours sous une forme ou sous une autre que par son fonctionnement interne, fonctionnement qui, bien sûr, mais seulement secondairement, se trouve avoir aussi des conséquences sur sa ou ses fonctions cognitives. Une simulation nous paraît ainsi devoir être prioritairement caractérisée par ce qu’elle est – ou fait – de manière interne plutôt que par ce qu’elle fait au sens d’une fonction cognitive quelconque qu’elle assurerait toujours et qu’on en attendrait prioritairement de l’extérieur : à ce titre, nous proposons de dire qu’\textit{elle est avant tout un traitement spécifique sur des symboles et qui prend toujours la forme d'au moins deux phases distinctes. 1) une phase opératoire [...] 2) une phase d'observation [...]}} \autocite[33-34]{Varenne2013}

\paragraph{La nécessité de repenser la représentativité des modèles}

La V\&V a toujours mis en avant le fait que la modélisation soit un processus incrémental tout à fait nécessaire pour obtenir un modèle de simulation satisfaisant, que cela soit dans les analyses de Naylor, ou d'Hermann. Ce dernier se réfère dès 1967 au principe de parcimonie, une méthode qui implique une abstraction, une simplification du système à représenter, et qui pour lui met logiquement et automatiquement en péril la représentativité. \Anote{Herman_parcimonie} 

%Une parcimonie hérité du principe d'Ockham dont on sait qu'elle n'est en aucun cas un synonyme de simplicité dans sa mise en oeuvre, celle-ci nécessitant au contraire un effort intellectuel important pour déterminer quelles sont les hypothèses réellement représentatives du problèmes à analyser. %Sur le plan de complexité, Poincarré ou le prix nobel d'économie Herbert Simon à fait état plusieurs fois des capacités d'expression du complexe rendu possible par l'usage de la simulation, et cela même avec des modèles simples.\autocite{Banos2013a}

%Une description de la construction des modèles qui coincide avec ce qui a été dit auparavant sur l'importance de la nature de l'objectif poursuivie sur la perception de cette \enquote{représentativité}, et le fait que cette dernière ne fasse pas systématiquement la valeur du modèle - tant soit peu qu'on arrive à fixer une valeur - 

Dans ce que l'on comprend de l'analyse d'Hermann, la perte de représentativité attendue d'un modèle de simulation qui n'est plus strictement dirigé vers la prédiction est compensée par un gain relatif à l'objectif poursuivi qui change la nature de la validation attendue : détection d'alternatives à un comportement, mise en avant de processus simplifiés pour l'éducation, construction de théorie, etc.

Il est donc logique de voir Hermann proposer dans la suite de son analyse de repenser la notion de représentativité et la notion de validation au regard de l'objectif poursuivi par le modélisateur. Il en résulte la généralisation de cette activité de validation dont le résultat se dessine à présent sous le couvert d'un objectif et dans le jeu d'une confrontation entre deux représentations, deux construits prenant pour cible le système modélisé et le système observé. 

\foreignquote{english}{A simulation or game is the partial representation of some independent system. Usually we are interested in simulation as a means for increasing our understanding of the system it is intended to copy. Therefore, the representativeness of a simulation or game becomes extremely important in assessing its value. The process of determining how well one system replicates properties of some other system is called validation.[...] In the present analysis however, validation will be defined more broadly as any comparison between the representation of a system and specified criteria.} \autocite[216]{Hermann1967}

\subsubsection{Le problème de la validation ramené à une confrontation des représentations entre système modélisé et système observé}
\label{ssec:confrontation_sysmodelise_sysobserve}

%\hl{repetition ?}
%La question de la représentativité d'une simulation est un sujet délicat à traiter car sa valeur se dessine à l'intersection d'au moins deux activités, la construction d'un modèle opérationel et la construction d'une grille d'évaluation, deux activités dont on s'apercoit par la suite qu'elles sont en réalité étroitement liées. 

\paragraph{Quelles hypothèses pour quelle représentativité ?}
\label{p:hypothese_representativite}

Si cette \enquote{representativité} ne semble plus intervenir dans la valeur du modèle que sous une forme beaucoup plus partielle, quelle est la part de représentativité acceptable que l'on peut attendre pour qu'une hypothèse soit considérée comme explicative ? Autrement dit quelles sont les modalités qui guident l'introduction maitrisée d'une part d'empirie dans un modèle, par l'existence d'un seuil caractérisant le potentiel de représentativité à atteindre pour chaque hypothèse ? Pour l'ensemble du modèle ? 

\Anotecontent{naylor_etonnement}{On pourra peut être être étonné de retrouver la démarche de Naylor dans les approches subjectives sachant la description qu'on en a fait au préalable. Mais il y a bien une part de subjectivité dans cette démarche, l'application de chacune des étapes de la multi-stage validation faisant quand même appel à une forme d'expertise pour constituer le jeu des hypothèses que l'on estime valable en vue du test final de comparaison aux données.}

L'acceptation d'un gradient de valeur pour juger de la validation rompt avec la méthode \enquote{binaire} proposée par Naylor, la validation d'un modèle passant à présent par l'acceptation subjective d'un seuil de représentativité relatif à l'objectif poursuivi. Avec pour conséquence notable qu'une \foreignquote{english}{[...] simulation or game relatively valid for one objective may be not be equally valid for another.}

Si la notion de seuil n'est pas explicitement abordée par Hermann, c'est pourtant sous cette acceptation que la \textit{V\&V} actuelle va reprendre ce concept. Avec la position suivante, celui de se fixer un seuil de représentativité général à atteindre \textit{a priori}.

\foreignquote{english}{\textbf{Principle 2: The outcome of simulation model VV\&T should not be considered as a binary variable where the model is absolutely correct or absolutely incorrect } [...] The outcome of model VV\&T should be considered as a degree of credibility on a scale from 0 to 100, where 0 represents absolutely incorrect and 100 represents absolutely correct. 

\textbf{Principle 3: A simulation model is built with respect to the study objectives and its credibility is judged with respect to those objectives } [...] The study objectives dictate how representative the model should be. Sometimes, 60\% representation accuracy may be sufficient; sometimes, 95\% accuracy may be required depending on the importance of the decisions that will be made based on the simulation results. Therefore, model credibility must be judged with respect to the study objectives.}\autocite[15-16]{Balci1998}

La position de \textcite[166]{Sargent2010}, tout en étant relativement similaire, propose une vision plus fine et plus réaliste ou le seuil de précision attendu est attaché aux variables de sorties. Un point important sur lequel nous reviendrons plus longuement dans la suite de cette partie. \hl{ref vers la bonne partie}

\foreignquote{english}{A model should be developed for a specific purpose (or application) and its validity determined with respect to that purpose.[...] A model is considered valid for a set of experimental conditions if the model’s accuracy is within its acceptable range, which is the amount of accuracy required for the model’s intended purpose. This usually requires that the model’s output variables of interest (i.e., the model variables used in answering the questions that the model is being developed to answer) be identified and that their required amount of accuracy be specified. The amount of accuracy required should be specified prior to starting the development of the model or very early in the model development process.}\autocite[166]{Sargent2010}

\begin{figure}[h]
\begin{sidecaption}[fortoc]{ On remarquera la forte présence des techniques présentés par Hermann dans la synthèse proposé par Balci en 1986 \autocite{Balci1986}}[fig:S_syntheseBalci]
  \centering
 \includegraphics[width=.9\linewidth]{subjective_balci.png}
  \end{sidecaption}
\end{figure}

Ces deux citations permettent de montrer au passage comment la vision de la validation défendue par Hermann a été intégrée dans une forme très approchante par des acteurs de la \textit{V\&V} comme Balci ou Sargent, dont on a vu précédemment les définitions dans la section \ref{ssec:def_generique_validation}. Ces deux derniers sont en réalité les acteurs majeurs d'une synthèse (voir la figure \ref{fig:S_syntheseBalci}) opérée dans les années 1980-1990 \autocite{Nance2002}, dont on peut dire qu'elle est marquée par un retour à une certaine forme de neutralité (voir par exemple le rejet des aspects philosophiques décrits décrits dans la section \ref{ssec:def_generique_ validation}  qui se double d'un jargon technique spécifique à l'établissement d'un processus qualité exploitable pour l'ingénierie) . Des adaptations qui permettent probablement de mieux accepter en son sein des typologies de techniques aussi différentes que celle de Naylor\Anote{naylor_etonnement} ou Hermann. Régulièrement révisées, \textcite{Balci1998} fait ainsi état dans sa dernière taxonomie d'un catalogue de 75 techniques différentes dans lequel peuvent piocher les modélisateurs en fonction de leurs besoins. 

On se rend bien compte que dans le cadre des sciences humaines et sociales la possibilité de fixer par avance ce type de seuil n'a pas de sens, surtout dans un cadre explicatif.

%\textit{Que faut il entendre ici par partiellement ? Quels sont les leviers permettant au géographe de compenser cette perte de représentativité par un gain en compréhension sur le système à étudier ? }

Pour mieux comprendre quel est l'enjeu de cette délimitation entre un modèle réaliste et un modèle abstrait il faut évoquer cette tension permanente qui nourrit les choix du modélisateurs dans la construction d'un modèle explicatif. Deux attracteurs possibles et apparemment opposés, avec d'une part la volonté de se rattacher à une forme de réalisme au travers de l'injection d'une part maitrisée de réalité tout au long du processus de construction \Anote{durand_observation}, et d'autre part une force qui nous pousse au contraire à se détacher de cette même empirie pour ne retenir que le matériel susceptible de servir l'objectif du modèle.

La sociologue et épistémologue \textcite{Bulle2005} a bien formalisé ce dilemme dans la nécessité pour tout modélisateur de positionner son modèle sur un gradient opposant le réalisme des causes des modèles explicatifs \Anote{bulle_modele_explicatif}, au réalisme des effets des modèles descriptifs. 

Pour mieux comprendre quelles connaissances peut-on attendre d'un tel positionnement sur ce gradient, le mieux est encore de commencer par évoquer un de ses extrêmes, en invoquant par exemple le modèle universellement connu de Schelling. De par sa portée d'application extrêmement générale et la nature très abstraite de ses paramètres celui-ci constitue en soi un extrême intéressant pour comprendre où se situe encore l'explication lorsque le détachement de la réalité est à ce point éloigné. Sur ce point, les analyses de \textcite{Bulle2005} et \textcite{Phan2008, Phan2010} se réfèrent principalement à l'essai de \textcite{Sugden2002} pour évoquer quels types de relations entre les deux mondes peut on attendre de ce type de modèle épuré. 

Les résultats qui dérivent de la mise en dynamique des règles dans le modèle de Schelling sont d'une telle universalité, d'une telle robustesse qu'il n'est plus question de confronter les résultats ainsi obtenus à la réalité. A cet égard le potentiel explicatif de ce type de modèle s'oppose selon \textcite{Bulle2005} à tout réalisme empirique. De ce point de vue, \enquote{le modèle n'est pas tant une abstraction de la réalité qu’une réalité parallèle [...] bien que le monde du modèle soit plus simple que le monde réel, celui-ci n'est pas une simplification de l'autre. Le modèle est réaliste dans le même sens qu'un roman peut être appelé réaliste [...] les personnages et les lieux sont imaginaires, mais l'auteur doit nous convaincre qu'ils sont crédibles } \autocites[131]{Sugden2002}[10]{Phan2008}

L'effet d'une telle recombinaison d'hypothèses revient à mettre en oeuvre un \enquote{monde crédible} où l'inférence inductive est mobilisée pour identifier des similitudes significatives entre les deux mondes. \autocites{Livet2006, Phan2008}. Tout le travail réside donc dans l'interprétation prudente qui peut être faite entre ces résultats d'un monde factice et d'une réalité.

Un processus commun utilisé dans toute oeuvre de fiction pour piquer la curiosité de l'observateur, la mise en exergue volontaire d'une tendance du monde réel dans un monde imaginaire permettant d'entamer une réflexion sur l'existence, la portée, la nature de cette même tendance dans le monde réel. Les villes ou les sociétés mis en avant dans des oeuvres de fiction cinéma ou dans la littérature ne sont jamais que des mondes plus ou moins crédibles (Gotham City, 1984, Matrix, la série Black Mirror, etc. car la liste est longue ...)  pour mettre en avant un discours, ou des tendances du monde réel sur lequel doit porter le questionnement; (http://www.influxpress.com/imaginary-cities/ , \href{http://cybergeo.revues.org/1170#tocto1n9?}{cybergeo})

Si le discours scientifique n'a clairement pas cette obligation ludique, il n'en reste pas moins que ce processus de reconstruction crédible est déjà un outil formidable pour questionner les processus à l'oeuvre dans le monde réel \Anote{ruffat_samuel_ville}. Mais cette ambiguïté de lecture a déjà mené à de nombreux malentendus, d'une part envers le grand public (Voir forrester, mais également \Anote{deffuant_debat}) qui pourrait prendre des résultats de simulation pour la réalité avec tout les conséquences que cela suppose, mais également parfois entre scientifiques provenant de divers horizons. Ainsi après la lecture de la critique par \textcite{Chattoe2011} de l'article de \textcite{Yanoff2009}, il ressort toute la difficulté d'évaluer la méthodologie et le travail réalisé autour d'un modèle au travers d'une seule publication, notamment lorsque la fonction cognitive recherchée par les modélisateurs n'est pas décrite explicitement, ce qui provoque aussi ce décalage entre attente du lecteur et le processus réel de recherche qui sous-tend la construction du modèle. \hl{dp: TROP ALLUSIF}

\textit{Doit on se contenter de ce seul mode explicatif ? Existe t il un moyen pour renforcer la confiance dans la capacité explicative des hypothèses ainsi mobilisés ? } 

%% DEBUT - EN ATTENTE DE LA REPONSE DE VARENNE %%

\textcite{Bulle2005} evoque bien l'existence de modèle à cheval entre potentialité explicative et potentialité descriptive. Ainsi \enquote{appliquée aux processus sociaux réels, la simulation peut allier au potentiel descriptif offert par l’imitation d’effets empiriquement observables, le potentiel explicatif que lui confère la mise en œuvre de relations causales effectives. }

A la différence de modèles trop simples qui n'offrent que de maigres accroches avec la réalité, c'est donc par la réintroduction maitrisée de l'empirie dans les modèles de simulation construits que l'on peut espérer la mise en route progressive d'un processus de validation.

Seulement conformément au type de problèmes que l'on a déjà pu effleurer en traitant de philosophie des sciences dans la section \ref{sssec:philo_sciences}, le processus de validation se heurte rapidement à la différence de nature entre les résultats produits par des hypothèses \textit{reconstruites} et le monde réel. Tout comme le substrat est artificiel, le résultat produit par cette dynamique reste le produit d'un monde reconstruit -in silico- L'existence de ce nouveau niveau d'empirie amène les épistémologues comme Varenne à parler ici d'\enquote{expérience concretes du second genre} faisant alors de la simulation une \enquote{quasi-expérimentation} \autocites{Varenne2001, Varenne2007, Phan2008}

On en déduit que quelque soit notre placement sur ce gradient, il est effectivement vain de chercher à valider un modèle en usant d'un quelconque \enquote{seuil de suffisance} caractérisant \enquote{l'injection de réalisme à atteindre qui autoriserait une inférence certaine sur le monde réel}, puisque de toute façon cette inférence s'appuie sur un résultat \enquote{artificiel} forcément discutable. \Anote{bulle_modele_autonome} \Anote{phan_livet_modele} 

La démonstration précédente nous indique plusieurs pistes de réflexions.

D'une part l'objectif de réalisation d'un modèle au réalisme uniquement structurel n'a pas de sens, même avec beaucoup d'hypothèses, car elle ne permet en aucun cas de garantir la justesse d'une comparaison entre données empiriques et simulés, et n'offre donc aucun critère d'arrêt pertinent dans l'activité de modélisation.

D'autre part à moins de retomber dans les débats philosophiques évoqués dans la section \hl{xxx}, elle nous oblige à penser le modèle pour ce qu'il est vraiment, non pas une construction guidée par la validation, mais la construction d'un raisonnement appuyé par une simplification orienté par et pour un but. Peu importe alors le fait qu'une divergence s'installe entre le monde tel qu'on l'observe et le système modélisé, au contraire.

\foreignquote{english}{In all probability some distributions of events or some kinds of hypotheses will produce results with unacceptable divergence between the operating model and the observable universe. Although these incongruous may not pinpoint the inadequacy in the model, they should provide a diagnosis of the general area which seems unrepresentative.} \autocite[226]{Herman1967}

Mais ce terme de \enquote{simplification} souvent employé reste d'emploi ambigue, la modélisation nécessitant comme le dit \textcite{Haggett1965} non pas tant la mise en oeuvre d'une simplification aveugle, qu'une idéalisation guidé par la volonté de mettre à nu des propriétés du système observé. \textcite{Brunet2000}, pour qui la modélisation est également un processus de recherche, propose même pour éviter toute confusion sur les termes de dénuder la définition de modèle de cette fausse directivité, le modèle devenant dans sa version la plus épurée une \enquote{représentation formalisée d'un phénomène}; le terme \enquote{représentation} intégrant alors toute la complexité sous jacente à une telle formalisation : \enquote{Il va de soi que cette représentation passe par plusieurs filtres, qui tous tendent des pièges : la perception du phénomène, sa représentation, la construction d'un modèle, l'interprétation du sens de ce modèle et la capacité du modèle à rendre compte du phénomène.}

Un point de vue semble t il partagé par Varenne pour qui le terme simplification est  \endquote{[...] un glissement d’attribution indu. Puisque l’usage du modèle est relatif (à un observateur et à un questionnement), on ne peut dire que le modèle doit être un objet simple en lui-même ou dans l’absolu. Il convient donc de regarder sous quel aspect exactement il doit apparaître simplificateur, sous quel aspect il devient un outil facilitateur, un outil de facilitation.}

Dès lors, {[...] on comprend déjà qu’un modèle n’est pas ce qui est recherché en tant que tel, mais ce qui facilite la recherche d’information au sujet d’un système réel ou fictif, cela dans le cadre d’un processus à visée de représentation, de connaissance, de conceptualisation, de conception ou encore de transformation. Il est le moyen plus que la fin. C’est pourquoi je m’aventurerai, à partir de maintenant, à user plutôt du terme de facilitation que de celui de simplification [...]} (voir également la section \ref{ssec:rapell_termes_generiques}) \autocite{Varenne2008}

Comme déjà évoqué par les géographes, ce n'est pas tant \enquote{le modèle} que ce qu'il y a \enquote{dans le modèle} qui nous intéresse \autocites{Sanders2000, Besse2000}, et il faut rajouter par là même aussi l'histoire justifiant de cette configuration. 

Reste alors à explorer comment cette \enquote{facilitation} s'exprime au travers de la construction du modèle de simulation. Seulement comment analyser la pertinence d'une représentation prise en dehors de son contexte, les choix intervenant lors d'une modélisation ne répondant pas à une logique universelle pré-établie, étant comme on l'a vu motivé et modifié par un (ou plusieurs) objectif. Varenne a identifié une vingtaine de ces fonctions de facilité de médiation pouvant motivé la construction ou l'utilisation générale d'un modèle. Or il me semble qu'une fois rapporté au modèle, on est bien obligé de constater l'impact que peut avoir cette diversité d'objectifs dans le choix menant à différentes représentation d'une même hypothèse dans le modèle. Etait-ce pour dénoter une entité réelle (un agent = un individu, une ville, une innovation ) ? Est ce dans le but de simplifier pour la compréhension ? pour les performances ? pour répondre au principe de parcimonie ? Ou les trois à la fois ? (une population agents homogène devenant une equation de croissance par exemple ) Etait-ce un choix fait à la suite parmi une multitudes d'autre essais (différentes équations plus ou moins représentative du phénomène à considérer) ? etc. Il n'y a aucune raison pour que les mécanismes intégrés aux modèles soit homogènes. Dans le cas d'un modèle de migration inter-ville, il est en effet plus intéressant de mobiliser les populations de façon aggrégé si on s'intéresse aux règles intervenant dans la dynamique d'interactions entre les villes, par contre, si il s'agit d'observer l'impact que peuvent avoir des règles de comportements sur ces interactions, ce niveau peut devenir pertinent; cette aproche ne chassant évidemment pas la première, au contraire les couplages étant bienvenu. Normalement tout ces choix devrait être explicité, ce qui est rarement le cas, vu la complexité d'une tâche qui apelle pour être sérieuse l'analyse d'une activité de raisonement accompagnant le modèle dont les jalons de reflexion ont bien souvent disparu. \autocite{Varenne2013b}

La tendance à la pluriformalisation \Anote{pluriformaliser} permise par les modèles multi-agent ne vient pas non plus faciliter cette tâche, car ces modèles de simulation qui peuvent déjà intégrer -et c'est d'ailleur pour cela qu'ils ont autant de succès- sans problème une hétérogénéité d'échelle, de niveau d'abstraction, de modèles, doivent aussi compter avec l'intégration de formalismes mobilisant des temporalités et/ou des échelles différentes \autocites{Varenne2008,Varenne2012a}. Ces couplages n'étant pas toujours évident, y compris au niveau informatique ou des artefacts, c'est à dire l'apparition de mécanismes non prévu et difficile à expliquer d'un point de vue purement théorique, peuvent venir rapidement venir perturber les belles ontologies réalisés en amont. 

Même les modélisateurs ont parfois du mal à s'y retrouver, par exemple il n'est pas toujours évident d'expliquer pourquoi on a choisit de coupler pour certain mécanismes le formalisme agents avec celui des équation différentielle ? Il faut alors comprendre que dans certains cas, c'est aussi ce qui a pu motiver le modèle, l'intérét de la pluriformalisation étant justement ce qu'il faut démontrer en comparaisons des approches traditionnelles prisent séparement. Plusieurs réflexions ont montré qu'il s'agissait d'un type de modélisation en devenir et en voie de démocratisation, les formalismes pour la simulation informatiques (multi-agents, micro-simulation, ac) ou mathématiques (systèmes dynamiques) utilisés n'ayant jamais eu vocation à s'opposer (approche individu - centré contre approche mathématique traditionnelle) comme on aimerait parfois nous le faire croire \autocites{Sanders2013, Banos2013}. 

%% FIN - EN ATTENTE DE LA REPONSE DE VARENNE %%

Une façon de dépasser cette problématique de la validation est d'accepter le fait que le réalisme des hypothèses ne soit plus vraiment un objectif, mais plutôt la réalisation conséquente d'une expertise qui tient essentiellement de l'angle théorique choisi pour éclairer un problème.

Autrement dit, la confiance établie dans les capacités explicatives des hypothèses choisies ne se juge pas tant dans la comparaison des résultats attendus avec le réel observé, que dans l'exploration du monde crédible ainsi simulé en fonction de critères experts construit sur une observation du réel, dans l'espoir d'en dégager une connaissance qui doit encore être vérifiée \Anote{denise_geopoint}. 

Le problème est ici en quelque sorte inversé, ce n'est plus une qualification directe du réel qui est visé par le modèle, mais le modèle qui est visée par notre compréhension du réel au travers de critères experts. Ceux-ci viennent questionner et mettre en tension ce monde virtuel en lui imposant de nouvelles contraintes, révélant par là même les forces et les faiblesses de nos hypothèses dans le modèle. On oppose dans la construction du modèle un jeu d'hypothèses susceptible de produire des résultats attendus, à la réalité des conclusions apportés par la mise en oeuvre effective d'une dynamique que l'on contraint volontairement.

A ce titre, et en s'inspirant de la remarque faites par \textcite{Bulle2005} à ce sujet, il sera toujours nécessaire et légitime de questionner la pertinence des rapports mesurés entre les liens causaux proposés dans le modèle et le ou les critères qui sont censés en rendre compte.

On retrouve ces réflexion dans les termes de \enquote{validation interne} et \enquote{validation externe} introduit par \autocite{Amblard2006}, qui est un des rares publications abordant de façon assez précise le passage d'une \enquote{validation} à une \enquote{évaluation} des modèles de simulations en sciences humaines et sociales. La relation d'inter-dépendance entre validation interne et validation externe y est clairement exposé, à travers l'impact des analyses de sensibilités sur la structuration des modèles, et l'exploration des classes de comportements émergentes observés, les deux se rapportant au final à une comparaison faisant intervenir des critères d'évaluation, des fait stylisés ou des données.

\enquote{L'analyse de sensibilité, si elle peut s'appliquer pour tester la robustesse des résultats d'un modèle, peut également être utilisée pour tester la robustesse de la structure du modèle. En modifiant les hypothèses réalisées dans le modèle, par exemple en modifiant les structures organisationnelles, le modélisateur obtient des indices relatifs à la stabilité de son modèle et de ses hypothèses. Ces indices lui permettent précisément de jauger l'importance du choix d’une hypothèse et l’influence de son remplacement par une autre sur un aspect particulier du modèle. [...] Une autre propriété importante qu'il s'agit d'étudier au cours de cette étape de validation interne, concerne les classes de comportements produites par le modèle. Les simulations multi-agents produisent ce qui est assez communément appelé des « comportements
émergents » (voir chapitres 14, 16 et 17), c'est-à-dire des comportements qui ne sont pas exprimables en utilisant uniquement les hypothèses réalisées sur les comportements individuels}. Deux propriétés qui font ainsi écho à une méthode de la validation externe qui \enquote{[...] consiste à rapprocher les classes de comportements (identifiées lors de la validation interne) à des comportements saillants du système-cible : les faits stylisés. Ce rapprochement, s’il peut être fait avec des faits stylisés identifiés a posteriori (permettant par exemple de découvrir dans les phénomènes empiriques, des comportements stylisés qui auraient pu passer inaperçus), possède, on le sent bien, plus de force lorsque les faits stylisés sont déterminés avant même la modélisation comme des comportements que l'on cherche à reproduire par le modèle ou dont on se servira comme un critère de validation parmi d'autres (rétrodiction).}

Ces deux points nous incite ainsi à pratiquer une évaluation à contrepied de la démarche habituelle, alternant validation interne, puis externe. On met alors de coté un instant la validation interne comme exploration non dirigé des comportements du modèle pour se concentrer sur une exploration, moins complète, ou c'est le désir de rapprochement qui vient piloter cette fois ci l'exploration des comportements. Les modélisateurs peuvent en effet introduire les hypothèses dans un modèle de simulation dans le but de satisfaire, ou \textbf{de ne pas satisfaire} ces critères. En effet on a bien précisé que l'objectif de correspondance avec les données n'était plus la priorité dans l'établissement de tels modèles de compréhension, sinon pourquoi ne pas se contenter d'un modèle de Gibrat pour expliquer la hierarchies des systèmes de villes ? 

Autrement dit, il s'agit de mobiliser de façon volontaire cette tension entre hypothèses du modèle et critères d'évaluations mis en place pour en rendre compte afin de savoir si oui ou non cette question valait la peine d'être posé. Reste qu'il faut avoir les moyens techniques de pouvoir répondre à cette question. % Avant de revenir à cela, question de la proof of possibility, impossibility

Il semble que cette mise en tension se satisfait assez bien d'un cadre d'analyse basé sur l'activité de modélisation, ou les critères et les hypothèses, et c'est bien pour cela que l'on mobilise la simulation, ne sont pas tous nécessairement connu à l'avance. Ce qui laisse la place au cours de cette confrontation à l'avénement d'une certaine surprise, à même de produire une connaissance, et de guider le choix des modélisateurs à chaque nouvelle étape du modèle. 

% Les critères toutefois entretiennent un lien avec les hypothèses dont ils sont censé rendre compte, on peut donc imaginer que la parcimonie exprimé dans la construction des modèles puisse faire en lien des critères porteurs de cette connaissance exprimés sur le comportement du modèle. 

% Une remarque d'autant plus valable lorsque on sais que cette histoire se construit en confrontation avec la construction et la mise en oeuvre progressive des critères constitutif du système ciblé.

% Ce qui pose effectivement toujours la question de la nature des connaissances attendues dans une telle perspective.

\paragraph{L'abduction, un phénomène clef moteur dans l'activité de modélisation}

La présence d'une hypothèse dans le modèle se justifie donc tout à la fois par l'expertise du modélisateur que par son adéquation, ou sa non adéquation \textbf{potentielle} avec différents critères de validation. La subjectivité de l'expérimentateur joue sur les deux tableau, et donne à voir dans cette subtile inter-dépendance qui relie le choix des hypothèses et le choix des critères une forme incertitude quand au résultat assez difficile à prévoir et quantifier.

Que se passe-t-il lorsque le potentiel explicatif d'une hypothèse pourtant appuyé par des résultats empirique constaté dans le système observé s'avére invalidé par une analyse de sensibilité ou un critère d'évaluation ? Et cela, alors même que l'experimentateur considère celle-ci comme étant indispensable dans le développement d'une dynamique donné ? Que se passe-t-il au contrare lorsqu'un critère d'évaluation est atteint alors que cela n'était pas attendu au vu de la structure du modèle ? 

%La fonction heuristique de la simulation pouvant s'exprimer tout autant dans cette \enquote{surprise} d'une divergence entre le potentiel investit dans les hypothèses et les critères selectionnés, que dans la surprise suivant l'introduction de nouveaux critères contraignant le modèle, et remettant en cause ce même potentiel de représentation investit dans certaines hypothèses. 

Il y a une divergence nécessaire entre la volonté du modélisateur de rendre compte d'un système observé par un réseau d'hypothèse qui lui parait parcimonieux, nécessaire et cohérent d'un point de vue thématique (le potentiel investit), et la réponse effective apporté par la mise en dynamique de ces causalités lues au travers des critères selectionnés pour en rendre compte. % la possibilité d'infirmer ou d'affirmer de nouvelle connaissances, avec le développement de nouveaux critères, de nouvelles hypothèses ayant jusque là échappé aux raisonnement du modélisateur.

\foreignquote{english}{In developing a game or simulation, the designer is required to be explicit about the nature and relationships between the units in the operating system and their counterparts in the observable universe. He must specify the conditions which cause a relationship to vary. In constructing an operating model a connection between previously unrelated findings may be discovered. Alternatively, a specific gap in knowledge my be pinpointed and hypotheses required by the model my be advanced to provide an explanation.} \autocite[219]{Hermann1967}

La surprise volontaire ou involontairement produite au cours de cette divergence, et qui accompagne généralement l'activité de modélisation, revient sous le nom d'abduction, le terme venant de Charles S. Peirce \autocites{Besse2000, Banos2013, Phan2006, Livet2014} 

Une capacité dont on a déjà vu en citant Hacking \autocites{Hacking1983,Hacking2003, Hacking2006} qu'elle tenait plus d'une propriété inhérente à l'humain, existant de façon préalable à ses créateurs Aristote, ou Peirce. Un modèle de cognition remis au gout du jour ces dernières années, et qui parait correspondre assez bien, est celui du \enquote{cerveau statisticien},  \enquote{cerveau prédictif}, ou \enquote{cerveau bayésien} pour qui \enquote{penser c'est avant tout prédire}. Cette machine à inférer permanente, construisant des logiques qui lui sont propre à partir du peu d'informations qui lui sont donnés directement ou indirectement, quitte à rapeller en urgence d'ancien schéma, fournit comme on pourrait s'en douter plus de mauvaises prédictions que de bonnes. Mais peu importe, ce qui est important ici, c'est sa capacité à apprend rapidement de ces erreurs. 
%Cette logique bayésienne qui consiste à formuler une hypothèse a priori de façon consciente ou inconsciente \Anote{kauffman}, prise de façon rapide ou lente, basé sur nos connaissances passés ou sur notre environnement présent, pour la confronter et la réévaluer au yeux de la réalité de façon itérative correspond assez bien il me semble à ce que l'on pourrait apeller \enquote{abduction}, apellée également \enquote{inférence de la meilleure explication}. 
Ainsi pour Stanislas Dehaene, partisant de ce modèle cognitif, \enquote{Ce que Pierce appelle l'abduction n'est rien d'autre que ce que les sciences cognitives contemporaines nomme l'inférence bayésienne et qui consiste à mener un raisonnement probabiliste en sens inverse afin de remonter aux causes cachées d'une série d'observations.}

Cette théories qui touche à l'ensemble des disciplines oeuvrant dans le champs des sciences cognitives sont mieux décrites par exemple par Stanislas Dehaene dont les cours sont disponibles sur le \href{http://www.college-de-france.fr/site/stanislas-dehaene}{@site} du Collège de France. Toutefois dans l'utilisation des modèles de simulation d'une part ce n'est pas le monde réel qui nous surprend, mais ce qui se passe dans le modèle de simulation, et d'autre part il est plus intéressant d'adopter une démarche active et créative dans la mobilisation des hypothèses, afin de maximiser la surprise plutot que de la miminiser, de dépasser son horizon de connaissance plutot que de s'y conforter. 

% Proof of possibility ? 
Comme le résume bien Banos dans son HDR, \enquote{l’esprit même de l’abduction au sens de Peirce, désignant cette capacité de l’être humain à générer des hypothèses temporaires à partir de l’information incomplète dont il dispose. Appliquée à la démarche scientifique, l’abduction renvoie ainsi à la capacité du scientifique à se mettre en position d’étonnement, à se laisser guider par la recherche de l’inattendu et plus généralement à laisser libre cours à sa créativité.} \autocite{Banos2013}

Comme on pouvait alors si attendre, les motifs développés par les modélisateurs soutenant cette approche sont très loin de ceux attendus pour la prédiction, ou c'est d'abord la robustesse des résultats qui prime, car \enquote{[...] dans le cas d’une modélisation compréhensive, où l’objectif est d’apprendre des propriétés du système que l’on reconstruit \textit{in silico}, le fait d’avoir, pour une partie des conditions expérimentales des comportements très erratiques du système, loin de discréditer le modèle nous apprend au contraire des éléments de son fonctionnement et nous permet même d’anticiper le fonctionnement du phénomène modélisé. Si sous certaines conditions le modèle est très instable c’est une information très enrichissante sur le modèle et sur le phénomène considéré.} \autocite{Amblard2010} 

Force aussi de constater que cette incrémentalité dans la construction d'un modèle de simulation ne suit pas vraiment un modèle linéaire de développement, et s'accompagne, au moins dans les sciences humaines et sociales d'une activité de raisonnement en partie imprévisible. Ainsi il peut paraitre paradoxal pour un modélisateur débutant de voir à quel point il est important de \enquote{malmener} les modèles que l'on a précédemment construit avec raison et parcimonie. L'important ici nous dit \textcite{Amblard2010}, c'est que cela participe à l'élaboration de la compréhension ou de l'explication des phénomènes considérés. Car comme le dit dans son tout premier principe \textcite[65]{Banos2013}, modéliser c'est avant tout apprendre.

D'autre définitions de l'abduction \Anote{abduction_definitions} permettent de mettre en valeur d'autres de ces propriétés, la création en est une, mais avec celle-ci vient aussi la selection. Si on se rapporte aux processus de cognition, la conscience apparaitrait par exemple comme un filtre discrétisant, selectif, d'une pensée inconsciente fluctuante et résolument continue. On peux supposer que lors de la modélisation, ce type de processus est également à l'oeuvre, et il n'est pas rare lorsqu'on construit un modèle de simulation d'avoir à choisir, ou à ne pas choisir, avec plusieurs hypothèses ou implémentation d'hypothèses alternatives. Le deuxième choix expliquant aussi en partie pourquoi les interfaces utilisateurs de nombreux modèles de simulation Netlogo sont aussi riches en boutons de selection.

%Cela soulève la possibilité d'hypothèse explicative concurrente ou inter-dépendante dans l'apparition d'un phénomène, dont certaine échappe forcément au seul modélisateur géographe du fait par exemple de la nature inter-disciplinaire des objets engagés, auquel il faut encore appliquer une selection plus consciente en décidant de mettre plus ou moins en avant des hypothèses susceptible de surprise. 

%La surprise ne vient donc pas seulement du modèle, mais aussi de ce que font les autres manipulant les modèles, surtout dans le contexte inter-disciplinaire ou nous évoluons, les limites de nos connaissances se manifestant assez vite lorsqu'il s'agit d'étudier un phénomène ou un objet partagé, comme les villes par exemple. 


%% A REPLACER

% S'exprime dans une dynamique ? 
\paragraph{Quels critères d'évaluation pour quelle mesure des hypothèses ?}

On a montré que l'objectif d'une adéquation avec le système observé n'avait pas de sens, mais on n'a pas évoqué les modalités de ce rapprochement dans le rapport d'évaluation existant entre hypothèses et critères quantitatif mobilisés dans le modèle. Pourquoi ne pas envisager ici une validation externe basé sur une comparaison empirique et terme à terme des hypothèses constitutive entrant dans la structure causale du modèle  ?

Pour \textcite{Batty2001} et du fait l'\textit{Observational Dilemna}, cette solution n'apparait pas faisable en général. \foreignquote{english}{In principle, each element of this process should be explicit and should be capable of being validated with observed data. In practice, this is rarely if ever the case. The data set would be too large, it would be impossible to collect in its entirety, it may be impossible to even observe and measure. Yet the processes are known to be important. Other criteria must thus be used.} Outre donc la question de la disponibilités des données en sciences humaines et sociales, l'\textit{Observational Dilemna} démontre l'impossibilité de s'abstraire de cette intrication entre cause et effet lorsqu'on observe un phénomène (complexe) en sciences humaines et sociales. De fait, cette caractéristique des systèmes complexes suppose aussi l'impossibilité d'établir l'unicité des hypothèses avancées pour décrire un phénomène, mais aussi par extension celle des critères d'évaluation, qui reste eux aussi des construits formulés sur la base d'une observation du système observé.

On peut ajouter à çà une autre limite concernant les données, on a effet vu que toutes les hypothèses du modèles n'avait aucune raison de se placer au même niveau d'abstraction, ou d'appartenir au même formalisme. Il est par exemple déjà difficile d'accéder à des données dans une fenêtre spatio-temporelle donné, est-il possible d'en avoir sur plusieurs fenetres, voire plusieurs fenetres simultanées ? 

Pour \autocite{Amblard2006}, si cette validation externe basé sur une comparaison quantitative avec les données n'est pas impossible, elle n'en reste pas moins difficile à mettre en oeuvre de façon systématique, pour les raisons évoqué ci-dessus, l'appel à des critères d'évaluation plus qualitatif, dont il faut déjà justifier la construction, restent les plus évidents à mettre en oeuvre. Un point de vue assez logiquement partagé par \textcite{Batty2001} \foreignquote{english}{ [...] complex systems models have multiple causes which display a heterogeneity of processes that are impossible to observe in their entirety. The focus is on more qualitative evaluation of a model’s plausibility in ways that relate to prior analysis of the model’s structure.} 

% A deplacer et a remettre dans le flux au desssus ? 

Une des autres originalité dans l'analyse d'Hermann réside dans les remarques très juste et très précoce qu'il a formulé sur la relativité des hypothèses et des critères mobilisé dans les modèles. Celui-ci est en effet tout à fait conscient qu'il ne s'agit pour l'une comme pour l'autre que d'assertions sur la réalité, comme nous l'avons déjà discuté auparavant. Il propose donc d'essayer de faire au mieux. En adoptant une validation multi-critère \Anote{methode_hermann} intégrant les objectifs ayant guidé la construction du modèle il espère ainsi renforcer le crédit qu'il est possible d'apporter aux simulations. \foreignquote{english}{We have arrived at the position, then, that multiple validity criteria are needed because of the error of measurement and because of the recognition that criteria can be only assertions about \enquote{reality}} 

Comme on pouvait toutefois s'y attendre, cette impossibilité d'admettre l'unicité des critères pour juger la structure causale mobilisé s'insère dans un questionnement plus large. Faut-il effectivement juger la valeur des hypothèses constituantes de cette structure uniquement vis à vis de la réponse à ces critères ?

Si on en croit \textcite[17]{Besse2000}, pas vraiment, car cela serait oublier qu'\enquote{Une hypothèse possède une signification propre, avant même d’avoir été engagée dans l’aventure hautement improbable des programmes de validation. Cela nous conduit à reconnaître dans l’activité scientifique un moment de la production du sens, a coté du mouvement vers l’établissement des vérités.}

L'abduction de part l'argument naturaliste évolutionniste qu'on lui prête,dépasse le simple cadre de la logique dans lequel de toute façon elle posait déjà problème, et n'intervient donc pas comme un moyen de preuve : \enquote{ On nous propose plutot d’envisager l’ensemble des démarches par lesquelles les chercheurs s’orientent vers les hypothèses qui semblent plausibles, en éliminant celles qui ne peuvent etre considérées comme pertinance. } A ce titre, \enquote{La démarche abductive permet un authentique gain de sens, une progression dans l’élucidation}

Un argument supplémentaire pour parler d'évaluation plutôt que de validation \autocite{Amblard2006}, car celle-ci s'inscrit dans un projet parallèle à l'activité de construction du modèle, dont la mise en œuvre implique sinon la construction au moins l'existence préalable d'hypothèses, et d'indicateurs pertinents sur le système observé; une expertise cumulé qui dépasse de loin en durée et en travail le seul projet de construction d'un modèle, et fait souvent intervenir un système de modèle dans une démarche de construction des connaissances de portée beaucoup plus large que cette seule construction de modèles de simulation. Un point que l'on a déjà abordé dans le paragraphe \ref{decorreler_validation} pour justifier d'une décorrélation des problématiques de la Validation vue sous l'angle réducteur de cette seule activité de modélisation multi-agents.

Que cela soit les paramètres, valeur de paramètres, hypothèses mobilisés, choix d'implémentation des hypothèses, critères d'évaluations ( fait stylisés ou données ) construit ou choisi, tout ces étapes ne peuvent être mis en oeuvre si on n'accepte pas de voir l'activité de modélisation pour la simulation comme parti prenante d'une activité de construction des connaissances plus globales. C'est ce qui rend aussi difficile l'évaluation de publication soutenant l'originalité d'un modèle de simulation par un public n'ayant pas connaissance de ce système de modèles et des interactions complexes et réflexives qui relient ceux ci, que cela soit en amont ou en parallèle de la construction des modèles de simulations.

Cette démarche globale a été plusieurs fois théorisé par les géographes \autocites{Besse2000, Sanders2000, Mathian2014}, et une application plus explicite des relations que peut entretenir un modèle de simulation avec d'autres type de modélisation (statistique, spatiales) peut être vu dans la thèse de Clémentine Cottineau \autocite{Cottineau2014a, Cottineau2014b}. 

De plus, il reste difficile donc d'éliminer une hypothèse présente dans le modèle en fonction de sa seule mise en défaut observés à un instant $t$ donné dans la construction d'un modèle, notamment lorsque la présence de celle ci fait sens du point de vue des objectifs qui ont été fixés par le modélisateur. 

D'autant plus qu'il faut aussi prendre en compte cette double dynamique dans lequel opère la construction et la complexification des modèles et des indicateurs pour en rendre compte. Une hypothèse valable à un instant $t$ ne le sera peut etre plus à un instant $t + 1$, ou inversement. 

Peut être n'était-ce simplement pas le moment pour intégrer cette hypothèse au modèle, celui-ci étant encore trop simple ? Peut être manquait-il des interactions pour que sa dynamique soit révélé ? Peut être que l'indicateur devant rendre compte de cette dynamique n'est pas adapté ? Peut être que l'implémentation proposé n'était tout simplement pas la plus adapté à ce moment là ? etc. 

Ce qui a mon sens soulève ici plusieurs remarques : 
- il est vraiment difficile de savoir ce qui va se passer avec l'intégration ou le retrait des hypothèses, ou des critères d'évaluation dans un modèle de simulation si on ne dispose pas d'un outil permettant d'évaluer systématiquement chacune de ces modifications,
- cette évaluation doit être mis en place de façon immédiate, dès que les premières questions sont posés à la structure causale du modèle, afin de ne pas biaisé le raisonnement construit par la prolongation d'une phase de \textit{face validity} pouvant très vite devenir problématique de part les redéveloppements qu'elle suppose dans le futur.
- malgré cela, il faut bien voire qu'une exploration des comportements du modèle, même complète, ne fera pas disparaitre ce problème, qui tient avant tout de l'avancement du raisonnement dans la construction du modèle.

Il reste donc à gérer cette possibilité de réengager les hypothèses et les critères à différents moments dans la construction des modèles, et soutenir une activité de construction cumulative qui ne soit pas \enquote{oublieuse} de cette autre espace temps dans lequel se construise les hypothèses et les différents critères mobilisés. 

Cette variabilité exprimé dans la construction et la paramétrisation des structures causales et des critères associés renvoie à ce phénomène bien connu des modélisateurs en sciences humaines et sociales, à savoir l'équifinalité.

% La question des modes de constructions
%Une solution élégante à été proposé par plusieurs auteurs, sous la forme d'une famille de modèle.

\paragraph{Equifinalité}

On a vu au cours de notre argumentation que la recherche d'un réalisme structurel ne pouvait suffire à vérifier un modèle. Il n'est pas non plus possible d'obtenir, voire même de formuler, des critères quantitatif susceptible de permettre la mise en place d'une telle vérification terme à terme entre hypothèse et critères mobilisés. La mise au jour exhaustive et transparente des dynamiques animant la structure causale de nos systèmes complexes (qui ne serait donc plus complexe) par la calibration ou l'exploration complète des modèles, si elle était possible, n'enleverait également en rien la possibilité de valider les critères avancés, car un tout autre jeu d'hypothèses, d'implémentation d'hypothèses, de paramètres, de valeur de paramètre pourrait très bien conduire au même résultat.


\foreignquote{english}{\textcite{Oreske1994} crisply describe the problem: it is impossible to verify the representational truth of any model of an open system. There  is a many to one relationship between the structure of models and the behaviour they produce, so that many models can account for the same observed outcome. This is the equifinality problem. One common (incorrect) response to the problem is to examine the internal consistency of the model, and to assume that internal consistency guarantees a true representation of reality.} 

Cette équifinalité, quant on la regarde sous cette forme, peut être à la fois considéré comme une limite dans l'établissement de vérité, voire une faille exploitable par les critiques de la simulation. Seulement, comme on a déjà pu le préssentir, la question de la preuve ou de la vérité n'est pas au coeur des préoccupations des chercheurs en sciences humaines et sociales, qui font appel à la simulation pour une tout autre raison. 



La recherche 

Il est en effet impossible de prouver qu'il n'y a pas un tout autre ensemble d'hypothèses et de fait stylisés qui puissent être mobilisé pour rendre compte d'un phénomène. 

L'équifinalité est donc à ce titre une limitation indépassable à la connaissance qui peut être déduite de nos modèles. Pourtant, si on reprend l'objectif avancé par \autocite{Varenne2014},  \enquote{[...] la fécondité propre à la géographie de modélisation contemporaine et à ses différentes formes de manifestation tient en grande partie à sa capacité à affronter cette question de la sous-détermination, à comprendre qu’il ne s’agit plus tant pour elle de chercher des théories que de développer des modèles aux fonctions épistémiques multiples.} 

L’existence de théories alternatives multiples est une constante dans l’histoire des sciences humaines. L'étude de l'objet social est un construit contextuel qui se nourrit d'une multiplicité des point de vues. C'est à ce titre que Jean-Claude Passeron \autocite{Passeron2006} nous met en garde contre une tentative de vérification des modèles qui serait décorrélée de tout contexte historique. Le terme \enquote{vérification} \foreignquote{english}{[...] stands for absolute thruth } \autocites{David2009, Oreskes1994} et se rapporte avant tout ici à la notion d'équifinalité \autocite{OSullivan2004} 

Pour lui le faillibilisme poppérien qui se cache derrière la méthode hypothético-déductive ne peut pas s'appliquer à la construction de théorie dans le cadre des sciences humaines et sociales. 

Toutefois il faut quand même accepter l'existence d'une base commune pour discuter de ces échanges entre la géographes et les autres disciplines, en posant collectivement la question de la \enquote{cumulativité} \Anote{pumain_cumulativité} des connaissances en sciences humaines et sociales. Comme l'indique \textcite{Pumain2005} dans un article dédié à ce sujet, \enquote{La condition indiquée par J.C. Passeron (\enquote{ la sociologie n’a pas et ne peut prendre la forme d’un savoir cumulatif, c’est-à-dire d’un savoir dont un paradigme théorique organiserait les connaissances cumulées }, 1991, p. 364) n’est-elle pas excessivement exigeante ? Les connaissances des sciences dites \enquote{ dures }, expérimentales, sont-elles vraiment organisées dans un même paradigme théorique ? [...] La multiplicité des contextes différents, dans l’espace et dans le temps, est aussi invoquée par J.C. Passeron comme un obstacle rédhibitoire à la comparaison des cas et donc à la cumulativité des connaissances.} 

Toutefois pour Denise Pumain, qui a déjà experimenté avec d'autres géographes la possibilité de ces transferts entre disciplines des sciences humaines et sciences  \autocites{Pumain1989,Sanders1992, Dastes1998}, il ne faudrait donc pas tomber dans un excès de relativisme tel que l'on trouve dans certaines postures postmoderne. Il est possible de travailler à la mise en place de méthodes \Anote{pumain_methode} propre à faire converger ces disciplines vers l'articulation et l'enrichissement de concepts, d'objets au travers de nouvelle grilles de lecture venant supporter la constitution d'un savoir, qui ne sacrifie si possible ni l'originalité, ni la diversité des points de vues engagés. Alors nous dit Denise Pumain, \enquote{Nous pourrions ainsi, tout en produisant des formalismes nouveaux, illustrer la question de la complexité d’une façon bien plus éclairante [...] La complexité d’une notion serait mesurée par la diversité des regards disciplinaires nécessaires à son élaboration, à l’intelligibilité des objets ou des processus étudiés, selon un objectif donné de précision des énoncés et des contextes}

Le modèle de simulation parait être un excellent support pour l'application et la discussion concrete autour de ces hypothèses, nouvelles, pouvant émerger de la mise en place d'un cadre commun. Les projets fortement inter-disciplinaire que sont par exemple Archeomedes, TransMonDyn, Alpage, ou GeoDivercity \autocite{Chapron2014} semblent tous démontrer quelle fertilité en terme de formalismes, de modèles de simulation, et de connaissances produites peut avoir une telle remise à plat.

L'etude de cette problématique de l'équifinalité à l'orée des débats ayant lieu dans une communauté inter-disciplinaire telle que celle gravitant autour du journal JASSS est également intéressante car elle introduit chez les sociologues un cadre pour penser la construction et l'évaluation des modèles, d'origines assez ancienne, qui intègre certains des éléments discutés précédemment : \enquote{les mécanismes générateurs}.

%Débat CONTE / EPSTEIN, et le retour aux mécanismes générateurs

Une entrée par les critiques récentes formulés sur ce cadre historique des \enquote{Science Générative} initialement formulé par Epstein est un bon exemple pour montrer que la prise en compte de l'équifinalité, à elle seule, n'est effectivement pas suffisante pour justifier de la crédibilité des modèles, et peux même dans certains cas fournir une base argumentaire qui permet de réduire la portée explicative des modèles et de décrédibiliser l'utilisation de la simulation en sciences sociales.

C'est la faille emprunté par \textcite{Yanoff2008} qui s'appuie sur le modèle des Anasazi pour proposer une critique générale des \textit{Artificial Societies}, un terme dont il faut dire par avance qu'il est désuet, étant donné la diversité de modèle opérant aujourd'hui dans la simulation en science et sociale.

Le modèles des Anasazi \autocites{Dean2000, Epstein2002} ne représente déjà à cette époque et en SHS qu'un type de modèle de simulation parmis une multitude. Le motto bien connu d'Epstein pour une \textit{generative social science} \foreignquote{english}{If you didn't grow it, you didn't explain its emergence} \autocite{Epstein2006} apparait par contre pour de nombreux modélisateurs comme une source d'inspiration, cela malgré son age et ses défaut, plusieurs fois analysés et cartographiés aux travers d'analyses de la dynamique interne \autocites{Janssen2009, Stonedahl2010, Schmitt2013}[151]{Schmitt2014}. Grunne-Yanof n'ignore probablement pas donc que lorsqu'il s'attaque à ce motto sur ce modèle assez symbolique, il vise en réalité une communauté et un spectre d'application de ce type de modèle beaucoup plus large. 

Ce défi que tacle Grüne-Yanoff sans vraiment le nommer, c'est l'équifinalité,  et plus précisément l'équifinalité telle quel est exprimé dans le cadre de cette science générative définis par Epstein.

Si Chattoe reconnait que l'existence d'un critère unique n'est effectivement pas suffisant pour juger de la qualité des hypothèses du modèle, l'attaque mené par Grüne-Yanoff sur ce point envers les Anasazi reste une attaque \textit{ad-hoc}, dont les conclusions ne peuvent en aucun cas être généralisé à la méthode utilisée pour construire les modèles de simulation en sciences humaines et sociales. Celui-ci ne faisant d'ailleurs dans sa démonstration aucun cas de l'existence d'une telle méthodologie sur lesquels les auteurs du modèle aurait pu se baser pour la construction du modèle \Anote{yanof_equi_a}, or celle-ci existe bel et bien dans les ouvrages de références, une erreur que \textcite{Chattoe2011} juge difficilement pardonnable lorsqu'on s'adresse ainsi à toute une communauté, avec son histoire, ses méthodes, ses codes, ses discussions, ses ouvrages et articles de références. 

La phrase d'\textcite{Epstein1999} \textit{If you didn’t grow it, you didn’t explain it.} est moins ambigue si on regarde le papier de clarification publié par l'auteur en 2006 : 

\foreignquote{english}{The scientific enterprise is, first and foremost, \textbf{explanatory} [...] If you didn’t grow it, you didn’t explain it. It is important to note that we reject the converse claim. Merely to generate is not necessarily to explain (at least not well). A microspecification might generate a macroscopic regularity of interest in a patently absurd—and hence non-explanatory—way. For instance, it might be that Artificial Anasazi [Axtell, et al. (2002)] arrive in the observed (true Anasazi) settlement pattern stumbling around backward and blindfolded. But one would not adopt that picture of individual behavior as explanatory. In summary, \textbf{generative sufficiency is a necessary, but not sufficient condition for explanation.}} \autocite{Epstein2006}

La générativité n'a jamais été pour lui une condition suffisante à l'explication, et l'équifinalité est un concept bien connu de l'auteur, qui renvoie pour cet effort de selection la balle à chacune des disciplines. 

\foreignquote{english}{Of course, in principle, there may be competing microspecifications with equal generative sufficiency, none of which can be ruled out so easily. The mapping from the set of microspecifications to the macroscopic explanandum might be many-to-one. In that case, further work is required to adjudicate among the competitors. [...] In any event, the first point is that the motto is a criterion for explanatory candidacy. There may be multiple candidates and, as in any other science, selection among them will involve further considerations.} \autocite{Epstein2006}

Que faut-il en retenir ? Tant que les modèles publiés ne montre pas plus d'efforts pour décrire à la fois les démarches de modélisations ayant permis la construction des critères et des hypothèses, et l'activité d'évaluation qui autorisent leur présences dans les modèles, le risque de voir ce type de publication se reproduire n'est pas écarté. 

Parmis les autres critiques de cette publication, on citera celle \textcite{Elsenbroich2012}. Celle-ci insiste à la fois sur le fait que les problèmes avancés par Grüne-Yanoff ne sont en rien spécifique à la modélisation multi-agents, mais revient surtout sur la partie explication avancé par ce dernier en lui donnant raison sur un point.

Elle est d'accord pour dire que la simulation multi-agents, pas plus que les sciences sociales, ne peux effectivement fournir de chaine causale complète prise au sens classique de la causalité. Toutefois, il existe selon-elle un autre cadre d'analyse qui permet aujourd'hui de dépasser cette limitation, et de produire quand même une explication, avec le transfert aux sciences sociales et à la modélisation multi-agents des thèses du biologiste Machamer \autocite{Machamer2000}

Avant de rentrer plus dans le détail sur ce cadre d'analyse, il semble que le point de vue d'Elsenbroich rejoigne donc la critique qu'a formulé Conte2007 à l'égard de la théorie d'Epstein lors d'une revue de son livre \autocite{Epstein2007} (faut il voir là une différence ancré dans l'histoire de ces deux courants simultanés, européen et américain ?). 

Par son acceptation des thèses de Machamer, elle rejoint de fait les partisant du courant de modélisateurs portant actuellement le cadre d'analyse dit des \enquote{mécanisme générateurs} \autocites{Hedstrom2010, Conte2007, Manzo2007}, s'opposant à la  \enquote{generative social science} \autocite{Epstein1999}  Pour \textcite[698]{Livet2014} c'est deux visions s'affrontent, mais sur quelle base exactement ? 

\enquote{Si une telle simulation « générative » peut être vue comme une condition nécessaire pour une science sociale computationnelle, elle ne suffit pas à fournir une explication ultime du phénomène. Tout d’abord, aux fonctions de la simulation doit correspondre un processus causal (Conte, 2007). De plus, ce type de modèle permet d’identifier un candidat explicatif pour ce phénomène, sans que ce soit nécessairement la seule explication possible, ni même forcément l’explication pertinente dans tous les cas de figure. La position extrême de Joshua M. Epstein a été critiquée pour la modélisation à base d’agents par Michael W. Macy et Andreas Flache dans leur ouvrage de synthèse sur la sociologie analytique (2009), où l’on préfère la notion plus large de \enquote{mécanismes générateurs}} \autocite{Livet2014}


Conte soulève deux critiques envers le cadre formulé par Epstein, pour elle : 
- il faut associer une théorie des causes à cette émergence au risque sinon d'obtenir une explication fausse ou ad hoc.
- il faut pouvoir reconstituer une chaine d'événément qui vont des causes aux effets, sinon l'explication par génération n'est qu'une simple reproduction de l'effet.


I will suggest that producing causes
and their link to effects must be hypothesized independent of generation: rather than wondering "which are the sufficient conditions to
generate a given effect?", the scientist should ask herself what is a general, convincing explanation, and only afterwards, she should
translate it into a generative explanation.

Ce que nous dit Conte c'est qu'il faut éviter à tout pris une explication ad-hoc; or dans le cadre prévu par Epstein, rien ne semble interdir la formulation d'une seule règle permettant dans son expression dynamique (growing) de reproduire l'explanandum; Ce n'est pas suffisant, il faut pour Conte que les causes avancées ne sont explicatives que si on a réussi à reconstituer la chaine de causalité complète qui va de cette cause à la production de l'événement. Autrement dit il faut éviter de mettre en oeuvre des règles qui n'apporte rien d'autre que la reproduction du phénomène, elle ne sont que des boites noires ou des raccourcis peu informatives.

They look for an informative explanation, which incorporates additional understanding of the level of reality that the phenomena of study belong to. In our example, this means an explanation adding further understanding of social individuals.

Generative explanation requires a theory of the causes from which to grow the effect, otherwise the explanation is irrelevant and a hoc.

Generative explanation requires a theory of the linked chain of events from those causes to effects, otherwise there is no
generative explanation but mere reproduction of the effect.

Il semblerait que ce soit les sociologues qui portent ce cadre d'analyse depuis un certain temps, au travers notamment des travaux de Coleman, mais aussi de Boudon. Quel rapport donc avec la biologie ? 




%- Les problèmes identifiés comme des problèmes de données liés aux ABM dans les Anasazi par Grüne-Yanoff sont applicables en réalité à toutes les sciences humaines : l'absence, l'incomplétude, l'incertitude des données, et l'impossibilité de mesurer des phénomènes empecherai l'obtention d'une chaine de causalité complète, l'inférence abductive et la possibilité d'une explication concurrente renvoie automatiquement à une explication causale partielle, et rend la possibilité d'une chaine de causalité complète impossible.
%- les hypothèses en entrée n'ont jamais été falsifié pour approcher les données en sortie, comme le suppose Grüne-Yanoff.
 

tout comme d'ailleur cela n'avait pas convaincu ...  qui voyait dans cet multiplicité de combinaisons possible un aveux de faiblesse dans la capacité causale de ces modèles. 



La levée de ce point montre bien à quelle point  \textcite{Yanoff2008} avait oublié de préciser que la construction et l'évaluation d'hypothèses de comportement crédible était clairement l'intérét du modèle Anasazi, et non pas juste la réplication \enquote{stupide} d'une régularité macroscopique par tout les moyens. 

Toutefois même évoqué ainsi, ce problème de l'équifinalité ne semble pas satisfaire \textcite{Conte2007}, 







L'équifinalité offre ce support pour confronter nos théories sur un objet social qu'il est impossible de tout façon impossible de voir dans son unicité.


Si on comprend les enjeux d'un tel projet, se pose alors les moyens de sa réalisation; la systématisation des évaluations avait déjà été annoncé comme un outil devant être mobilisé dès la pose des premières hypothèses, mais elle devient absolument nécessaire pour rendre cette fouille de modèles réaliste, et passé peut être à une échelle supérieure, celle de la construction et de l'étude de famille de modèles comme premier élément de réponse intégrateur de la pluralités des points de vues.

A ce titre, le recours au calibrage, et la recherche de cohérence interne dans les dynamiques pourraient passer pour une tentative de mieux définir par ce biais les processus en jeu dans un contexte réel. Pour \autocite{OSullivan2004} cet argument est encore un leurre, car toujours au vu de l'équifinalité, si ces procédures améliorent bien la connaissance du modèle, absolument aucune garantie ne peut être donnée sur l
