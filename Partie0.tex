% -*- root: These.tex -*-
\graphicspath{{FigureIntroduction/}}

\chapter{La simulation de modèles au cœur de la construction des connaissances en géographie}

\startcontents[chapters]
\Mprintcontents

%\epigraph{Nous sommes comme un patient qui sort d'un coma aussi long que la vie d'une étoile.
%Ce dont nous ne pouvons nous souvenir, nous devons le redécouvrir }{---  \textup{Robert Charles Wilson}  Axis}

\epigraph {L'humanité se compose de plus de morts que de vivants } { --- \textup{Auguste Comte}}

\epigraph {La connaissance commence par la découverte de quelque chose que l'on ne comprend pas.  } { --- \textup{Frank Herbert}}

\epigraph {Seeking and staying on a research frontier is a most exacting task. It is now very clear that, in this age of specialization, special knowledge and specialized concepts are not sufficient to hold a science on the frontier.}{ --- \textup{Ackerman 1963}}

% Citer quelque part l'edito de Denise Pumain !

La géographie est partie prenante des bouleversements considérables introduits par la numérisation dans l’ensemble des pratiques scientifiques depuis à peine deux décennies, et cela à plusieurs titres. Les manifestations les plus évidentes tiennent à la prolifération des informations individuelles \enquote{géolocalisées} désormais disponibles sur toutes sortes de support, et notamment, ce qui est entièrement nouveau, en situations de mobilité \autocite{FenChong2012}. Les dispositifs techniques de repérage comme le GPS et l’ouverture des systèmes d’information géographique à l’interactivité grâce à la version 2.0 d’Internet donnent lieu au développement d’une \enquote{géographie volontaire} \autocite{Goodchild2007}, qui conduit à diffuser auprès du grand public des pratiques et des savoir-faire jusqu’ici réservés aux professionnels. Le très grand nombre des institutions privées ou publiques qui partagent ce nouvel engouement pour l’inscription spatiale de leurs activités, tout comme la croissance fabuleuse des \enquote{ réseaux sociaux } sur Internet  contribuent à l’immense développement de ce qu’il est convenu d’appeler, sans traduction en français, les \textit{ big data }. Ces masses de données très labiles, évoluant souvent en temps réel, qu’il est relativement facile de collecter à différents niveaux d’agrégation, posent de nouveaux défis aux géographes en termes de traitement de ces informations, tout autant qualitatives que quantitatives.

Les méthodes classiques de résumé des connaissances par la modélisation et la visualisation doivent être considérablement transformées pour s’adapter à cette nouvelle donne. Mais il serait dommageable de ne pas appuyer notre réflexion sur les pratiques passées pour dessiner un horizon des transformations à venir. Avant d’en arriver au propos de cette thèse, il nous semble indispensable d’opérer un retour sur les expériences de modélisation qui ont été conduites depuis plus de soixante ans dans le cadre paradigmatique général de la systémique. Notre sujet de thèse et notre hypothèse de recherche principale s’inscrivent en effet dans une longue histoire collective dont il nous faut repérer les forces et les faiblesses afin de construire une grille d'évaluation a même de justifier cette démarche que nous avons adoptée.

% -*- root: These.tex -*-

\section{L'entrée de la simulation comme méthode pour les sciences sociales}

L'effort militaire Etats-Uniens a non seulement entrainé dans ses retombées le développement des outils informatiques mais aussi l'institutionnalisation d'entreprises de connaissances appuyées sur ces nouveaux outils tels que le MIT, ou dans un autre genre la RAND corporation, et autres diverses formations.

Ces institutions nouvelles ont largement contribué à susciter des rencontres inter-disciplinaires qui vont favoriser la pénétration des idées de l'école néo-positiviste puis du paradigme systémique, jusque dans les sciences humaines et sociales.

En parallèle, le dogme du déterminisme scientifique hérité de la pensée mécanique plonge de nombreuses disciplines en \enquote{crises} \autocite[20-23]{Pouvreau2013} que l'on pense aux lois de Boltzman pour la thermodynamique, ou au principe d’indétermination d'Heisenberg pour la mécanique quantique. Cette remise en cause est contemporaine de l'émergence d'une pensée \enquote{holiste} (ou pensée de la \enquote{totalité}) qui se construit en confrontation avec la démarche réductionniste classique.

C'est dans ce contexte que l'irruption de l'ordinateur (section \ref{sec:apparition_outil_informatique}) vient bouleverser en tout point le rapport des scientifiques aux données. Si les données de recensements focalisent les premiers travaux inter-disciplinaires, les usages s'étendent rapidement à de nouvelles thématiques propres aux différentes disciplines. Un usage particulier de l'ordinateur va accrocher la curiosité de plusieurs chercheurs en sciences humaines et sociales, la capacité à mettre en oeuvre des expériences d'un genre nouveau, sur un support virtuel qui permet de projeter des hypothèses dans le temps pour observer l'évolution de systèmes dont on ne peut étudier leur comportement dans la réalité, pour de multiples raisons.  

Afin de donner aux lecteurs des éléments de compréhension pour poursuivre la lecture du manuscrit, la partie suivante (section \ref{sec:apparition_simu_science_sociales}) commence par exposer quelques premières définitions générales sur les modèles et la simulation (section \ref{ssec:rapell_termes_generiques}) qui tiennent à la fois compte du contexte historique propre à leur apparition, mais s'intègrent également dans des grilles de classification plus récentes et volontairement plus englobantes, fruit des longs travaux menés par des épistémologues spécialistes de la simulation.

Après avoir constaté l'usage très anciens des modèles de simulation pour l'expérimentation, cet engouement sera précisé par la collecte de nombreuses références dans de multiples disciplines des sciences humaines et sociales (section \ref{ssec:engouement_sciencesociale}) ; autant de pointeurs pour qui voudra poursuivre ses recherches dans ce vaste océan bibliographique, dominé par de nombreuses références croisées du fait de l'inter-disciplinarité et de la position centrale affiché par quelques auteurs.

En géographie la découverte des modèles de simulation coïncide avec l'établissement d'une véritable \enquote{révolution quantitative} (section \ref{sec:premier_modele_geo}), où l'usage de l'ordinateur accompagne, et rend possible même, cette transformation de la discipline voulue par une partie des géographes. C'est dans cette période 1960-1970 d'abondance des modèles (section \ref{ssec:revol_modele} ) que les premiers modèles de simulation émergent du fait de courants qui semblent diverger dans leurs intérêts : les plannificateurs de la RAND, et les universitaires inspirés par Hägerstrand. L'occasion de voir ici quelle redéfinition des termes sont proposés par les géographes, et de présenter plus en détail les acteurs de ces deux courants de modélisation.

S'ensuit une crise de confiance envers les outils ( section \ref{sec:critiques_simulation} ), et plus particulièrement envers la simulation, qui touche les disciplines des sciences sociales (section \ref{ssec:disciplines_touches}) à des dates, des degrés et pour des raisons très différentes, dont on essaye de rassembler dans la section \ref{sec:critiques_simulation} quelques témoignages disponibles. La géographie (section \ref{ssec:crise_mutation}), bien que touchée elle aussi dans le courant des années 1970 par des critiques sur ses outils, ses méthodologies, ses résultats présente toutefois dans ses rangs la présence des éléments actifs d'une transformation (auteurs, publications, etc) qui laisse deviner pour les années suivante un changement de paradigme explicatif sur lequel nous reviendronts plus en détail dans la section \ref{ssec:transition_annee70}. Un glissement plus qu'une rupture, qui ouvre de toutes nouvelles perspectives thématiques, méthodologiques et techniques aux futurs géographes modélisateurs. Ces mêmes modélisateurs qui apparaissent un peu partout sur la planète, faisant suite à l'essaimage de cette révolution à l'international. Les Français découvrent brutalement dans les années 1970 ce bloc de 15 années d'expériences - positives et négatives - acccumulées par les pionniers. On parlera donc plutôt ici d'une transformation que d'une véritable crise, les modèles de simulation n'ayant jamais réellement disparu de la discipline. 

\hl{A reformuler pour être plus percutant en liaison avec le chapitre 2 validation}
Toutefois, et c'est ce qui nous amènera à questionner de façon beaucoup plus précise l'évolution paradigmatique que subit la géographie au regard de la \enquote{Validation} dans le chapitre suivant; car l'apparition de nouvelles problématiques qui coïncide avec ce changement de paradigme vient en réalité se rajouter à celles déjà existantes, dont on déjà constaté qu'elles avaient étés un frein à une adoption plus généralisée de la simulation dans les sciences sociales (section \ref{ssec:disciplines_touches}). Ainsi malgré l'évolution et la démocratisation des outils informatiques et des plateformes de modélisation, l'accès à la ressource informatique (plateforme outils, formations limités, puissawnce disponibles) continue par la suite d'être un facteur limitant pour le développement de modèles plus complexes, mais aussi pour le calibrage et l'exploration efficace des comportements exprimés par ceux-ci, qui nécessite des compétences bien au delà du bagage technique initial des géographes. 

%Il manque l'écologie, cf unwin1992 121


\subsection{Irruption de l'outil informatique }
\label{sec:apparition_outil_informatique}

%GUllahorn cite le recueil 
%Supprime l'histoire du BIG DATA , qui est un anachronisme plutot >> 

L'accumulation et l'exploitation de données numériques est une problématique récurrente pour les géographes et les sciences humaines en général. Ainsi depuis les années 1950-1960 les spécialistes de sciences humaines et sociales ont régulièrement signalé l'importance de l'outil informatique pour le traitement de leur données désormais informatisées, notamment depuis les premières grandes récoltes de données informatisées sur la population. \autocite{Kao1963, Hagerstrand1967b} \autocite[386]{Barnes2011}. 

On pourra citer à ce propos \textcite{Gullahorn1966} lorsqu'il pointe l'importance pour les sciences humaines et sociales du recueil \textit{Computer methods in the analysis of large-scale social systems} qui retrace les discussions issues d'un des tous premiers grands rassemblements inter-disciplinaires organisés par le MIT. Cette conférence pilotée par un sociologue de la section \foreignquote{english}{Urban Studies} du MIT \autocite{Beshers1965} propose de faire le point sur les nouvelles méthodologies et techniques quantitatives et leur utilisations dans les différentes disciplines en science sociales, avec cette volonté marquée de reprendre le contrôle sur la construction des modèles, \footnote{Un point de vue parmi les nombreux dans ce livre, celui de l'éditeur \textcite[194]{Beshers1965} : \foreignquote{english}{The development of a simulation model must by by persons intimately familiar with the subject matter. This principle has been violated in the past by excessive delegation of responsability to mathematicians and programmers interested primarly in questions of structure and style.} } afin de faire face à ce qui apparaissait comme une manne de données nouvelles, les données américaines de recensement \textit{U.S Census}. Une question brûlante d'actualité à l'ère du \textit{Big data}, car cinquante ans après, et des centaines d'innovations techniques plus tard, peut-on enfin dire que les scientifiques ont pris le plein contrôle de leurs données et des outils associés permettant la construction des modèles ?

\begin{figure}[!h]
\begin{sidecaption}[fortoc]{Le \enquote{champignon informationnel} proposé par Frédéric Kaplan est révélateur de l'augmentation du champ d'expérimentation rendu possible par la numérisation des données, puis la simulation numérique.}[fig:I_Champi]
 \centering
 \includegraphics[width=\linewidth]{champignonKaplan.png}
 \legend{Legendary table}
  \end{sidecaption}
\end{figure}

Mais ce serait une erreur que de limiter l'application de ces nouveaux outils aux seuls stockages numériques récents, et ne pas citer l'importance du volume de connaissances accumulées ces derniers siècles par certaines sciences sociales telles que l'archéologie ou encore la géographie. Les lacunes dans l'information sont depuis longtemps une problématique récurrente pour de nombreuses disciplines en Sciences Humaines et Sociales. L'outil computationnel a permis dès lors qu'il a été disponible d'envisager de combler ces lacunes efficacement. Voir la figure ci dessous \ref{fig:I_Champi} \footnote{Voir l'article sur son blog \href{http://fkaplan.wordpress.com/2013/03/14/lancement-de-la-venice-time-machine/}{@FrédéricKaplan}}

%\begin{figure}[tb]
%\raggedright
%\begin{sidecaption}{This is a subcaption just for illustration purposes. This is a subcaption just for illustration purposes. 
%Champignon Informationnel de Frédéric Kaplan. Page number is \LARGE\textbf{\thepage}}[fig:test]
%\includegraphics[width=\linewidth]{champignonKaplan.png}
%\end{sidecaption}
%\end{figure}

La classification automatique des données par l'ordinateur mais aussi la construction de modèles et leur simulation (au sens d'abord mathématique et parfois algorithmique du terme) apparaît rapidement comme un enjeu pour la géographie. La simulation apparaît comme un outil de construction de connaissance absolument naturel et nécessaire pour confronter et construire les théories en rapport avec ces données \autocite{Kao1963, Hagerstrand1967b}. L'image de cette communauté inter-disciplinaire agitant et confrontant ses problématiques méthodologiques, techniques, théoriques dans un but de progression commun, fait écho à des revendications plus récentes \footnote{On pensera notamment à la communauté ABM inter-disciplinaire qui gravite autour de la revue JASSS fondée en  1990}. En réalité cet esprit de partage tient d'une \enquote{volonté commune} qui apparaît quasiment avec l'apparition et la démocratisation des techniques de simulation. C'est ainsi que l'on trouve trace des efforts de cette communauté de chercheurs dans plusieurs ouvrages tels que \autocite{Beshers1965,Naylor1966,Dutton1971,Guetzkow1962,Guetzkow1972}.

Une citation d'un météorologiste du MIT, tout à fait remarquable par sa lucidité, anticipe ce qui sera le principal argument de l'emploi de la simulation en sciences humaines, à savoir un substitut à l'expérimentation \foreignquote{english}{I have argued that in the near future the social science will remain largely empirical and that simulation can serve as a device for making experiments \textbf{in vitro}. I think that this use is more important, at this time, than the massive making of models and that the principal contribution of simulation lies in the direction of intelligent, vivacious empiricism} \autocite{Fleisher1965}

%Forrester1969 à ce sujet "In the social sciences failure to understand systems is often blamed on inadequate data... The barrier is deficiency in the existing theory of structure." \autocite[355]{Batty1976}

\subsection{Les conditions d'apparition de la simulation dans les différentes sciences sociales }
\label{sec:apparition_simu_science_sociales}

\subsubsection{Bref rappel autour des définitions de modèles et de simulations}
\label{ssec:rapell_termes_generiques}

Nous apportons ici une petite digression afin de préciser quelle acception de la simulation nous souhaitons mettre en oeuvre dans notre thèse.

\paragraph{Définitions générales du terme \enquote{modèle}}

\textcite{Varenne2013} ont entrepris une classification de la richesse historique associée aux termes de modèle et de simulation.

La première définition généraliste et aussi la plus couramment encore rencontrée dans la littérature est probablement celle de Marvin Minsky établie en 1965 \autocite{Varenne2008} \autocite[15]{Varenne2013}  : \enquote{ Pour un observateur B, un objet A* est un modèle d’un objet A, dans la mesure où B peut utiliser A* pour répondre à des questions qui l’intéressent au sujet de A } \autocite{Minsky1965}

A partir de cette définition très formelle, Franck Varenne \autocite{Varenne2008} relève dans une analyse plus moderne du terme les cinq points suivants : 
\begin{enumerate}
  \item Le modèle n'est pas nécessairement une représentation
  \item Le modèle doit son existence à l'existence d'un observateur subjectif, et d'un questionnement lui aussi subjectif
  \item Le modèle est un objet qui a une vie propre, une existence autonome
  \item L'existence du modèle est justifiée par l'existence d'une \enquote{fonction de facilitation}
  \item Cette caractérisation minimale permet l'établissement d'une typologie
\end{enumerate}

Franck Varenne propose dans des travaux plus récents \autocite{Varenne2013} d'associer à cette définition les travaux de Mary S. Morgan et Margaret Morrison qui replacent et caractérisent le rôle du modèle dans une enquête de connaissance par sa fonction de médiation (point 4 de la liste), une façon de faire écho à la problématique motivant la construction de modèles établie dans la définition de Minsky.

Un modèle est ainsi défini comme \enquote{un objet médiateur qui a pour fonction de faciliter une opération cognitive dans le cadre d'un questionnement orienté}, opération cognitive qui peut être de cognition pratique (manipulation, savoir-faire, apprentissage de gestes, de techniques, de conduites, etc.) ou théorique (récolte de données, formulation d'hypothèses, hypothèses de mécanismes théoriques, etc.) \autocite{Varenne2013}

Les travaux actuels de \textcite{Varenne2008, Varenne2013} dénombrent pas moins de cinq familles pour un total de vingt grandes fonctions, ce qui permet de situer efficacement la ou les problématiques - rien n’empêche les fonctions de se recouper - qui motivent la construction d'un modèle. 

Nous verrons dans la section \ref{ssec:revol_modele} que les géographes modélisateurs ont mis dans leur définition davantage l'accent sur le rôle et les résultats attendus des modèles, plutôt que sur ces aspects formels.

\paragraph{Définition générale du terme \enquote{simulation}}

Le terme \enquote{simulation}, tout comme le terme \enquote{modèle}, est porteur d'une polysémie qui remonte aux alentours de l'accélération de sa diffusion en 1960 \footnote{ \textcite[343-350]{Morgan2004} propose une analyse intéressante de la diversité d’acception pouvant sous tendre l'emploi du terme \enquote{simulation} en se basant sur l'état de l'art réalisé par \textcite{Shubik1960a} en 1960, mais on peut aussi citer des sources plus directes comme les rapports fait par les instituts scientifiques militaires proche de l'OR : \foreignquote{english}{ The term \enquote{simulation} has recently become very popular, and probably somewhat overworked. There are many and sundry definitions of simulation, and a review and study of some of these should help in gaining a better perspective of the broad spectrum of simulation.} \autocite{Harman1961}}, mais nous retenons ici simplement les acceptions qui concernent la simulation computationnelle.

Bien que la simulation apparaisse sous sa première forme computationnelle dans la technique de Monte-Carlo et les travaux de Von Neumann et Ulman \autocite{Eckhardt1987}, il faut attendre les années 1960 et les avancées techniques nécessaires pour que son utilisation semble utile. L'historienne en économie \textcite{Morgan2004} estime que le mot se diffuse vraiment dans la communauté inter-disciplinaire, et en économie, aux alentours de 1960. Elle souligne le rôle central de Martin Shubik, un des pères de la théorie des jeux \footnote{voir sa \href{http://blogs.library.duke.edu/rubenstein/2012/12/18/the-martin-shubik-papers-from-early-game-theory-to-the-strategic-analysis-of-war/}{@Biographie}} dans la construction de ce débat autour de la notion \footnote{Shubick est aussi présent à un des tout premiers symposiums sur le sujet organisés par \textit{American Economic Review} \autocite{Shubik1960b}, où il retrouve d'autres pionniers de son époque, comme \textcite{Orcutt1960}, et Clarkson aidé de Simon \autocite{Clarkson1960}}, comme celui qui a servi à la fois d'intermédiaire important dans la rencontre entre les différents acteurs de l'économie expérimentale et de l'informatique, mais aussi comme celui tout aussi important de prospecteur au travers des vastes études bibliographiques qu'il a réalisées sur le sujet \autocite{Shubik1960a, Shubik1972} \autocite{Morgan2004}.

Par la suite d'autres conférences et ouvrages vont proposer de délimiter, toujours dans une construction inter-disciplinaire, cet objet \enquote{simulation}, comme on peut le voir dans \autocite{Guetzkow1962, Borko1962, Guetzkow1972, Dutton1971}. La simulation computationnelle est rapidement reconnue par les disciplines en sciences sociales ou les sciences du comportement comme un outil important pour la construction et l'extension de théories (\textit{theory-building} ou \textit{model-building} selon la fonction définie pour le modèle), de par sa capacité à manipuler certes des symboles mathématiques, mais aussi des symboles de plus haut niveau d'abstraction, propre à l'établissement de règle \autocite[924-925]{Clarkson1960}. Dans notre étude les modèles de simulation seront évoqués dans leur dimensions avant tout numérique ou algorithmique (cf. dirigés par des règles) \autocite[36-38]{Varenne2013}.

\subsubsection{La simulation vue comme laboratoire virtuel d'expérimentation, une analogie ancienne}
\label{ssec:labo_virtuelle}

Parmi la vingtaine de fonctions épistémiques recensées par \textcite[14-23]{Varenne2013} motivant la construction de modèles de simulation, la caractéristique la plus souvent exprimée pour l'époque en sciences sociales est sûrement cette capacité à pouvoir \enquote{expérimenter} sur les modèles en mobilisant des processus et des interactions sélectionnés et animés dans le cadre d'une dynamique, d'un temps mimant celui des systèmes cibles \footnote{Plusieurs auteurs, comme \autocite[462]{Gullahorn1965}, \autocite[296]{Doran1970}, \autocite[294-295]{Batty1976} semblent faire référence implicite ou explicite à cette action de \enquote{plonger le modèle dans le temps}. Hors \autocite[31]{Varenne2013} indique que cette dénotation se rapporte principalement au temps du système cible, et non pas au temps du modèle, qui peut être simulé autrement (en usant par exemple d'un tirage probabiliste). Cette référence n'est donc pas un marqueur permettant de caractériser en elle-même la notion de \enquote{simulation de modèle}}, et cela même dans des conditions difficiles caractérisées par l'absence ou l'inconsistance des données, les expérimentations réelles impossibles ou difficiles, etc. mais pas seulement, car la simulation de modèles a aussi vocation à simplifier certaines simulations physiques coûteuses, ou trop limitées dans l'expression de nouvelles hypothèses. Ce lien entre simulation et expérimentation, complexe du fait de la relation entretenue entre le modèle et la réalité, est aussi ancien que la technique elle-même, Von Neumann affirmant dès le départ sa volonté de remplacer par des simulations sur ordinateur certaines techniques coûteuses de simulation physique \autocite[15]{Winsberg2013}.

\Anotecontent{laboVirtuel}{Les récentes et au moins tout aussi récurrentes critiques sur l'apport d'une telle expérimentation dans les sciences sociales montrent qu'il est intéressant de développer quels sont véritablement ces points de similitudes et de divergences entre l'expérimentation physique et virtuelle, ne serait ce que pour construire une argumentation lisible à destination des nouveaux modélisateurs. Des sociologues des sciences comme Bruno Latour ou Ian Hacking ont développé ces vingt dernières années une véritable épistémologie des pratiques de laboratoire centrées autour de la démarche expérimentale, des réflexions qu'il nous faut prendre absolument prendre en compte pour toute analyse qui se voudrait plus poussée sur cette notion, comme en témoignent les travaux récents des épistémologues spécialisé dans la simulation comme Winsberg, ou \textcite[204]{Varenne2012}}

Régulièrement employée dans la littérature, cette fonction d’expérimentation revient également sous la forme de \enquote{laboratoire virtuel}, un terme qui prend selon les époques des teintes légèrement différentes, et cela quelles que soient les techniques sous-jacentes de support à la simulation des modèles.\Anote{laboVirtuel}

Cette analogie ancienne entre simulation et laboratoire virtuel est illustrative d'une réalité dont on aurait bien du mal à nier l'existence tant celle ci est persistante dans cette littérature. Parfois le terme est invoqué directement, parfois il est implicite au discours présenté. Pour ne citer que quelques auteurs pionniers dans l'historique de la notion, les premier ouvrages collectifs en simulation et science sociale de \textcite{Borko1962, Guetzkow1962, Guetzkow1972}; les rapports et états de l'art des instituts scientifiques militaires américains \autocite{Harman1961}, \footnote{La légende veut que l'idée d'appliquer la simulation aux \textit{Behavioral Science} viendrait d'un déjeuner entre Guetzkow et des physiciens nucléaire lors de son séjour au Carnegie, pour en savoir plus : \href{http://www.hawaii.edu/intlrel/pols635f/Guetzkow/hg.html}{@Harold} } \footnote{L'ouvrage de 1962, difficile à trouver, contient des re-publications de publications inédites dans plusieurs disciplines : Orcutt en économie \foreignquote{english}{Simulation of economic systems}, Coleman en sociologie \foreignquote{english}{Analysis of social structures and simulation of social processes with electronic computers}, Abelson en psychologie et science politique \foreignquote{english}{Simulmatics project}, Hovland en psychologie sociale avec \foreignquote{english}{Computer simulation of thinking} } et les travaux de Herbert Simon et Alan Newell \autocite{Newell1961}; et de façon plus localisé, en économie \textcite[915]{Shubik1960b}, en psychologie sociale \textcite{Abelson1968} \footnote{Un auteur connu aussi pour avoir échangé aussi avec \textcite{Boudon1967} sur la simulation à la même période, voir  \textcite{Padioleau1969}}, \textcite{Fleisher1965} météorologue, le couple d'Anthropologues/sociologues du comportement \textcite{Gullahorn1965}, l'archéologue anthropologue et informaticien \textcite{Doran1970}, la physicienne biostatisticienne et démographe \textcite{Sheps1971}, l'informaticien \textcite[3-4]{Forrester1971}, l'économiste informaticien \textcite{Naylor1966}, le professeur de science régionale \textcite[271]{Harris1966}, l'urbaniste \textcite[295]{Batty1976} sans oublier plus récemment \textcite{Epstein1996}, l'écologue \textcite{Grimm2006}, et encore sûrement bien d'autres auteurs. Une longue liste qui témoigne de l'intérêt pour cet outil bien au delà d'un simple critère de démarcation disciplinaire, technique, ou encore temporel; une hypothèse que nous allons développer par la suite.

\subsubsection{Un engouement pour la simulation qui touche l'ensemble des sciences sociales}
\label{ssec:engouement_sciencesociale}

Cet engouement pour la simulation de modèles touche toute les sciences sociales ou presque. Nous dressons dans les paragraphes qui suivent une brève énumération des travaux qui en témoignent pour la période 1950-1970.

Suite au mouvement Cybernétique, à la convergence des travaux sur l'intelligence artificielle et les sciences cognitives, les premiers travaux qui visent la démonstration de la faisabilité de la simulation dans les sciences sociales viennent de Newell, Shaw, et Simon à la fin des années 1950 \autocite{Gullahorn1965} \footnote{Avec plusieurs tentatives pour la construction d'une machine universelle de résolution de problème (\foreignquote{english}{Logic Theorist program} en 1957 et \foreignquote{english}{General Problem Solver} en 1959). Ce programme s'avère également être la première pierre posée de l'intelligence artificielle, en formation à l'intersection de la naissance encore récente des sciences cognitives et de l'informatique. Cette machine est conçue pour mimer les capacités de résolution de l'esprit humain, et permet enfin d'exprimer et de questionner les théories comportementales dans un langage informatique alors plus précis et moins ambigu que le langage naturel. Le programme est ainsi capable de résoudre des problèmes aussi différents que de jouer aux échecs, de résoudre des problèmes mathématiques, ou de retrouver des motifs dans des données.} A ces travaux s'ajoutent ceux, en psychologie linguistique de l'équipe gravitant autour de \textcite[280-416]{Borko1962}, en psychologie des comportements sociaux de \textcite{Hovland1960}, d'\textcite{Abelson1961, Abelson1968} ou du couple \autocite{Gullahorn1965a} qui utilisent la simulation de modèle pour formuler, vérifier des théories sur la psychologie des individus et les modalités de leurs interactions avec les autres dans diverses situations \footnote{Les applications sont menées à des échelles très diverses, ainsi alors que le modèle Homonculus développé par le couple Gullahorn tente de mieux comprendre les stratégies de résolution de conflits avec la programmation de comportements au niveau individuel \autocite{Gullahorn1965}, le projet \textit{Simulmatics} mené par \textcite{Abelson1961} vise quand à lui l'étude du comportement de groupes d'électeurs en cas de conflit d'opinion (ou \foreignquote{cross-pressure}) pour tenter en fonction d'un échantillon de population d'analyser l'impact de stratégies politiques, une demande de J.F.Kennedy pour la campagne de 1960 aux États-Unis}, ce que \textcite{Ostrom1988} appellera \foreignquote{latin}{a posteriori} les \foreignquote{english}{complex human processes}.

%La formation d’ingénieur de Coleman l’amène à prendre comme modèle un physicien : « My real hero is not Isaac Newton, but James Clerk Maxwell. He took Newtonian theory and developped from it a theory of gases, the Maxwell-Boltzmann distribution law of molecular velocities. I was fascinated by Maxwell because he was also concerned with the micro-macro problem. He had a very simple and neat theoretical framework of dimensionless molecules of any gas acting according to the law of motion, each with a certain mass and velocity. And from this he constructed a theory of gas. » (Coleman dans Swedberg, 1990, pp. 55-56).

En \textbf{sociologie}, la simulation émerge dans les années 1960-1970 selon \textcite[50]{Manzo2005}, sous l'impulsion de pionniers dans la sociologie mathématique comme \textcite{Boudon1967} en France \footnote{Selon \textcite[61]{Manzo2005}, Boudon a très tôt supporter l'idée des modèles de simulation comme support à l'explication, comme il témoigne à propos de ces écrits des années 60-80 : \enquote{À ce moment, j’avais publié divers écrits sur l’individualisme méthodologique, la théorie de l’action, la rationalité et les \enquote{modèles générateurs}. Mes travaux sur l’éducation m’avaient en effet convaincu que ni l’analyse multivariée ni les méthodes statistiques d’\enquote{analyse des données} ne permettaient d’expliquer les régularités statistiques qui sont le pain quotidien du sociologue : il fallait tenter plutôt de les engendrer à partir d’hypothèses sur les logiques de comportement des acteurs.} \autocite[391]{Boudon2003}}, ou James Samuel Coleman aux Etats-Unis \footnote{C'est à l'université de Columbia sous la direction du sociologue Robert Merton et du mathematicien-sociologue Paul Lazarsfeld, des acteurs influents dans l'application des méthodes quantitatives à la sociologie \autocite{Lazarsfeld1954} aux États-Unis mais également en France (il collabore avec Boudon sur plusieurs projets, d'enseignements et de publications) et à l'international \autocite{Lecuyer2002}, que James Coleman publie en 1964 \textit{Introduction to Mathematical Sociology} \autocite{Coleman1964}, un ouvrage devenu une référence en sociologie quantitative dont on peut lire un résumé élogieux dans la \textit{Revue francaise de Sociologie} réalisé par \textcite{Boudon1966} en 1966.} et de simulation, celui-ci considérant cette dernière {[...] as a half-way point between verbal speculative theory and formal theory, aiding in the development of such theory through concretizing the functioning of \foreignquote{english}{social processes}. \autocite[36]{Guetzkow1972}}.

Celui-ci travaille sur des simulations liées à ses recherches sur l'éducation au début des années 1960 aux États-Unis, dont il a déjà publié des travaux dans l'ouvrage inter-disciplinaire de \textcite{Guetzkow1962} en 1962, et qu'il publie ensuite \autocite{Coleman1965} dans une des premières revues abordant la méthode de simulation en sociologie, un numéro spécial des \textit{Archives Européennes de Sociologie} introduit par \textcite{Boudon1965} en 1965. Dans ce  numéro figure également une des premières traductions de la simulation de diffusion d'Hägerstrand \autocite{Hagerstrand1965} utilisant la technique de Monte-Carlo, un modèle qui recoupe les préoccupations du vaste courant inter-disciplinaire dit des SNA (Social Network Analysis) \autocite{Bernard2005}, qui touche tout autant aux structures de parenté (voir le paragraphe suivant pour des références en Anthropologie), qu'à la géographie (Hägerstrand à Lünd), ou à la sociométrie (modèle du sociologue mathématicien Coleman \textcite{Coleman1957}, mais également modèle de \textcite{Rapoport1961}, un biomathématicien de Chicago et confondateur avec Boulding, Gerard et Von Bertalanffy de la société pour l'étude des systèmes généraux, ou GST) \footnote{Une analyse croisée entre des modèles de différentes disciplines sur la diffusion des innovations, contenant notamment les modèles d'Hägerstrand et de Rapoport a été publiée en 1968 dans la revue \textit{Lund Studies in Geography} par \textcite{Brown1968}}. Du point de vue français, outre l'analyse de Boudon sur ce sujet dans le numéro spécial de 1965, on trouve également une revue de ces mêmes avancées en simulation du côté de la sociologie politique (qui recoupe la psychologie sociale américaine), un état de l'art réalisé par \textcite{Padioleau1969} dans la \textit{Revue francaise de sociologie} en 1969.

\Anotecontent{archeo_stat}{Des transferts parfois étonnants en provenance d'autres disciplines, comme le montre cette citation : \foreignquote{english}{Similar trends are apparent in allied subjects such as anthropology and social geography. In particular, location analysis has influenced archaeologists, with its emphasis on the study of all aspects of a population and its environment, and on the use of quantitative methods and models (Haggett 1963)} \autocite{Doran1970}}

\Anotecontent{archeo_systemique}{Une analyse a posteriori confirme l'apport de la systémique dans la construction des modèles de simulation, comme en témoigne \textcite[5]{Lake2013} et de façon plus précoce \textcite{Aldenderfer1998} en 1988. \foreignquote{english}{One of the theoretical hallmarks of the \textit{New Archaeology} was the systems approach \autocite{Aldenderfer1991}, and a result of its adoption was the use of computer simulation to model whole societies or significant portions of them.}}

\Anotecontent{whallon_simulation}{\foreigntextquote{english}[Whallon1972, 38]{The techniques and procedures of computer simulation so closely parallel the current thinking and processes of model-building of many archaeologists that the lateness and limits of their application are surprising.}}

%% FIXME ORTHOGRAPHE
En archéologie, dans la très claire retrospective historique faite par Gary Lock en 2003\autocite{Lock2003} sur l'histoire de l'archéologie computationelle, l'auteur s'attache à bien séparer au moins deux sinon trois époques aux méthodologies et aux outils différents. En adoptant une posture un peu simplificatrice on peut donc affirmer que si l'archéologie pre-années 1960 se base principalement sur la récolte de données empiriques et la mise en exergue de pattern dans ces même données pour générer la plupart de ces explications, une rupture dans la discipline se dessine dès les années 1960-70 avec l'avénement d'un courant d'archéologie proclamant une \enquote{ new archeology} (ou \emph{processualism}). Rejettant un empirisme beaucoup trop subjectif, celle ci vante le retour à la seule \enquote{ Méthode Scientifique } pour générer des explications. 

\Anotecontent{wilcock_stat}{On trouve un récit plus détaillé de l'arrivée des méthodes statistiques en archéologie dans la publication de \autocite{Wilcock1997}}

\Anotecontent{caa}{Il est intéressant de noter que ces quelques archéologues pionniers en informatique ont très vite créés leur propres canaux de diffusion en angleterre. Si de multiples conférences pour le développement des aspects computationels en archéologie existent à la charnière 1960-1970 (Rome, New-York, Marseille) \autocite{Wilcock1997}, ce n'est qu'en 1973 que se forme sous le patronage de quelques chercheurs anglais la première \foreignquote{english}{Computer Applications and Quantitative Methods in Archaeology Conference} \href{http://caaconference.org/about/}{@CAA}. Celle-ci se tient sa première édition à Birmingham, et deviendra par la suite en 1992 une conférence à portée internationale. La particularité de cette conférence, qui existe toujours, est son inter-disciplinarité; le comité d'organisation militant toujours pour la rencontre et le dialogue entre  archéologues, mathématiciens et informaticiens. A l'ocasion des 40 ans de la conférence en 2012, le projet \foreignquote{english}{Personnal-Histories Project} à permis la collecte et la mise à disposition de témoignages vidéo des pionniers sur le site de \href{http://www.sms.cam.ac.uk/collection/750864}{@Cambridge}}

Si la critique de 1962 opéré par \textcite{Binford1962} cristalise pour beaucoup cette rupture, 1968 est également considéré comme une année particulièrement importante pour la structuration de ce courant dans la discipline. L'avénement de plusieurs publications phares vient souligner l'émergence progressive dans les années 1960-70 de nouveaux outils \Anote{caa}, à la fois computationels comme les statistiques \Anote{wilcock_stat} ou la simulation \autocite{Clarke1968} , ou plus conceptuels avec l'ancrage de la \foreignquote{english}{New Archeology} dans la pensée systémique \autocites{Clarke1968, Flannery1968, Binford1968} \Anote{archeo_systemique}. Des avancées qui fournissent un véritable support à ce changement des pratiques dans la discipline.

\Anotecontent{doran_intuition}{\foreignquote{english}{There has now been a wide variety of experiments involving computer processing of archaeological data. Clarke (1968) describes many of them, and another valuable source is Cowgill (1967). I do not propose to discuss these experiments here, important though they are. [...] In this final section I shall briefly present the computer in what seems to me to be a much more promising and interesting role, which has as yet received rather little attention from archaeologists, even though in some ways it can be regarded as the practical equivalent of systems theory. I mean the use of a computer to construct and test a \enquote{simulation} of some complex system evolving in time. [...] Indeed, one of the great advantages of using a computer program to simulate evolving systems is that a much wider range of possibilities can be accommodated than can be expressed mathematically.} \autocite{Doran1970}}

Si \textcite{Binford1968} représente le point de vue américain, Clarke présente en angleterre et à la même période (\textcite{Clarke1968} est édité en 1968 par Binford) un point de vue un peu différent sur la New-Archeology \autocite{Binford1983}. Clarke est en effet sous l'influence des idées animant le campus de Cambridge, un haut lieu de changement ayant déjà accueilli une autre révolution, celle de la \textit{New-Geography} \Anote{archeo_stat}. C'est dans cet environnement que Clarke publie en 1968 un premier livre \foreignquote{english}{Analytical Archeology} qui démontre le potentiel que pourrait avoir les statistiques spatiales, les modèles et la simulation stochastique en archéologie \autocites{Clarke1968, Clarke1972} (ce dernier meurt jeune en 1976). Il est accompagné dans ses travaux par l'expertise, la volonté et les intuitions pionnières \Anote{doran_intuition} de James Doran \autocite{Doran1970} qui écrit également avec Hodson en 1975 l'ouvrage devenu référence \foreignquote{english}{Mathematics and Computers in Archaeology} \autocite{Doran1975} faisant état des travaux utilisant les toutes dernières techniques computationelles à la fois en traitement de données, et en simulation (Chaîne de Markov, Monte-Carlo, langage pour la simulation Dynamo, GPSS, etc.)

Ce militantisme qui semble recevoir un écho positif tout au long des années 1970 \footnote{On pourra trouver plus d'informations sur les premiers travaux dans les ouvrages cités précédemments, et via des retrospéctive plus récente comme celle de \autocite{Kohler2011}, ou \autocite{Lake2013}}, certains auteurs comme \textcite[38]{Whallon1972} n'hésitant pas à définir\Anote{whallon_simulation} la simulation comme un prolongement naturel à la pratique existante de construction des modèles. Cette mise en oeuvre de programmes pionniers se poursuit avec une diversification dans les usages jusqu'au début des années 80 et constitue une première phase d'appréhension de la simulation, plus qu'une adoption massive par la discipline. \autocite{Lake2013}

% VOir aussi Mathematics and Computers in Archaeology doran 1975, partie sur la simulation cf http://books.google.fr/books?id=ZAPvXcnz0kkC&pg=PA369&lpg=PA369&dq=The+computer+in+archaeology:+A+critical+survey+whallon&source=bl&ots=6et-F8jHab&sig=gQWgTIHRuO2ICqMJtrRdGovo9gs&hl=fr&sa=X&ei=OskxU5W5Nen20gW0_4DIDA&ved=0CGUQ6AEwBQ#v=onepage&q=whallon&f=false

%\autocite{Clarke1987}

A la croisée de plusieurs disciplines, sociologie, anthropologie et géographie on trouve les modèles de variation de population, ou modèles démographiques dont les hypothèses sont amenées à varier selon des facteurs biologiques, économiques, spatiaux faisant souvent appel à une dynamique des interactions humaines impossible à expérimenter dans la réalité. \footnote {\foreignquote{english}{To understand how changes in the size and composition of human populations occur, it is essential to study the determinants of these changes and the interrelations among them. The impossibility of investigating these relationships experimentally stimulates the formulation of models, as a means of enhancing our understanding of the process.} \autocite{Sheps1971}} Dans cette branche se côtoient donc macro-simulation, micro-simulation et modèle analytique hérités des premiers démographes mathématiciens, comme le plus connu d'entre eux, Lotka dont les premières publications sur le sujet datent de 1907 \autocite[355]{Veron2009}.

%%FIXE CLEMENTINE : c’est intéressant, que la plupart des travaux pionniers que tu cites  apparaissent dans les urban studies.  Est-ce que c’est la ville qui est si complexe ou une dépendance au chemin des méthodes dans les champs d’études ? ou parce que urban studies est particulièrement interdisciplianaire que ça a dépasser les barrières des affiliations disciplinaires ?

Les modèles TRIM, puis DYNASIM (entre 69 et 75) développés par Orcutt et son équipe à l'\foreignquote{english}{Urban Institute} sont pionniers \autocite{Orcutt1957, Orcutt1960, Orcutt1976}, et inspirent différents modèles dynamiques en démographie avec les travaux de \textcite{Perrin1964}, \textcite{Sheps1971}, et \textcite{Ridley1966} avec REPSIM aux États-Unis,  \textcite{Hyrenius1964} en Suède, \textcite{Horvitz1971} avec POPSIM, ou encore SOCSIM basé sur les travaux en anthropologie de \textcite{Gilbert1966}, qui viennent compléter efficacement les modèles analytiques inspirés des travaux de Lotka \autocite{Sheps1971}, père entre autre de la démographie mathématique moderne. Coïncidence de l'histoire, ou inspiration commune, Hägerstrand apportera de façon parallèle en géographie, et dans la même décennie \autocite{Hagerstrand1952, Hagerstrand1967}, une vision micro similaire, à cela près qu'elle y ajoute un ancrage spatial des individus.

Dans le cas de l'anthropologie, qui partage un tronc commun avec nombre de problématiques en archéologie, et en psychologie, on retiendra le manuel édité par \textcite{Hymes1965} retranscrivant une conférence de 1962. Celui-ci contient deux articles importants pour la discipline, celui de \textcite{Gullahorn1965} et celui complémentaire de \textcite{Hays1965}. L'intégration de la simulation dans l'arsenal méthodologique prend part selon \textcite[274]{Bentley2009} d'un mouvement ayant pour objectif de mieux comprendre les contraintes sociales et culturelles dans les processus démographiques en général. Dans ce cadre par exemple de l'étude de la parenté ou \foreignquote{english}{kinship}, l'application de la simulation donne lieu à plusieurs expériences pionnières \autocite{Dyke1981} en simulation comme celle de \textcite{Kunstadter1963}, mais aussi de \textcite{Gilbert1966}. Cet engouement continuera dans les années 1970 \autocite{Read1999} avec des simulations mettant en œuvre des processus stochastiques dynamiques comme par exemple dans les travaux de \textcite{Howell1978} et \textcite{Thomas1973}.

%\autocite{Costopoulos2007} . %Antony Wallace également, levy strauss 1955: les mathématiques de l'homme...

%En utilisant la simulation non pas comme un solveur d'équation mais en utilisant la puissance des opérateurs symboliques à sa disposition pour la mise en temporalités de systèmes d'interaction dans des sociétés passées, Doran décrit une vision de la simulation qui n'est pas sans rapeller le multi-agent d'aujourd'hui. Une conception de la simulation reprise et concrétisée par DH Thomas en 1972.\footnote{La discussion sur  \href{www.jiscmail.ac.uk/cgi-bin/webadmin?A2=ind04\&L=simsoc\&F=\&S=\&P=39083} {@SimSOC}} 


% http://books.google.fr/books?id=G8sA95bz5pwC&pg=PA143&lpg=PA143&dq=%22Social+Physics%22+stewart+cybernetics&source=bl&ots=FsOC2mqHvr&sig=cS914G7pelGvgG6bG32fKsmWWPc&hl=fr&sa=X&ei=yTRAU5m7OIOH0AXwtYEY&ved=0CDsQ6AEwAQ#v=onepage&q=%22Social%20Physics%22%20stewart%20cybernetics&f=false
% "Social Physics" stewart cybernetics
% http://www.eoht.info/page/Princeton+Department+of+Social+Physics
% http://books.google.fr/books?id=F84mS2nnjWsC&pg=PA105&lpg=PA105&dq=geographer+reino+ajo&source=bl&ots=buVSBElr7Y&sig=_NXU0Py2goM2c6fVi1To3dUwqHQ&hl=fr&sa=X&ei=0DBAU_niJqrO0AWk44CwCw&ved=0CG8Q6AEwBw#v=onepage&q=geographer%20reino%20ajo&f=false
% http://www.persee.fr/web/revues/home/prescript/article/ingeo_0020-0093_1957_num_21_5_6491_t1_0223_0000_5#
% http://www.eoht.info/page/Social+physics
% Contributions to "social Physics" reino ajo
% Stewart, J.Q. "The Development of Social Physics"

\subsection{Les premiers modèles de simulation en géographie}
\label{sec:premier_modele_geo}

\subsubsection{Une \enquote{révolution quantitative} au cœur de multiples convergences}
\label{ssec:revol_quanti}

L'apparition et la diffusion de ces techniques quantitatives n'est pas le résultat d'une convergence unique, mais bien d'une succession de moments dont la fréquence et l'étalement temporel est difficile à cerner et empêche sur ce sujet toute exhaustivité. 

On retiendra toutefois plusieurs grands facteurs, à la fois généraux, et d'autres plus spécifiques à la géographie, dont certains qui peuvent paraître étonnamment antinomiques. Une convergence qui s'illustre dans la richesse et la diversité des transformations qui touche la discipline géographique entre 1950 et 1970, un constat déjà établi par bien d'autres auteurs \autocite{Varenne2014}

%[28-29]Claval2003
%http://books.google.fr/books?id=s5xjIsejTjkC&pg=PA28&lpg=PA28&dq=h%C3%A4gerstrand+positivism&source=bl&ots=FrIMA95glO&sig=9Knqs1cLfJJefcc30qwsIMDzW-s&hl=fr&sa=X&ei=UMVDU86hJ-mS1AWPmIDoCw&ved=0CC4Q6AEwADgK#v=onepage&q=h%C3%A4gerstrand%20positivism&f=false

\paragraph{L'influence de l'école néo-positiviste}

Le néo-positivisme, néo-empirisme, positivisme logique selon les étiquettes, est un mouvement philosophique important, sinon peut être le plus important, entre les deux guerres. Ce cercle dont on trouve les premières traces dans les années 1908 à Vienne, est organisé autour de grands débats, dont la caractéristique est d'être fréquenté par un grand nombre d'intellectuels, issus de plusieurs disciplines. Tout au long de son évolution caractérisée par différentes phases (avec une apogée durant la troisième phase entre 1928-1934), de multiples courants d'opinions \textcite[126]{Ouelbani2006} vont être amenés à se côtoyer, du fait des débats internes, mais aussi des critiques extérieures au cercle. C'est donc à ce titre que \textcite[11]{Ouelbani2006}, préfère parler de \enquote{programme néo-positiviste} \footnote{Le programme de Carnap tient en quatre points selon Dahms, cités par \textcite{Ouelbani2006} : (i) la réduction de la philosophie à une théorie de la connaissance; (ii) la distinction des sciences, non plus en sciences de la nature et sciences humaines, mais en sciences empiriques et analytiques: (iii) le logicisme comme programme de réduction des sciences analytiques; (iv) le réductionnisme comme programme de réduction des sciences synthétiques ou empiriques.} plutôt que d'un réel courant unifié.

Inspirés des sciences naturelles, et plus particulièrement d'une observation des sciences physiques et mathématiques, les tenants du programme néo-positiviste sont motivés par l'unification des sciences, et pensent l'application d'un tel programme incontournable pour fonder des sciences sociales véritablement \enquote{scientifiques}. \textcite[1-20]{Ouelbani2006}

Les positivistes logiques ont ceci de particulier qu'ils raisonnent sur des démonstrations logiques encapsulant les énoncés observationnels décrits dans une logique formelle qu'ils veulent non ambiguë. Entre empirisme et logicisme, ce programme réductionniste \footnote{Voir la définition du programme donné par Carnap dans la note précédente.} fait porter toute la connaissance sur l'expérience; ce qui mène avec l'aide de l'analyse logique et mathématique à l'élimination de toute métaphysique, et de toute structure a priori (anti-kantien) dans la construction des énoncés d'observation. Ainsi l'inférence déductive se fait seulement sur des énoncés d'observations qui sont \foreignquote{latin}{a posteriori} tout à fait justifiables, et donc mobilisables dans celle-ci seulement si ils cohérents.

Ian Hacking \autocite{Hacking1983} a ,selon Orain \footnote{Voir les notes de \href{http://www.esprit-critique.net/article-12642840.html}{@cours}, dispensés sur le blog \enquote{esprit critique} de Olivier Orain} très bien saisi ce qui fait les axes communs \footnote{Le positivisme peut se définir par quelques idées forces. (1) L’importance accordée à la vérification (ou à une variante comme la falsification) : une proposition n’a de sens que si l’on peut, d’une quelconque manière, établir sa vérité ou sa fausseté. (2) La priorité accordée à l’observation : ce que nous pouvons voir, toucher ou sentir fournit, sauf pour les mathématiques, la matière ou le fondement le plus appréciable de la connaissance. (3) L’opposition à la cause : dans la nature, on ne trouve pas de causalité dépassant ou surpassant la constance avec laquelle des événements d’un certain type sont suivis par des événements d’un autre type. (4) Le rôle mineur joué par l’explication : expliquer peut contribuer à organiser des phénomènes mais le pourquoi reste sans réponse. On peut seulement remarquer que le phénomène se produit régulièrement de telle ou telle manière. (5) Opposition aux entités théoriques : les positivistes ont tendance à être non réalistes parce qu’ils limitent la réalité à ce qui est observable mais aussi parce qu’ils s’opposent à la causalité et se méfient des explications. Leur rejet de la causalité les fait douter de l’existence des électrons simplement parce que ces derniers ont une action causale. Ils soutiennent qu’il s’agit là seulement de régularités constantes entre phénomènes. (6) L’opposition à la métaphysique est finalement le dénominateur commun entre les points (1) à (5) ci-dessus. Propositions invérifiables, entités inobservables, causes, explications profondes, tout cela, dit le positiviste, est objet de métaphysique et doit être abandonné. \autocite[82]{Hacking1983}.} des différentes relectures du terme positivisme. Une parenté qui dans le cas du programme néo-positiviste est difficile à isoler tant l'acceptation par les proches (comme Popper) ou membres du programme (certain préféreront même le terme empirisme logique) est amené à varier, on pourra ainsi se référer à la classification proposée par Hacking pour en savoir plus sur ce sujet. \autocite[81-86]{Hacking1983}

L'apogée du groupe à Vienne est de courte durée, avec les pressions du régime nazi et l'annexion de l'Autriche, le groupe est dissous. De nombreux acteurs du mouvement sont alors contraints à l'exil, et nombreux sont ceux qui vont aux États-Unis. A ce moment-là, ce programme philosophique est alors quasiment inconnu des philosophes pragmatistes américains, mais paradoxalement c'est sur ce nouveau territoire qu'il va trouver un très bon accueil. 

C'est sur cette philosophie pragmatique depuis longtemps installée (Peirce, Dewey) que vient se greffer ce nouveau programme, jusqu'à finalement quasiment l'éclipser. Un transfert que l'on n'imagine pas totalement unilatéral, et il est presque évident que le discours originel viennois tire largement profit d'une philosophie pragmatiste compatible dans ses fondements \footnote{ Ainsi selon \textcite[149]{Ouelbani2006} Carnap aurait été rassuré en 1935, date de son arrivée aux Etats-Unis, \enquote{ [...] de trouver une ambiance philosophique différente,en ce sens que les jeunes philosophes étaient intéréssés par des méthodes scientifiques et logiques}}. C'est ce que \textcite[123]{Wilson1995} affirme en disant que les pragmatistes \foreignquote{english}{[...] had created the conditions in which logical positivism and other analytic philosophies could flourish and ultimately displace them as the dominant voice in mid-century philosophical debates} mais aussi les conditions de son dépassement \foreignquote{english}{Pragmatism, then, not only created the conditions in which logical positivism and analytic philosophy could flourish in the United States, it also contained the seeds of the postanalytic philosophies that have attempted to move beyond [...] }. Ce programme va se diffuser à la fois sur les bancs des universités, mais aussi via les grands instituts scientifiques après guerre qui font publicité de cette science \foreignquote{english}{mainstream}, organisée aux Etats-Unis autour de l'ordinateur. 

La RAND fait partie de ces instituts fondés après guerre, qui approche dès 1947 les sciences sociales \autocite{Rand106}, et n'hésitent pas à mettre en avant par la suite les stars de la philosophie positiviste de l'époque comme Reichenbach \autocite[384-385]{Barnes2011} .

\paragraph{L'apparition de mouvements inter-disciplinaires fédérateurs}

L'apparition de grands mouvements de convergence inter-disciplinaires et leur intérêt pour l'application de nouveaux concepts et techniques aux sciences sociales, dont certains prennent par la suite la forme de paradigme du fait de leur portée d'application : Cybernétique de Wiener, \textit{projet} de la \foreignquote{english}{General System Theory} de Bertalanffy \autocite[9]{Pouvreau2013} s'organise autour de grandes structures de recherches comme le MIT, la RAND, qui favorisent les collaborations par la mise en place d'équipe de travail pluri-disciplinaire.

Parmi les ramifications directes de ces coopérations, on citera par exemple la \enquote{social physics} de Stewart \autocite{Stewart1947}. Du fait des liens développés à l'université de Pennsylvanie, lieu de ses études, et siège de la fondation de la science régionale d'Isard en 1954, Stewart sera amené avec sa rencontre avec Warntz, un géographe atypique qui plonge très tôt dans l'inter-disciplinarité, à publier dans la revue \textit{Regional Sciences} \autocite{Stewart1958}.

Les retombées de ces interactions sur la géographie sont importantes \footnote{ A condition de ne pas oublier qu'une partie de ces concepts existent de façon sous-jacente aux disciplines, ce qui explique parfois leur rapidité d'acceptation. C'est le cas de l'approche systémique développée par la cybernétique quand elle ne fait pas qu'apposer un nom commun sur des concepts déjà étudiés, mais fait alors écho à des révolution méthodologiques en attente d'être activée. \textcite[5]{Batty1976} résume la situation ainsi \foreignquote{english}{The idea of systems being described in terms of structure and behaviour, in terms of input and output, and the notion of purposeful control of such systems in terms of negative and positive feedbacks, appeared to many social scientists an ideal description of their systems of interest and thus the approach has come to be used in more-or-less all of the social sciences}.}, et fournissent tout autant : (i) des concepts généraux en correspondances avec les débats qui animent l'ensemble des sciences : sensibilité aux conditions initiales, équifinalité, bifurcation et catastrophe, boîte noire, rétro-causalité, hiérarchie d'emboîtement, etc.) , (ii) un catalogue d'isomorphisme supplémentaire dont la correspondance reste à évaluer dans notre discipline \autocite{Wilson1969}, (iii)  une méthodologie et une typologie des modèles tirés de la recherche opérationnelle \autocite{Ackoff1962} \footnote{Une discipline proche du projet Bertalanffien en bien des aspects, comme le défend \autocite[801]{Pouvreau2013}} et largement revendiqués par les géographes dans la décennie 1960-70, un constat tiré de la lecture d'états de l'art \autocite{Kohn1970}, ou d'ouvrage phare sur le sujet comme \autocite{Berry1964a, Haggett1965}, (iv) la découverte d'une nouvelle science mathématique de la dynamique en correspondance avec ces nouveaux concepts, accessible soit par un vocabulaire graphique opérationnalisable \autocite{Forrester1961}, soit par des langages de programmation plus traditionnels !

On citera parmi les pionniers d'une exploration volontaire de cette convergence en géographie, Haggett en 1965 \autocite{Haggett1965}, Chorley avec la géomorphologie en 1962 \autocite{Chorley1962}, Berry avec les villes en 1964 \autocite{Berry1964a}

\paragraph{Les influences des \enquote{passeurs de modèles}}

\Anotecontent{footnote_kant}{Edgar Kant (1902-1978) un géographe déjà rompu aux méthodes statistiques en Estonie \autocite{Chabot1937} - où il avait déjà pu appliqué ses méthodes - s'est expatrié d'Allemagne avec dans ses bagages les travaux de Christaller, Lösch, etc. Tuteur d'Hagerstrand il le forme aux différentes méthodes qui vont se répercuter sur ses travaux de thèse.}

Ces influences se sont réalisées à l'échelle internationale par Torsten Hägerstrand, Edgar Kant \Anote{footnote_kant}, Christaller et Lösch \autocite[119]{Berry1970}, ou dans un cadre plus national avec le travail de traduction ou de mise à disposition de textes originaux par les économistes et géographes Lösh, et Isard.

\paragraph{La conjoncture politique favorable}

L'impact de la conjoncture politique et l'importance de grands \textit{Think-Tank} comme la RAND, et du MIT qui remobilisent en sortie de guerre des armées d'ingénieurs alors désoeuvrés sur des missions plus scientifiques. On soulignera à la même période le rôle joué chez les géographes par Ullmann, Harris, Ackerman dans la transformation institutionnelle de la géographie, dont la qualité en tant que corps de métier a pu être remarquée en temps de guerre. Cela se traduit sur la durée par un financement de la marine (\textit{Office Of Naval Research}), qui profite aussi de la nouvelle \textit{Regional Science} fondée par Isard. On trouvera plus d'informations sur ces inter-relations entre instituts après guerre et leur impact sur l'établissement d'une géographie quantitative dans les publications de \textcite{Barnes2006a}.

\subsubsection{D'une révolution quantitative à une révolution des modèles}
\label{ssec:revol_modele}

De cette \enquote{révolution quantitative} aux origines on le voit multiples, certains auteurs préfèrent ne retenir qu'une certaine essence de cette volonté nomothétique. Cette \enquote{révolution des modèles} comme préfère en parler \textcite{Wilson1970, Varenne2014} fait ici écho à cette déferlante de modèles qui apparaissent dans la décennie 1960-1970, et dont on trouve un recensement quasi exhaustif dans plusieurs ouvrages de référence \autocite{Haggett1965,Chorley1967}.

Une fois révélée cette profusion d'approches sous jacentes à l'emploi, parfois confus, d'un même terme, s'ensuit chez les géographes une tentative de classification, de définition de cette pratique de modélisation. Il en ressort des typologies, l'évocation de divers substrats ( analogique, iconique, symbolique ) la plupart du temps empruntés dans les ouvrages de spécialistes alors disponibles. Ainsi les deux sources d'inspirations de \textcite[106]{Berry1963}, \textcite{Haggett1965} sont à ce moment-là des références issues d'un rapprochement avec la Recherche Opérationnelle (RO) \footnote{On en trouve trace également dans des collectes bibliographiques à destination des enseignements comme \autocite{Greer1972}}, une discipline pionnière dont le développement après-guerre oeuvre pour l'application et la diffusion de méthodes numériques en vue de résoudre des problèmes extrêmement diversifiés. On retiendra des auteurs comme \textcite{Ackoff1962} (déjà cité par Ackerman en 1958) ou \textcite{Kemeny1962}

% détails typologies ?
\paragraph{Une autre définition des modèles et de la modélisation}
\label{p:autre_def_modele}

Alors que dans les faits beaucoup de choses ont changé sur le plan des pratiques, des techniques, des institutions, la référence à des définitions datant de 1965 reste après les années 1990 tout à fait acceptable \autocites{Dastes2001b, Antony2013}[295]{Bailly1995}, et sert encore comme base de travail solide pour établir de nouvelles réflexions \autocite{Brunet2000}. 

Comme nous le rappelle dès 1965 Peter Haggett, le modèle est pour les géographes avant tout un construit. En s'appuyant sur la typologie et la réflexion d'Ackoff, il définit ainsi dans \textit{l'analyse spatiale en géographie humaine} : \enquote{En construisant un modèle (\textit{model building}), on crée une représentation idéalisée de la réalité afin de faire apparaître certaines de ses propriétés } \autocite[30]{Haggett1965}. 

% Brunet2000 définit également le modèle comme "processus de recherche" p28

A la différence de la définition donnée par Varenne \footnote{Franck Varenne propose un panorama beaucoup plus large et générique de la notion de modèle dans son ouvrage \textit{Théorie,Réalité, Modèle} paru en 2012. \autocite{Varenne2012}} et inspirée de Minsky (section \ref{ssec:rapell_termes_generiques}), celle de Haggett en 1965 met l'accent sur l'activité même de modélisation. Ce faisant, ce n'est plus tant la fonction définitive du modèle qui est mise en exergue ( \enquote{le pourquoi} motivant la sélection des propriétés saillantes) mais sa dimension en tant que construit.

%modélisation = diachrnoqiue, temp long
%synchronique = extraction modele; temps court

Pour \textcite[36]{Langlois2005}, \enquote{le terme de \textit{modélisation} désigne à la fois l'activité pour produire un modèle et le résultat de cette activité.} Le concept de modélisation est donc \enquote{[...] plus large que celui de modèle, car il recouvre l'activité humaine qui aboutit au modèle achevé, alors que le modèle est un objet (concret ou abstrait), volontairement dépouillé de l'activité qui l'a créé.} 

Ainsi en généralisant encore un peu plus les propos de Langlois, l'activité de modélisation est un processus qui s'inscrit dans un temps long, alors que le modèle peut être vu comme le résultat d'une extraction correspondant à un instantané de cette activité. Ainsi de l'ensemble des choix qui ont constitué sa formation, le modèle ne porte plus après extraction qu'une histoire partielle de sa construction. Dans ce processus, toute opération cognitive qui n'est pas explicitement relatée est alors perdue dans cette compression d'informations.

%%FIXME CLEMENTINE : ça me fait penser à un article de Drogoul et al, 2003 : ou clairement, la modélisation est du temps long ET du collectif puisqu’il y a 3 rôles pour 3 modèles : thématique, conptuel, implémenté.

Un processus qui n'est pas limité à la seule construction de modèle de simulation, et s'applique à la construction de n'importe quel modèle, comme le présente très bien \textcite[32-33]{Haggett1965} lorsqu'il évoque les deux voies possibles de construction de modèles théoriques : Dans la \textbf{première méthode}, que l'on pourrait qualifier de complexification progressive, \enquote{[...] le chercheur aborde \enquote{furtivement} un problème; il pose d'abord des postulats très simples et introduit peu à peu des complications, en se rapprochant toujours davantage de la réalité. Ainsi procède Thünen (1875) dans le modèle d'utilisation du sol qu'il présente dans son livre \textit{Der Isolierte Staat} (chap. 6, section 2) [...]}; méthode qui autorise la divergence, le retour en arrière sur les hypothèses. Si au départ Thünen \enquote{[...] Dans cet \enquote{Etat isolé} [...] suppose d'abord l'existence d'une seule ville, d'une plaine uniforme horizontale, d'un seul moyen de transport, et d'autres faits tout aussi simples[...]}, celui-ci \enquote{[...] brouille ensuite cette image en réintroduisant les objets mêmes qu'il avait tout d'abord supposés inactifs : sol de nature différente, marchés entre lesquels on peut choisir, moyens de transport divers.} La \textbf{seconde méthode}, symétrique, \enquote{[...] consiste à transformer la réalité par une série de généralisations simplificatrices}, qui permet comme dans le modèle de Taffe et Morrill (voir la description faite par \textcite[93-96]{Haggett1965}) de généraliser sur une base d'observations empiriques un certain nombre d'étapes stylisées qui interviennent dans le développement des voies de communication au Ghana.

%FIXME INTRODUIRE LE PASSAGE DU MODELE A LA SIMULATION DE MODELE, ET FAIRE UNE TRANSITION CORRECTE AVEC LA PARTIE D'APRES
Quand aux modalités guidant cette incrémentalité, celles-ci restent au demeurant très mystérieuses, et semblent plus relever au premier abord d'un art \autocite{Tocher1963, Axelrod1997} que d'une pratique véritablement rationalisée.

Le substrat de référence qui nous intéresse pour supporter les modèles est évidemment l'ordinateur. Or, si on se réfère au compte rendu réalisé par \textcite{Haggett1969} en 1969, celui-ci nous indique qu'à cette période l'ordinateur intervient dans au moins quatre usages qui font écho aux méthodes modernes considérées comme nécessaires selon \textcite{Claval1977} à l'évolution  de la géographie : (i) statistiques multivariées, (ii) surfaces de tendances, (iii) graphismes, (iv) simulation. 

Si on se réfère à la grille de fonctions établie par \autocite{Varenne2014}, celui-ci classe les modèles de cette époque comme étant en grande partie des modèles d'analyses de données, ou des modèles théoriques à visée explicative. Sur cette base, il faut pour être exhaustif également prendre en compte les modèles à visée prédictive pour l'aide à la décision, même si cela fait plus référence aux travaux réalisés dans le cadre des grands programmes de planification de la RAND, où les géographes mobilisés sont plus soumis aux directives d'ingénieurs que de chercheurs.

%Spécificité de l'objet d'étude "Le spatial et le temporel", objet d'étude des géographes
% FIXME : TRAVAILLER LE RACCROCHEMENT ENTRE LES DEUX 
%Partant de la grille proposé par Varenne \autocite{Varenne2013} il est possible de proposer un positionnement du modèle tel qu'on l'emploi le plus souvent aujourd'hui en géographie humaine quantitative; et de préciser le substrat sur lequel nous greffons différentes fonctions de médiations.

Afin d'illustrer l'importance de l'outil \enquote{simulation de modèles} dans la construction géographique théorique, et à condition d'accepter un découpage flou, on identifie deux grands moments innovants pour l'outil simulation en géographie, moments qui se juxtaposent partiellement dans l'espace et dans le temps.

D'une part il y a l'apparition et la rencontre au début des années 1960 de deux pôles académiques innovants avec d'un coté les universitaires américains et suédois, et d'autre part il y a cette montée en puissance simultanée des instituts de planning aux USA, pilotés par des \textit{Think-Tanks} comme la \textit{RAND corportation}, qui commande la construction de plusieurs modèles de simulations urbains entre 1959 et 1968 \autocite[307]{Batty1976}. 

\paragraph{La rencontre entre les pionniers américains et suédois}

Ce premier moment prend appui sur les fondements de ce que l'on appelle aujourd'hui \enquote{la révolution quantitative}, notamment du fait du caractère international et multi-site de cette contestation. \textcite{Gould2004} propose toutefois de s'attarder en particulier sur deux premiers foyers importants dans cette \enquote{révolte}. Le premier socle se situe dans quelques universités de l'ouest des Etats-Unis \autocite{Gould2004} parmi lesquelles Washington, Iowa et NorthWestern à Chicago; le deuxième socle est en Suède avec l'université de Lund; une liste à laquelle il faudra ajouter par la suite Cambridge qui va dans un troisième temps propulser sur le devant de la scène les \textit{terrible twins} Chorley et Haggett que l'on ne présente plus.

C'est à l'université de Iowa et de Washington, sous la direction de Ed Ullmann et William Garrison, considéré comme l'un des premiers \footnote {Le premier cours serait daté de 1954 sous l'intitulé (Geog 426: Quantitative Methods in Geography) } à voir l'intérêt général de l'usage de l'ordinateur pour la géographie, qu'à la fin des années 1950 se forme un groupe d'étudiants qui va marquer le renouveau de la géographie. L'innovation des thématiques abordées dans les publications, mais aussi des formations proposées va de pair avec l’entraînement mutuel qui anime cette équipe de jeunes doctorants, formés à l'inter-disciplinarité. Brian Berry, William Bunge, Richard Morrill, Duane Marble, Waldo Tobler etc. bientôt rejoints par Torsten Hägerstrand sont ainsi parmi les premiers à mettre en pratique les techniques computationnelles les plus récentes. \footnote{ On trouvera un aperçu de cette dynamique dans les articles plus généraux sur l'usage de l'ordinateur et des simulations en géographie à cette période dans les articles de \textcite{Haggett1969} et \textcite{Marble1972}}

Le déplacement de Torsten Hägerstrand de l'université de Lund aux Etats-Unis mérite une attention particulière, tant son impact sera important sur la discipline. Deux années après sa première publication en anglais en 1957, Hägerstrand est aussitôt repéré et invité par Garrison en 1959 à présenter ses travaux novateurs à une période, rappelons le, où la géographie est encore majoritairement idiographique en Angleterre mais aussi aux Etats-Unis. La rencontre a lieu à Washington dans un séminaire intitulé \foreignquote{english}{simulation modelling of the diffusion of innovation}. Encore réalisées à la main lors de sa venue à Washington \footnote{ \textcite{Barnes2006a} indique que le premier ordinateur sur le campus serait daté de 1955, un IBM 604}, les premières simulations Monte-Carlo \footnote{Pour la petite histoire, c'est via un voyage aux États-Unis que le physicien Karl Erik Frödberg, un ami d'enfance de Torsten Hägerstrand, récupère un texte polycopié présenté par John Von Neumman et Stanislas Ulam sur les méthodes de Monte-Carlo. Alors appliquées au calcul de l'épaisseur des chapes de béton pour les centrales nucléaires, la technique est utilisée pour pallier à une résolution impossible de ce problème via les approches mathématiques classiques. Hägerstrand ayant déjà travaillé à l'étude de l'émigration en 1949, trouvera dans cette technique un écho innovant à sa problématique d'alors, la propagation des idées et des innovations dans l'agriculture suédoise. \autocite[26-28]{Gould2004}]} impressionnent les disciples de Garrisson, notamment Morrill \autocite[120]{Unwin1992}, qui à la suite de cette expérience va partir plusieurs mois en Suède \autocite{Morril2005}, ce qui lui inspirera d'autres développements s'appuyant sur cette technique, avec une application notamment sur le ghetto de Seattle \autocite{Marble1972}.

Il est difficile de savoir si les travaux pionniers (voir \ref{ssec:crise_mutation}) de l'économiste Orcutt \autocite{Orcutt1957, Orcutt1960} qui prennent aussi un niveau micro pour étude, et utilisent la technique Monte-Carlo pour les simulations, ont percolé jusqu'aux oreilles de Garrison, déjà bien renseigné par ailleurs sur le plan de la recherche en économie par sa proximité avec Isard, ou si ces travaux usant de Monte-Carlo paraissent totalement novateurs à ce moment là; reste que la démonstration de ce couplage efficace entre nouvelles techniques et nouvelles questions impressionne \autocite[120]{Unwin1992}, et fait dire à \textcite{Morril2005} et \textcite{Gould1970} tout l'impact que ces travaux ont eu sur ses contemporains.

\Anotecontent{marble_hagerstrand}{A propos des échanges entres cette équipe de géographes développeur et Hägerstrand, voici un passage tiré d'un échange privé avec Duane Marble en Aout 2015. A la question suivante \foreignquote{english}{When and how did you have the idea to develop the first Hagerstrand program in Fortran ? And when you develop this program during the 1960's (and published in 1967 if i understand) did you work with Hagerstrand and Frödberg which give you some source code, or you get some informations with the trip made by Morril in Sweden when it work on the Ghetto ? or perhaps this is a totally independent work of your own after your read first translation of the Hagerstrand thesis/ paper in english, but \enquote{On Monte Carlo Simulation of Diffusion} is only published in 1967, so i suppose you work on this before this later translation, because i see you publish some paper about MIF in 1963, isn't it ? } Duane Marble répondra \foreignquote{english}{The development of the Hagerstrand model in Fortran was part of a joint research activity by Dr. Pitts and myself. At that time, you must understand, that there was no viable substitute for Fortran. This was an independent work although we both were in correspondence with Hagerstrand from time to time. The creation of the program was part of a larger project that involved field work in South Korea. I had left the University of Washington before Hagerstrand came to spend a term as a visiting professor, but I did meet him briefly while he was in the United States. My main contact with him occurred during the year I spent teaching in Sweden. This was not at Hagerstrand’s university but I did spend some time lecturing at Lund and I also saw him at various places around Sweden.}}

Il faudra attendre quelques années pour que les simulations soient effectives; en Suède, probablement en langage machine sur le premier ordinateur de l'université de Lund nommé SMIL(\foreignquote{sweden}{Siffermaskinen i Lund} ou \foreignquote{english}{The Number Machine in Lund}) construit sous la principale influence de Carl-Erik Froberg et que l'on sait utilisé très rapidement par Hägerstrand \footnote{\autocite[32-33]{Lindgren2008} Sten Henriksson relate \hl{(traduction à revoir)} à propos d'Hägerstrand : \foreignquote{english}{First Torsten Hägerstrand , he was active then in the mid - 50s , he was , shall we say, one of the world's leading human geographers and devoted himself to simulate stuff on SMIL , he was a childhood friend of Carl-Erik Froberg and was one of the first to use SMIL -56 and there are others such as these early adopters who have been proactive.}, suivi du témoignage de Axel Ruhe plus précis sur ses premier travaux : \foreignquote{english}{I will mention two of them, I do not know how much research it has led to , and was the geographic data processing Torsten Hägerstrand 59 who was a professor of human geography , I remember we ran a program on SMIL for possible locations of the Öresund bridge , how much shipping would be developed if we had it here or there. And then it was the location of the cinemas, roughly the same as going over the Öresund Bridge but on a smaller scale. It was a study of school children going to school and then also examined if they used the nearest way or another}}, un ami d'enfance de Froberg; et en Fortran aux Etats-Unis par le duo Pitts(1963), Marble(1967) \Anote{marble_hagerstrand} \autocite{Morril2005, Marble1972, Pitts1963}. Le modèle est traduit et publié en 1965 en Europe dans les \textit{Archives Européennes de Sociologie} \autocite{Hagerstrand1965}, et en 1967 \autocite{Hagerstrand1967a} aux États-Unis le plus connu sur cette technique. 

%%FIXME Ajouter témoignage de DUANE au dessus en rapport avec les deux dates !

Suite à cette publication de 1967, la spatialisation des processus de diffusion décrits par Hägerstrand vont inspirer le développement d'autres travaux en géographie et dans d'autres disciplines où le thème est déjà abordé au niveau macro, en sociologie avec la diffusion d'innovation chez Coleman, en épidémiologie \autocite{Cliff1981, Cliff2000} où la diffusion de processus a déjà été étudiée (Bailey 1957, Bartlett 1960) \autocite{Pitts1963, Morrill1968}, mais aussi dans les études de migration motivées en géographie par les travaux de Morrill, Pitts et Marble, dérivé de \autocite{Wolpert1965}, mais aussi de ceux de Cavalli-Sforza en 1962. 

\paragraph{L'influence de l'économie, entre travaux universitaires et commandes des instituts étatiques}

D'autres techniques de simulation, à la fois déterministes et probabiliste, sont également introduites à cette période en géographie, comme les méthodes de programmation linéaires, ou l'utilisation de chaînes de Markov \autocites{Marble1964, Clark1965} 

La percolation de ces techniques se fait en premier lieu via des mathématiciens \footnote{On pourra se référer à des ouvrages sur l'importance du complexe militaro-industriel de la RAND pour étudier son impact sur les mathématiques, et la science en général, du fait des larges financements, et des relations complexes qui existent entre les chercheurs et ces instituts} vers les économistes \autocite{Samuelson1952}, et son introduction chez les géographes est à chercher ensuite du côté des ouvrages pionniers d'Isard \autocite{Isard1956} \autocite{Isard1958} et son disciple Stevens \autocite{Stevens1958}.

Compte tenu de la proximité entre Isard, Ullman, Ackerman et Garrisson \autocite{Barnes2004} qui vont initier \autocite[120]{Unwin1992} par la suite plusieurs générations de géographes en s'appuyant sur des modèles d'économie spatiale dans le cadre des \textit{regional sciences}, il est normal de retrouver ces techniques opérationnelles innovantes \footnote{ \foreigntextquote{english}[Unwin1992, 120]{Garrisson argued that the use of algebraic notation and linear programming methods enable problems of location structure to be given an operational character, and that problems couched in such terms \enquote{\textit{display the price interdependencess associated with the location system in a manner which was not possible before}}}} assez rapidement dans les publications des géographes américains, comme cette première application assez connue de Garrison et Marts en 1958 \autocite{Garrison1958}.

%http://www.aag.org/cs/membership/tributes_memorials/gl/golledge_reginald

\paragraph{L'écho des premiers travaux individu centrés}

Un peu plus tard, et dans la continuité des travaux déjà réalisés dans les simulations mettant en œuvre des discrétisations de l'espace comme celle d'Hägerstrand, se sont les automates cellulaires qui apparaissent dans la continuité des travaux de von Neumann sur la théorie des jeux, dans les sciences sociales avec Sadoka (1949;1971) et Schelling(1969;1971) \autocite{Ganguly2003}, qui se diffuse par la suite en géographie principalement avec les travaux de Tobler. \autocites{Tobler1970b,Tobler1979}. On trouve une description plus détaillée de cette période dans \autocite{Louail2010}

%L'introduction de la dimension spatiale et temporelle est importante ici ... 

%Du coté des géographes, les pionniers Suédois de l'école de Lund et Américains de l'école de Washington saisissent dès 1960-70 cette opportunité d'accélérer la résolution de modèles explicatifs déjà éprouvés avec du papier et du crayon en usant des tout premiers ordinateurs; car c'est à cette époque que sont justement développés les premiers langages informatiques génériques, et même si ceux ci sont d'abord réservés à quelques élites pionnières ayant accès à du matériel et aux multi-compétences adaptés, très vite de jeunes chercheurs formés à l'interdisciplinarité vont permettre la diffusion de ce savoir faire (Morril, Marble, Tobler, etc.). 

Cet engouement constaté pour la simulation de modèles dans les sciences sociales est suivi peu après par une crise de confiance dont il existe peu de témoignages directs. Il faut le remettre en perspective dans un historique propre à chaque discipline, sur le plan spatial et temporel, ce qui rend extrêmement difficile la définition de ces contours. Plusieurs auteurs, la plupart du temps les pionniers, font toutefois l'état des difficultés rencontrées.

% Citer troizsch avec son schéma

\subsection{Une crise de confiance envers l'outil ?}
\label{sec:critiques_simulation}
\Anotecontent{starbuck_footnote}{Starbuck met l'effet de tassement des publications sur les trois dernières années sur le compte du nombre croissant de publications, impossibles à comptabiliser.}

Deux travaux de \textcite{Dutton1971} et \textcite{Starbuck1983} identifient un ralentissement des publications à partir de 1970. L'étude de 1971 est inédite, et consiste à éplucher et classer de façon exhaustive la littérature portant sur la simulation. Plus de 12000 publications en anglais pourront être classées, et plus de 2000 papiers seront identifiés traitant spécifiquement de la simulation en \foreignquote{english}{Human Behavior}. Si ce ralentissement dans la publication de simulations n'est pas forcément observable en 1969, date qui marque l'arrêt de l'étude \Anote{starbuck_footnote}, Starbuck constate par contre en 1983 la quasi-absence de nouvelles publications sur le sujet, voire pire, la remise au goût du jour de modèles de plus de 20 ans.

Une surprise qui finalement n'en est pas une, car dans l'étude de Starbuck en 1971, moins de la moitié des publications ne proposait aucun modèle implémenté, la plupart des études se bornant à une discussion méthodologique.

Pour appuyer et résumer ce constat assez terrible, Starbuck cite John McLeod, un scientifique pionnier qui travaille depuis plusieurs décennies dans des journaux dédiés à la simulation ( \textit{Simulation} créé en 1963 , et \textit{Simulation and gaming} en 1970): \foreignquote{english}{According to  John McLeod who has been involved with Simulation magazine for two decades, one primary reason for the methodology's sorry state is that simulators have overstated its capabilities and so, subsequently, disapointed their audiences.}

\subsubsection{Les principales disciplines touchées en science sociales}
\label{ssec:disciplines_touches}

\Anotecontent{temoignagne_lake}{\foreigntextquote{english}[Lake2013]{However, as already noted, archaeological simulation did not entirely die out during the 1980s, so it is worth considering the exact nature of this resurgence. In fact, I estimate that approximately ten archaeological simulation studies were undertaken in the 1980s and thirteen in the 1990s. Clearly neither is a large number in absolute terms, but nor is the increase anything approximating an order of magnitude.[...]  What we learn from them is that the resurgence of simulation in the 1990s was more a matter of perception that of the actual numbers of models being built.}}

\Anotecontent{temoignage_archeo_alden}{Pour ne prendre qu'un exemple des témoignages relevés chez les pionniers, celui d'Aldenderfer en 1988 expliquant que \foreigntextquote{english}[Aldenderfer1998]{During the 1980s, relatively few archaeologists continued to advocate whole-society modeling [...] While much of Doran's work has been widely cited within the relatively small community of mathematically inclined archaeologists, his work has had relatively little influence beyond this small circle}}

En \textbf{archéologie}, plusieurs témoignages \autocite[6-7]{Lake2013} font état d'une période de relative inactivité \Anote{temoignage_archeo_alden} qui démarre au début des années 1980. Après avoir cru pendant longtemps cette période comme une période morte, celle ci est aujourd'hui caractérisée par \textcite{Lake2013} comme une période de maturation bénéfique, marqué par un changement de discours , car plusieurs modèles déboucheront sur des résultats importants sont développés durant les années 1980, et seulement publiés après 1990. \Anote{temoignage_lake} donne également plusieurs pistes pour expliquer les facteurs à l'origine de cette inactivité, dont une particulièrement intéressante dans le cas de Hodder, qui est amené à la fin des années 1970 à faire un volte-face vis à vis des espoirs qu'il avait mis dans l'outil simulation. 

\foreignblockquote{english}[Lake2013,7]{In the introduction to his 1978 [...], Hodder expressed optimism about the utility of simulation in archaeology (Hodder 1978, p. viii), yet just four years later, in one of the founding works of postprocessual archaeology, he rejected the positivist inferential strategy and functionalism of the New Archaeology [...] (Hodder 1982). Ironically, the results of Hodder's own simulations (Hodder and Orton 1976) were a contributory factor in his volte-face because they revealed how the problem of equifinality could profoundly undermine attempts to quantitatively test hypotheses about settlement pattern and trade mechanisms}

A la problématique d'assimilation des techniques dont la complexification mathématique et conceptuelle courant des années 1970 ne cesse d'isoler les pionniers, vient se greffer la critique d'un mouvement émergent dans l'archéologie \foreignquote{english}{post-processualist} critiquant la \textit{New Archeology} qui va constituer un véritable frein au développement de modèle. Un mouvement appuyé par un fondement commun, et la chute du dogme néo-positiviste nait en parallèle en géographie une frange de géographes radicaux qui remet en cause à travers l'usage des modèles l'idéologie néo-positiviste courant des années 1970 (voir section )

Autre discipline, et même constat affiché par \textcite{Ostrom1988} en \textbf{psychologie sociale}. Alors que qu'il revendique en 1988 l'importance de la simulation comme un \foreignquote{english}{third way system} pour faire de la science, il fait également un constat assez accablant sur la place aujourd'hui tenue par cette pratique de modélisation dans le courant \foreignquote{english}{mainstream} de la psychologie. Ainsi dit-il \footnote{ \foreignquote{english}{Despite the clear relevance of these models to  social psychology, the simulation approach had not caught the imagination of main stream social psychologists. Very few simulations had appeared in the core journals of the field prior to the publication of Abelson's chapter. […] At the time of Abelson's writing, simulation models had not made much contact with the dominant empirical pursuits of the field. } \autocite[382]{Ostrom1988}}, force est de constater le peu de retours rapportés par la communauté face aux manifestes des pionniers tels que \textcite{Gullahorn1965} ou \textcite{Abelson1968} 

En \textbf{anthropologie}, \textcite{Dyke1981} \footnote{ \foreignquote{english}{Since that time there has been a considerable increase in the number of publications whose results have depended on simulation studies. Despite this increase, it is probably fair to say that simulation has received at best only a cautious acceptance in anthropology.} \autocite{Dyke1981} } dresse de son coté un maigre bilan, et donne, malgré l'augmentation du nombre de publications sur ce sujet, quelques éléments d'explications pour justifier ce désengagement de l'outil dans sa discipline, parmi lesquels figurent un possible effet de mode exagérant les capacités réelles de la simulation, et la problématique de la validation des modèles. \footnote{\foreignquote{english}{The initial enthusiasm for a newly acquired ability to model complex systems, characteristic of the early days of anthropological simulation, more often than not led to an exaggeration of the capabilities and usefulness of computer models.In retrospect it seems clear that much of this excess could have been avoided had more attention been paid to testing (particularly to validation). The literature of the past 4 or 5 years, however, gives ample evidence that the situation has changed. Those who continue to use simulation seem to have paid much more attention to the problem of validation and tend to be more modest in their claims of utility.}}

%%FIXME PAGE REF DYKE

Il semblerait donc que la pratique de la simulation en sciences sociales (sauf peut-être le cas spécifique de la géographie, traité dans la section suivante) se concentre dans les années 1980 sur de petites communautés de chercheurs, disposant de fortes compétences techniques initiales, qui vont continuer à travailler, à proposer des modèles, et à acquérir de nouvelles techniques et méthodologies en parallèle d'un courant plus \textit{mainstream} intégrant seulement les nouvelles capacités offertes par les ordinateurs mais délaissant l'aspect simulation. Dès les années 1990, plusieurs chercheurs pionniers, comme Jim Doran, réapparaissent conjointement avec l’avènement d'une nouvelle innovation dans les techniques de simulation; en partie dérivée des progrès en intelligence distribuée; un retour qui se fera avec plus de succès.

Parmi ces témoignages et en s'appuyant également sur les livres références \autocite{Naylor1966,Guetzkow1972,Dutton1971}, voici une liste forcément non exhaustive d'arguments évoqués par les auteurs pour justifier de cette baisse effective dans la confiance envers l'utilisation des simulations de modèles : \textbf{(1)} l'effet de mode initial qui exagère largement les capacités de l'outil pour expliquer ou prédire \textbf{(2)} les effets négatifs d'un rattachement volontaire ou involonaire à l'idéologie néo-positiviste, un programme épistémologique vivement critiqué courant des années 1970 dans plusieurs disciplines des sciences sociales, \textbf{(3)} la non-adéquation entre la richesse d'expression des théories sociales et la concision/réduction mathématique, \textbf{(4)} l'absence de standard de validation prenant en compte le cadre thématique, voire l'absence de validation tout court, \textbf{(5)} la non-adéquation avec un courant théorique \textit{mainstream} réfractaire, \textbf{(6)} les capacités encore limitées des ordinateurs de l'époque, pour le stockage des données, pour l'exécution des programmes, pour l'exécution des analyses sur les modèles, et pour les réplications nécessaires à la validation, \textbf{(7)} l'ignorance ou la difficulté à mettre en oeuvre les techniques adéquates, va de pair avec le manque de formation/compétence pour ces nouveaux outils dans la discipline, et rend difficile l'exploitation et la construction des modèles, \textbf{(8)} l'existence de parcours et de stratégies de publications scientifiques non adaptés pour ce nouvel objet de recherche limite sa diffusion : concentration sur les seuls résultats du modèle, peu ou pas de suivi dans l'évaluation des modèles sur le long terme.

Certains arguments sont clairement conjoncturels, beaucoup se recoupent, et d'autres englobent toutes les dimensions, comme la problématique de la \enquote{validation} qui aborde des questions de fond sur tous les plans, technique, méthodologique, philosophique et institutionnel. Un constat qui n'a rien de nouveau, et dont on peut déjà entendre en 1970 de la bouche des spécialistes, qu'il sera un des problèmes les plus difficiles à résoudre dans le futur. \autocites{Hermann1967,Naylor1967,Guetzkow1972}  %\hl{Pour herman, voir Padioleau p209 p205, + scepticisme de boudon, voir citation p205}

% TYPOLOGIE A REVOIR SUREMENT POUR MIEUX LA RANGER

\subsubsection{Une mutation dans la construction des modèles en géographie ?}
\label{ssec:crise_mutation}

Fruit des diverses influences citées auparavant, le cas de la géographie en particulier est traité un peu à part des autres disciplines en sciences humaines et sociales, car plus qu'une crise les années 1970-80 semblent -- une hypothèse à prendre toutefois avec prudence -- avant tout avoir constituées le socle fertile d'un renouvellement dans l'activité de modélisation, empruntant la voie de la mathématisation avec semble-t-il plus de facilité que d'autres sciences sociales. Une explication à chercher peut-être dans les fondements de la révolution quantitative, car pour \textcite{Gould1970} \foreignquote{english}{The intellectual revolution in geography since the middle and late fifties rests upon two main supporting pillars - men and machines.}

\begin{figure}[h]
\begin{sidecaption}[fortoc]{Une image de la série 7094 pris dans la collection de photographies historiques sur le site \href{http://www-03.ibm.com/ibm/history/exhibits/mainframe/mainframe_album.html}{@IBM} }[fig:I_IBM]
  \centering
 \includegraphics[width=.8\linewidth]{IBM7094.jpg}
  \end{sidecaption}
\end{figure}

En parallèle du perfectionnement des machines et de leur puissance de calcul apparaissent des langages de programmation qui vont faciliter la construction et la diffusion des méthodes de simulation. Nous n'envisageons pas d'en faire un historique complet, mais nous en donnons un aperçu dans l'encadré \enquote{Les premiers langages de programmation}.

\begin{framewithtitle}[Les premiers langages de programmation]{ Les premiers langages de programmation }

La période 1955 - 1965 est une période où la simulation est reconnue comme une méthode de résolution d'un certain nombre de problèmes difficilement tractables mathématiquement.\autocite{Nance1993, Ackoff1961} Des programmes de développement visant à mettre en place des modèles de représentation, de description nécessaire et facilitant la construction de simulations se multiplient. Deux classes de langage informatiques vont voir le jour durant cette période, et vont continuer à se développer et à s'influencer chacune de leur coté jusqu'à encore aujourd'hui. D'une part, les langages de plus haut niveau qui apparaissent ont pour vocation de se positionner comme une alternative plus expressive que l'assembleur. Dans cette optique le premier compilateur FORTRAN apparaît en 1957,  Algol en 1958, Cobol en 1959, et Lisp 1958. Ces langages et leur successeurs sont d'usage assez générique, et permettent de décrire correctement tout types de programmes. Toutefois à l'époque de leur apparition ils sont d'accès relativement difficiles pour une personne non initiée, ce qui nous amène au développements sur la même période d'une deuxième catégorie de langage, plus spécialisée dans la construction spécifique de modèle de simulation. \autocite[239]{Naylor1966}

A la même époque, des langages spécialisés dans l'expression des simulations apparaissent, et pour la plupart s'appuient et évoluent en parallèle des développements des langages classiques sur lesquels ils s'appuient. Ces SPL ( \foreignquote{english}{Simulation Programming Langages}) comme Simula en 1962, ou bien Dynamo en 1958 ont ceci d'intéressant qu'ils ont très largement accompagné les formidables avancées conceptuelles de cette époque et cela au travers des différentes disciplines. Ainsi la première période 1955-1960 est marquée par la mise au point de GSP (\foreignquote{english}{General Simulation Program}) par Owen et Tocher \autocite{Tocher1960}. Celui-ci est considéré comme le tout premier langage mis au point pour faciliter la description de simulation sur ordinateur. Un effort que Tocher va accompagner d'une publication phare en 1963 dans le livre \foreignquote{english}{Art of Simulation} \autocite{Tocher1963} . Vient ensuite une autre génération de langage en 1960-1965 comme GPSS (\foreignquote{english}{General Purpose System Simulator}), Simscript (développé sous l'impulsion de la RAND corporation), et la première version du langage SIMULA, qui donnera naissance à la fin des années 1960 à Simula-67, un langage qui aura un impact dépassant largement la classe des SPL, et inspirera les créateurs des futurs langages objets comme Alan Kay, auteur plus connu comme le créateur du premier langage objet SmallTalk. 

%% FIXME ORTHOGRAPHE DEUX PARAGRAPHE CI DESSOUS
On trouve plus d'information sur cette période spécifique abordé sous l'angle de l'ingénierie logicielle dans les publications de \textcites{Nance2013,Nance1993, Araten1992, Nance2002} et en consultant les \href{http://informs-sim.org/}{@archives} de la WSC (Winter Simulation Conference). Cette dernière, si elle n'est pas la première à aborder cette thématique (le \textit{System Simulation Symposium} en 1957 selon Nance), est la première à vouloir péréniser le débat à un niveau national \autocite{Nance2002}. Fondé en 1967 \autocite{Crain1992, Araten1992} celle-ci jouit aujourd'hui d'une très large visibilité au niveau international, notamment car elle abrite les publications de pionniers et de membres importants pour la discipline simulation. On pourra citer par exemple Sargent et Balci, des pionniers dans la construction de la discipline de la Validation \& Verification, qui participe et publie régulièrement pour cette conférence. 

Dernièrement les \textit{procedings} héberge le récit sur plusieurs années d'un projet \autocite{Nance2013} réunissant les acteurs important dans l'histoire de la simulation autour d'une fondation oeuvrant pour la récolte de témoignages vidéo, audio et la préservation, mise à disposition de tous des documents initiaux fondateurs, le \href{http://d.lib.ncsu.edu/computer-simulation/}{@Computer-Simulation-Archive} hébergé par la \textit{North Carolina State University}

\end{framewithtitle}

\Anotecontent{marble_computer_historycdc}{ \textcite[3]{Marble1967} déclare dans son recueil de programme de 1967 avoir écrit des routines pour le CDC 3400, que l'on suppose rapidement traduit en CDC 6400. Une procédure qui semble courante, comme en témoigne \textcite{Goldberg1968} pour le package \textit{SPURT} dédié à la simulation créé et utilisé (apparemment même par des géographes) au \textit{Vogelback Computing Center} alors sous la direction de Mittman. En 2010 Marble écrit \foreignquote{english}{Northwestern, when I arrived, was just opening its new Vogelback Computing Center and had acquired high end computing technology in the form of a Control Data Corporation (CDC) 6400. Aside from the “super”computer, the most significant component of Northwestern's computing infrastructure, in my eyes, was clearly Vogelback's Director of Computing, Dr. Benjamin Mittman. Ben was the originator of computer chess as a competitive programming activity, and he put together a generally excellent support staff at Vogelback. He was immensely helpful on a personal level to those of us who were working on the CDC mainframe. Ben also made sure that a number of useful software packages (e.g., the BMD statistical analysis package, linear programming software for solution of the transportation problem, etc.) were made freely available to all Vogelback users.} \autocite{Marble2010} Anecdote amusante sur le personnage cité par Marble, Benjamin Mittman est aussi un acteur important dans le développement et la structuration de la communauté créant des programmes d'échecs sur ordinateur, et accueille au \textit{Vogelback Computing Center} les étudiants David J. Slate, Larry R. Atkin, et Keith Gorlen ayant donné naissance au programme pionnier \textit{CHESS} \autocite{Mittman1971} s'executant sur le tout récent \href{http://computerchess.tumblr.com/post/56345790213/playing-chess-at-vogelback-computing-center}{@CDC6400}, vainqueur plusieurs années d'affilé dans les premières compétitions d'échec organisés à l'époque par l'ACM \href{https://chessprogramming.wikispaces.com/ACM+North+American+Computer+Chess+Championship}{@ACM}}

\Anotecontent{ibm604650}{Voir \href{http://www.aag.org/cs/garrison}{@Garrison} et la page \href{https://www.washington.edu/uwit/history/}{@historique} du service IT (information technology) de l'université de Washington}

D'un point de vue technique \textcite{Haggett1969} cite comme véritable point de départ dans la discipline la démocratisation de l'accès à la ressource informatique après 1961, avec la diffusion d'une deuxième génération d'ordinateurs dans les grands centres de calculs, en partant notamment de la série IBM 7094, le \textit{Vogelback Computing Center} ouvert en 1965 à Northwestern avec un CDC 3400 apparemment très vite completé avec la sortie du CDC 6400 (600 cartes perforés minutes ! Une bonne occasion pour apprendre à utiliser correctement le matériel de perforation \autocite{Fisk2005}) sur lequels vont travailler des pionniers comme Marble \Anote{marble_computer_historycdc}. Des ordinateurs que l'on imagine beaucoup plus accessibles et performants que la précédente série IBM 604 et 650 à \textit{vacuum tube} utilisé au début des années 1960 à l'université de Washington\Anote{ibm604650}, des précurseurs qui seront rapidement remplacés, par exemple par l'IBM 1620 enfin utilisable avec le langage Fortran I \autocite[66]{Berry2005}. 

\Anotecontent{ordinateur_actuel}{En comparaison, les ordinateurs actuels contiennent au minimum 4Go de mémoire, soit 4 194 300 KB.}

\Anotecontent{numac}{Un témoignage recoupé par les administrateurs du centre \href{http://archive.michigan-terminal-system.org/discussions/how-did-sites-learn-about-and-make-the-decision-to-use-mts/3numac}{@NUMAC} (\textit{Northumbrian Universities Multiple Access Computer, UK}), la primo installation dans une université sur le territoire anglais, et probablement européen (si on s'en tient aux témoignages...) d'un IBM 360/67 au lieu des \textit{mainframes} jusque là anglais étant lié à la volonté des administrateurs d'utiliser et de se former sur un des premiers système de temps partagé américain MTS (Michigan Terminal System) alors compatible avec la série d'IBM 360/67.}

\Anotecontent{atlas}{On trouve une copie du \textit{Flower Reports} ainsi qu'un listing par années des équipements présents dans les universités de Grande Bretagne sur le site internet de \href{http://www.chilton-computing.org.uk/}{@Chilton} qui héberge depuis de nombreuses années des facilités de calcul pour les universitaires \href{http://www.chilton-computing.org.uk/acl/society/computing/1965.htm}{@ATLAS} }

%DEBUT ORTHOGRAPHE EMILIE

En Grande Bretagne, en 1951 il y'aurait 4 ordinateurs seulement. En plus des universités déjà pionnières (Edinburgh, Londres, Oxford, Cambridge) et sous l'impulsion du \textit{Flowers Report} les universités s'équipe progressivement à partir du milieu des années 1960 suivant ce plan établi nationalement. On trouve un peu partout dans les universités \Anote{atlas} un pattern assez similaire à celui de l'université de Bristol décrit par \textcite{Haggett1969}, c'est à dire une hierarchie de machines qui va par ordre croissant de puissance de l'IBM 1620 et PDP 8 (Faculty level), l'Elliott 503 (University level), English Electric System 4-75 (South West Regional Computer Centre). Viennent ensuite les machines plus souvent remplacées des différents centres de traitements nationaux comme le \textit{SRC Atlas Computing Centre} de Chilton, le \textit{National Computing Centre} de Manchester, et bien d'autres centres plus petit, la plupart du temps accessibles à distances aux universitaires par divers terminaux. \textcite{Rhind1989} témoigne de la présence à l'université d'Edinburgh en 1967 d'une English Electric System KDF9 de puissance probablement similaire à celle évoqué par Haggett à Bristol. Depuis 1969 et jusqu'à 1973, soit un an à peine avant la commercialisation des tout premiers \textit{microcomputers}, \foreignquote{english}{The most sophisticated computing system in British universities was the NUMAC one, serving the whole of the universities of Newcastle and Durham: this IBM 360/67 had 1 Mb of main memory.} \Anote{numac} Un autre 360/65 est également installé la même année à l'\textit{University College} de Londres.

%FIN ORTHOGRAPHE EMILIE

En Nouvelle-Zélande, Golledge nous indique que l'installation sur le territoire de la firme IBM semble précéder de peu la formation des pionniers \autocite[94]{Bailly2000}, et au début des années 1960 l'université de Canterbury se porte acquéreur d'un flambant neuf IBM 1620 doté de 32K de mémoire.\Anote{ordinateur_actuel}

\Anotecontent{histoire_suede}{Une récolte de documents publique nationale a été organisé par le \textit{tekniskamuseet} de Suède, les textes sont disponibles à l'adresse internet suivante \href{http://www.tekniskamuseet.se/it-minnen}{@tekniskamuseet}}

En Suède \Anote{histoire_suede}, trois ordinateurs sont construits dans le courant des années 1950-60 : SARA par la société Saab à Linköping, DASK à l'institut scientifique de Copenhague, et SMIL à l'université de Lund \autocite{Persson2007}. Carl Erik Frödberg, un ami d'enfance de Hägerstrand, fait partie avec Eric Stemme des consultants amenés à échanger sur le sol américain avec les leaders du domaine (Neumman, etc.) afin de démarrer le programme suédois.  SMIL est capable de compiler de l'Algol, et c'est probablement sur celui-là que Hägerstrand assisté de Frödberg a pu exécuter ses premiers programmes. En 1969, un Univac 1108 est acheté pour faire suite à SMIL \autocite[33-34]{Lindgren2008}.

En France, en 1955 il y a exactement six ordinateurs \autocite[3]{Armatte2008}, mais c'est seulement en 1970 que l'université Paris 1, centre de référence pour les géographes pionniers quantitativistes, se dote d'un ordinateur Philips et d'un terminal en contact avec le calculateur d'Orsay.

Toutefois, on ne peut parler d'une véritable démocratisation de l'outil informatique chez les chercheurs qu'avec l'apparition dans les années 1974 aux États-Unis des premiers postes informatiques individuels \autocite[221]{Ceruzzi2000}, et il faudra encore attendre le milieu des années 1980 pour que cette technologie se diffuse véritablement et touche le grand public.

A cette période la mise en oeuvre de modèles de simulation est fortement limitée par des problématiques humaines et techniques \autocites{Haggett1969}[387]{Marble1972}, dont on peut constater dans les ouvrages inter-disciplinaires vus dans la section précédente, qu'elle ne touche pas en réalité que la géographie \autocite{Guetzkow1972}.

C'est toutefois dans cette période où les compétences informatiques nécessaires à la programmation se font encore très rares, les langages de programmation multiples et peu stables, le matériel coûteux et peu disponible (nécessitant des opérateurs de saisie, temps d'utilisation partagé entre différentes disciplines, accessible seulement localement), que des packages de programmes sont peu à peu publiés et mis à disposition des chercheurs via les réseaux universitaires \autocite{Haggett1969}. 

Au niveau de ces réseaux de diffusion de programmes, selon \textcite[20-21]{Greer1972} deux sont à noter : \textit{the State Geological Survey of University of Kansas (Computer Contributions)}  et \textit{ the Department of Geography of the University of Nottingham U.K. (Computer Applications in the Natural and Social Sciences) }. Au niveau des progiciels, \textcite[20-21]{Greer1972} identifie en 1972 trois pôles universitaires importants : Iowa \autocite{Wittick1968}, Northwestern \autocite{Marble1967}, Michigan \autocite{Tobler1970c}\footnote{Ces progammes sont malheureusement impossibles à trouver, et les publications ne sont disponibles que sous la forme d'archives numérisées non exploitables (\textit{Google Books}), ou sous format papier dans les universités correspondantes. Un travail reste à faire pour sauvegarder et mettre ce bien commun à disposition de tous les géographes.}. En effet, des pionniers comme Marble ou Tobler mettent à disposition dans le courant des années 1960 différentes routines informatiques en libre accès, \textcite[3]{Marble1967} parle de 150 routines développées jusqu'à 1967, et cela seulement à Northwestern dans le département de géographie. Le premier \textit{Statistical package for Social Science} pour les sciences sociales (ou \href{http://www.spss.com.hk/corpinfo/history.htm}{@SPSS}) date quant à lui de 1968 \autocite{Barnes2011}, alors que sort à la même date l'ouvrage \foreignquote{english}{best-of} de \textcite{Berry1968} \foreignquote{english}{Spatial Analysis: a Reader in Statistical Geography}, qui offre une vision d'ensemble des derniers développements statistiques et mathématiques.

\Anotecontent{programmes}{Particulièrement difficiles à trouver en dehors des Etats-Unis, voici un exemple des rapports disponibles dans les \href{http://findingaids.library.northwestern.edu/catalog/inu-ead-nua-archon-989}{@archives} de la \textit{Northwestern University Library} contenants les précieux programmes et les rapports d'avancements de ces ingénieurs géographes : a) \textit{Duane F. Marble and Sophia R. Bowlby, Computer Programs for the Operational Analysis of Hagerstrand Type Spatial Diffusion Models, Research Report No. 27, February, 1968} ; b) \textit{Duane F. Marble, Some Computer Programs for Geographic Research, Special Publication No. 1, August, 1967 } c) \textit{ Forrest R. Pitts, Hager III and Hager IV: Two Monte Carlo Computer Programs for the Study of Spatial Diffusion Problems, Research Report No. 2, October, 1965}}

En faisant régulièrement état de leur avancements dans divers rapports ou publications\Anote{programmes}, les pionniers Marble, Morrill, Pits et Bowlby \autocite{Pitts1963} qui se placent dans la continuité des premiers travaux relatifs aux processus de diffusion d'Hägerstrand \autocite{Hagerstrand1953, Hagerstrand1967a} donnent ainsi à voir les efforts et les difficultés auxquelles la petite équipe doit faire face pour améliorer les programmes, ou les adapter à des problématiques différentes.

Sur un tout autre front, celui du développement des \textit{large scale models} \autocites[8]{Batty1976}, les universitaires géographes sont plus souvent cités comme spectateurs qu'acteurs \autocites[9]{Batty1994}[153]{Batty1989}, cela même si quelques universitaires arrivent à décrocher des contrats importants \autocite{Barnes2006a} pour des études plus pratiques, comme \textcite{Garrison1959}, nottamment du fait que les objectifs poursuivis sont relativement différents, la planification et la prédiction prenant plus souvent le pas sur la curiosité et l'explication scientifique. Toutefois, et si on en croit \textcite{Haggett1969} la communauté universitaire semble attendre beaucoup des retombées de ces grands programmes, qui disposent de moyens humains et économiques importants pour développer des programmes et collecter des données.

Si le requiem de \textcite{Lee1973} a bien eu un effet non négligeable sur la construction et la publication de tels modèles du coté des planificateurs \footnote{Seulement trois modèles seront publiés dans le même journal à la suite de cet article ...}, force est de constater que la construction de modèles de simulation pour la théorie urbaine ne disparaît pas dans cette période \autocite[11-12]{Batty1994}, et s'appuie au contraire sur l'apprentissage de ses échecs pour se réinventer dans les années qui suivent. A ce titre, \textcite{Harris1994} soulève dans une relecture très critique de l'article de Lee, l'ignorance ou la méconnaissance de l'auteur vis-à-vis des débats qui agitent déjà depuis plusieurs années la simulation de modèles urbains \autocites{Batty1971, Wilson1970, Orcutt1957, Harris1968}. Ce faisant, Harris accuse Lee d'enfoncer des portes ouvertes et de porter des accusations que certains jugeront par la suite prématurées vis-à-vis du préjudice subi, touchant à cœur une discipline d'à peine une décennie et encore en phase d'apprentissage. \autocite[p11]{Batty1994}.

Ce mouvement de modélisation doit faire face à l'expression de ces limitations pour se reconstruire, limitations dont on sait par avance qu'elles ne seront pas seulement levées par la seule amélioration des techniques. Ainsi pour \textcite[11]{Batty1976}, de façon plus importante que tous les autres problèmes, c'est la révélation dans l'observation de cette richesse et de cette complexité d'interactions des facteurs causaux à l’œuvre dans l'évolution et la structuration des phénomènes urbains qui va le plus contribuer à la réévaluation des formes de modélisation. \footnote{Une analyse qu'il reprend dans son article de 2001 \autocite{Batty2001}, axée essentiellement autour de l'évolution de l'articulation des mécanismes internes aux modèles et aux répercussions que cela entraîne dans la construction et la validation des modèles.}

D'une part l'emploi de théories trop simplistes, induit indirectement la nécessité d'un retour à une démarche inductive plus exploratoire \footnote{On notera par exemple le témoignage de \textcite{Boyce1988} lorsqu'il dit à propos des chercheurs engagés dans cette voie \foreignquote{english}{Some, including myself, turned to more empirically oriented research activities, perhaps in the hope of strengthening the foundation of future models}}, jusque là mise de coté. 

D'autre part, une autre voie d'évolution possible pour les modèles vient des travaux existants réalisés dans d'autres disciplines universitaires ou dans le monde industriel. Ainsi différentes équipes de développements sont déjà bien identifiées dans la communauté des économistes comme \textcite{Orcutt1960} et son premier modèle micro \foreignquote{english}{bottom-up} développé à l'\textit{Urban Institute}, les démographes sur les modèles de migrations inspirés des travaux d'Orcutt comme REPSIM, puis POPSIM; sans oublier l'apport de \textcite{Forrester1961} sur l'optimisation industrielle, une des branches opérationnelles d'inspiration la plus directe du projet systémique au début des années 1960 \autocites{Cohen1961}[911]{Shubik1960b}.

% et Hagerstrand ? 
La \enquote{micro-simulation} initiée par Orcutt, qui semble effectivement passer outre l'extinction annoncée par Lee en 1973, rencontre même un certain succès durant toutes les années 1970 comme en témoigne la mise en place de nombreux programmes nationaux au début des années 1980. \autocite{Baroni2007} Une réponse à cette survie peut être avancée dans le positionnement innovant d'Orcutt pour faire face aux résultats décevants des \textit{Large Scale Models} de son époque, opérant pour la plupart à un niveau macro et fournissant des résultats hautement agrégés difficiles à exploiter dans un cadre prédictif, et finalement peu représentatifs de la diversité des systèmes économiques \autocites{Birkin2012, Baroni2007}. Si les critiques de Lee peuvent pour la plupart être mobilisées pour critiquer les modèles issus de la micro-simulation (complexité des modèles, absence d'objectifs clairement posés, volume des données à mobiliser, complexité des calculs, coût de construction, absence de résultats, etc.), il n'en reste pas moins que la proposition d'Orcutt introduit avec une approche plus \textit{bottom-up} une dimension explicative absente jusque là. En répondant à l'observation de Lee sur l'absence d'extraction de connaissances micro quelque soit la complexité injectée dans les modèles macro, Orcutt ouvre d'une certaine façon la voie à des développements théoriques beaucoup plus riches que ne le permettaient à l'époque les seuls modèles macro, faisant ainsi de son modèle un instrument pour \foreignquote{english}{consolidating past, present, and future research efforts of many individuals in varied areas of economics and sociology into one effective and meaningful model; an instrument for combining survey and theoretical results obtained on the micro-level into an all-embracing system useful for prediction, control, experimentation, and analysis on the aggregate level} \autocite[122]{Cohen1961}.

D'un autre coté, cette micro-simulation telle que déjà théorisée par Hägerstrand dans sa version spatiale ou par Orcutt dans sa version économique, va étonnamment et cela pendant plusieurs années rester un courant ayant peu d'impact sur le développement des modèles urbains en économie spatiale \autocite[5]{Sanders2006}, et cela malgré plusieurs appels d'un coté \autocite{Hagerstrand1970} ou de l'autre \autocite[5]{Isard1998}. De façon indépendante et dans un univers somme toute limité par de fortes contraintes techniques et financières, ces travaux vont toutefois dans leurs lentes et multiples convergences donner naissance autant à des modèles universitaires qu'à des programmes nationaux (DYNASIM et CORSIM pour Orcutt aux Etats-Unis, SVERIGE en Suède, etc.). Pour finir cette parenthèse sur la micro-simulation par une petite transgression temporelle, si peu de modèles existent encore dans les années 1990, plusieurs publications récentes font état d'un inversion de la tendance ces vingt dernières années \autocite{Lenormand2013}, avec une augmentation (et une diversification ? ) croissante des modèles, sûrement liée à des capacités de développements informatiques plus importants, tant du point de vue des données, que de la puissance d’exécution qui admet l'importance croissante du parallélisme, idéale pour simuler des entités individuelles. \autocites[5]{Sanders2006}{Lenormand2013}

Cette crise, qui on l'a vu touche avant tout les instituts de planification américains couverts par la RAND, va fournir \textit{post mortem} le terreau nécessaire à la transformation d'une discipline dont le rayonnement dans la communauté scientifique à l'international ne va aller qu'en s'amplifiant après 1970 (voir la carte \ref{fig:S_carte_wegener}).

\begin{figure}[h]
\begin{sidecaption}[fortoc]{La carte des centres de recherches les plus actifs à la fin des années 1980, début des années 1990 selon \textcite{Wegener1994}}[fig:S_carte_wegener]
  \centering
 \includegraphics[width=.9\linewidth]{carte_wegener.png}
  \end{sidecaption}
\end{figure}

C'est le cas par exemple au Royaume-Uni où sont récupérés les modèles américains ayant donné de bons résultats, comme celui de Lowry \autocite{Lowry1964}, pour servir de base à de nouveaux travaux mettant en perspective l'influence ou les progrès d'autres courants disciplinaires en contact avec la géographie. 

Le mouvement du professeur Orcutt \autocite{Orcutt1957}, mais aussi celui de Forrester \autocite{Forrester1961, Forrester1969} font dans leur implémentation dynamique alors écho aux travaux initiaux du géographe Hägerstrand, et poussent dans cette période de reconstruction toute une partie des géographes à réintégrer la dimension temporelle à des modèles d'optimisation statique en échec. \autocite[p295]{Batty1976}.

La diffusion et la généralisation du programme systémique permettent aux géographes d'accéder à tous les outils conceptuels et surtout opérationnels \autocite{Forrester1969} nécessaires pour penser, modéliser et simuler les systèmes géographiques au travers de leurs interactions complexes, en intégrant dans leurs analyses cette hétérogénéité d'échelle caractéristique des objets géographiques, comme peut l'être par exemple la région.

Si les universitaires américains semblent rater le coche de cette transformation, en Europe plusieurs écoles viennent à se former, comme la \foreignquote{english}{social physics} de \autocite{Wilson1970} dont l'émergence est considérée comme un moment important dans le renouveau des modèles urbains \autocite{Griffith2010}; mais également d'autres écoles comme celle de Peter Allen, qui s'appuient sur l'évolution des mathématiques et le transfert méticuleux de concepts observés en physique pour construire des modèles à la fois spatiaux et dynamiques capables de simuler de façon plus réaliste les interactions complexes intervenant dans la formation et l'évolution des villes. \autocite[11]{Batty1976, Batty2001} \autocite[27-28]{Pumain2003} \footnote{ Pumain liste au moins trois intérêts qui découlent de cette phase d'acquisition du projet systémique : a) le dépassement de l'opposition idiographique et nomothétique, b) l'histoire et les particularités des entités géographiques vues comme expression originale de trajectoires et de bifurcations, c) le dépassement de la rigidité des trajectoires biographiques historiques par l'emploi des simulations}

%FIXME : COUPER LE PARAGRAPHE EN DEUX AVEC UN NOUVEAU TITRE INTRODUISANT LES LIMITATIONS INFORMATIQUES.
\textit{Dans quelles mesures les problématiques levées à la fin de la section précédente ( section \ref{ssec:disciplines_touches}) sont-t-elles encore pertinentes après une telle évolution des pratiques dans la géographie ? }

L'amélioration de la formation des géographes européens, et notamment des géographes français dans les années 1970, permet à ceux-ci d'intégrer plus facilement les évolutions de l'informatique durant les années 1970-1990, leur garantissant ainsi une certaine autonomie de développement qui va donner lieu à plusieurs collaborations fructueuses avec les physiciens \autocite{Pumain1984}; 

Concernant l'accès à la ressource informatique pour construire et explorer les modèles de simulation, même si les conditions se sont améliorées avec la démocratisation de l'ordinateur, celle-ci reste un élément bloquant pour l'exécution et l'exploration des modèles de simulation.

Tout d'abord, il y a ce témoignage\footnote{Ce témoignage est issu d'un échange par mail en 2013} précieux de Duane Marble, un des pionniers modélisateurs américains, qui lorsqu'il est interrogé sur la suite des problématiques de validation des modèles de simulation détaillées dans son article de 1972 \autocite{Marble1972}, conforte d'une certain façon notre point de vue : \foreignquote{english}{As I recall, the situation in the 1980's had not changed very much. Simulation in human geography did not last long. Much of this was the result of a lack of computer capacity. Simply replicating Hagerstrand's diffusion model proved difficult and our attempt to inject a more explicit temporal element just would not work due to the computational load.}

Malgré les apports heuristiques indéniables qui vont avec l'utilisation de l'outil, on retrouve l'expression de difficultés concernant le calibrage des modèles plus complexes chez de nombreux auteurs pionniers modélisateurs \autocites{Batty1976, Pumain1998a}[400]{Sanders1984}, notamment pour ce qui concerne le calibrage des modèles, souvent difficile pour ces modèles dynamiques non linéaires soumis à de tels fluctuations dans leur comportements. Voici comment \autocite{Pumain1998a} résume les difficultés opérationnelles résultats de plusieurs années de travaux menés autour des modèles de simulation dynamiques non-linéaires opérant dans le cadre de la théorie de l'auto-organisation : \enquote{Les difficultés de calibrage, associées à la capacité élevée de bifurcation des modèles, ont été maintes fois décrites, de même que l’impossibilité de valider comme \enquote{meilleur ajustement} une configuration donnée de paramètres.}

Batty est probablement un des premiers géographes à faire ce travail d'état de l'art des techniques de calibrations disponibles et applicables à cette nouvelle classe de modèles urbains. Des méthodes de calibration basées sur des méta-heuristiques de type descente de gradient, sont déjà utilisées par les géographes comme \textcite[159]{Batty1976} pour résoudre des problèmes d'optimisations utilisant les sorties de modèles. Toutefois ces méthodes sont encore trop souvent limitées à des modèles à 1 ou 2 paramètres, et s'avèrent peu robustes face à des problèmes acceptant des minima locaux. % FIXME AJOUTER LA REMARQUE DOPENSHAW1989 TIRÉ DE MACMILLAN1989; ESSAYER DE SEPARER CORRECTEMENT AVEC UN INTERTITRE CETTE PARTIE DE LA PRÉCEDENTE

Une chose est sûre pour ce qui est de la recherche de paramètres, on perçoit très tôt chez certains géographes la nécessité d'optimiser cette étape, rendue improductive et dangereuse du fait de la non-linéarité des modèles \enquote{The trial and error method of searching for best-parameter values by running the model exhaustively through a range of parameter values or combinations thereof represents a somewhat blunt approach to model calibration.}

L'appel à l'utilisation de nouvelles méthodes pour l'exploration des modèles déjà lancés par \textcite{Batty1976}, est par la suite repris de façon implicite par \textcite{Openshaw1996}. Celui-ci publie en 1996 avec un collègue de Leeds un article sur les algorithmes génétiques, la méthode la plus efficace disponible alors pour explorer des espaces de paramètres de façon efficace. La conclusion est explicite :

\foreignquote{english}{The results demonstrate that even GA en ES can provide very good solutions for spatial interaction model calibration, albeit sometimes at the expense of considerable extra compute times. [...] It would also be worth considering the use of other forms of global optimization method; [....] As computer hardware becomes faster, the attraction of simple, relatively assumption-free, and highly robust approaches to global parameter estimation can only grow and allow the geographical model builder to worry less about the problems of parameter estimation and focus more on the task of model design.}\autocite{Openshaw1996}

La course à la puissance informatique nécessaire pour explorer et calibrer les modèles ne fait en réalité que commencer. Les méthodes sont encore en cours de développement, et leur usage s'avère extrêmement coûteux sur le plan informatique.

Se pose alors la question suivante, l'incapacité à calibrer un modèle de simulation n'est-elle pas un problème qui limite de facto l'évolution en crédibilité de l'outil simulation ? 

Concernant ce problème plus large de la validation, dont le calibrage n'est qu'une facette, le changement de paradigme explicatif et l'ouverture sur la complexité a soulevé un débat qui dépasse en réalité la seule problématique technique. Il ne suffit plus de garantir un résultat pour que le modèle soit considéré comme valide, sa structure causale est elle aussi considérée comme le résultat d'un processus social, et dont la contenance doit normalement être validée terme à terme avec le domaine empirique; or c'est celle-là même qu'on ne peut observer dans le cadre d'un système complexe. (cf \textit{observational dilemna} de \textcite[296]{Batty1976} :

\foreignquote{english}{Perhaps the major problem concerns the ability to observe or monitor the urban system. Unlike the physical sciences in which the effect of critical variables on the system of interest can be isolated in the laboratory, such a search for cause and effect is practically impossible in social systems. Thus, there are many instances when it is difficult, if not impossible, to disentangle one cause from another in the changing behaviour of such systems. This is a fundamental limitation which is referred to here as the observational dilemma.}

La dernière phrase d'Openshaw prend alors tout son sens, et en nous rappelant que la construction de modèle est un processus incrémental, il fait indirectement écho à l'évaluation elle aussi incrémentale d'une structure causale où chaque mécanisme lorsqu'il est ajouté/enlevé, remet en cause l'exploration précédente. Dès lors, la systématisation de cette calibration devient \textbf{le seul moyen de garantir une construction} qui serait faite en tout connaissance de cause, en mesurant, et donc en discutant l'apport de chacune des hypothèses durant le processus de calibration.

La dépendance à la ressource informatique se renforce en réalité encore un peu plus avec la nécessité d'explorer les modèles, non plus lorsqu'ils sont terminés, mais dès que la première brique est posée. 

%%FIXME : A MODIFIER POUR COLLER AVEC LE PARAGRAPHE PRÉCÉDENT %%
%Paradoxalement il donne aussi à voir les limites des approches proposées pour létude de l'homme dans son environnement, et offre ainsi le matériel idéal pour appuyer la formulation critique des géographes radicaux marxistes, un mouvement qui s'amplifie dès le début des années 1970 en parallèle avec la conjoncture politique nationale et mondiale. \autocite{Golledge2006}


%Ainsi les progrès fulgurants de l'informatique, l'apparition de nouveaux langages exclusivement orientés pour la simulation comme Dynamo, la prise de conscience tout au long des années 1960-70 des défauts de cette première génération de modèles, et les changements d'objectifs de la discipline \autocite[12]{Batty1994} \autocite{Boyce1988} autorisent (voire recommandent) la formation de nouveaux modèles. Ceux-ci sont conçus comme plus parcimonieux, autorisant les démonstrations plus abstraites \autocite{Forrester1969}, plus orientées vers la compréhension des mécanismes à l’œuvre que sur la prédiction (un retour sur les modèles théoriques est opéré), intégrant plus facilement l'hétérogénéité dans la nature des dynamiques (rétro-action, non linéarité) des processus \autocite{Forrester1969, Wilson1970, Allen1978}, et ouvert à l'intégration d'autres dimensions explicatives à l'oeuvre dans la formation des processus, comme ceux déjà explorés l'individu et le temps \autocite{Hagerstrand1967a,Orcutt1957,Forrester1961}. En lisant les articles de Pred, d'Olsson \autocite{Olsson1969,Olsson1970}, de Curry, on percoit chez les nouveaux économistes spatiaux cette volonté de changement, avec la reintroduction de la stochasticité et des modèles probabilistes, l'intégration du temps dans les modèles, mais aussi les causalités multiples.

% PLUSIEURS points développement méthodologiques accompagnant renouvellement théoriques accompagnant nouvelle géographie : Hagerstrand , Orcutt -> causalité + individualisme méthodologique,  Forrester -> complexité
% Hagerstrand premiere utilisation montecarlo en science sociale, vient a Washington et rencontre Morril... qui pour Benko Stromayer marque troisieme theme dominant le bouleversement quantitatif) Gould2004

% simulation permet de développer cette causalité ...
% Systeme dynamique, non linéarité, permet avancée fondamentale dans les questionnements, révélateur aussi de l'apport des techniques / méthodologies...
% Basculement vers explicatif !


% -*- root: These.tex -*-

\section{La validation des modèles de simulation}
\label{sec:constante_problematique}

Les termes \enquote{Validation \& Verification} tels que définis par les institutions de normalisation sont conçus comme génériques et valables pour des disciplines autres que l'ingénierie logicielle (section \ref{ssec:def_generique_validation}). Dans ce sous ensemble de pratiques, la simulation dispose de sa propre branche historique, dans laquelle des spécialistes raffinent et organisent depuis les années 60 ces notions en mettant en oeuvre des typologies d'outils et des méthodologies de conception et d'évaluation standardisées. \autocite{Nance2002} Si aujourd'hui ces définitions ont évolués et sont parfois reprises pour encadrer des travaux en sciences humaines et sociales, il faut savoir que dans les années 1960-70 celle ci était en l'état peu compatible avec les mutations en cours dans la modélisation en géographie.

Dans l'histoire de la géographie américaine, le début des années 1970 est marqué comme une période d'émergence de nouveaux courants de géographie. (section \ref{ssec:transition_annee70}) Si il n'est pas question ici de relater en détail cette construction d'une géographie radicale, humaniste ou comportementale, on retiendra seulement que ces courants se forment principalement à la convergence de problématiques politiques (crises économique nationales et internationales, guerres), de revendications théoriques (rejet des méthodes quantitatives et du \enquote{fétichisme spatial} \footnote{\foreignquote{english}{Any approach that treats space as sufficiently autonomous to social processes that ‘no change in the social process or spatial relations could alter the fundamental structure of space’} \autocite[712]{Gregory2009}} ) et/ou méthodologiques (retour de l’herméneutique).

Les acteurs prônant une démarche scientifique teinté de néo-positivisme largement inspiré des sciences physiques sont alors la cible idéale de ces nouveaux acteurs, et vont subir un large front de critique.

Gregory, dont on mobilise le point de vue pour critiquer la vision néo-positiviste / positiviste en géographie, utilise ce dernier argument de façon conjointe avec la pensée d'Habermas pour charger les dérives entraînées par les méthodes quantitatives, et proposer un autre style de pensée axé sur la réconciliation d'un point de vue structuraliste, phénoménologique et critique pour entre autre éviter l'écueil du \enquote{fétichisme spatial}. A la lecture d'ouvrage comme ceux de Gregory, dont la démarche de dépassement n'est pas sans levée des critiques pertinentes, il nous semble a posteriori que sa vision du mouvement quantitatif est en partie biaisé, d'une part parce que la réalité des pratiques peut tout à fait s'éloigner des discours tenus par quelques leaders d'opinion, tel qu'Harvey ou Bunge, et d'autres part parce que les critiques externes au mouvement, comme Gregory font mine d'ignorer une partie des transformations qui opère depuis le début des années 1970 en interne dans les pratiques visés.

Ainsi, afin de montrer que la discipline géographique n'a pas attendu l'émergence de tels discours parfois extrémistes, nous avons aperçu dans la section \ref{ssec:crise_mutation} que les modèles de simulation économiques spatialisés, ont adopté au vu de leurs maigres résultats une démarche plus explicative permise entre autre par l'évolution des moyens de simulations, et que cette confrontation avec la problématique de validation a été formulée comme centrale par les modélisateurs pionniers et cela de façon explicite dans des ouvrages collectifs abordant cette question \autocite{Marble1972}. Si sur le fond il n'y a rien de critiquable à vouloir développer un autre style de pensée en opposition de certains excès constatés relatifs aux usages de ces nouvelles méthodes quantitatives, sur la forme il en résulte chez certains géographes l'émergence d'un amalgame malheureux qui associe un peu trop rapidement méthode quantitative positiviste, et modèle d'inspiration économique néo-libéraliste \autocite[61-64]{Paterson1984}. Une dualité opposant géographe (et géographie) qualitativiste/quantitativiste encore brandit aujourd'hui comme un processus supposé constructif alors qu'il n'en est rien \autocite{Sheppard2001}.

La section \ref{sssec:realite_neopositiviste} propose de déconstruire avec les arguments disponibles ce point de vue qui voit dans l'application pratique de la méthodologie néo-positiviste un support crédible à l'explication dans la construction de modèles en géographie. Une fois cette proposition écartée, la question du devenir des pratiques de \foreignquote{english}{model-building} mobilisées par la géographie quantitative doit être vue sous un autre angle, qui dépasse la seule critique des méthodes de la géographie radicale, celui de la réification du paradigme systémique comme expression formelle adaptée à l'analyse complexe des objets géographiques \ref{sssec:progressive_systemique}.

\hl{ A finir intro section}

%Au coeur de la théorie des \enquote{système ouverts} les concepts d'équifinalité, de hierarchisation de statistique sont dans leur opérationalisation \ref{subsec:operationaliser_concept} autant d'incitation à utiliser les récents progrès de l'informatique des années 1950-60 pour explorer un univers, non pas tant complexe dans sa description (comme en témoigne Simon, des problèmes complexes peuvent très bien être dérivé de règle simple) mais dans la multiplicité d'approche (trajectoire, échelles, interactions) qu'elles permettent.

Outre le fait que cette ouverture s'accompagne d'innovations méthodologiques permettant l'opérationalisation des concepts, s'ouvrent en parallèle avec la chute du néo-positivisme de nouveaux débats autour de l'explication \autocite{Hedstrom2010} à la fois chez les praticiens (les \enquote{mécanismes générateurs} de Boudon, les \foreignquote{english}{causal-mechanisms} plus récents des biologistes, les \foreignquote{english}{generative mechanisms} d'Epstein) mais également chez les philosophes des sciences en biologie (Salmon, Machamer, etc.) où les thèses de Popper-Hempel, bien que souvent citées, sont en réalité rarement appliquées ou même appliquables dans les faits. \autocite{Bechet2013}

Un retour sur la démarche de construction des modèles en géographie s'avère nécessaire pour comprendre les éléments qui nous ont échappé dans la continuité de cette problématique qu'est la validation des modèles. En s'appuyant sur les témoignage de \autocite{Batty2001, Pumain2003} on parvient très bien à décrire ce basculement opéré à la charnière des années 1970, alors même que les géographes accèdent peu à peu aux concepts opérant dans le paradigme systémique \autocite{Harvey1969}, et que l'insuffisance des démarches de construction de modèles devient prégnante.

L'enjeu ici est d'autant plus important qu'il se double d'une réalité opérationelle, faisant des problématiques de sous-détermination (Quine) ou d'équifinalité (Bertalanffy) des concepts tout à fait tangibles, dont la manipulation déborde du cercle des philosophes des sciences pour venir parasiter les débats des modélisateurs en SHS, dont la qualité des explications avancées doit s'adapter à cet horizon infranchissable, et se réinventer dans des discours, des méthodologies plus spécifiques.




% -*- root: These.tex -*-

\subsection{La difficile validation des modèles de simulation explicatif en géographie}
\label{ssec:validation_compatible_shs}

\subsubsection{La lecture multiple des problématiques liés à la validation}
\label{sssec:triple_lecture}

%Si le modélisateur est au courant des simplifications opérés dans les hypothèses censés représenter ,  la dynamique de construction introduit dans l'activité de construction des modèles une incertitude supplémentaire qui nous oblige à repenser l'activité de validation.

\paragraph{Les définitions de la validation en V\&V}
\label{ssec:def_generique_validation}

Les termes \foreignquote{english}{Validation \& Verification} ou \textit{V\&V} proviennent à l'origine de l'ingénierie des systèmes, et peuvent être rattachés au concept de \enquote{qualité} tel qu'il est défini par la famille de règles ISO établies par l'organisation mondiale de normalisation.

Décomposable en plusieurs branches cette discipline à part possède une branche dédiée à l'expertise logicielle. De ce fait, il n'existe pas réellement de définition ni de théories ou méthodologies officiellement acceptables, l'acceptation des termes pouvant varier fortement selon les branches d'application.

On trouve toutefois quelques références dans des livres dédiés à la terminologie standard pour la \enquote{gestion de projet} dans un large panel de disciplines, telle que le PMBOK (\textit{A guide to the project Management Body of Knowledge}) \autocite{PMBOK2013}. Résultats d'un travail certifié par des associations ou des organismes étatiques tels que IEEE et ANSI, ce dernier propose une définition générale de ces termes pour l'ingénierie logicielle :

\foreignquote{english}{Verification and validation (V\&V) processes are used to determine whether the development products of a given activity conform to the requirements of that activity and whether the product satisfies its intended use and user needs.}

et revient ensuite plus spécifiquement sur les termes :

\begin{itemize}
\item \textbf{Validation} \foreignquote{english}{The assurance that a product, service, or system meets the needs of the customer and other identified stakeholders. It often involves acceptance and suitability with external customers. Contrast with verification.}
\item \textbf{Verification} \foreignquote{english}{The evaluation of whether or not a product, service, or system complies with a regulation, requirement, specification, or imposed condition. It is often an internal process. Contrast with validation.}
\end{itemize}

Les termes tels qu'ils sont définis sont finalement bien trop généraux pour envisager de les appliquer tels quels dans notre domaine de compétence. Dérivé de la branche de l'\textit{Operational Research (OR)}, les auteurs de la communauté restreinte des \textit{systems analysis or modelling and Simulation (M\&S) } engagent dès les années 1960-70 des efforts pour standardiser ces définitions pour la simulation.

\Anotecontent{first_time_validation}{La citation de Churchman par \textcite{Naylor1966} est tiré de \autocite[165]{Nance2002} : \foreignquote{english}{\foreignquote{english}{X simulates Y} is true if, and only if, (a) X and Y are formal systems, (b) Y is taken to be the real system, (c) X is taken to be an approximation to the real system and (d) the rules of validity in X are non-error-free.} \autocite{Nance2002} }

Parmi les différents auteurs participant de ce mouvement ( Naylor, Finger, Oren, Hermann, Zeigler, Nance, Banks, Gass, Balci, Sargent, etc.), \textcite{Naylor1966} est considéré avec West Churchman (1963) comme un des tout premier à avoir attiré et cristalisé \Anote{first_time_validation} dans de multiples publications l'attention sur cette problématique importante de la V\&V.

Cet économiste formé à l'informatique dans la branche des \foreignquote{english}{management sciences} \autocite{Stricklin1985} est un des premiers en 1967 \autocite{Naylor1967} à publier dans un article nommé \foreignquote{english}{Verification of Computer simulation models} une méthode abordant spécifiquement la question de la crédibilité des connaissances qui peuvent être apportées par un modèle de simulation. Une méthode qu'il va mettre spontanément en tension avec les débats qui agitent la communauté des philosophes à cette même période.

Malgré ces efforts et sa volonté de porter le débat loin dans la communauté inter-disciplinaire (voir les premiers ouvrage collectifs sur l'usage de la simulation dans les \enquote{behavior science} \autocite{Dutton1971, Guetzkow1972} \hl{A verifier}), la démarcation entre les deux termes est encore peu claire \autocites[165]{Nance2002}[3]{Balci1986}. \footnote{\foreignquote{english}{Thomas Naylor, a coauthor of the book cited above, deserves credit for drawing major attention to the validation issue in the 1960s: Is the model actually representing the truthful behavior of the referent system? His work, above and in later publications (Naylor 1971, Naylor and Finger 1967), exerted a major influence in framing validation within different philosophical perspectives. Numerous techniques that can be used were identified or developed. While the issues of both verification and validation were of concern from the early days of simulation, often no clear distinction was made between the two terms.} \autocite[165]{Nance2002}}

\Anotecontent{balci_standard}{\foreignquote{english}{A uniform, standard terminology is yet nonexistent. A recent literature review \autocite{Balci1984} indicated the usage of 16 terms [...] Except some early papers which appearead between 1966 and 1972, model verification and model validation have been most of the time consistently defined reflecting the following differentiation : \textbf{model verification} refers to building the model right; and \textbf{model validation} refers to building the right model. \autocite{Balci1986}}}

Il faudra attendre le début des années 1980 pour qu'un standard émerge, grâce à des financements étatiques \autocite{Balci1986}, mais également du fait des efforts fournis par des auteurs comme Sargent et Balci \autocite{Nance2002}, qui collectent et organisent dans une typologie cohérente l'existant statistique et méthodologique, une activité qu'ils poursuivent encore aujourd'hui \autocite{Balci1998}.\Anote{balci_standard}

Pour \autocite[22]{Oberkampf2010} \foreignquote{english}{A Key milestone in the early work by the OR community was the publication of the first definitions of V\&V by the Society of Computer Simulation (SCS) in 1979 \autocite{Schlesinger1979}}, un des instituts avec la U.S GAO (U.S General Accounting Office) à fournir des spécifications en 1979 \autocite{Balci1986}

\begin{itemize}
\item \textbf{Model Verification} \foreignquote{english}{substantiation that a computerized model represents a conceptual model within specified limit of accuracy.}
\item \textbf{Model Validation} \foreignquote{english}{substantiation that a computerized model within its domain of applicability possesses a satisfactory range of accuracy consistent with the intended application of the model.}
\end{itemize}

\begin{figure}[h]
\begin{sidecaption}[fortoc]{Un des tout premiers schémas issus de la publication de la SCS \autocite{Oberkampf2010,Schlesinger1979}}[fig:S_VV]
  \centering
 \includegraphics[width=.7\linewidth]{schelinger_schema1979.png}
  \end{sidecaption}
\end{figure}

Même si elles sont plus anciennes et de portée moins générale, ces définitions de la \textit{V\&V} semblent plus pertinentes, car évoquées plus régulièrement par les chercheurs en sciences sociales; les travaux les plus cités étant ceux de \textcite{Kleijnen1995}, ou \textcite{Sargent2010} qui placent leurs travaux dans la continuité de ces définitions. L'avancée de leurs travaux peut être suivie en feuilletant les \textit{Proceedings of the Winter Simulation Conference} où la problématique de la \textit{V\&V} est réévaluée régulièrement au regard des nouvelles connaissances. Ce schéma \ref{fig:S_VV} est devenu un classique repris et régulièrement amendé \autocite{Sargent2010}. Voici la lecture qu'en fournit \autocite{Oberkampf2010}

\foreignquote{english}{The \textbf{conceptual model} comprises all relevant information, modelling assumptions, and mathematical equations that describe the physical process or process of interest. [...] The SCS defined \textbf{qualification} as \enquote{Determination of adequacy of the conceptual model to provide an acceptable level of agreement for the domain of intended application}. The \textbf{computerized model} is an operational computer program that implements a conceptual model using computer programming. Modern terminology typically refers to the computerized model as the computer model or code.}

Ce schéma a la particularité suivante, il \foreignquote{english}{ [...] emphasizes that \textbf{verification} deals with the relationship between the conceptual model and computerized model and that \textbf{validation} deals with the relationship between the computerized model and reality. These relationships are not always recognized in other definitions of V\&V [...]}

\Anotecontent{Kleijnen_def}{\foreignquote{english}{This paper uses the definitions of V \& V given in the classic simulation textbook by Law and Kelton (1991, p.299): \enquote{Verification\textbf{Verification} is determining that a simulation computer program performs as intended, i.e., debugging the computer program .... \textbf{Validation} is concerned with determining whether the conceptual simulation model (as opposed to the computer program) is an accurate representation of the system under study}. Therefore this paper assumes that verification aims at a \enquote{perfect} computer program, in the sense that the computer code has no programming errors left (it may be made more efficient and more user friendly). Validation, however, can not be assumed to result in a perfect model, since the perfect model would be the real system itself (by definition, any model is a simplification of reality). The model should be \enquote{good enough}, which depends on the goal of the model.}}

\Anotecontent{Sargent_def}{\foreignquote{english}{\textbf{Model verification} is often defined as \enquote{ensuring that the computer program of the computerized model and its implementation are correct} and is the definition adopted here. \textbf{Model validation} is usually defined to mean \enquote{substantiation that a computerized model within its domain of applicability possesses a satisfactory range of accuracy consistent with the intended application of the model} \autocite{Schlesinger1979} and is the definition used here. A model sometimes becomes accredited through model accreditation. Model accreditation determines if a model satisfies specified model accreditation criteria according to a specified process. A related topic is model credibility. Model credibility is concerned with developing in (potential) users the confidence they require in order to use a model and in the information derived from that model. A model should be developed for a specific purpose (or application) and its validity determined with respect to that purpose [...]A model is considered valid for a set of experimental conditions if the model’s accuracy is within its acceptable range, which is the amount of accuracy required for the model’s intended purpose.}}

Autrement dit, \foreignquote{english}{The OR community clearly recognized, as it still does today, that V\&V are tools for assessing the accuracy of the conceptual and computerized models.} Un avis partagé par \textcite{Kleijnen1995} \Anote{Kleijnen_def} , \textcite{Balci1998}, et \textcite{Sargent2010} \Anote{Sargent_def} mais également des auteurs de références sur le sujet dans les sciences humaines et sociales \autocite{Amblard2006} \hl{Prend le bout de texte la dessus}.

Seulement, cette forme de relâchement sur la correspondance entre réalité et modèle, et ce positionnement plus relativiste de la validation n'a pas toujours été une évidence; les premières définitions de Naylor par exemple, sont toujours usitées, et continuent si on en croit des auteurs comme \textcite{Kleindorfer1998} à semer le trouble dans certaines disciplines.

\Anotecontent{VV_philout}{ \foreignquote{english}{During the last two decades a workable and constructive approach to the concepts, terminology, and methodology of V\&V has been developped, but it was based on pratical realities in business and government, \textbf{not} the issue of obsolute thruth in the philosophy of nature} \autocite{Oberkampf2010}
\foreignquote{english}{A very old philosophical question is: do humans have accurate knowledge of reality or do they have only flickering images of reality, as Plato stated? In this paper, however, we take the view that managers act as if their knowledge of reality were sufficient. Also see Barlas and Carpenter (1990), Landry and Oral (1993), and Naylor, Balintfy, Burdick and Chu (1966, pp.310-320).} \autocite{Kleijnen1995}
\foreignquote{english}{With the strong interest in verification from the software engineering community, this contrasting but complementary explanation of the term was quite important. The effort to place valida- tion in a cost-risk framework moved the concept from a philosophical explanation in earlier works to a form more useable for simulation practitioners.} \autocite[165-166]{Nance2002}}

Mais en excluant ainsi de son analyse la partie subjective et philosophique de la \enquote{Validation}\Anote{VV_philout} pour se concentrer sur la seule partie opérationnelle, ces méthodologies restent pour le modélisateur une coquille vide décevante, qui demande encore à être incarnée thématiquement. Autrement dit, ces méthodes si elles prennent bien en compte la dimension dynamique et incrémentale nécessaire à la construction d'un modèle de simulation qui tendrait vers une réalité en accord avec la question posée, l'organisation des connaissances nécessaires pour guider ce processus reste à la lecture de ces typologies une opération quelque peu énigmatique pour les modélisateurs géographes. On retombe sur une des critiques soulevées précédemment dans la section \ref{sec:critiques_simulation} sur l'absence constatée dans les publications de méthodologie standard pour la validation qui prendrait en compte les problématiques spécifiques d'une discipline. \footnote{Aujourd'hui des disciplines comme l'écologie proposent des méthodologies plus spécifiques, comme la méthode POM proposé par Grimm sur lequel nous reviendront par la suite \hl{mettre une ref et un appel à la section}}

Une position compréhensible pour ces auteurs oeuvrant pour la standardisation, alors même que ces termes sont toujours d'usages toujours assez variables. Une des conséquences visibles tient dans ces incompréhensions et ces débats terminologiques sans fin \autocite{David2009} que l'on observe parfois en marge des discussions inter-disciplinaires. Cette gamme d'acceptions différentes tient souvent au transfert hasardeux des terminologies entre l'ingénierie des M\&S, la philosophie des sciences, et la thématique d'un chercheur en sciences sociales qui se retrouve en position intermédiaire de ces deux derniers. Un exercice d'équilibriste périlleux, car comme le fait remarquer \textcite{Kleijnen1995} en citant astucieusement une note de bas de page de \textcite{Barlas1990}, en philosophie il est tout à fait possible de voir la signification des deux termes inversée! \hl{Expliquez mieux que verification pourrait se traduire en philosophie pour certains par representation de la vérité, du “reel”, alors que le fait même de modéliser implique qu’on en soit loin}

\paragraph{La philosophie des sciences}

Il ne s'agit pas de se lancer ici dans un exposé historique des courants et débats s'étant succédés dans cette discipline, mais d'amener de façon illustrative et avec quelques références récentes l'émergence ces 20 dernières années d'une \enquote{épistémologie de la simulation} reprenant (en parasitant parfois le débat comme on l'a cité au dessus) de son point de vue certains débats évoqués chez les praticiens de la simulation; la question de validation étant comme on l'a vu dans le chapitre 1 un sujet de longue date chez les praticiens de la simulation, mais aussi chez les premiers acteurs fondateurs de la V\&V.

\hl{redite : L'objectif n'est donc pas tant de développer une argumentation critique exposant l'ensemble de ces points de vues, car ce n'est pas l'objet de cette thèse, que de tenter de s'insérer (et non de s'enfermer) dans ces réflexions en spécifiant en quoi celle ci diffère, néglige ou font peu écho à nos pratiques et réflexion historique en sciences sociales.}

Le premier obstacle avec laquelle les acteurs supportant cette nouvelle épistémologie doivent cohabités est évidemment la contre-argumentation questionnant cette même necessité d'opérer une nouvelle sous-division épistémologique. Car existe-t-il réellement des spécificité à la connaissance dérivé de l'étude de l'objet simulation, et si oui quelles sont elles réellement ? Autrement dit, existe t il une différence fondamentale entre les questionnements déjà posés dans le cadre d'une épistémologie des modèles et ceux évoqués dans le cadre d'une épistémologie de la simulation ?

\Anotecontent{frilosite_philoScience}{\foreignquote{english}{As computer simulation methods have made their way into novel disciplines, the issue of their trustworthiness for generating new knowledge has often loomed large, especially when they have competed for attention with experiments or analytically tractable modeling methods. The relevant question is always whether or not the results of a particular computer simulation are accurate enough for their intended purpose.[...] Given our long-standing preoccupation with issues of confirmation, it might seem obvious that philosophers of science would have the resources to easily approach these questions.} \autocite{Winsberg2013}}

Parmis les auteurs ouvertement favorable à la création d'une nouvelle épistémologie, on citera entre autre les efforts de \autocites{Winsberg2001, Winsberg2009, Winsberg2013} qui pousse dans chacune de ses publications les \enquote{philosophes des sciences} à sortir de la seule étude de la \enquote{théorie de la confirmation} pour aller vers un terrain un peu plus aventureux \Anote{frilosite_philoScience}, celui de l'étude de la crédibilité des explications et des hypothèses dans leur dépendance au contexte.

Il propose de résumer l'originalité d'une telle épistémologie en évoquant l'inférence spécifique que produisent l'étude simultanée de trois point sur la simulation. \foreignquote{english}{ \textcite{Winsberg2001} argued that, unlike the epistemological issues that take center stage in traditional confirmation theory, an adequate EOCS must meet three conditions. 
downward, motley, and autonomous.[...] These three features were meant to be offered as conditions of adequacy; for which any adequate epistemology of simulation must account. Against the background of the growing use of simulation in the sciences, an adequate epistemology for the philosophy of science needs to explain the fact that simulation results and computational models are often taken to be reliable despite these three features. Winsberg (2001) argues that simulation requires a new epistemology precisely because traditional stories in philosophy of science about how knowledge claims get credentialed cannot explain them.}

Cette typologie a soulevé un certain nombre de critiques chez les philosophes des sciences, dont la plus longue et la plus argumenté est surement celle de \textcite{Frigg2009} dont on trouve le résumé des points saillants dans les publications de \textcites{Winsberg2009, Winsberg2013} mais également de bien d'autres auteurs qui se réfèrent à ce débat pour se positionner \textcites{Yanoff2010, Eckhart2010}.

Le deuxième point de débat intéressant réside dans le qualificatif souvent donné à la simulation de \enquote{laboratoire virtuel pour l'expérimentation}. Si les philosophes des sciences ne peuvent que s'incliner face au constat d'une telle banalisation du terme, dont nous avons donné nous même un aperçu de son ancienneté d'usage dans les sciences sociales dans le chapitre 1; il existe quand même chez les philosophes la volonté de mettre à l'épreuve les fondements et les conséquences pour la connaissance extraite d'une telle analogie.

\Anotecontent{HackingCartwright}{\enquote{Nos deux livres ont plus d'un point commun. L'un et l'autre accordent peu d'importance à la vérité des théories et avouent un faible pour certaines entités théoriques. Cartwright soutient que seules les lois phénoménologiques de la physique parviennent à la vérité tandis que, dans la partie B de ce livre, je fais remarquer que la science expérimentale est plus indépendante de la théorie que ce que l'on veut bien généralement admettre. Nous ne partons pas des mêmes postulats anti-théoriques car elle considère les modèles et les approximations alors que c'est surtout l'expérience qui m'intéresse, mais nos conceptions convergent.}\autocite{Hacking1983}}

\Anotecontent{Phan_Varenne_theorie}{\foreignquote{english}{Consequently, in the first neo-positivist epistemology, models were viewed not as autonomous objects, but as theoretically driven derivative instruments. Following the modelistic turn in mathematical logic, the semantic epistemological conception of scientific models persisted to emphasize on theory.} \autocite{Phan2010}}

Un débat d'autant plus actif qu'on assiste depuis ces 20 dernières années à un véritable renouveau des questionnements dans le cadre d'une \enquote{épistémologie de l'expérimentation} jusqu'alors relativement peu considéré par la majorité des philosophes des sciences \Anote{Phan_Varenne_theorie}. \textcites{Phan2008, Phan2010} citent ainsi les contributions importantes d'auteurs comme Fischer(1996), Galison (1987, 1997), Franklin (1986, 1996), Morrisson(1993, 1999), mais également les efforts de Hacking (1983) et Cartwright.

\Anotecontent{def_cartwright}{\enquote{Disons qu'il y a des théories, des modèles et des phénomènes. Il serait normal de penser que les modèles sont doublement des modèles. Ils sont modèles pour les phénomènes et modèles pour la théorie. [...] Le réalisme scientifique est ici tout particulièrement concerné. Cartwright est pour l'essentiel anti-réaliste à propos des théories. Pour cela, elle s'appuie en partie sur les modèles. Elle fait remarquer que non seulement les modèles ne peuvent être déduits de la théorie qui les englobe, mais plus encore que les physiciens utilisent à leur gré divers modèles qui, sans pourtant se recouper, cohabitent tous au sein de la même théorie. Et cependant,ces modèles sont les seules représentations formelles disponibles des lois phénoménologiques que nous tenons pour vraies. Elle affirme que seules ces lois phénoménologiques nous permettent d'avancer. Toutes les modélisations de ces lois ne peuvent être vraies ensemble puisqu'elles ne sont pas compatibles. Et rien ne permet de penser qu'un modèle est supérieur à un autre. Aucun n'est vraiment justifié par la théorie qui le porte. Plus encore, les modèles ont tendance à résister aux changements de théorie, c'est-à-dire que le modèle est conservé même si la théorie s'avère inadéquate. Il y a plus de vérité locale dans les modèles incompatibles que dans les théories, pourtant plus sophistiquées.[...] L'idéal de la science n'est pas l'unité mais dans une abondance et diversité de plus en plus grandes.} \autocite[350]{Hacking1983}}

\Anotecontent{def_hacking}{\enquote{Le \textit{réaliste à propos des entités} affirme que bon nombre d'entités théoriques existent vraiment. L'anti-réaliste s'oppose à ces entités qui ne sont pour lui que fictions, constructions logiques ou éléments d'un processus intellectuel d'appréhension du monde. Un anti-réaliste moins dogmatique dirait que nous n'avons pas, et ne pouvons avoir, de raison de supposer que ces entités ne sont pas des fictions. Peut-être existent-elles,mais le présupposer n'est pas nécessaire à notre compréhension du monde. 

Le \textit{réaliste à propos des théories} dit que les théories
sont soit vraies, soit fausses et ce indépendamment de ce que nous percevons : la science, elle au moins, vise à obtenir la vérité et la vérité est le monde tel qu'il est. L'anti-réaliste dit des théories qu'elles sont au mieux
prouvées, adéquates, opératoires, acceptables - quoi-que incroyables, entre autres qualificatifs possibles. } \autocite[59]{Hacking1983}}

On retiendra principalement pour notre argumentaire cette propriété d'indépendance retrouvé de l'expérimentation par rapport à la théorie \Anote{def_cartwright}, dont on peut trouver un très bon manifeste dans les écrits de \textcite{Hacking1983} et Cartwright \Anote{def_hacking}, ces derniers se positionnant comme des antiréalistes des théories, tout en étant des réalistes des entités théoriques. Un point de vue très bien résumé à la fois dans \textcite{Hacking1983} et \textit{Théorie, Réalité, Modèle} de \textcite[226-231]{Varenne2012}

Sur la notion de modèle dans sa relation à l'expérimentation, il semblerait qu'un consensus se dégage chez les philosophes \autocites{Morgan2009, Varenne2013} autour du modèle perçu comme un \enquote{médiateur autonome} articulant théorie, pratiques et données dans un contexte spécifique d'une question et d'un cadre technico-social. \autocite[2]{Phan2010}

Il y a probablement un point intéressant à développer entre cette argument du modèle autonome, et les récents travaux en sciences sociales pour qualifier au travers d'une grille de lecture \autocites{Banos2013a, Sanders2013} le positionnement \autocites{Banos2013, Schmitt2013} et le déplacement des modèles de simulation au travers d'une part de leur construction \autocite{Cottineau2014b}, mais également de leur réutilisation \autocite{Schmitt2014}. Une autre façon de démontrer en quoi cette capacité à cumuler de façon flou différentes fonctions épistémiques donné dans la spécification minimale de Varenne pour la simulation \autocite{Varenne2013} est intéressante dès lors qu'il s'agit de tracer la trajectoire disciplino-temporelle de certains modèles : daisyWorld \autocite{Dutreuil2013}, Schelling \autocite {Bulle2005}, SugarScape, etc.)

\Anotecontent{winsberg_exper_simu_link}{Another unique feature of the epistemology of simulation is the ease with which it can draw inspiration from the epistemology of experiment.}

Les acteurs pronant comme Winsberg une épistémologie de la simulation n'hésite alors pas à débattre pour ce qui est des différents parallèle que l'on peut tracer avec les réflexions de cette communauté. \Anote{winsberg_exper_simu_link}.

Pour ne pas se perdre dans les différents points de vues sur le sujet et bénéficier d'une vue plus large incluant les réflexions des praticiens, on pourra se référer au travail opéré par \textcite{Varenne2001} dans son article \textit{What does a computer simulation prove?}, qui propose une lecture du débat au travers de d'une typologie soulevant trois grandes thèses : I - La simulation est elle un outils commes les autres \textit{A simulation is only a tool} ? II - ou bien l'équivalent fusionnel d'une expérimentation classique (\textit{A simulation is an experiment}) ? III - ou se positionne-t-elle comme médiateur entre la théorie et expérimentation ? (\textit{A computer simulation is an intermediate between theory and experiment})? 

%L'expérimentation mène sa vie propre et entretient diverses relations avec la spéculation, le calcul, la construction de modèles, l'invention et la technologie. Mais alors que le calculateur, le spéculateur et le constructeur e modèles peuvent être anti-réalistes, l'expérimentateur, lui, doit être réaliste. p18 

On trouve donc un grand nombres de travaux, toutes disciplines confondues (les philosophes des sciences ne sont pas les seuls à se poser ce type de question, comme nous verrons par la suite), qui tentent d'établir par le biais de différentes grilles de lecture l'appartenance de ce \enquote{nouveau?} mode d'expérimentation à une des catégories de cette grille. \textit{Pourquoi ? Au delà du jeu d'esprit, quel est l'enjeu motivant une telle comparaison ?}

\Anotecontent{moto_hacking}{Une remarque qui renvoie d'ailleur explicitement à sa lecture du moto d'Hacking \foreignquote{english}{experiments have a life of their own} et à la notion d'autonomie (\textit{autonomous}) de sa synthèse précédemment, qui marque le fait que dans certains cas (impossibilité d'observation, manque de données), la simulation doit faire la preuve des connaissances (\textit{background knowledge}) apportés sur appel de ses propres ressources.}

\Anotecontent{experimental_warranting_belief}{\foreignquote{english}{The central idea of this thread is that experiments are the canonical entities that play a central role in warranting our belief in scientific hypotheses, and that therefore the degree to which we ought to think that simulations can also play a role in warranting such beliefs depends on the extent to which they can be identified as a kind of experiment} \autocite{Winsberg2009}}

Partant du fait que l'expérimentation joue un grand rôle dans l'établissement d'une crédibilité pour les hypothèses avancés, il s'agit de mesurer à quel point la simulation serait susceptible d'apporter les mêmes garanties dès lors qu'on accepte de la voir comme une sorte d'expérimentation.\Anote{experimental_warranting_belief}

On s'appuiera dans la suite de cette argumentation sur la lecture de Winsberg, un philosophe des sciences que l'on estime plutôt partisan de la III thèse dans la classification ci dessus. Ce dernier s'appuie largement sur les travaux d'Hacking, mais aussi Galison pour construire sa réflexion, par exemple en arguant\foreignquote{english}{ [...] that some of the techniques that simulationists use to construct their models get credentialed in much the same way that Hacking says that instruments and experimental procedures and methods do; the credentials develop over an extended period of time and become deeply tradition-bound.} \autocites{Winsberg2003, Winsberg2013} \Anote{moto_hacking}

Winsberg résume ce débat en deux thèses opposés : \foreignquote{english}{Identity Thesis} qui consiste à dire que la simulation est littéralement une expérimentation, et \foreignquote{english}{Epistemology Identity Thesis} qui consiste à penser qu'il existe une dépendance entre les garanties de crédibilité qui pourront être accordé par les résultats de la simulation et leur capacité à être plus ou moins définie en tant qu'expérience. Si la première thèse semble assez bien correspondre au point I de la classification de Varenne, la deuxième semble être une sous-variation du point I.

La plupart des auteurs cités par la suite dans ce débat sont des philosophes des sciences spécialisé en économie (Guala , Morgan, Maki, Simon ) qui rejettent comme Winsberg (plus spécialisé en physique) assez naturellement ces deux thèse \autocite{Winsberg2009}, mais avec des arguments assez différents, qu'il convient d'évoquer pour bien comprendre la complexité de ce débat, assez théorique. 

\Anotecontent{maki_phan}{\foreignquote{english}{For Mäki, abstractions in models are similar to abstractions in experiments as they both can be interpreted as a kind of isolation [...] This analogy between models and experiments is called \enquote{isolative analogy} by Guala (2008). From Mäki’s standpoint, a model can be said to be experimented in its explanatory dimension: the finality of such a model is to explore the explanatory power of some causal mechanism taken in isolation.} \autocite{Phan2008}}

Parmis les différents point de vue existant, on citera par exemple le sous-débat de l'\foreignquote{english}{isolative analogy} relaté ici au travers des publications de \textcite{Phan2008, Phan2010} apellant les points de vue de Morgan et Guala contre Maki (2005). Ce dernier voit dans la similitudes entre isolement théorique du modèle comme expérience de pensée et isolement expérimental \Anote{maki_phan} la possibilité de rejoindre une des deux thèses évoqués par Winsberg, établissant d'une façon ou d'une autre que \textit{les modèles sont des expériences, et les expériences des modèles}. Mais ce type d'argument, et on le suppose tout ceux qui se rapportent à l'évocation d'analogies pour justifier d'une équivalence de puissance épistémique se heurterai, comme on va le voir, à une différence fondamentale.

\Anotecontent{guala_phan_winsberg}{Winsberg résume le point de vue de Guala(2002) ainsi \foreignquote{english}{Guala argues that simulation differ fundamentally from experiments in that the object of manipulation in an experiment bears a material similarity to the target of interest, but in a simulation, the similarity between object and target are merely formal.}, mais on peut trouver une version réactualisé en 2008 dans l'article de \textcite[4.2]{Phan2010} \foreignquote{english}{In a simulation, one reproduces the behavior of a certain entity or system by means of a mechanism and/or material that is radically different in kind from that of a simulated entity (...) In this sense, \enquote{models simulate} whereas \enquote{ experimental systems} do not. Theoretical models are conceptual entities, whereas experiments are made of the same \enquote{stuff} as the target entity they are exploring and aiming at understanding}\autocite[14]{Guala2008}}

\textcite{Phan2010} et \textcite{Winsberg2013} cite le point de vue de Guala (2002, 2008), partagé par Morgan(2002, 2005) et se référant aux travaux de Simon (1969). Ceux-ci s'appuient sur une différence de relation qui existe entre système à étudier et système cible dans chacun des deux cas. En effet, dans le cas des expérience, la comparaison s'appuie avant tout sur une similarité matérielle, alors que dans le cas de la simulation la comparaison est limité à une comparaison formelle entre les objets.\Anote{guala_phan_winsberg}

\Anotecontent{Winsberg_critique_morvan}{\foreignquote{english}{Interestingly, while Morgan accepts this argument against the identity thesis, she seems to hold to a version of the epistemological dependency thesis. She argues, in other words, that the difference between experiments and simulations identified by Guala implies that simulations are epistemologically inferior to real experiments - that they have intrinsically less power to warrant belief in hypotheses about the real world.} \autocite[841]{Winsberg2013}}

Morgan(2002, 2005) accepte le point de vue Guala et Simon, mais s'en sert pour réduire indirectement le pouvoir épistémique de la simulation. Un argument bien résumé par \textcite{Phan2008} \enquote{Pour Morgan (2005) modèles et expériences partagent des fonctions de médiateurs et peuvent fonctionner \textit{sur un mode expérimental}, mais les expériences \textit{réelles} offrent un \textit{pouvoir épistémique} d'investigation de la réalité empirique plus fort.} Ce qui fait dire à Winsberg que Morgan serait indirectement plutot partisan de sa deuxième thèse.\Anote{Winsberg_critique_morvan}
\Anotecontent{winsberg_mereformal}{\hl{A compléter avec ce que dit Winsberg2013}}

Pour \textcite{Winsberg2009} le flou des arguments avancé par Morgan et Guala  (\textit{material similarity}, \textit{mere formal similarity}) ne permet pas d'exclure complétement et définitivement la première thèse.\Anote{winsberg_mereformal} Celui-ci se range malgré tout du coté de Guala, et préfère là aussi rejetter cette thèse, mais à la faveur de sa propre argumentation; ce qui lui permet de rejetter à la fois l'argument Morgan pointant l'infériorité épistémique de la simulation, et la deuxième thèse. Il argue que les simulations et l'expérience diffère principalement par la nature du \textit{background knownledge}, c'est à dire protocoles et les connaissances mobilisés.

Des modélisateurs et épistémologues en sciences sociales beaucoup plus proche de nos pratique comme Phan et Varenne trouve un argument convaincant dans ce dernier point, car \foreignquote{english}{Aujourd'hui, comme le souligne Winsberg, la crédibilité des modèles de simulation repose largement sur la \textit{confiance} que nous pouvons avoir dans les compétences des modélisateurs, informaticiens, expérimentateurs et observateurs, ainsi que dans les composants ou plateformes qui supportent les expériences de simulation.} \textcite{Phan2008}

\Anotecontent{gilbert_critique}{\foreignquote{english}{\enquote{[t]he major difference is that while in an experiment, one is controlling the actual object of interest (for example, in a chemistry experiment, the chemicals under investigation), in a simulation one is experimenting with a model rather than the phenomenon itself.} \autocite[14]{Gilbert2005}. But this doesn't seem right. [...] It is false that real experiments always manipulate exactly their targets of interest. In fact, in both real experiments and simulations, there is a complex relationship between what is manipulated in the investigation on the one hand, and the real-world systems that are the targets of the investigation on the other. In cases of both experiment and simulation, therefore, it takes an argument of some substance to establish the ‘external validity’ of the investigation – to establish that what is learned about the system being manipulated is applicable to the system of interest. Mendel, for example, manipulated pea plants, but he was interested in learning about the phenomenon of heritability generally \autocite{Winsberg2013}}}

\Anotecontent{guala_morgan_reality_experiments}{\foreignquote{english}{The identity thesis itself has drawn criticism from Guala (2002) and Morgan(2002). Guala begins by dismissing what he takes to be a poor argument against it. The poor argument goes something like this : simulations are not at all like real experiments because real experiments manipulate the real-world systems that are the very target of the investigation, while simulation merely manipulate \enquote{models} of the target system. What both Guala and Morgan correclty point out is that it is, quite generally speaking, false.}}

Autre sous-débat évoqués par \textcite{Winsberg2013}, on suppose en partie en réponse à sur son article précédent et très similaire \autocite{Winsberg2009}, la critique de l'\textit{identity thesis} comme évoqué par Gilbert et Troitzsch (1999), dont il pense \Anote{gilbert_critique}, en accord avec Guala (2002) \autocite{Winsberg2009} mais également Morgan et Parker \autocite{Winsberg2013} qu'elle est un argument trop faible pour rejeter l'\textit{identity thesis}. \Anote{guala_morgan_reality_experiments} 

Si les arguments de Winsberg semblent convaincant, \textcites{Peschard2010b, Peschard2013} tente dans une analyse critique d'en montrer les biais, et apporte dans son article des objections tout à fait crédible issue de son domaine d'expertise. Pour ne citer qu'un de ces argument, si il existe bien un intermédiaire de mesure issue d'un modèle, comme l'indique Winsberg, il existe également un sous système en prise directe avec la réalité physique de ce monde. En conclusion, elle estime que si il y a bien une certaine forme de similarité entre cibles épistémiques de la simulation et de l'expérience, pour elle ces activités ne peuvent pas être épistémiquement équivalentes, ce qui n'empeche en rien selon elle la coopération fructeuse des deux approches. \hl{Ajouter une footnote avec explication}

\textcite{Winsberg2013} résume le point de vue de \autocite{Peschard2010} ainsi, \textit{Thus, simulation is distinct from experiment, according to her, in that its epistemic target (as opposed to merely its epistemic motivation) is distinct from the object being manipulated.} Autrement dit, même si la motivation menant à l'expérience est bien eloigné (la motivation), l'objet manipulé dans une expérience est bien celui du monde physique, alors que dans le cas de la simulation c'est l'ordinateur. Or autant la motivation peut apprendre de l'objet manipulé dans le monde physique, autant il n'est pas ici dans notre intérêt d'apprendre sur l'ordinateur en tant qu'objet. Dans ce cas là on pointe une différence, mais on peut également appeler selon \textcite{Winsberg2013} et Morrisson (2009) l'argument inverse pointant au contraire une similarité. L'objet expérimenté étant le plus souvent choisi en tenant compte justement de sa capacité de \textit{surrogate} rapport à la question que l'on se pose effectivement, un point commun entre la construction de simulation et d'expérimentation. 

Winsberg conclu en ajoutant que l'expérimentation, contrairement à ce que l'on pourrait penser, n'est pas forcément et immédiatement plus crédible si on ne lui ajoute pas un bagage de connaissance : \textit{Experiments are not automatically more reliable than simulations, despite their differences. [...] It would seem that there are identifiable differences between ordinary experiments and simulations, but there is nothing about these differences that makes one or the other intrinsically more epistemically powerful.}  \autocites{Winsberg2009, Winsberg2013}

\textcite{Varenne2001} avance alors un autre argument intéressant : \foreignquote{english}{Indeed, when you read (Von Neumann 1951), you see that analog models are inferior to digital models because of the accuracy control limitations in the first ones. Following this argument, if you consider a prototype, or a real experiment in natural sciences, is it anything else than an analog model of itself? The test on the prototype is a real experiment. But is it something different and better than the handling of an analog model? So the possibilities to make sophisticated and accurate measures on this model - i.e. to make sophisticated real experiment - rapidly are decreasing, while your knowledge is increasing. These considerations are troublesome because it sounds as if nature was not a good model of itself and had to be replaced and simulated to be properly questioned and tested! It looks as if it was not possible any more to end a paper on simulation by reassuringly using the traditional word: \enquote{Simulation will never replace real experiments”.} }

Ces derniers paragraphes montrent que le débat est loin d'être fixé, et il semblerait là encore que ce soit la définition du contexte d'application qui détermine le mieux la capacité explicative de la simulation, car comme le dit Winsberg \enquote{l'impossibilité d'expérimenter} existe dans bien des disciplines, comme les sciences sociales, mais également la biologie ou la physique, ou les tentatives de reconstitution simulé d'univers ou d'étoiles dans des super calculateur de plus en plus puissant montre qu'il existe un interet explicatif à cette pratique. On pensera notamment aux projets d'expérimentation récents extremement complexe et couteux en physique (laser megajoule de bordeaux, projet ITER pour la fusion).

Et c'est sur ce point que l'argumentation de la plupart des philosophes des sciences est tout à la fois aussi intéressant que problématique. Pour continuer sur Winsberg, celui ci traite de ces problématiques en se positionnant uniquement du point de vue des sciences physiques. Un fait dont il reconnait prudement les conséquences que peuvent avoir l'inclusion d'un contexte différent sur sa synthèse : \foreignquote{english}{Parker (forthcoming) has made the point that the usefulness of these conditions is somewhat compromised by the fact that it is overly focused on simulation in the physical sciences, and other disciplines where simulation is theory-driven and equation-based. This seems correct. In the social and behavioral sciences, and other disciplines where agent-based simulation (see 2.2) are more the norm, and where models are built in the absence of established and quantitative theories, EOCS probably ought to be characterized in other terms.

For instance, some social scientists who use agent-based simulation pursue a methodology in which social phenomena (for example an observed pattern like segregation) are explained, or accounted for, by generating similar looking phenomena in their simulations (Epstein and Axtell 1996; Epstein 1999). But this raises its own sorts of epistemological questions. What exactly has been
accomplished, what kind of knowledge has been acquired, when an observed
social phenomenon is more or less reproduced by an agent-based simulation?
Does this count as an explanation of the phenomenon? A possible explanation?
(see e.g., Grüne-Yanoff 2007).

It is also fair to say, as Parker does (forthcoming), that the conditions outlined above pay insufficient attention to the various and differing purposes for which simulations are used (as discussed in 2.4). [...] Indeed, it is also fair to say that much more work could be done in classifying the kinds of purposes to which computer simulations are put and the constraints those purposes place on the structure of their epistemology.}

Des philosophes des sciences ont donc saisi cette opportunité de critiquer l'approche de Winsberg pour soulever dans des tentatives de typologies parfois intéressantes \autocite{Eckhart2010} les points de divergences que soulève l'utilisation d'une philosophies des sciences naturelle inadapté à la simulation en sciences sociales. Malheureusement, au cours de ces mêmes lectures, on constate que cette critique se retourne vers les modélisateurs et praticiens des sciences sociales, et mène cette fois ci dans une analyse incomplète du contexte historique au mieux à des interprétations erronés (voir le débat animé entre \autocite{Yanoff2008}  \autocites{Elsenbroich2012, Chattoe2011}), et au pire à des approximations et  conseils de mise en oeuvre totalement déplacé \autocite{Eckhart2010} vis à vis de disciplines qui disposent comme on l'a vu d'une véritable histoire autour de l'usage des méthodes computationelles.

Car bien que la recherche des points communs et des différences entre réalité de l'expérimentation physique et virtuelle apparaisse comme un débat intéressant, il faut bien avouer que celui ci ne peut que difficilement s'adapter à la quasi absence d'expérimentation au sens classique dans les sciences sociales. Ainsi, même si la simulation partage certaines des propriétés de l'expérimentation classique, il y a quand même quelque chose de paradoxal à vouloir absolument analyser le rapport de la simulation à l'expérimentation alors même que c'est cette absence qui justement motive son utilisation dans notre discipline, hormis peut etre pour mettre plus en avant cette incapacité à formuler un unique cadre fédérateur par un tel débat. Comme le dit très justement \textcite{Phan2008} {[...] les sciences économiques et sociales sont plus volontiers concernées par l’opposition entre \enquote{le modèle et l’enquête}  (Gérard-Varet et Passeron, 1995) que par celle entre \enquote{l’expérience et le modèle} (Legay, 1997)}

La notion de modèle vue comme médiateur autonome entre théorie et modèle doit elle aussi être repensé pour les sciences humaine, et la géographie; car les théories si elles peuvent exceptionnelement servir à dériver des modèles, celle ci ne peuvent qu'être difficilement rapporté à leur équivalent en science physique \autocite{Pumain1997}.

D'un coté les sciences physiques semble encore viser l'etablissement d'un cadre fédérateur alors qu'il semble que les théories et les modèles en sciences sociales - hormis peut être le cas particulier de l'économie - soit au contraire pourvoyeur de richesse dans leur capacité à apporter un nouvel éclairage sur un phénomène observé. \hl{a préciser peut etre}

A cela il faut ajouter que le modèle en géographie opère dans un cadre épistémique particulier qui n'est pas forcément celui de toute les sciences humaines. Ainsi, bien que les notions et le rapport entre les notions d'observation du \textit{singulier} et du \textit{général} soient théoriquement à la portée de toute disciplines \autocite{Dastes1992}, il semblerait que la géographie trouve un intérét particulier pour la constitution de sa démarche explicative à articuler des éléments de connaissance pris dans les grandes familles explicatives historique, écologique, et spatiale; justifiant ainsi de niveaux d'explication plus ou moins en interaction mobilisant chacun des déterminants de nature différentes. Avec la possibilité d'intégrer à tout moment dans l'explication les résidus qui tiennent d'un dialogue entre méchanismes généraux et singularité historique, écologique ou spatiale. Car quelque soit le registre explicatif choisi il reste dans les deux cas \textit{ [...] une part d'explication relevant de ce que l'on peut qualifier de singularités locales, non prédictibles à partir de mécanismes généraux, mais nécessitant d'appréhender l'histoire spécifique du lieu}. La conséquence étant une diversité de modèles support de l'explanan (l’explication que l’on propose du phénomène auquel on s’intéresse) dont l'évolution sur la forme et le fond n'a eu de cesse d'éclairer l'explanandum sous un jour différent. \autocite{Dastes1992, Sanders2000, Sanders2013} \hl{Mais il y a aussi le multi-échelles, etc.}

L'éclairage sur la méthodologie sous-jacente à la construction des modèles, pourtant un élément au coeur du raisonnement dans la discipline géographique depuis la révolution quantitative, a encore moins de chance d'être évoqué dans ces publications philosophiques, au détriment d'une réflexion statique plus axé sur la nature de l'objet simulation, et de sa relation au monde.

Or l'évolution des réflexions touchant l'activité de modélisation se construit il me semble à une échelle de reflexion tout à fait différente, celle contextualisé de pratiques guidés par une activité de résolution de questions spécifique à l'analyse spatiale, dont la mise en oeuvre s'appuie sur une chaine de traitements flexibles utilisant à bon escient et de façon cumulative l'arrivée historique de nouveaux outils et avec eux leur capacité à renouveller les questionnements : les statistiques, les modèles, les simulation.

\Anotecontent{ce qui n'est pas sans nous rapeller les difficultés évoqués dans le chapitre 1 sur l'inadéquation et le danger que représente les modes de transmissions actuels}

\Anotecontent{remarque_Varenne_2001}{\foreignquote{english}{The second thesis of this article is that none of the three categories of arguments could be applied to contemporary sciences in general, whatever their objects, their methods and the moment of their history we consider. None of these three categories could be considered as the only true one. We cannot have a general point of view on the value of computer simulations, because of the different implications and meanings of mathematics in the different fields of science, and because of the various philosophies of nature at stake. This fact remains true for a given field throughout its own history, because the role of mathematics and the definition of the studied object evolve: You cannot find a unique and stable value that would be given to its simulation uses once for all. Again and hopefully, this thesis illustrates the fact that it does not belong to the historian to decide on the value of computer simulation in a given field but to the scientists themselves. These preliminary reflections prove the importance to investigate the intellectual history of contemporary sciences and not only their sociological construction nor their philosophical general insights.}\autocite{Varenne2001}}

Ces rapides remarques nous éclaire sur la latence qui existe entre la réflexion récentes d'épistémologues comme Winsberg, Grüne-Yanoff et la réalité théorique et pratique en géographie. \textcite{Varenne2001} avait déjà bien cerné dans la synthèse faite en 2001 qu'il n'y avait pas dans sa classification une position meilleure ou plus convaincante qu'une autre, la réponse se trouvant comme pour la notion de modèle avant tout dans l'étude du contexte, et donc de l'histoire des disciplines face à cet objet simulation. \Anote{remarque_Varenne_2001} Cela ne veut pas dire que les débats évoqués précédemment en sont automatiquement invalidés, seulement qu'il faut être probablement plus regardant vis à vis des remarques générales et des conclusions beaucoup trop hative qui peuvent parfois en découler. 

Les travaux croisés de praticiens (Pumain, Sanders, Banos), d'épistémologues ou historien des sciences propre à la géographie (Orain, Besse, Robic, Cuyala) ou s'en approchant (Varenne, Phan) permettent d'une part d'apposer un premier filtre sur ces réflexions génériques pour s'y référer prudemment, et d'autre part d'innover en questionnant nos démarches dans ce qu'elles ont d'originales, cette fois ci appuyé sur une lecture des pratiques certe pas toujours parfaite mais pouvant au moins être qualifié de \textit{bottom-up} 


\hl{Communauté histoire JASSS ? }

%\hl{Un travail conséquent à la croisée de différentes approches, les travaux d'historiens et épistémologues des sciences somme Orain, Besse, Cuyala, et la lecture plus spécifique de l'évolution des méthodes numériques puis computationelles et de leur apports d'un point de vue pratique et théorique dont on trouve source à la fois dans les nombreux travaux des praticiens, mais également dans des travaux de plus long cours comme celui qu'est en train de réaliser Varenne dans son HDR.  == REDITE}

%\hl{ont su voir rapidement l'intérét de développer plus en avant les spécificités attachés à la simulation en science sociale, déjà riche de réflexion sur les apports successifs et cumulatifs de techniques de simulations \autocites{Banos2013, Varenne2008}, en s'intégrant au débat d'une communauté inter-disciplinaire structuré autour de la modélisation agent, qui émerge dans les sciences sociales au début des années 1990. Ce débat par contre ne fait semble-t-il que commencer dans le courant plus \textit{mainstream} des philosophe des sciences.}

%tel que celle des géographes pratiquant la simulation depuis les années 1950, tels que celle qui a émergé autour de la modélisation agent dans les années 1990. 

%Ainsi comme on a pu le voir dans le chapitre 1, le terme laboratoire virtuel pour l'expérimentation apparait très tot dans les sciences sociales, et des auteurs ont pour l'époque déjà donnés de très bonne raisons pour l'emploi de ce terme; les aspects dynamiques de la simulation en faisait partie.

\hl{=> Transition apport de Varenne par rapport à tout ce bazar}


C'est là que le travail de Varenne réalisé au cours des années 2000 \autocites{Varenne2008, Varenne2013} apparait assez audacieux, en proposant une typologie de fonctions épistémiques flexible et cumulable, il propose une grille de lecture permettant d'intégrer à la fois la diversité des approches dans les disciplines (inter) mais également l'évolution de ces même approches à l'intérieur d'une discipline (intra). Un découplage qui permet également une définition plus fine des rapports que peuvent entretenir les disciplines entre le modèle et la simulation.

\hl{ Varenne, la simulation comme expérience de second genre, la possibilité d'un rapport à l'empirie ... (a voir si je rentre la dedans maintenant ou si je garde ça pour plus tard dans la partie construction de modèle de simulation}



\hl{--------------- Pas fini, tu peux sauter à la partie d'après :) ------}


 %La typologie de Varenne est intéressante car elle sous entend une grande partie des sous débats ou raffinements qui peuvent exister sur ce thème, \autocite{Eckhart2010}

%Et c'est vrai que des propriétés intéressantes développés par Hacking comme l'autonomie des modèles et de ce fait l'autonomie des résultats, est un concept intéressant lorsqu'on le rattache à la vie des modèles de simulations tels que nous les construisons.


 %On pourra également arguer que c'est bien là le problème des sciences de la complexité, c'est qu'il est difficile sinon impossible de rendre compte du fonctionnement global d'un système en étudiant seulement les éléments qui le constitue, coupés de tout ou partie de leur interactions%, avec pour effet l'intrication des causes et des effets.


%++ Innovation en géographie des ABÙ, par rapport aux système dynamique outre la flexibilité exposé, c'est l'apport de la pluriformalisation et la possibilité de formuler (ou pas) un rapprochement entre entité virtuelle et réelle (dénotation interne / externe de Varenne); avec tout les dangers qu'un tel rapprochement suppose... cf les individu micro pour les sociologues, les villes pour les géographe, etc. Mais les modèles restent des modèles causaux, ou ce qui est dans le modèle compte plus pour l'explication que le modèle en lui meme en tant qu'instantané ++

%Mais j'aimerais revenir à présent sur l'apport historique d'Hermann à ces débats, un acteur important dans l'histoire de la V\&V, et dont il me semble on mesure encore l'actualité des questionnements qu'il souleve en 1967.



\subsubsection{synthèse}


Si la communauté \textit{Models\&Simulations} propose aujourd'hui un cadre d'analyse cohérent avec la dynamique attendue chez les géographes pour la construction des modèles, il lui manque toutefois une incarnation géographique qu'il va falloir extraire de nos propres exigences de construction.

Pour comprendre comment la notion de validation se construit en marge de ces deux discours, il faut revenir sur ce qui fait sens dans l'explication pour les géographes. En repartant des transformations que subit la géographie quantitative dans les années 1970 au contact du paradigme systémique, prise dans une nouvelle réflexion des objets géographiques , dont la percolation chez les géographes s'observe dans la nouveauté le champs lexical, les méthodes, mais également les techniques.

Au delà des outils il y a un fond commun, à la construction de modèle en géographie, et qui n'est que très rarement traité dans ces lectures, celui de la mise en oeuvre de la construction. Nous verrons qu'il y a des raisons à cela, la dépendance au contexte en est une, et cette activité hautement flexible, bien qu'influencé par la qualité du substrat technique, pose dans la mobilisation des hypothèses des questions génériques à ce dernier : l'hypothèse que j'ai choisi est elle représentative ?

Il ya une forme de permanence dans les questions posés par la construction d'un modèle ou d'un modèle de simulation qui apelle à la construction d'une vision de la validation plus appliqué en géographie.

(ce qui veut dire qu'il lui faudra un support, et cela viendra par la suite)

\hl{--------------- Pas fini, tu peux sauter à la partie d'après :) ------}




% -*- root: These.tex -*-

\subsection{Une lecture pluri-disciplinaire des problématiques liés à la validation}
\label{ssec:triple_lecture}

%Si le modélisateur est au courant des simplifications opérés dans les hypothèses censés représenter ,  la dynamique de construction introduit dans l'activité de construction des modèles une incertitude supplémentaire qui nous oblige à repenser l'activité de validation.


Notre tentative pour pointer quelque une des transformations touchant l'activité modélisatrice dans le giron universitaire entre 1950 et 1970 recoupe bien évidemment la description de vagues d'innovations déjà identifiées par ailleurs, et nous aurons l'ocasion de revenir plusieurs fois sur cette période charnière que sont les années 1970 par la suite. 

L'originalité du travail réside plutôt dans l'identification par une communauté de modélisateurs déjà très hétérogènes d'un ensemble de facteurs limitant le développement et la diffusion de l'activité de modélisation appuyé par l'usage de l'ordinateur. 

Fait quelque peu destabilisant pour un modélisateur cotoyant encore cette expression dans les publications en 2015, le \enquote{problème de la validation} apparait en effet très tôt parmis ces différents facteurs. 

Une question nous vient alors très rapidement à l'esprit, et mérite d'être posé, même si elle on verra par la suite qu'elle est un peu naïve. Comment se fait il que ce problème apparu il a presque 50 ans de façon quasi conjointe avec l'invention et l'adoption de la simulation par différentes communautées de pionniers modélisateurs soit encore aussi présente aujourd'hui, par exemple dans le cadre des publications de communautées fortement inter-disciplinaire comme JASSS ? 

En réalité, s'agit il vraiment du même problème ? Car entre la révolution quantitative partisanne d'une révolution plus globale qui voie les principaux verrou informatique s'effacer ou se déplacer, que faut il encore entendre d'une telle expression ? 

La difficulté d'analyser ce terme à la fois dans ses évolutions temporelles,  et dans sa diversité d'inscription disciplinaire 


\textbf{en interne}

par une communauté hétérogènes de modélisateurs comme dommageable pour la diffusion de cette activité de modélisation sur ordinateur. 


	détour d'une analyse rapide des mutations qu'a subit la modélisation en géographie au détour des années 1970, il est d'ores et déjà visible que la progression informatique n'apparait plus comme la seule contrainte 

Proposer une étude exhaustive de ce terme est délicat

\subsubsection{Les définitions de la validation en V\&V}
\label{sssec:def_generique_validation}

Les termes \foreignquote{english}{Validation \& Verification} ou \textit{V\&V} proviennent à l'origine de l'ingénierie des systèmes, et peuvent être rattachés au concept de \enquote{qualité} tel qu'il est défini par la famille de règles ISO établies par l'organisation mondiale de normalisation.

Décomposable en plusieurs branches cette discipline à part possède une branche dédiée à l'expertise logicielle. De ce fait, il n'existe pas réellement de définition ni de théories ou méthodologies officiellement acceptables, l'acceptation des termes pouvant varier fortement selon les branches d'application.

On trouve toutefois quelques références dans des livres dédiés à la terminologie standard pour la \enquote{gestion de projet} dans un large panel de disciplines, telle que le PMBOK (\textit{A guide to the project Management Body of Knowledge}) \autocite{PMBOK2013}. Résultats d'un travail certifié par des associations ou des organismes étatiques tels que IEEE et ANSI, ce dernier propose une définition générale de ces termes pour l'ingénierie logicielle :

\foreignquote{english}{Verification and validation (V\&V) processes are used to determine whether the development products of a given activity conform to the requirements of that activity and whether the product satisfies its intended use and user needs.}

et revient ensuite plus spécifiquement sur les termes :

\begin{itemize}
\item \textbf{Validation} \foreignquote{english}{The assurance that a product, service, or system meets the needs of the customer and other identified stakeholders. It often involves acceptance and suitability with external customers. Contrast with verification.}
\item \textbf{Verification} \foreignquote{english}{The evaluation of whether or not a product, service, or system complies with a regulation, requirement, specification, or imposed condition. It is often an internal process. Contrast with validation.}
\end{itemize}

Les termes tels qu'ils sont définis sont finalement bien trop généraux pour envisager de les appliquer tels quels dans notre domaine de compétence. Dérivé de la branche de l'\textit{Operational Research (OR)}, les auteurs de la communauté restreinte des \textit{systems analysis or modelling and Simulation (M\&S) } engagent dès les années 1960-70 des efforts pour standardiser ces définitions pour la simulation.

\Anotecontent{first_time_validation}{La citation de Churchman par \textcite{Naylor1966} est tiré de \autocite[165]{Nance2002} : \foreignquote{english}{\foreignquote{english}{X simulates Y} is true if, and only if, (a) X and Y are formal systems, (b) Y is taken to be the real system, (c) X is taken to be an approximation to the real system and (d) the rules of validity in X are non-error-free.} \autocite{Nance2002} }

Parmi les différents auteurs participant de ce mouvement ( Naylor, Finger, Oren, Hermann, Zeigler, Nance, Banks, Gass, Balci, Sargent, etc.), \textcite{Naylor1966} est considéré avec West Churchman (1963) comme un des tout premier à avoir attiré et cristalisé \Anote{first_time_validation} dans de multiples publications l'attention sur cette problématique importante de la V\&V.

Cet économiste formé à l'informatique dans la branche des \foreignquote{english}{management sciences} \autocite{Stricklin1985} est un des premiers en 1967 \autocite{Naylor1967} à publier dans un article nommé \foreignquote{english}{Verification of Computer simulation models} une méthode abordant spécifiquement la question de la crédibilité des connaissances qui peuvent être apportées par un modèle de simulation. Une méthode qu'il va mettre spontanément en tension avec les débats qui agitent la communauté des philosophes à cette même période.

Malgré ces efforts et sa volonté de porter le débat loin dans la communauté inter-disciplinaire (voir les premiers ouvrage collectifs sur l'usage de la simulation dans les \enquote{behavior science} \autocite{Dutton1971, Guetzkow1972} \hl{A verifier}), la démarcation entre les deux termes est encore peu claire \autocites[165]{Nance2002}[3]{Balci1986}. \footnote{\foreignquote{english}{Thomas Naylor, a coauthor of the book cited above, deserves credit for drawing major attention to the validation issue in the 1960s: Is the model actually representing the truthful behavior of the referent system? His work, above and in later publications (Naylor 1971, Naylor and Finger 1967), exerted a major influence in framing validation within different philosophical perspectives. Numerous techniques that can be used were identified or developed. While the issues of both verification and validation were of concern from the early days of simulation, often no clear distinction was made between the two terms.} \autocite[165]{Nance2002}}

\Anotecontent{balci_standard}{\foreignquote{english}{A uniform, standard terminology is yet nonexistent. A recent literature review \autocite{Balci1984} indicated the usage of 16 terms [...] Except some early papers which appearead between 1966 and 1972, model verification and model validation have been most of the time consistently defined reflecting the following differentiation : \textbf{model verification} refers to building the model right; and \textbf{model validation} refers to building the right model. \autocite{Balci1986}}}

Il faudra attendre le début des années 1980 pour qu'un standard émerge, grâce à des financements étatiques \autocite{Balci1986}, mais également du fait des efforts fournis par des auteurs comme Sargent et Balci \autocite{Nance2002}, qui collectent et organisent dans une typologie cohérente l'existant statistique et méthodologique, une activité qu'ils poursuivent encore aujourd'hui \autocite{Balci1998}.\Anote{balci_standard}

Pour \autocite[22]{Oberkampf2010} \foreignquote{english}{A Key milestone in the early work by the OR community was the publication of the first definitions of V\&V by the Society of Computer Simulation (SCS) in 1979 \autocite{Schlesinger1979}}, un des instituts avec la U.S GAO (U.S General Accounting Office) à fournir des spécifications en 1979 \autocite{Balci1986}

\begin{itemize}
\item \textbf{Model Verification} \foreignquote{english}{substantiation that a computerized model represents a conceptual model within specified limit of accuracy.}
\item \textbf{Model Validation} \foreignquote{english}{substantiation that a computerized model within its domain of applicability possesses a satisfactory range of accuracy consistent with the intended application of the model.}
\end{itemize}

\begin{figure}[h]
\begin{sidecaption}[fortoc]{Un des tout premiers schémas issus de la publication de la SCS \autocite{Oberkampf2010,Schlesinger1979}}[fig:S_VV]
  \centering
 \includegraphics[width=.7\linewidth]{schelinger_schema1979.png}
  \end{sidecaption}
\end{figure}

Même si elles sont plus anciennes et de portée moins générale, ces définitions de la \textit{V\&V} semblent plus pertinentes, car évoquées plus régulièrement par les chercheurs en sciences sociales; les travaux les plus cités étant ceux de \textcite{Kleijnen1995}, ou \textcite{Sargent2010} qui placent leurs travaux dans la continuité de ces définitions. L'avancée de leurs travaux peut être suivie en feuilletant les \textit{Proceedings of the Winter Simulation Conference} où la problématique de la \textit{V\&V} est réévaluée régulièrement au regard des nouvelles connaissances. Ce schéma \ref{fig:S_VV} est devenu un classique repris et régulièrement amendé \autocite{Sargent2010}. Voici la lecture qu'en fournit \autocite{Oberkampf2010}

\foreignquote{english}{The \textbf{conceptual model} comprises all relevant information, modelling assumptions, and mathematical equations that describe the physical process or process of interest. [...] The SCS defined \textbf{qualification} as \enquote{Determination of adequacy of the conceptual model to provide an acceptable level of agreement for the domain of intended application}. The \textbf{computerized model} is an operational computer program that implements a conceptual model using computer programming. Modern terminology typically refers to the computerized model as the computer model or code.}

Ce schéma a la particularité suivante, il \foreignquote{english}{ [...] emphasizes that \textbf{verification} deals with the relationship between the conceptual model and computerized model and that \textbf{validation} deals with the relationship between the computerized model and reality. These relationships are not always recognized in other definitions of V\&V [...]}

\Anotecontent{Kleijnen_def}{\foreignquote{english}{This paper uses the definitions of V \& V given in the classic simulation textbook by Law and Kelton (1991, p.299): \enquote{Verification\textbf{Verification} is determining that a simulation computer program performs as intended, i.e., debugging the computer program .... \textbf{Validation} is concerned with determining whether the conceptual simulation model (as opposed to the computer program) is an accurate representation of the system under study}. Therefore this paper assumes that verification aims at a \enquote{perfect} computer program, in the sense that the computer code has no programming errors left (it may be made more efficient and more user friendly). Validation, however, can not be assumed to result in a perfect model, since the perfect model would be the real system itself (by definition, any model is a simplification of reality). The model should be \enquote{good enough}, which depends on the goal of the model.}}

\Anotecontent{Sargent_def}{\foreignquote{english}{\textbf{Model verification} is often defined as \enquote{ensuring that the computer program of the computerized model and its implementation are correct} and is the definition adopted here. \textbf{Model validation} is usually defined to mean \enquote{substantiation that a computerized model within its domain of applicability possesses a satisfactory range of accuracy consistent with the intended application of the model} \autocite{Schlesinger1979} and is the definition used here. A model sometimes becomes accredited through model accreditation. Model accreditation determines if a model satisfies specified model accreditation criteria according to a specified process. A related topic is model credibility. Model credibility is concerned with developing in (potential) users the confidence they require in order to use a model and in the information derived from that model. A model should be developed for a specific purpose (or application) and its validity determined with respect to that purpose [...]A model is considered valid for a set of experimental conditions if the model’s accuracy is within its acceptable range, which is the amount of accuracy required for the model’s intended purpose.}}

Autrement dit, \foreignquote{english}{The OR community clearly recognized, as it still does today, that V\&V are tools for assessing the accuracy of the conceptual and computerized models.} Un avis partagé par \textcite{Kleijnen1995} \Anote{Kleijnen_def} , \textcite{Balci1998}, et \textcite{Sargent2010} \Anote{Sargent_def} mais également des auteurs de références sur le sujet dans les sciences humaines et sociales \autocite{Amblard2006} \hl{Prend le bout de texte la dessus}.

Seulement, cette forme de relâchement sur la correspondance entre réalité et modèle, et ce positionnement plus relativiste de la validation n'a pas toujours été une évidence; les premières définitions de Naylor par exemple, sont toujours usitées, et continuent si on en croit des auteurs comme \textcite{Kleindorfer1998} à semer le trouble dans certaines disciplines.

\Anotecontent{VV_philout}{ \foreignquote{english}{During the last two decades a workable and constructive approach to the concepts, terminology, and methodology of V\&V has been developped, but it was based on pratical realities in business and government, \textbf{not} the issue of obsolute thruth in the philosophy of nature} \autocite{Oberkampf2010}
\foreignquote{english}{A very old philosophical question is: do humans have accurate knowledge of reality or do they have only flickering images of reality, as Plato stated? In this paper, however, we take the view that managers act as if their knowledge of reality were sufficient. Also see Barlas and Carpenter (1990), Landry and Oral (1993), and Naylor, Balintfy, Burdick and Chu (1966, pp.310-320).} \autocite{Kleijnen1995}
\foreignquote{english}{With the strong interest in verification from the software engineering community, this contrasting but complementary explanation of the term was quite important. The effort to place valida- tion in a cost-risk framework moved the concept from a philosophical explanation in earlier works to a form more useable for simulation practitioners.} \autocite[165-166]{Nance2002}}

Mais en excluant ainsi de son analyse la partie subjective et philosophique de la \enquote{Validation}\Anote{VV_philout} pour se concentrer sur la seule partie opérationnelle, ces méthodologies restent pour le modélisateur une coquille vide décevante, qui demande encore à être incarnée thématiquement. Autrement dit, ces méthodes si elles prennent bien en compte la dimension dynamique et incrémentale nécessaire à la construction d'un modèle de simulation qui tendrait vers une réalité en accord avec la question posée, l'organisation des connaissances nécessaires pour guider ce processus reste à la lecture de ces typologies une opération quelque peu énigmatique pour les modélisateurs géographes. On retombe sur une des critiques soulevées précédemment dans la section \ref{sec:critiques_simulation} sur l'absence constatée dans les publications de méthodologie standard pour la validation qui prendrait en compte les problématiques spécifiques d'une discipline. \footnote{Aujourd'hui des disciplines comme l'écologie proposent des méthodologies plus spécifiques, comme la méthode POM proposé par Grimm sur lequel nous reviendront par la suite \hl{mettre une ref et un appel à la section}}

Une position compréhensible pour ces auteurs oeuvrant pour la standardisation, alors même que ces termes sont toujours d'usages toujours assez variables. Une des conséquences visibles tient dans ces incompréhensions et ces débats terminologiques sans fin \autocite{David2009} que l'on observe parfois en marge des discussions inter-disciplinaires. Cette gamme d'acceptions différentes tient souvent au transfert hasardeux des terminologies entre l'ingénierie des M\&S, la philosophie des sciences, et la thématique d'un chercheur en sciences sociales qui se retrouve en position intermédiaire de ces deux derniers. Un exercice d'équilibriste périlleux, car comme le fait remarquer \textcite{Kleijnen1995} en citant astucieusement une note de bas de page de \textcite{Barlas1990}, en philosophie il est tout à fait possible de voir la signification des deux termes inversée! \hl{Expliquez mieux que verification pourrait se traduire en philosophie pour certains par representation de la vérité, du “reel”, alors que le fait même de modéliser implique qu’on en soit loin}

\subsubsection{La philosophie des sciences}
\label{sssec:philo_sciences}

Il ne s'agit pas de se lancer ici dans un exposé historique des courants et débats s'étant succédés dans cette discipline, mais d'amener de façon illustrative et avec quelques références récentes l'émergence ces 20 dernières années d'une \enquote{épistémologie de la simulation} reprenant (en parasitant parfois le débat comme on l'a cité au dessus) de son point de vue certains débats évoqués chez les praticiens de la simulation; la question de validation étant comme on l'a vu dans le chapitre 1 un sujet de longue date chez les praticiens de la simulation, mais aussi chez les premiers acteurs fondateurs de la V\&V.

\hl{redite : L'objectif n'est donc pas tant de développer une argumentation critique exposant l'ensemble de ces points de vues, car ce n'est pas l'objet de cette thèse, que de tenter de s'insérer (et non de s'enfermer) dans ces réflexions en spécifiant en quoi celle ci diffère, néglige ou font peu écho à nos pratiques et réflexion historique en sciences sociales.}

Le premier obstacle avec laquelle les acteurs supportant cette nouvelle épistémologie doivent cohabités est évidemment la contre-argumentation questionnant cette même necessité d'opérer une nouvelle sous-division épistémologique. Car existe-t-il réellement des spécificité à la connaissance dérivé de l'étude de l'objet simulation, et si oui quelles sont elles réellement ? Autrement dit, existe t il une différence fondamentale entre les questionnements déjà posés dans le cadre d'une épistémologie des modèles et ceux évoqués dans le cadre d'une épistémologie de la simulation ?

\Anotecontent{frilosite_philoScience}{\foreignquote{english}{As computer simulation methods have made their way into novel disciplines, the issue of their trustworthiness for generating new knowledge has often loomed large, especially when they have competed for attention with experiments or analytically tractable modeling methods. The relevant question is always whether or not the results of a particular computer simulation are accurate enough for their intended purpose.[...] Given our long-standing preoccupation with issues of confirmation, it might seem obvious that philosophers of science would have the resources to easily approach these questions.} \autocite{Winsberg2013}}

parmi les auteurs ouvertement favorable à la création d'une nouvelle épistémologie, on citera entre autre les efforts de \autocites{Winsberg2001, Winsberg2009, Winsberg2013} qui pousse dans chacune de ses publications les \enquote{philosophes des sciences} à sortir de la seule étude de la \enquote{théorie de la confirmation} pour aller vers un terrain un peu plus aventureux \Anote{frilosite_philoScience}, celui de l'étude de la crédibilité des explications et des hypothèses dans leur dépendance au contexte.

Il propose de résumer l'originalité d'une telle épistémologie en évoquant l'inférence spécifique que produisent l'étude simultanée de trois point sur la simulation. \foreignquote{english}{ \textcite{Winsberg2001} argued that, unlike the epistemological issues that take center stage in traditional confirmation theory, an adequate EOCS must meet three conditions. 
downward, motley, and autonomous.[...] These three features were meant to be offered as conditions of adequacy; for which any adequate epistemology of simulation must account. Against the background of the growing use of simulation in the sciences, an adequate epistemology for the philosophy of science needs to explain the fact that simulation results and computational models are often taken to be reliable despite these three features. Winsberg (2001) argues that simulation requires a new epistemology precisely because traditional stories in philosophy of science about how knowledge claims get credentialed cannot explain them.}

Cette typologie a soulevé un certain nombre de critiques chez les philosophes des sciences, dont la plus longue et la plus argumenté est surement celle de \textcite{Frigg2009} dont on trouve le résumé des points saillants dans les publications de \textcites{Winsberg2009, Winsberg2013} mais également de bien d'autres auteurs qui se réfèrent à ce débat pour se positionner \textcites{Yanoff2010, Eckhart2010}.

Le deuxième point de débat intéressant réside dans le qualificatif souvent donné à la simulation de \enquote{laboratoire virtuel pour l'expérimentation}. Si les philosophes des sciences ne peuvent que s'incliner face au constat d'une telle banalisation du terme, dont nous avons donné nous même un aperçu de son ancienneté d'usage dans les sciences sociales dans le chapitre 1; il existe quand même chez les philosophes la volonté de mettre à l'épreuve les fondements et les conséquences pour la connaissance extraite d'une telle analogie.

\Anotecontent{HackingCartwright}{\enquote{Nos deux livres ont plus d'un point commun. L'un et l'autre accordent peu d'importance à la vérité des théories et avouent un faible pour certaines entités théoriques. Cartwright soutient que seules les lois phénoménologiques de la physique parviennent à la vérité tandis que, dans la partie B de ce livre, je fais remarquer que la science expérimentale est plus indépendante de la théorie que ce que l'on veut bien généralement admettre. Nous ne partons pas des mêmes postulats anti-théoriques car elle considère les modèles et les approximations alors que c'est surtout l'expérience qui m'intéresse, mais nos conceptions convergent.}\autocite{Hacking1983}}

\Anotecontent{Phan_Varenne_theorie}{\foreignquote{english}{Consequently, in the first neo-positivist epistemology, models were viewed not as autonomous objects, but as theoretically driven derivative instruments. Following the modelistic turn in mathematical logic, the semantic epistemological conception of scientific models persisted to emphasize on theory.} \autocite{Phan2010}}

Un débat d'autant plus actif qu'on assiste depuis ces 20 dernières années à un véritable renouveau des questionnements dans le cadre d'une \enquote{épistémologie de l'expérimentation} jusqu'alors relativement peu considéré par la majorité des philosophes des sciences \Anote{Phan_Varenne_theorie}. \textcites{Phan2008, Phan2010} citent ainsi les contributions importantes d'auteurs comme Fischer(1996), Galison (1987, 1997), Franklin (1986, 1996), Morrisson(1993, 1999), mais également les efforts de Hacking (1983) et Cartwright.

\Anotecontent{def_cartwright}{\enquote{Disons qu'il y a des théories, des modèles et des phénomènes. Il serait normal de penser que les modèles sont doublement des modèles. Ils sont modèles pour les phénomènes et modèles pour la théorie. [...] Le réalisme scientifique est ici tout particulièrement concerné. Cartwright est pour l'essentiel anti-réaliste à propos des théories. Pour cela, elle s'appuie en partie sur les modèles. Elle fait remarquer que non seulement les modèles ne peuvent être déduits de la théorie qui les englobe, mais plus encore que les physiciens utilisent à leur gré divers modèles qui, sans pourtant se recouper, cohabitent tous au sein de la même théorie. Et cependant,ces modèles sont les seules représentations formelles disponibles des lois phénoménologiques que nous tenons pour vraies. Elle affirme que seules ces lois phénoménologiques nous permettent d'avancer. Toutes les modélisations de ces lois ne peuvent être vraies ensemble puisqu'elles ne sont pas compatibles. Et rien ne permet de penser qu'un modèle est supérieur à un autre. Aucun n'est vraiment justifié par la théorie qui le porte. Plus encore, les modèles ont tendance à résister aux changements de théorie, c'est-à-dire que le modèle est conservé même si la théorie s'avère inadéquate. Il y a plus de vérité locale dans les modèles incompatibles que dans les théories, pourtant plus sophistiquées.[...] L'idéal de la science n'est pas l'unité mais dans une abondance et diversité de plus en plus grandes.} \autocite[350]{Hacking1983}}

\Anotecontent{def_hacking}{\enquote{Le \textit{réaliste à propos des entités} affirme que bon nombre d'entités théoriques existent vraiment. L'anti-réaliste s'oppose à ces entités qui ne sont pour lui que fictions, constructions logiques ou éléments d'un processus intellectuel d'appréhension du monde. Un anti-réaliste moins dogmatique dirait que nous n'avons pas, et ne pouvons avoir, de raison de supposer que ces entités ne sont pas des fictions. Peut-être existent-elles,mais le présupposer n'est pas nécessaire à notre compréhension du monde.

Le \textit{réaliste à propos des théories} dit que les théories
sont soit vraies, soit fausses et ce indépendamment de ce que nous percevons : la science, elle au moins, vise à obtenir la vérité et la vérité est le monde tel qu'il est. L'anti-réaliste dit des théories qu'elles sont au mieux
prouvées, adéquates, opératoires, acceptables - quoi-que incroyables, entre autres qualificatifs possibles.} \autocite[59]{Hacking1983}}

On retiendra principalement pour notre argumentaire cette propriété d'indépendance retrouvé de l'expérimentation par rapport à la théorie \Anote{def_cartwright}, dont on peut trouver un très bon manifeste dans les écrits de \textcite{Hacking1983} et Cartwright \Anote{def_hacking}, ces derniers se positionnant comme des antiréalistes des théories, tout en étant des réalistes des entités théoriques. Un point de vue très bien résumé à la fois dans \textcite{Hacking1983} et \textit{Théorie, Réalité, Modèle} de \textcite[226-231]{Varenne2012}

Sur la notion de modèle dans sa relation à l'expérimentation, il semblerait qu'un consensus se dégage chez les philosophes \autocites{Morgan2009, Varenne2013} autour du modèle perçu comme un \enquote{médiateur autonome} articulant théorie, pratiques et données dans un contexte spécifique d'une question et d'un cadre technico-social. \autocite[2]{Phan2010}

Il y a probablement un point intéressant à développer entre cette argument du modèle autonome, et les récents travaux en sciences sociales pour qualifier au travers d'une grille de lecture \autocites{Banos2013a, Sanders2013} le positionnement \autocites{Banos2013, Schmitt2013} et le déplacement des modèles de simulation au travers d'une part de leur construction \autocite{Cottineau2014b}, mais également de leur réutilisation \autocite{Schmitt2014}. Une autre façon de démontrer en quoi cette capacité à cumuler de façon flou différentes fonctions épistémiques donné dans la spécification minimale de Varenne pour la simulation \autocite{Varenne2013} est intéressante dès lors qu'il s'agit de tracer la trajectoire disciplino-temporelle de certains modèles : daisyWorld \autocite{Dutreuil2013}, Schelling \autocite {Bulle2005}, SugarScape, etc.)

\Anotecontent{winsberg_exper_simu_link}{Another unique feature of the epistemology of simulation is the ease with which it can draw inspiration from the epistemology of experiment.}

Les acteurs pronant comme Winsberg une épistémologie de la simulation n'hésite alors pas à débattre pour ce qui est des différents parallèle que l'on peut tracer avec les réflexions de cette communauté. \Anote{winsberg_exper_simu_link}.

Pour ne pas se perdre dans les différents points de vues sur le sujet et bénéficier d'une vue plus large incluant les réflexions des praticiens, on pourra se référer au travail opéré par \textcite{Varenne2001} dans son article \textit{What does a computer simulation prove?}, qui propose une lecture du débat au travers de d'une typologie soulevant trois grandes thèses : I - La simulation est elle un outils commes les autres \textit{A simulation is only a tool} ? II - ou bien l'équivalent fusionnel d'une expérimentation classique (\textit{A simulation is an experiment}) ? III - ou se positionne-t-elle comme médiateur entre la théorie et expérimentation ? (\textit{A computer simulation is an intermediate between theory and experiment})? 

%L'expérimentation mène sa vie propre et entretient diverses relations avec la spéculation, le calcul, la construction de modèles, l'invention et la technologie. Mais alors que le calculateur, le spéculateur et le constructeur e modèles peuvent être anti-réalistes, l'expérimentateur, lui, doit être réaliste. p18 

On trouve donc un grand nombres de travaux, toutes disciplines confondues (les philosophes des sciences ne sont pas les seuls à se poser ce type de question, comme nous verrons par la suite), qui tentent d'établir par le biais de différentes grilles de lecture l'appartenance de ce \enquote{nouveau?} mode d'expérimentation à une des catégories de cette grille. \textit{Pourquoi ? Au delà du jeu d'esprit, quel est l'enjeu motivant une telle comparaison ?}

\Anotecontent{moto_hacking}{Une remarque qui renvoie d'ailleur explicitement à sa lecture du moto d'Hacking \foreignquote{english}{experiments have a life of their own} et à la notion d'autonomie (\textit{autonomous}) de sa synthèse précédemment, qui marque le fait que dans certains cas (impossibilité d'observation, manque de données), la simulation doit faire la preuve des connaissances (\textit{background knowledge}) apportés sur appel de ses propres ressources.}

\Anotecontent{experimental_warranting_belief}{\foreignquote{english}{The central idea of this thread is that experiments are the canonical entities that play a central role in warranting our belief in scientific hypotheses, and that therefore the degree to which we ought to think that simulations can also play a role in warranting such beliefs depends on the extent to which they can be identified as a kind of experiment} \autocite{Winsberg2009}}

Partant du fait que l'expérimentation joue un grand rôle dans l'établissement d'une crédibilité pour les hypothèses avancés, il s'agit de mesurer à quel point la simulation serait susceptible d'apporter les mêmes garanties dès lors qu'on accepte de la voir comme une sorte d'expérimentation.\Anote{experimental_warranting_belief}

On s'appuiera dans la suite de cette argumentation sur la lecture de Winsberg, un philosophe des sciences que l'on estime plutôt partisan de la III thèse dans la classification ci dessus. Ce dernier s'appuie largement sur les travaux d'Hacking, mais aussi Galison pour construire sa réflexion, par exemple en arguant\foreignquote{english}{ [...] that some of the techniques that simulationists use to construct their models get credentialed in much the same way that Hacking says that instruments and experimental procedures and methods do; the credentials develop over an extended period of time and become deeply tradition-bound.} \autocites{Winsberg2003, Winsberg2013} \Anote{moto_hacking}

Winsberg résume ce débat en deux thèses opposés : \foreignquote{english}{Identity Thesis} qui consiste à dire que la simulation est littéralement une expérimentation, et \foreignquote{english}{Epistemology Identity Thesis} qui consiste à penser qu'il existe une dépendance entre les garanties de crédibilité qui pourront être accordé par les résultats de la simulation et leur capacité à être plus ou moins définie en tant qu'expérience. Si la première thèse semble assez bien correspondre au point I de la classification de Varenne, la deuxième semble être une sous-variation du point I.

La plupart des auteurs cités par la suite dans ce débat sont des philosophes des sciences spécialisé en économie (Guala , Morgan, Maki, Simon ) qui rejettent comme Winsberg (plus spécialisé en physique) assez naturellement ces deux thèse \autocite{Winsberg2009}, mais avec des arguments assez différents, qu'il convient d'évoquer pour bien comprendre la complexité de ce débat, assez théorique. 

\Anotecontent{maki_phan}{\foreignquote{english}{For Mäki, abstractions in models are similar to abstractions in experiments as they both can be interpreted as a kind of isolation [...] This analogy between models and experiments is called \enquote{isolative analogy} by Guala (2008). From Mäki’s standpoint, a model can be said to be experimented in its explanatory dimension: the finality of such a model is to explore the explanatory power of some causal mechanism taken in isolation.} \autocite{Phan2008}}

parmi les différents point de vue existant, on citera par exemple le sous-débat de l'\foreignquote{english}{isolative analogy} relaté ici au travers des publications de \textcite{Phan2008, Phan2010} apellant les points de vue de Morgan et Guala contre Maki (2005). Ce dernier voit dans la similitudes entre isolement théorique du modèle comme expérience de pensée et isolement expérimental \Anote{maki_phan} la possibilité de rejoindre une des deux thèses évoqués par Winsberg, établissant d'une façon ou d'une autre que \textit{les modèles sont des expériences, et les expériences des modèles}. Mais ce type d'argument, et on le suppose tout ceux qui se rapportent à l'évocation d'analogies pour justifier d'une équivalence de puissance épistémique se heurterai, comme on va le voir, à une différence fondamentale.

\Anotecontent{guala_phan_winsberg}{Winsberg résume le point de vue de Guala(2002) ainsi \foreignquote{english}{Guala argues that simulation differ fundamentally from experiments in that the object of manipulation in an experiment bears a material similarity to the target of interest, but in a simulation, the similarity between object and target are merely formal.}, mais on peut trouver une version réactualisé en 2008 dans l'article de \textcite[4.2]{Phan2010} \foreignquote{english}{In a simulation, one reproduces the behavior of a certain entity or system by means of a mechanism and/or material that is radically different in kind from that of a simulated entity (...) In this sense, \enquote{models simulate} whereas \enquote{ experimental systems} do not. Theoretical models are conceptual entities, whereas experiments are made of the same \enquote{stuff} as the target entity they are exploring and aiming at understanding}\autocite[14]{Guala2008}}

\textcite{Phan2010} et \textcite{Winsberg2013} cite le point de vue de Guala (2002, 2008), partagé par Morgan(2002, 2005) et se référant aux travaux de Simon (1969). Ceux-ci s'appuient sur une différence de relation qui existe entre système à étudier et système cible dans chacun des deux cas. En effet, dans le cas des expérience, la comparaison s'appuie avant tout sur une similarité matérielle, alors que dans le cas de la simulation la comparaison est limité à une comparaison formelle entre les objets.\Anote{guala_phan_winsberg}

\Anotecontent{Winsberg_critique_morvan}{\foreignquote{english}{Interestingly, while Morgan accepts this argument against the identity thesis, she seems to hold to a version of the epistemological dependency thesis. She argues, in other words, that the difference between experiments and simulations identified by Guala implies that simulations are epistemologically inferior to real experiments - that they have intrinsically less power to warrant belief in hypotheses about the real world.} \autocite[841]{Winsberg2013}}

Morgan(2002, 2005) accepte le point de vue Guala et Simon, mais s'en sert pour réduire indirectement le pouvoir épistémique de la simulation. Un argument bien résumé par \textcite{Phan2008} \enquote{Pour Morgan (2005) modèles et expériences partagent des fonctions de médiateurs et peuvent fonctionner \textit{sur un mode expérimental}, mais les expériences \textit{réelles} offrent un \textit{pouvoir épistémique} d'investigation de la réalité empirique plus fort.} Ce qui fait dire à Winsberg que Morgan serait indirectement plutot partisan de sa deuxième thèse.\Anote{Winsberg_critique_morvan}
\Anotecontent{winsberg_mereformal}{\hl{A compléter avec ce que dit Winsberg2013}}

Pour \textcite{Winsberg2009} le flou des arguments avancé par Morgan et Guala  (\textit{material similarity}, \textit{mere formal similarity}) ne permet pas d'exclure complétement et définitivement la première thèse.\Anote{winsberg_mereformal} Celui-ci se range malgré tout du coté de Guala, et préfère là aussi rejetter cette thèse, mais à la faveur de sa propre argumentation; ce qui lui permet de rejetter à la fois l'argument Morgan pointant l'infériorité épistémique de la simulation, et la deuxième thèse. Il argue que les simulations et l'expérience diffère principalement par la nature du \textit{background knownledge}, c'est à dire protocoles et les connaissances mobilisés.

Des modélisateurs et épistémologues en sciences sociales beaucoup plus proche de nos pratique comme Phan et Varenne trouve un argument convaincant dans ce dernier point, car \foreignquote{english}{Aujourd'hui, comme le souligne Winsberg, la crédibilité des modèles de simulation repose largement sur la \textit{confiance} que nous pouvons avoir dans les compétences des modélisateurs, informaticiens, expérimentateurs et observateurs, ainsi que dans les composants ou plateformes qui supportent les expériences de simulation.} \textcite{Phan2008}

\Anotecontent{gilbert_critique}{\foreignquote{english}{\enquote{[t]he major difference is that while in an experiment, one is controlling the actual object of interest (for example, in a chemistry experiment, the chemicals under investigation), in a simulation one is experimenting with a model rather than the phenomenon itself.} \autocite[14]{Gilbert2005}. But this doesn't seem right. [...] It is false that real experiments always manipulate exactly their targets of interest. In fact, in both real experiments and simulations, there is a complex relationship between what is manipulated in the investigation on the one hand, and the real-world systems that are the targets of the investigation on the other. In cases of both experiment and simulation, therefore, it takes an argument of some substance to establish the ‘external validity’ of the investigation – to establish that what is learned about the system being manipulated is applicable to the system of interest. Mendel, for example, manipulated pea plants, but he was interested in learning about the phenomenon of heritability generally \autocite{Winsberg2013}}}

\Anotecontent{guala_morgan_reality_experiments}{\foreignquote{english}{The identity thesis itself has drawn criticism from Guala (2002) and Morgan(2002). Guala begins by dismissing what he takes to be a poor argument against it. The poor argument goes something like this : simulations are not at all like real experiments because real experiments manipulate the real-world systems that are the very target of the investigation, while simulation merely manipulate \enquote{models} of the target system. What both Guala and Morgan correclty point out is that it is, quite generally speaking, false.}}

Autre sous-débat évoqués par \textcite{Winsberg2013}, on suppose en partie en réponse à sur son article précédent et très similaire \autocite{Winsberg2009}, la critique de l'\textit{identity thesis} comme évoqué par Gilbert et Troitzsch (1999), dont il pense \Anote{gilbert_critique}, en accord avec Guala (2002) \autocite{Winsberg2009} mais également Morgan et Parker \autocite{Winsberg2013} qu'elle est un argument trop faible pour rejeter l'\textit{identity thesis}. \Anote{guala_morgan_reality_experiments} 

Si les arguments de Winsberg semblent convaincant, \textcites{Peschard2010b, Peschard2013} tente dans une analyse critique d'en montrer les biais, et apporte dans son article des objections tout à fait crédible issue de son domaine d'expertise. Pour ne citer qu'un de ces argument, si il existe bien un intermédiaire de mesure issue d'un modèle, comme l'indique Winsberg, il existe également un sous système en prise directe avec la réalité physique de ce monde. En conclusion, elle estime que si il y a bien une certaine forme de similarité entre cibles épistémiques de la simulation et de l'expérience, pour elle ces activités ne peuvent pas être épistémiquement équivalentes, ce qui n'empeche en rien selon elle la coopération fructeuse des deux approches. \hl{Ajouter une footnote avec explication}

\textcite{Winsberg2013} résume le point de vue de \autocite{Peschard2010} ainsi, \textit{Thus, simulation is distinct from experiment, according to her, in that its epistemic target (as opposed to merely its epistemic motivation) is distinct from the object being manipulated.} Autrement dit, même si la motivation menant à l'expérience est bien eloigné (la motivation), l'objet manipulé dans une expérience est bien celui du monde physique, alors que dans le cas de la simulation c'est l'ordinateur. Or autant la motivation peut apprendre de l'objet manipulé dans le monde physique, autant il n'est pas ici dans notre intérêt d'apprendre sur l'ordinateur en tant qu'objet. Dans ce cas là on pointe une différence, mais on peut également appeler selon \textcite{Winsberg2013} et Morrisson (2009) l'argument inverse pointant au contraire une similarité. L'objet expérimenté étant le plus souvent choisi en tenant compte justement de sa capacité de \textit{surrogate} rapport à la question que l'on se pose effectivement, un point commun entre la construction de simulation et d'expérimentation. 

Winsberg conclu en ajoutant que l'expérimentation, contrairement à ce que l'on pourrait penser, n'est pas forcément et immédiatement plus crédible si on ne lui ajoute pas un bagage de connaissance : \textit{Experiments are not automatically more reliable than simulations, despite their differences. [...] It would seem that there are identifiable differences between ordinary experiments and simulations, but there is nothing about these differences that makes one or the other intrinsically more epistemically powerful.}  \autocites{Winsberg2009, Winsberg2013}

\textcite{Varenne2001} avance alors un autre argument intéressant : \foreignquote{english}{Indeed, when you read (Von Neumann 1951), you see that analog models are inferior to digital models because of the accuracy control limitations in the first ones. Following this argument, if you consider a prototype, or a real experiment in natural sciences, is it anything else than an analog model of itself? The test on the prototype is a real experiment. But is it something different and better than the handling of an analog model? So the possibilities to make sophisticated and accurate measures on this model - i.e. to make sophisticated real experiment - rapidly are decreasing, while your knowledge is increasing. These considerations are troublesome because it sounds as if nature was not a good model of itself and had to be replaced and simulated to be properly questioned and tested! It looks as if it was not possible any more to end a paper on simulation by reassuringly using the traditional word: \enquote{Simulation will never replace real experiments”.} }

Ces derniers paragraphes montrent que le débat est loin d'être fixé, et il semblerait là encore que ce soit la définition du contexte d'application qui détermine le mieux la capacité explicative de la simulation, car comme le dit Winsberg \enquote{l'impossibilité d'expérimenter} existe dans bien des disciplines, comme les sciences sociales, mais également la biologie ou la physique, ou les tentatives de reconstitution simulé d'univers ou d'étoiles dans des super calculateur de plus en plus puissant montre qu'il existe un interet explicatif à cette pratique. On pensera notamment aux projets d'expérimentation récents extremement complexe et couteux en physique (laser megajoule de bordeaux, projet ITER pour la fusion).

Et c'est sur ce point que l'argumentation de la plupart des philosophes des sciences est tout à la fois aussi intéressant que problématique. Pour continuer sur Winsberg, celui ci traite de ces problématiques en se positionnant uniquement du point de vue des sciences physiques. Un fait dont il reconnait prudement les conséquences que peuvent avoir l'inclusion d'un contexte différent sur sa synthèse : \foreignquote{english}{Parker (forthcoming) has made the point that the usefulness of these conditions is somewhat compromised by the fact that it is overly focused on simulation in the physical sciences, and other disciplines where simulation is theory-driven and equation-based. This seems correct. In the social and behavioral sciences, and other disciplines where agent-based simulation (see 2.2) are more the norm, and where models are built in the absence of established and quantitative theories, EOCS probably ought to be characterized in other terms.

For instance, some social scientists who use agent-based simulation pursue a methodology in which social phenomena (for example an observed pattern like segregation) are explained, or accounted for, by generating similar looking phenomena in their simulations (Epstein and Axtell 1996; Epstein 1999). But this raises its own sorts of epistemological questions. What exactly has been
accomplished, what kind of knowledge has been acquired, when an observed
social phenomenon is more or less reproduced by an agent-based simulation?
Does this count as an explanation of the phenomenon? A possible explanation?
(see e.g., Grüne-Yanoff 2007).

It is also fair to say, as Parker does (forthcoming), that the conditions outlined above pay insufficient attention to the various and differing purposes for which simulations are used (as discussed in 2.4). [...] Indeed, it is also fair to say that much more work could be done in classifying the kinds of purposes to which computer simulations are put and the constraints those purposes place on the structure of their epistemology.}

Des philosophes des sciences ont donc saisi cette opportunité de critiquer l'approche de Winsberg pour soulever dans des tentatives de typologies parfois intéressantes \autocite{Eckhart2010} les points de divergences que soulève l'utilisation d'une philosophies des sciences naturelle inadapté à la simulation en sciences sociales. Malheureusement, au cours de ces mêmes lectures, on constate que cette critique se retourne vers les modélisateurs et praticiens des sciences sociales, et mène cette fois ci dans une analyse incomplète du contexte historique au mieux à des interprétations erronés (voir le débat animé entre \autocite{Yanoff2008}  \autocites{Elsenbroich2012, Chattoe2011}), et au pire à des approximations et  conseils de mise en oeuvre totalement déplacé \autocite{Eckhart2010} vis à vis de disciplines qui disposent comme on l'a vu d'une véritable histoire autour de l'usage des méthodes computationelles.

Car bien que la recherche des points communs et des différences entre réalité de l'expérimentation physique et virtuelle apparaisse comme un débat intéressant, il faut bien avouer que celui ci ne peut que difficilement s'adapter à la quasi absence d'expérimentation au sens classique dans les sciences sociales. Ainsi, même si la simulation partage certaines des propriétés de l'expérimentation classique, il y a quand même quelque chose de paradoxal à vouloir absolument analyser le rapport de la simulation à l'expérimentation alors même que c'est cette absence qui justement motive son utilisation dans notre discipline, hormis peut etre pour mettre plus en avant cette incapacité à formuler un unique cadre fédérateur par un tel débat. Comme le dit très justement \textcite{Phan2008} {[...] les sciences économiques et sociales sont plus volontiers concernées par l’opposition entre \enquote{le modèle et l’enquête}  (Gérard-Varet et Passeron, 1995) que par celle entre \enquote{l’expérience et le modèle} (Legay, 1997)}

La notion de modèle vue comme médiateur autonome entre théorie et modèle doit elle aussi être repensé pour les sciences humaine, et la géographie; car les théories si elles peuvent exceptionnelement servir à dériver des modèles, celle ci ne peuvent qu'être difficilement rapporté à leur équivalent en science physique \autocite{Pumain1997}.

D'un coté les sciences physiques semble encore viser l'etablissement d'un cadre fédérateur alors qu'il semble que les théories et les modèles en sciences sociales - hormis peut être le cas particulier de l'économie - soit au contraire pourvoyeur de richesse dans leur capacité à apporter un nouvel éclairage sur un phénomène observé. \hl{a préciser peut etre}

A cela il faut ajouter que le modèle en géographie opère dans un cadre épistémique particulier qui n'est pas forcément celui de toute les sciences humaines. Ainsi, bien que les notions et le rapport entre les notions d'observation du \textit{singulier} et du \textit{général} soient théoriquement à la portée de toute disciplines \autocite{Dastes1992}, il semblerait que la géographie trouve un intérét particulier pour la constitution de sa démarche explicative à articuler des éléments de connaissance pris dans les grandes familles explicatives historique, écologique, et spatiale; justifiant ainsi de niveaux d'explication plus ou moins en interaction mobilisant chacun des déterminants de nature différentes. Avec la possibilité d'intégrer à tout moment dans l'explication les résidus qui tiennent d'un dialogue entre méchanismes généraux et singularité historique, écologique ou spatiale. Car quelque soit le registre explicatif choisi il reste dans les deux cas \textit{ [...] une part d'explication relevant de ce que l'on peut qualifier de singularités locales, non prédictibles à partir de mécanismes généraux, mais nécessitant d'appréhender l'histoire spécifique du lieu}. La conséquence étant une diversité de modèles support de l'explanan (l’explication que l’on propose du phénomène auquel on s’intéresse) dont l'évolution sur la forme et le fond n'a eu de cesse d'éclairer l'explanandum sous un jour différent. \autocite{Dastes1992, Sanders2000, Sanders2013} \hl{Mais il y a aussi le multi-échelles.}

\hl{A mieux introduire ici} En 1986, lors d'un débat sur les apports de l'AI en géographie, avec les apports de la branche discrete de la simulation, Couclelis montre que les considération philosophiques abordés jusqu'ici sont en réalité très vite abordés par les géographes \foreignquote{english}{A further insight to emerge from discrete model theory, which has some interesting philo- sophical implications, deserves a few comments. The widespread belief that proposing a model corresponds to the assertion that the real phenomena must have some similar structure is contradicted by the sharp distinction between structure and behavior drawn in discrete model theory. Although structure governs behavior, the converse is not true, so that obtaining a model that reproduces some behavior well does not entitle one to make any inferences about the \enquote{real} structure of the phenomenon represented. In fact, it is doubtful whether we may talk about the structure of real phenomena in other than a metaphorical sense. A real system may be no more than the universe of potentially acquirable data.} \autocite{Couclelis1986}

L'éclairage sur la méthodologie sous-jacente à la construction des modèles, pourtant un élément au coeur du raisonnement dans la discipline géographique depuis la révolution quantitative, a encore moins de chance d'être évoqué dans ces publications philosophiques, au détriment d'une réflexion statique plus axé sur la nature de l'objet simulation, et de sa relation au monde.

Or l'évolution des réflexions touchant l'activité de modélisation se construit il me semble à une échelle de reflexion tout à fait différente, celle contextualisé de pratiques guidés par une activité de résolution de questions spécifique à l'analyse spatiale, dont la mise en oeuvre s'appuie sur une chaine de traitements flexibles utilisant à bon escient et de façon cumulative l'arrivée historique de nouveaux outils et avec eux leur capacité à renouveller les questionnements : les statistiques, les modèles, les simulation.

\Anotecontent{ce qui n'est pas sans nous rapeller les difficultés évoqués dans le chapitre 1 sur l'inadéquation et le danger que représente les modes de transmissions actuels.}

\Anotecontent{remarque_Varenne_2001}{\foreignquote{english}{The second thesis of this article is that none of the three categories of arguments could be applied to contemporary sciences in general, whatever their objects, their methods and the moment of their history we consider. None of these three categories could be considered as the only true one. We cannot have a general point of view on the value of computer simulations, because of the different implications and meanings of mathematics in the different fields of science, and because of the various philosophies of nature at stake. This fact remains true for a given field throughout its own history, because the role of mathematics and the definition of the studied object evolve: You cannot find a unique and stable value that would be given to its simulation uses once for all. Again and hopefully, this thesis illustrates the fact that it does not belong to the historian to decide on the value of computer simulation in a given field but to the scientists themselves. These preliminary reflections prove the importance to investigate the intellectual history of contemporary sciences and not only their sociological construction nor their philosophical general insights.}\autocite{Varenne2001}}

Ces rapides remarques nous éclaire sur la latence qui existe entre la réflexion récentes d'épistémologues comme Winsberg, Grüne-Yanoff et la réalité théorique et pratique en géographie. \textcite{Varenne2001} avait déjà bien cerné dans la synthèse faite en 2001 qu'il n'y avait pas dans sa classification une position meilleure ou plus convaincante qu'une autre, la réponse se trouvant comme pour la notion de modèle avant tout dans l'étude du contexte, et donc de l'histoire des disciplines face à cet objet simulation. \Anote{remarque_Varenne_2001} Cela ne veut pas dire que les débats évoqués précédemment en sont automatiquement invalidés, seulement qu'il faut être probablement plus regardant vis à vis des remarques générales et des conclusions beaucoup trop hative qui peuvent parfois en découler. 

Les travaux croisés de praticiens (Pumain, Sanders, Banos), d'épistémologues ou historien des sciences propre à la géographie (Orain, Besse, Robic, Cuyala) ou s'en approchant (Varenne, Phan) permettent d'une part d'apposer un premier filtre sur ces réflexions génériques pour s'y référer prudemment, et d'autre part d'innover en questionnant nos démarches dans ce qu'elles ont d'originales, cette fois ci appuyé sur une lecture des pratiques certe pas toujours parfaite mais pouvant au moins être qualifié de \textit{bottom-up} 

%\hl{Un travail conséquent à la croisée de différentes approches, les travaux d'historiens et épistémologues des sciences somme Orain, Besse, Cuyala, et la lecture plus spécifique de l'évolution des méthodes numériques puis computationelles et de leur apports d'un point de vue pratique et théorique dont on trouve source à la fois dans les nombreux travaux des praticiens, mais également dans des travaux de plus long cours comme celui qu'est en train de réaliser Varenne dans son HDR.  == REDITE}

%\hl{ont su voir rapidement l'intérét de développer plus en avant les spécificités attachés à la simulation en science sociale, déjà riche de réflexion sur les apports successifs et cumulatifs de techniques de simulations \autocites{Banos2013, Varenne2008}, en s'intégrant au débat d'une communauté inter-disciplinaire structuré autour de la modélisation agent, qui émerge dans les sciences sociales au début des années 1990. Ce débat par contre ne fait semble-t-il que commencer dans le courant plus \textit{mainstream} des philosophe des sciences.}

%tel que celle des géographes pratiquant la simulation depuis les années 1950, tels que celle qui a émergé autour de la modélisation agent dans les années 1990. 

%Ainsi comme on a pu le voir dans le chapitre 1, le terme laboratoire virtuel pour l'expérimentation apparait très tot dans les sciences sociales, et des auteurs ont pour l'époque déjà donnés de très bonne raisons pour l'emploi de ce terme; les aspects dynamiques de la simulation en faisait partie.

\subsubsection{La Validation vue par une communauté de modélisateurs}
\label{sssec:communautes_jasss}

Depuis le début des années 1990 et la diffusion progressive du méta-formalisme agent \autocite{Treuil2008} dans les sciences humaines et sociales, les modélisateurs géographes peuvent, en plus des pratiques internes à la géographie, se tourner vers les discussions opérés dans une communauté d'acteurs internationaux et inter-disciplinaire. On trouvera sur ce sujet une tentative d'exploration des fondement historiques de ce mouvement dans l'annexe \hl{ref annexe}. Sorti des ouvrages fondateurs, c'est principalement autour du \textit{Journal of Artificial Societies and Social Simulation} (JASSS) fondé en 1998 que gravitent la plupart des discutants pertinents sur la problématique de la Validation. 

\paragraph{Persistance des guides méthodologiques et des évocations du problème de la validation}

Cette étape de validation, l'écueil le plus important surement, est pourtant souvent évoqué comme une étape cruciale dans bon nombres de guides méthodologiques pour \enquote{la bonne construction des modèles}, qu'il soit ancien \autocite[195]{Beshers1965} \autocite{Naylor1966, Naylor1967}, récent \autocite{Amblard2006, Gilbert2008}.

En prenant les écrits de \textcite[301]{Doran1975}, qui fait quasiment figure de co-créateur du mouvement de part son introduction des DAI à Gilbert \autocite{Gilbert1985} (voir également l'annexe déjà cité), on retrouve une filiation directe entre ces écrits de 1975 et 2000 sur cette question.

Ainsi on peut dire que \textcite[300-301]{Doran1975} dans \textit{Mathematical Models and Computer Simulations} est déjà au courant des travaux sur la question dans les sciences sociales, car il cite \autocite{Guetzkow1972}, et reprend à son compte un protocole de construction de modèle dont la description n'a finalement que peu bougé ces dernières années \foreignquote{english}{Any serious simulation study involves major effort at a number of stages : advanced planning; collection and organisation of suitable data; detailed specification of simulation; writing and initial \enquote{debugging} of the computer program; preliminary testing and validation of the program; it's use in the sequence of experiments designed to achieved specified objectives; and finally the study and interpretation of the results obtained. It is easy to underestimate the magnitude of the total effort required.It is all the more important to have a clear idea of what the simulation study is intended to achieve; either a broad investigation of the behaviour of simuland or, more likely, a determined attempt to examine the behaviour of certain variables of interest (cost? death rate? output? public approval?) and to discover to what extent they can be controlled.} 

Si la validation est décrite dans des termes classiques comme une comparaison avec les données, \textcite[301]{Doran1975} soulève par la suite quelles difficultés de l'expérimentation pouvant en définitive grever cette précédente tâche. Car comment valider corectement une simulation compte tenu de tout ces problèmes posés par l'expérimentation ? \foreignquote{english}{In any simulation, \textit{validation} is a matter of great importance. How can it be ensured that the model is indeed a reliable guide to reality ? [...] Once simulation has been validated it can be put to useful work. At this point a major problem appears. [...] any stochastic simulation must be run many times, and effectively one is sampling the behaviour of the variable of interest. This makes for much book keeping and for many complications.[...] Simulations poses two unexpected experimental problems. First, the number of variables and parameters in the simulation is liable to be very large, [...] Second, it may prove disconcertingly difficult to comprehend what is going on within the simulation, just as it is often difficult to comprehend a complex part of the real outside world. [...] These problems are of more than purely technical interest. They arise from the use of tool of sufficient complexity that it's details can extend human comprehension to the limit.}

En 2000, même si les termes se sont raffinés, et que de nouveau problèmes semblent avoir fait leur apparitions du fait des spécificités de l'outil (aspect cognitif par exemple), le lecteur n'est pas dépaysé sur le fond par la description que fait \textcite{Doran2000} des \foreignquote{english}{Hard problems in the use of agent-based modelling} : \foreignquote{english}{Skills and Time requierement, Which type of Model? , What level of Abstraction ?, Searching a massive parameter space ? The problem of validation ?}

Pour Openshaw, 

, une preuve qui vient s'ajouter à celle déjà évoqué dans le chapitre 1 pour justifier de l'enracinnement de cette problématique dans l'histoire de la simulation.


Si on observe bien une constance dans le développement des guides méthodologiques pour la construction de modèles de simulation \footnote{Rajouter comparaison entre deux suites de points}, l'étape ayant attrait à la validation ne reste que très rarement développé ou mis en oeuvre dans des exemples concrets \Anote{grimqualite}. Un exemple encore récent est la publication dans la revue JASSS (dont on rapelle qu'elle a été initié par Gilbert) à la critique cinglante de \autocite{Manzo2007a} vis à vis du guide méthologique établit récemment par \autocite{Gilbert2008}, très concis sur ces questions (7 pages au format A5). Or, si même les auteurs aussi cités et reconnus que Gilbert, qui ont la chance de publier pour la première fois cette méthode dans une collection aussi reconnu (\textit{Sage Quantitative Applications in the Social Sciences}), ne donne pas l'exemple, que doit on attendre des plus jeunes recrues s'essayant à cette technique ?

\hl{Transition}

Toutefois là ou naïvement, avec les évolutions de l'informatique, on aurait pu s'attendre à voir émerger dans les publications des applications concretes permettant d'aborder, même de façon incomplète, cette problématique, ce n'est pourtant pas du tout ce que l'on constate de façon générale.

Dans l'étude mené par \textcite{Heath2009} entre 1998 et 2008 sur 279 publications, l'auteur considère que seul 35 \% des modèles sont validés conceptuellement et informatiquement, même si une amélioration est à noter entre 2005 et 2008, ou ce chiffre monte à 43\% environ. Toutefois, si dans l'ensemble des modèles agents récupérés, on ne garde que les modèles agents classés en science sociale, ce chiffre chute fortement, avec 28\% de modèle validés, et quasiment 37\% de modèle qui n'aborde même pas ce problème \Anote{survey_heath}.

Si on regarde plus du coté des techniques, comme par exemple les analyse de sensibilités, cité régulièrement pour leur utilité dans la \enquote{validation interne} des modèles \autocite{Amblard2006}, le résultat n'est guère plus encourageant. Du moins si on en croit l'étude de \textcite{Thiele2014} entre 2009-2010 \Anote{survey_thiele}, mais également celle plus restreinte de \textcite{Cottineau2015} sur le volume JASSS de Mars 2014 \Anote{survey_cottineau}.

Finalement, sans même faire intervenir les critiques issue de débats plus épistémologiques (comme ceux que l'on a vu précédemment), cette seule insuffisance dans l'utilisation des moyens existants pour évaluer nos modèles suffit largement à prolonger un cercle vicieux où l'absence d'évaluation nourrit une perpétuelle remise en question de cet outil et de sa scientificité \Anote{serpent_mer}. Cette seule connaissances des dynamiques à l'oeuvre dans les modèles ne fait pas tout, mais elle offre déjà une base de discussion marqué par l'honneteté de cette démarche.

A l'image de certain auteurs, on pourra reprocher l'absence d'un protocole plus standard encadrant cette problématique. 

Cette discussion questionne aussi l'absence d'un protocole, d'un standard pour l'évaluation des modèles. %est cité comme un des nombreux écueils avec lequel se bat toujours la discipline, comme en témoigne les discussions réccurentes de nombreux auteurs sur un sujet dont la complexité touche à une dimension technique, que méthodologique ou philosophique. 

Autant d'auteurs \autocite{Richiardi2006} \autocite[198]{Fagiolo2007} \autocite{Moss2008} \autocite{Windrum2007} \autocite{Barlas1996} \autocite{Amblard2003} \autocite{OSullivan2004} \autocite{Doran2000} \autocite{Crooks2012} \autocite{Rouchier2013} dont il faudra par la suite développer les discussions.  


validation des modèles de simulation agents, on se rend compte qu'un certain nombre de problématiques persistent et limitent toujours la diffusion des modèles en dehors du cercle bienveillant de l'inter-disciplinarité \autocite{Richiardi2006}. Ce problème, loin d'être un isolat touchant uniquement les sciences humaines et sociales, existe également dans d'autres disciplines, comme en écologie, où même lorsque les modèles sont publiés, l'absence de protocole pour répliquer, évaluer le modèle est courant \autocite{Grimm1999}. \hl {ref à vérifier}

Un problème qui met un peu plus en danger des communautés de chercheurs dont on sais déjà leur isolement dans certain pays : archéologie, sociologie \autocite{Manzo2007}, et économie \autocites{Lehtinen2007, Richiardi2006} pour n'en citer que quelqu'un. 

cette critique récurrente de l'outil sur le plan de la scientificité, une faiblesse qui constitue toujours un danger pour la pérénité des pratiques inter-disciplinaire autour de la simulation agents \autocite[220]{Squazzoni2010}, a tel point que la communauté se dote de guides de survie pour se protéger des sceptiques. \autocite{Waldherr2013} Est ce là le seul fait d'une mauvaise communication autour de notre discipline comme le laisserait penser la lecture de ces travaux ? 


A : Les approches appliqués les plus sérieuse nous paraissent celle de Grimm, et celle de Behavior search, elles seront commenté en conclusion.

= Article de référence est clairement Amblard 2006

L'article de référence pour la validation de modèle agent en géographie est clairement celui d'Amblard2006, dans le livre dédié à l'agent. Celui-ci ne comportement pas malheureusement de volet applicatif.

= Les géographes interviennent dans cette communautée, exprime forme de détachement

A vouloir mettre en évidence un protocole de validation générique axé autour du seul modèle de simulation, on applanit sans le vouloir une démarche de construction des connaissances mobilisant un ensemble de modèle et de méthode. Il est important de rapeller sans cesse le poids de cet héritage disciplinaire à l'interface de questionnement plus génériques, comme le font régulièrement dans notre discipline Hélène Mathian, Léna Sanders, et plus récemment Cottineau2014b.

Les géographes modélisateurs, même si ils interviennent également au contact de cette communauté, continuent pour certains à développer une approche ou le SMA n'est qu'un outil parmis d'autres hérités d'une longue histoire de la géographie avec la simulation. La question de la validation s'inscrit dans une démarche beaucoup plus globale ou interviennent des multiplicité de modèles et de méthodes.

= Geographes \textbf{centré sur la démarche} et non sur l'outil

Pour raccrocher cette remarque avec les dernières analyses portant sur l'usage des techniques de simulation en géographie, les observateurs (Varenne) tout autant que les acteurs \autocite{Sanders2013}\Anote{sanders_couplage_spirale} de ces pratiques tendent à mettre en avant une tendance croissante à la pluri-formalisation croissante des modèles agents, preuve que cette multiplicité des formalismes et des techniques est de plus en plus considérée comme une richesse dans l'approche de systèmes complexes.

Dans son analyse sur la place de l'explication en analyse spatiale, \textcite{Sanders2000} voit plus dans le rapport de l'outil à la démarche adoptée (exploratoire ou hypothético-deductif) une question d'interprétation, ce qui contextualise encore un peu plus la place de l'outil \enquote{Ce sont en effet la manière dont l'outil est inséré dans une chaîne de réflexion et de traitement et l'interprétation des informations figurant en entrée et en sortie qui permettent de donner le statut descriptif ou explicatif d'une démarche.} Avant de conclure un peu plus loin à la fin de son analyse \enquote{[...]il n'y a pas de relation simple et fixe, ni entre les niveaux d'observation et la nature des explications, ni entre les outils de l'analyse spatiale et l'explication. La variété des approches permet de diversifier les éclairages sur les phénomènes que l'on cherche à expliquer. Nos démarches en analyse spatiale consistent ainsi davantage à \enquote{éclairer} qu'à \enquote{démontrer} et identifier des jeux de causalités bien stricts.}

Pour Maryvonne le Berre encore, \enquote{Il n’y a donc pas d’adaptation de la géographie à la technique mais recherche d’une technique adaptée à chaque objet d’étude géographique}. \textcite{Sanders2013} expose une idée assez similaire de pré-existance des concepts sur l'outils, car pour elle \enquote{La comparaison entre famille de modèles montre qu'il existe des décalages temporels entre la construction conceptuelle d'un champ et la mise au point d'outils appropriés pour la tester. Dans une certaine mesure on peut avancer que les outils \enquote{ont rattrapé} les concepts dans ce champ de recherche.}

= Reste qu'il faut quand meme un support à ce developpement


\subsubsection{synthèse}

% Faire la synthèse des trois approches, notamment en redescendant la partie déjà ecrite à la fin de la partie philo, c'est mieux de noter la spécificité de la géographie pour justifier le fait d'une introspection dans les habitudes de construction en géographie.

Si la communauté \textit{Models\&Simulations} propose aujourd'hui un cadre d'analyse cohérent avec la dynamique attendue chez les géographes pour la construction des modèles, il lui manque toutefois une incarnation géographique qu'il va falloir extraire de nos propres exigences de construction.

Pour comprendre comment la notion de validation se construit en marge de ces deux discours, il faut revenir sur ce qui fait sens dans l'explication pour les géographes. En repartant des transformations que subit la géographie quantitative dans les années 1970 au contact du paradigme systémique, prise dans une nouvelle réflexion des objets géographiques , dont la percolation chez les géographes s'observe dans la nouveauté le champs lexical, les méthodes, mais également les techniques.

Au delà des outils il y a un fond commun, à la construction de modèle en géographie, et qui n'est que très rarement traité dans ces lectures, celui de la mise en œuvre de la construction. Nous verrons qu'il y a des raisons à cela, la dépendance au contexte en est une, et cette activité hautement flexible, bien qu'influencé par la qualité du substrat technique, pose dans la mobilisation des hypothèses des questions génériques à ce dernier : l'hypothèse que j'ai choisi apporte elle un éclairage sur la question posé qui guide la construction du modèle ?

- Oui, et je peux le prouver 

- Non, mais cela ne prouve pas que cette hypothèse n'est en soit pas valable 

Il ya une forme de permanence dans les questions posés par la construction d'un modèle ou d'un modèle de simulation qui apelle à la construction d'une vision de la validation plus appliqué en géographie => ce qui veut dire qu'il lui faudra un support, et cela viendra par la suite

\hl{--------------- Pas fini, tu peux sauter à la partie d'après :) ------}




\subsection{De la validation à la construction des modèles de simulation par l'évaluation}
\label{ssec:evaluation_construction}

% Permanence des questions évoqués pour la construction d'un modèle de simulation, plus complexification de la validation liés à la pluriformalisation.

En montrant que la validation est dépendante au contexte, Hermann a permis de lever un certain nombre de questions remarquables par leur actualité dans le cadre de nos propre problématique de construction.  %La mise en avant d'une possibilité de validation dépendante à l'objectif nous oblige inévitablement à prendre en compte l'activité de construction comme activité validante.

\subsubsection{Des modalités de validation dépendante au contexte, l'apport d'Hermann à une première formalisation du problème}

\paragraph{Une vision de la validation différente chez les pionniers du mouvement S\&G}

Charles F. Hermann opère dans la branche des simulations appelées à l'époque par Shubik les \textit{Man-Machine Games} \autocite{Shubik1972}. Une catégorie de simulation qui intègre dans son exécution un couplage entre un ou plusieurs systèmes numériques et des humains, qui peuvent être amenés à interagir entre eux, ou avec les machines. Ce type de simulation de structure hétérogène est intéressante dans le sens où elle permet d'intégrer l'arbitraire humain dans une chaîne d'interaction complexe qui n'aurait pas pu être établie autrement, du fait de l'impossibilité de programmer des interactions et des réactions humaines face à des situations précises. Même si ce type de techniques est motivé par une multitude d'usages, ce n'est pas par hasard si elle se développe particulièrement au cours de la guerre froide aux Etats-unis, toujours sous la direction d'institutions militaires. Ce genre de techniques permettant par exemple de simuler et de reproduire des guerres au travers d'inter-relations diplomatiques et/ou économiques \autocite{Hermann1967b}, avec la possibilité de mesurer via des indicateurs adaptés l'importance et l'impact de différents scénarii sur le couple humain/machine.

Ce type de simulation est particulièrement représenté dans des publications qui traitent de la simulation au sens large, comme par exemple le journal \textit{Simulation and Gaming} ou \textit{S\&G} \autocite{Crookall2011}, dont l'activité remonte au début des années 1970. On retrouve parmi les auteurs ayant participé au développement de la discipline des personnalités importantes comme Guetzkow, Shubik, Coleman, etc. \autocite{Crookall2012}. Aujourd'hui, le terme à évolué vers ce que l'on pourrait probablement appeler des jeux sérieux, l'utilisation de l'ordinateur n'étant plus forcément un élément obligatoire dans ce type de simulation. Du côté des objectifs qui sont aujourd'hui susceptibles de motiver l'utilisation de ces techniques, \textcite{Shubik2009} définit une taxonomie en 6 objectifs : \textit{teaching, experimentation, entertainment, therapy and diagnosis, operations, training }

Cette présence d'une dimension humaine dans les simulations introduit une complexité qui touche forcément à plusieurs objets d'études des sciences humaines (psychologie, sociologie, etc.), et il n'est donc pas étonnant que l'on retrouve ce type de publication dès l'apparition des premiers ouvrages inter-disciplinaires sur la simulation, quand elle ne les pilote pas; Harold Guetzkow par exemple est un des personnages importants qui gravitent autour de Herbert Simon au GSIA (Graduate School of Industrial Administration) de Carnegie Tech dans les années 1950-56 \autocite{Guetzkow2004}, et qui a beaucoup oeuvré pour le développement de la simulation dans ces sciences politiques et psychologiques (\textit{Inter-Nation simulation laboratory}) \autocite{Janda2011, Druckman2010}. Celui ci s'inscrit exactement dans la même branche que Hermann, et apparaît deux fois comme premier éditeur dans des recueils de textes pluri-disciplinaires traitant de la simulation au sens large, preuve aussi de son implication dans le développement et la diffusion de ces techniques au delà de sa propre discipline \autocite{Guetzkow1962, Guetzkow1972}

\paragraph{L'apport du contexte dans l'évolution du sens attaché à l'activité de simulation}

Ce qui est intéressant dans ce type de simulations, c'est qu'elles forcent à penser la validation des modèles sous un angle qui doit nécessairement tenir compte de la variabilité inhérente aux comportements humains, par essence difficilement évaluables et réplicables. C'est de cette contrainte, et parce que \textcite{Hermann1967} s'intéresse aux modèles de simulation pour d'autres objectifs que la prédiction (\textit{teaching, training, theory-building}), que celui-ci développe à mon sens une vision de la validation beaucoup plus réaliste pour les sciences sociales que celle proposée à la même période par Naylor.

\foreignquote{english}{First, the validity of an operating system is affected by the purpose or use for which the game or simulation is constructed [...]}\autocite[217]{Hermann1967}

% Plus d'information à ajouter, soit sur la dite boucle (sachant que le conceptual correspond quand meme pas mal à ce que lon fait, voir Sargent2010), Si la boucle définit par les tenants de la \textit{V\&V} n'est pas inintéressante, et de façon générale résume bien le cycle de vie qui correspond à la construction d'une simulation, de nombreuses questions reste en suspens sur le choix et la mise en œuvre des techniques telles qu'elles sont décrites. La construction et la mise en oeuvre des critères en fait partie. Les objectifs sont cités dans la définitions mais on ne rentre pourtant pas dans le détail de la relation entre ces objectifs et la construction du modèle, qui est laissé à l'expertise de l'utilisateur, en cela Hermann ne propose pas mieux dans sa description d'une boucle modélisatrice que les dernières avancées portés par Sargent2010, toutefois sa réflexion est par son orientation, et par sa précocité de réflexion son intéressante il me semble à citer. les moyens technique de la mise en oeuvre par exemple ? 

%Dans l'explication sociologique, la réalité structurelle n'est pas forcément d'intérét pour la construction du modèle. (bulle)

%Cette observation amène Hermann à considérer que la validation des composantes de la structure mérite une attention tout aussi importante que la seule comparaison avec des données de sorties, notamment dans un cadre explicatif.  curl -k -o ~/backups/pinboard-backups/pinboard-$(date +\%y\%m\%d).json 'https://api.pinboard.in/v1/posts/all?&auth_token=username:APItokenhere&format=json'

En s'appuyant sur ce premier argument évoquant l'existence d'une dépendance liant processus de validation et objectif poursuivi par le modélisateur, Hermann semble \textit{de facto} mettre en défaut une définition de la simulation ayant comme première et unique vocation de représenter au mieux le système observé. Les modalités de la validation étant maintenant définies par rapport au contexte, la possibilité d'un critère unique pour juger de la validation de façon universelle paraît tout à fait improbable. Afin de montrer qu'il ne s'agit pas seulement d'une question de disponibilités des données, et pour amener par la suite sa proposition de méthode multi-critères, Hermann s'attaque donc en premier lieu à réduire la portée des confirmations apportées sur un système observé par l'emploi de la seule technique de validation basée sur la comparaison de données en sortie des modèles de simulation.

Pour montrer qu'il existe des limitations dans la confiance que l'on peut mettre dans la validation lorsqu'il s'agit de comparer des données historiques (dans le cas des simulations de reproduction de guerre, on parle ici plutôt de reproduire des séries d'événements historiques) -cela même si elles sont idéalement toute rendues disponible- aux données en sorties de simulation, \textcite{Hermann1967b} s'appuient sur les travaux de \textcite{Pool1965}.

\foreignquote{english}{This correspondence does not demonstrate that the simulation correctly represents the structure and processes that were operative in the historical occurence. We are speculating on the similarity between the historical and simulated inputs on the basis of the similarity of their outputs. Different relationships among various combination of properties in the simulation conceivably could produce outcomes like those in the historical situation.

A simulation of the 1960 national Presidential election predicted the percentage of the vote for each candidate - the outcome - with considerable success. The designers of that simulation observe, however, that \enquote{it may legitimaly be asked what in the equations accounted for this success, and whether there were parts of the equations in the simulation that contributed nothing or even did harm} Further analysis of the equations in the simulation revealed that the outcome was predicted despite the fact that at least one equation misrepresented aspects of voter turnout. Part of the structure was incorrect, but the simulated result still matched the actual outcome. Despite this difficulty, our confidence that the simulation has captured some aspects of the voting process is greater than it would have been if the simulation had failed to replicate the campaign outcome. Confidence in the simulation would increase further as the operating model demonstrated ability to produce outcomes that corresponded with various elections. In sum, the similarity between simulation and historical events can provide at best only indirect and partial evidence for the correctness of the simulated structures and processes that produced the outcome.}



Ce que nous dit Hermann ici, à la différence de Naylor, c'est que même dans le cas idéal ou toutes les données serait présente, ce mode classique de validation ne peut pas être suffisant, cela quelque soit l'objectif poursuivi par le modélisateur. Un constat que nous avions déjà acquis à la lecture des déboires des géographes avec les préceptes de validation néo-positivistes, associant dans une démarche de modélisation instrumentaliste prédiction et explication (section \ref{sssec:realite_neopositiviste}).

Ce constat reste encore valide aujourd'hui, car comme le rappelle très justement \textcite[32]{Bulle2005}, \enquote{ les problèmes posés en sciences humaines visent cependant, en général, la compréhension des phénomènes. Dans cette optique, l’objet premier de la modélisation n’est pas de faire \enquote{coïncider} les modèles construits avec la réalité qui est celle des effets. Le test par la prévision ne peut assurer des qualités explicatives des modèles.}

Un point de vue partagé par \textcite[106]{Amblard2006}, pour qui \enquote{[...] la recherche de similitudes avec les données, si elle peut être utile, ne peut absolument pas être un critère unique et définitif de validation}

Suivant ces conseil, si Forrester avait appliqué lors de la construction de son modèle \textit{Urban Dynamics} des analyses de sensibilités (voir le type de critère \textit{variable-parameter testing} de \autocite{Hermann1967}) tel que le propose Hermann, il aurait probablement conclu, comme ont pu faire ces détracteurs par la suite, à l'inutilité d'une bonne partie des hypothèses intégrés dans son modèle, qui s'avèrent en réalité très peu influente sur la dynamique observé en sortie des simulations.

Autre point important, l'existence de multiples objectifs de modélisation permet à Hermann certe de révéler la diversité et l'attachement de la validation à un contexte, mais surtout de noter d'une part comment la variation de ce dernier affecte les modalités de cette comparaison entre système simulé et système observé, et d'autre part comment cela affecte la perception du résultat engendré par cette comparaison.

\foreignquote{english}{The first comment is that the validation of an operating model cannot be separated from the purpose for which it is designed and used. [...] The second observation somewhat mediates the first. For the most part the various purposes for conducting games and simulations do not negate the need for criteria we can use to estimate the degree of fidelity with which one system (the operating model) reproduces aspects of another (the reference system). Given some purposes for using games and simulations (such as exploring nonexistent universes), finding appropriate criteria in the referent system is quite difficult. With other objectives, the value of the operating model may remain even if the fit between the model and various criteria representing the observable universe is poor (as in theory building).} \autocite[219]{Hermann1967}

Indirectement, on observe ici le transfert d'une définition de la simulation comme simple \enquote{type de modèle} vers la définition plus générale d'une simulation \enquote{ caractérisée non pas tant par l’unité d’une fonction cognitive qu’elle assurerait toujours sous une forme ou sous une autre que par son fonctionnement interne, fonctionnement qui, bien sûr, mais seulement secondairement, se trouve avoir aussi des conséquences sur sa ou ses fonctions cognitives. Une simulation nous paraît ainsi devoir être prioritairement caractérisée par ce qu’elle est – ou fait – de manière interne plutôt que par ce qu’elle fait au sens d’une fonction cognitive quelconque qu’elle assurerait toujours et qu’on en attendrait prioritairement de l’extérieur : à ce titre, nous proposons de dire qu’\textit{elle est avant tout un traitement spécifique sur des symboles et qui prend toujours la forme d'au moins deux phases distinctes. 1) une phase opératoire [...] 2) une phase d'observation [...]}} \autocite[33-34]{Varenne2013}

\paragraph{La nécessité de repenser la représentativité des modèles}

La V\&V a toujours mis en avant le fait que la modélisation soit un processus incrémental tout à fait nécessaire pour obtenir un modèle de simulation satisfaisant, que cela soit dans les analyses de Naylor, ou d'Hermann. Ce dernier se réfère dès 1967 au principe de parcimonie, une méthode qui implique une abstraction, une simplification du système à représenter, et qui pour lui met logiquement et automatiquement en péril la représentativité. \Anote{Herman_parcimonie} 

%Une parcimonie hérité du principe d'Ockham dont on sait qu'elle n'est en aucun cas un synonyme de simplicité dans sa mise en oeuvre, celle-ci nécessitant au contraire un effort intellectuel important pour déterminer quelles sont les hypothèses réellement représentatives du problèmes à analyser. %Sur le plan de complexité, Poincarré ou le prix nobel d'économie Herbert Simon à fait état plusieurs fois des capacités d'expression du complexe rendu possible par l'usage de la simulation, et cela même avec des modèles simples.\autocite{Banos2013a}

%Une description de la construction des modèles qui coincide avec ce qui a été dit auparavant sur l'importance de la nature de l'objectif poursuivie sur la perception de cette \enquote{représentativité}, et le fait que cette dernière ne fasse pas systématiquement la valeur du modèle - tant soit peu qu'on arrive à fixer une valeur - 

Dans ce que l'on comprend de l'analyse d'Hermann, la perte de représentativité attendue d'un modèle de simulation qui n'est plus strictement dirigé vers la prédiction est compensée par un gain relatif à l'objectif poursuivi qui change la nature de la validation attendue : détection d'alternatives à un comportement, mise en avant de processus simplifiés pour l'éducation, construction de théorie, etc.

Il est donc logique de voir Hermann proposer dans la suite de son analyse de repenser la notion de représentativité et la notion de validation au regard de l'objectif poursuivi par le modélisateur. Il en résulte la généralisation de cette activité de validation dont le résultat se dessine à présent sous le couvert d'un objectif et dans le jeu d'une confrontation entre deux représentations, deux construits prenant pour cible le système modélisé et le système observé. 

\foreignquote{english}{A simulation or game is the partial representation of some independent system. Usually we are interested in simulation as a means for increasing our understanding of the system it is intended to copy. Therefore, the representativeness of a simulation or game becomes extremely important in assessing its value. The process of determining how well one system replicates properties of some other system is called validation.[...] In the present analysis however, validation will be defined more broadly as any comparison between the representation of a system and specified criteria.} \autocite[216]{Hermann1967}

\subsubsection{Le problème de la validation ramené à une confrontation des représentations entre système modélisé et système observé}
\label{ssec:confrontation_sysmodelise_sysobserve}

%\hl{repetition ?}
%La question de la représentativité d'une simulation est un sujet délicat à traiter car sa valeur se dessine à l'intersection d'au moins deux activités, la construction d'un modèle opérationel et la construction d'une grille d'évaluation, deux activités dont on s'apercoit par la suite qu'elles sont en réalité étroitement liées. 

\paragraph{Quelles hypothèses pour quelle représentativité ?}
\label{p:hypothese_representativite}

Si cette \enquote{representativité} ne semble plus intervenir dans la valeur du modèle que sous une forme beaucoup plus partielle, quelle est la part de représentativité acceptable que l'on peut attendre pour qu'une hypothèse soit considérée comme explicative ? Autrement dit quelles sont les modalités qui guident l'introduction maitrisée d'une part d'empirie dans un modèle, par l'existence d'un seuil caractérisant le potentiel de représentativité à atteindre pour chaque hypothèse ? Pour l'ensemble du modèle ? 

\Anotecontent{naylor_etonnement}{On pourra peut être être étonné de retrouver la démarche de Naylor dans les approches subjectives sachant la description qu'on en a fait au préalable. Mais il y a bien une part de subjectivité dans cette démarche, l'application de chacune des étapes de la multi-stage validation faisant quand même appel à une forme d'expertise pour constituer le jeu des hypothèses que l'on estime valable en vue du test final de comparaison aux données.}

L'acceptation d'un gradient de valeur pour juger de la validation rompt avec la méthode \enquote{binaire} proposée par Naylor, la validation d'un modèle passant à présent par l'acceptation subjective d'un seuil de représentativité relatif à l'objectif poursuivi. Avec pour conséquence notable qu'une \foreignquote{english}{[...] simulation or game relatively valid for one objective may be not be equally valid for another.}

Si la notion de seuil n'est pas explicitement abordée par Hermann, c'est pourtant sous cette acceptation que la \textit{V\&V} actuelle va reprendre ce concept. Avec la position suivante, celui de se fixer un seuil de représentativité général à atteindre \textit{a priori}.

\foreignquote{english}{\textbf{Principle 2: The outcome of simulation model VV\&T should not be considered as a binary variable where the model is absolutely correct or absolutely incorrect } [...] The outcome of model VV\&T should be considered as a degree of credibility on a scale from 0 to 100, where 0 represents absolutely incorrect and 100 represents absolutely correct. 

\textbf{Principle 3: A simulation model is built with respect to the study objectives and its credibility is judged with respect to those objectives } [...] The study objectives dictate how representative the model should be. Sometimes, 60\% representation accuracy may be sufficient; sometimes, 95\% accuracy may be required depending on the importance of the decisions that will be made based on the simulation results. Therefore, model credibility must be judged with respect to the study objectives.}\autocite[15-16]{Balci1998}

La position de \textcite[166]{Sargent2010}, tout en étant relativement similaire, propose une vision plus fine et plus réaliste ou le seuil de précision attendu est attaché aux variables de sorties. Un point important sur lequel nous reviendrons plus longuement dans la suite de cette partie. \hl{ref vers la bonne partie}

\foreignquote{english}{A model should be developed for a specific purpose (or application) and its validity determined with respect to that purpose.[...] A model is considered valid for a set of experimental conditions if the model’s accuracy is within its acceptable range, which is the amount of accuracy required for the model’s intended purpose. This usually requires that the model’s output variables of interest (i.e., the model variables used in answering the questions that the model is being developed to answer) be identified and that their required amount of accuracy be specified. The amount of accuracy required should be specified prior to starting the development of the model or very early in the model development process.}\autocite[166]{Sargent2010}

\begin{figure}[h]
\begin{sidecaption}[fortoc]{ On remarquera la forte présence des techniques présentés par Hermann dans la synthèse proposé par Balci en 1986 \autocite{Balci1986}}[fig:S_syntheseBalci]
  \centering
 \includegraphics[width=.9\linewidth]{subjective_balci.png}
  \end{sidecaption}
\end{figure}

Ces deux citations permettent de montrer au passage comment la vision de la validation défendue par Hermann a été intégrée dans une forme très approchante par des acteurs de la \textit{V\&V} comme Balci ou Sargent, dont on a vu précédemment les définitions dans la section \ref{ssec:def_generique_validation}. Ces deux derniers sont en réalité les acteurs majeurs d'une synthèse (voir la figure \ref{fig:S_syntheseBalci}) opérée dans les années 1980-1990 \autocite{Nance2002}, dont on peut dire qu'elle est marquée par un retour à une certaine forme de neutralité (voir par exemple le rejet des aspects philosophiques décrits décrits dans la section \ref{ssec:def_generique_ validation}  qui se double d'un jargon technique spécifique à l'établissement d'un processus qualité exploitable pour l'ingénierie) . Des adaptations qui permettent probablement de mieux accepter en son sein des typologies de techniques aussi différentes que celle de Naylor\Anote{naylor_etonnement} ou Hermann. Régulièrement révisées, \textcite{Balci1998} fait ainsi état dans sa dernière taxonomie d'un catalogue de 75 techniques différentes dans lequel peuvent piocher les modélisateurs en fonction de leurs besoins. 

On se rend bien compte que dans le cadre des sciences humaines et sociales la possibilité de fixer par avance ce type de seuil n'a pas de sens, surtout dans un cadre explicatif.

%\textit{Que faut il entendre ici par partiellement ? Quels sont les leviers permettant au géographe de compenser cette perte de représentativité par un gain en compréhension sur le système à étudier ? }

Pour mieux comprendre quel est l'enjeu de cette délimitation entre un modèle réaliste et un modèle abstrait il faut évoquer cette tension permanente qui nourrit les choix du modélisateurs dans la construction d'un modèle explicatif. Deux attracteurs possibles et apparemment opposés, avec d'une part la volonté de se rattacher à une forme de réalisme au travers de l'injection d'une part maitrisée de réalité tout au long du processus de construction \Anote{durand_observation}, et d'autre part une force qui nous pousse au contraire à se détacher de cette même empirie pour ne retenir que le matériel susceptible de servir l'objectif du modèle.

La sociologue et épistémologue \textcite{Bulle2005} a bien formalisé ce dilemme dans la nécessité pour tout modélisateur de positionner son modèle sur un gradient opposant le réalisme des causes des modèles explicatifs \Anote{bulle_modele_explicatif}, au réalisme des effets des modèles descriptifs. 

Pour mieux comprendre quelles connaissances peut-on attendre d'un tel positionnement sur ce gradient, le mieux est encore de commencer par évoquer un de ses extrêmes, en invoquant par exemple le modèle universellement connu de Schelling. De par sa portée d'application extrêmement générale et la nature très abstraite de ses paramètres celui-ci constitue en soi un extrême intéressant pour comprendre où se situe encore l'explication lorsque le détachement de la réalité est à ce point éloigné. Sur ce point, les analyses de \textcite{Bulle2005} et \textcite{Phan2008, Phan2010} se réfèrent principalement à l'essai de \textcite{Sugden2002} pour évoquer quels types de relations entre les deux mondes peut on attendre de ce type de modèle épuré. 

Les résultats qui dérivent de la mise en dynamique des règles dans le modèle de Schelling sont d'une telle universalité, d'une telle robustesse qu'il n'est plus question de confronter les résultats ainsi obtenus à la réalité. A cet égard le potentiel explicatif de ce type de modèle s'oppose selon \textcite{Bulle2005} à tout réalisme empirique. De ce point de vue, \enquote{le modèle n'est pas tant une abstraction de la réalité qu’une réalité parallèle [...] bien que le monde du modèle soit plus simple que le monde réel, celui-ci n'est pas une simplification de l'autre. Le modèle est réaliste dans le même sens qu'un roman peut être appelé réaliste [...] les personnages et les lieux sont imaginaires, mais l'auteur doit nous convaincre qu'ils sont crédibles } \autocites[131]{Sugden2002}[10]{Phan2008}

L'effet d'une telle recombinaison d'hypothèses revient à mettre en oeuvre un \enquote{monde crédible} où l'inférence inductive est mobilisée pour identifier des similitudes significatives entre les deux mondes. \autocites{Livet2006, Phan2008}. Tout le travail réside donc dans l'interprétation prudente qui peut être faite entre ces résultats d'un monde factice et d'une réalité.

Un processus commun utilisé dans toute oeuvre de fiction pour piquer la curiosité de l'observateur, la mise en exergue volontaire d'une tendance du monde réel dans un monde imaginaire permettant d'entamer une réflexion sur l'existence, la portée, la nature de cette même tendance dans le monde réel. Les villes ou les sociétés mis en avant dans des oeuvres de fiction cinéma ou dans la littérature ne sont jamais que des mondes plus ou moins crédibles (Gotham City, 1984, Matrix, la série Black Mirror, etc. car la liste est longue ...)  pour mettre en avant un discours, ou des tendances du monde réel sur lequel doit porter le questionnement; (http://www.influxpress.com/imaginary-cities/ , \href{http://cybergeo.revues.org/1170#tocto1n9?}{cybergeo})

Si le discours scientifique n'a clairement pas cette obligation ludique, il n'en reste pas moins que ce processus de reconstruction crédible est déjà un outil formidable pour questionner les processus à l'oeuvre dans le monde réel \Anote{ruffat_samuel_ville}. Mais cette ambiguïté de lecture a déjà mené à de nombreux malentendus, d'une part envers le grand public (Voir forrester, mais également \Anote{deffuant_debat}) qui pourrait prendre des résultats de simulation pour la réalité avec tout les conséquences que cela suppose, mais également parfois entre scientifiques provenant de divers horizons. Ainsi après la lecture de la critique par \textcite{Chattoe2011} de l'article de \textcite{Yanoff2009}, il ressort toute la difficulté d'évaluer la méthodologie et le travail réalisé autour d'un modèle au travers d'une seule publication, notamment lorsque la fonction cognitive recherchée par les modélisateurs n'est pas décrite explicitement, ce qui provoque aussi ce décalage entre attente du lecteur et le processus réel de recherche qui sous-tend la construction du modèle. \hl{dp: TROP ALLUSIF}

\textit{Doit on se contenter de ce seul mode explicatif ? Existe t il un moyen pour renforcer la confiance dans la capacité explicative des hypothèses ainsi mobilisés ? } 

%% DEBUT - EN ATTENTE DE LA REPONSE DE VARENNE %%

\textcite{Bulle2005} evoque bien l'existence de modèle à cheval entre potentialité explicative et potentialité descriptive. Ainsi \enquote{appliquée aux processus sociaux réels, la simulation peut allier au potentiel descriptif offert par l’imitation d’effets empiriquement observables, le potentiel explicatif que lui confère la mise en œuvre de relations causales effectives. }

A la différence de modèles trop simples qui n'offrent que de maigres accroches avec la réalité, c'est donc par la réintroduction maitrisée de l'empirie dans les modèles de simulation construits que l'on peut espérer la mise en route progressive d'un processus de validation.

Seulement conformément au type de problèmes que l'on a déjà pu effleurer en traitant de philosophie des sciences dans la section \ref{sssec:philo_sciences}, le processus de validation se heurte rapidement à la différence de nature entre les résultats produits par des hypothèses \textit{reconstruites} et le monde réel. Tout comme le substrat est artificiel, le résultat produit par cette dynamique reste le produit d'un monde reconstruit -in silico- L'existence de ce nouveau niveau d'empirie amène les épistémologues comme Varenne à parler ici d'\enquote{expérience concretes du second genre} faisant alors de la simulation une \enquote{quasi-expérimentation} \autocites{Varenne2001, Varenne2007, Phan2008}

On en déduit que quelque soit notre placement sur ce gradient, il est effectivement vain de chercher à valider un modèle en usant d'un quelconque \enquote{seuil de suffisance} caractérisant \enquote{l'injection de réalisme à atteindre qui autoriserait une inférence certaine sur le monde réel}, puisque de toute façon cette inférence s'appuie sur un résultat \enquote{artificiel} forcément discutable. \Anote{bulle_modele_autonome} \Anote{phan_livet_modele} 

La démonstration précédente nous indique plusieurs pistes de réflexions.

D'une part l'objectif de réalisation d'un modèle au réalisme uniquement structurel n'a pas de sens, même avec beaucoup d'hypothèses, car elle ne permet en aucun cas de garantir la justesse d'une comparaison entre données empiriques et simulés, et n'offre donc aucun critère d'arrêt pertinent dans l'activité de modélisation.

D'autre part à moins de retomber dans les débats philosophiques évoqués dans la section \hl{xxx}, elle nous oblige à penser le modèle pour ce qu'il est vraiment, non pas une construction guidée par la validation, mais la construction d'un raisonnement appuyé par une simplification orienté par et pour un but. Peu importe alors le fait qu'une divergence s'installe entre le monde tel qu'on l'observe et le système modélisé, au contraire.

\foreignquote{english}{In all probability some distributions of events or some kinds of hypotheses will produce results with unacceptable divergence between the operating model and the observable universe. Although these incongruous may not pinpoint the inadequacy in the model, they should provide a diagnosis of the general area which seems unrepresentative.} \autocite[226]{Herman1967}

Mais ce terme de \enquote{simplification} souvent employé reste d'emploi ambigue, la modélisation nécessitant comme le dit \textcite{Haggett1965} non pas tant la mise en oeuvre d'une simplification aveugle, qu'une idéalisation guidé par la volonté de mettre à nu des propriétés du système observé. \textcite{Brunet2000}, pour qui la modélisation est également un processus de recherche, propose même pour éviter toute confusion sur les termes de dénuder la définition de modèle de cette fausse directivité, le modèle devenant dans sa version la plus épurée une \enquote{représentation formalisée d'un phénomène}; le terme \enquote{représentation} intégrant alors toute la complexité sous jacente à une telle formalisation : \enquote{Il va de soi que cette représentation passe par plusieurs filtres, qui tous tendent des pièges : la perception du phénomène, sa représentation, la construction d'un modèle, l'interprétation du sens de ce modèle et la capacité du modèle à rendre compte du phénomène.}

Un point de vue semble t il partagé par Varenne pour qui le terme simplification est  \endquote{[...] un glissement d’attribution indu. Puisque l’usage du modèle est relatif (à un observateur et à un questionnement), on ne peut dire que le modèle doit être un objet simple en lui-même ou dans l’absolu. Il convient donc de regarder sous quel aspect exactement il doit apparaître simplificateur, sous quel aspect il devient un outil facilitateur, un outil de facilitation.}

Dès lors, {[...] on comprend déjà qu’un modèle n’est pas ce qui est recherché en tant que tel, mais ce qui facilite la recherche d’information au sujet d’un système réel ou fictif, cela dans le cadre d’un processus à visée de représentation, de connaissance, de conceptualisation, de conception ou encore de transformation. Il est le moyen plus que la fin. C’est pourquoi je m’aventurerai, à partir de maintenant, à user plutôt du terme de facilitation que de celui de simplification [...]} (voir également la section \ref{ssec:rapell_termes_generiques}) \autocite{Varenne2008}

Comme déjà évoqué par les géographes, ce n'est pas tant \enquote{le modèle} que ce qu'il y a \enquote{dans le modèle} qui nous intéresse \autocites{Sanders2000, Besse2000}, et il faut rajouter par là même aussi l'histoire justifiant de cette configuration. 

Reste alors à explorer comment cette \enquote{facilitation} s'exprime au travers de la construction du modèle de simulation. Seulement comment analyser la pertinence d'une représentation prise en dehors de son contexte, les choix intervenant lors d'une modélisation ne répondant pas à une logique universelle pré-établie, étant comme on l'a vu motivé et modifié par un (ou plusieurs) objectif. Varenne a identifié une vingtaine de ces fonctions de facilité de médiation pouvant motivé la construction ou l'utilisation générale d'un modèle. Or il me semble qu'une fois rapporté au modèle, on est bien obligé de constater l'impact que peut avoir cette diversité d'objectifs dans le choix menant à différentes représentation d'une même hypothèse dans le modèle. Etait-ce pour dénoter une entité réelle (un agent = un individu, une ville, une innovation ) ? Est ce dans le but de simplifier pour la compréhension ? pour les performances ? pour répondre au principe de parcimonie ? Ou les trois à la fois ? (une population agents homogène devenant une equation de croissance par exemple ) Etait-ce un choix fait à la suite parmi une multitudes d'autre essais (différentes équations plus ou moins représentative du phénomène à considérer) ? etc. Il n'y a aucune raison pour que les mécanismes intégrés aux modèles soit homogènes. Dans le cas d'un modèle de migration inter-ville, il est en effet plus intéressant de mobiliser les populations de façon aggrégé si on s'intéresse aux règles intervenant dans la dynamique d'interactions entre les villes, par contre, si il s'agit d'observer l'impact que peuvent avoir des règles de comportements sur ces interactions, ce niveau peut devenir pertinent; cette aproche ne chassant évidemment pas la première, au contraire les couplages étant bienvenu. Normalement tout ces choix devrait être explicité, ce qui est rarement le cas, vu la complexité d'une tâche qui apelle pour être sérieuse l'analyse d'une activité de raisonement accompagnant le modèle dont les jalons de reflexion ont bien souvent disparu. \autocite{Varenne2013b}

La tendance à la pluriformalisation \Anote{pluriformaliser} permise par les modèles multi-agent ne vient pas non plus faciliter cette tâche, car ces modèles de simulation qui peuvent déjà intégrer -et c'est d'ailleur pour cela qu'ils ont autant de succès- sans problème une hétérogénéité d'échelle, de niveau d'abstraction, de modèles, doivent aussi compter avec l'intégration de formalismes mobilisant des temporalités et/ou des échelles différentes \autocites{Varenne2008,Varenne2012a}. Ces couplages n'étant pas toujours évident, y compris au niveau informatique ou des artefacts, c'est à dire l'apparition de mécanismes non prévu et difficile à expliquer d'un point de vue purement théorique, peuvent venir rapidement venir perturber les belles ontologies réalisés en amont. 

Même les modélisateurs ont parfois du mal à s'y retrouver, par exemple il n'est pas toujours évident d'expliquer pourquoi on a choisit de coupler pour certain mécanismes le formalisme agents avec celui des équation différentielle ? Il faut alors comprendre que dans certains cas, c'est aussi ce qui a pu motiver le modèle, l'intérét de la pluriformalisation étant justement ce qu'il faut démontrer en comparaisons des approches traditionnelles prisent séparement. Plusieurs réflexions ont montré qu'il s'agissait d'un type de modélisation en devenir et en voie de démocratisation, les formalismes pour la simulation informatiques (multi-agents, micro-simulation, ac) ou mathématiques (systèmes dynamiques) utilisés n'ayant jamais eu vocation à s'opposer (approche individu - centré contre approche mathématique traditionnelle) comme on aimerait parfois nous le faire croire \autocites{Sanders2013, Banos2013}. 

%% FIN - EN ATTENTE DE LA REPONSE DE VARENNE %%

Une façon de dépasser cette problématique de la validation est d'accepter le fait que le réalisme des hypothèses ne soit plus vraiment un objectif, mais plutôt la réalisation conséquente d'une expertise qui tient essentiellement de l'angle théorique choisi pour éclairer un problème.

Autrement dit, la confiance établie dans les capacités explicatives des hypothèses choisies ne se juge pas tant dans la comparaison des résultats attendus avec le réel observé, que dans l'exploration du monde crédible ainsi simulé en fonction de critères experts construit sur une observation du réel, dans l'espoir d'en dégager une connaissance qui doit encore être vérifiée \Anote{denise_geopoint}. 

Le problème est ici en quelque sorte inversé, ce n'est plus une qualification directe du réel qui est visé par le modèle, mais le modèle qui est visée par notre compréhension du réel au travers de critères experts. Ceux-ci viennent questionner et mettre en tension ce monde virtuel en lui imposant de nouvelles contraintes, révélant par là même les forces et les faiblesses de nos hypothèses dans le modèle. On oppose dans la construction du modèle un jeu d'hypothèses susceptible de produire des résultats attendus, à la réalité des conclusions apportés par la mise en oeuvre effective d'une dynamique que l'on contraint volontairement.

A ce titre, et en s'inspirant de la remarque faites par \textcite{Bulle2005} à ce sujet, il sera toujours nécessaire et légitime de questionner la pertinence des rapports mesurés entre les liens causaux proposés dans le modèle et le ou les critères qui sont censés en rendre compte.

On retrouve ces réflexion dans les termes de \enquote{validation interne} et \enquote{validation externe} introduit par \autocite{Amblard2006}, qui est un des rares publications abordant de façon assez précise le passage d'une \enquote{validation} à une \enquote{évaluation} des modèles de simulations en sciences humaines et sociales. La relation d'inter-dépendance entre validation interne et validation externe y est clairement exposé, à travers l'impact des analyses de sensibilités sur la structuration des modèles, et l'exploration des classes de comportements émergentes observés, les deux se rapportant au final à une comparaison faisant intervenir des critères d'évaluation, des fait stylisés ou des données.

\enquote{L'analyse de sensibilité, si elle peut s'appliquer pour tester la robustesse des résultats d'un modèle, peut également être utilisée pour tester la robustesse de la structure du modèle. En modifiant les hypothèses réalisées dans le modèle, par exemple en modifiant les structures organisationnelles, le modélisateur obtient des indices relatifs à la stabilité de son modèle et de ses hypothèses. Ces indices lui permettent précisément de jauger l'importance du choix d’une hypothèse et l’influence de son remplacement par une autre sur un aspect particulier du modèle. [...] Une autre propriété importante qu'il s'agit d'étudier au cours de cette étape de validation interne, concerne les classes de comportements produites par le modèle. Les simulations multi-agents produisent ce qui est assez communément appelé des « comportements
émergents » (voir chapitres 14, 16 et 17), c'est-à-dire des comportements qui ne sont pas exprimables en utilisant uniquement les hypothèses réalisées sur les comportements individuels}. Deux propriétés qui font ainsi écho à une méthode de la validation externe qui \enquote{[...] consiste à rapprocher les classes de comportements (identifiées lors de la validation interne) à des comportements saillants du système-cible : les faits stylisés. Ce rapprochement, s’il peut être fait avec des faits stylisés identifiés a posteriori (permettant par exemple de découvrir dans les phénomènes empiriques, des comportements stylisés qui auraient pu passer inaperçus), possède, on le sent bien, plus de force lorsque les faits stylisés sont déterminés avant même la modélisation comme des comportements que l'on cherche à reproduire par le modèle ou dont on se servira comme un critère de validation parmi d'autres (rétrodiction).}

Ces deux points nous incite ainsi à pratiquer une évaluation à contrepied de la démarche habituelle, alternant validation interne, puis externe. On met alors de coté un instant la validation interne comme exploration non dirigé des comportements du modèle pour se concentrer sur une exploration, moins complète, ou c'est le désir de rapprochement qui vient piloter cette fois ci l'exploration des comportements. Les modélisateurs peuvent en effet introduire les hypothèses dans un modèle de simulation dans le but de satisfaire, ou \textbf{de ne pas satisfaire} ces critères. En effet on a bien précisé que l'objectif de correspondance avec les données n'était plus la priorité dans l'établissement de tels modèles de compréhension, sinon pourquoi ne pas se contenter d'un modèle de Gibrat pour expliquer la hierarchies des systèmes de villes ? 

Autrement dit, il s'agit de mobiliser de façon volontaire cette tension entre hypothèses du modèle et critères d'évaluations mis en place pour en rendre compte afin de savoir si oui ou non cette question valait la peine d'être posé. Reste qu'il faut avoir les moyens techniques de pouvoir répondre à cette question. % Avant de revenir à cela, question de la proof of possibility, impossibility

Il semble que cette mise en tension se satisfait assez bien d'un cadre d'analyse basé sur l'activité de modélisation, ou les critères et les hypothèses, et c'est bien pour cela que l'on mobilise la simulation, ne sont pas tous nécessairement connu à l'avance. Ce qui laisse la place au cours de cette confrontation à l'avénement d'une certaine surprise, à même de produire une connaissance, et de guider le choix des modélisateurs à chaque nouvelle étape du modèle. 

% Les critères toutefois entretiennent un lien avec les hypothèses dont ils sont censé rendre compte, on peut donc imaginer que la parcimonie exprimé dans la construction des modèles puisse faire en lien des critères porteurs de cette connaissance exprimés sur le comportement du modèle. 

% Une remarque d'autant plus valable lorsque on sais que cette histoire se construit en confrontation avec la construction et la mise en oeuvre progressive des critères constitutif du système ciblé.

% Ce qui pose effectivement toujours la question de la nature des connaissances attendues dans une telle perspective.

\paragraph{L'abduction, un phénomène clef moteur dans l'activité de modélisation}

La présence d'une hypothèse dans le modèle se justifie donc tout à la fois par l'expertise du modélisateur que par son adéquation, ou sa non adéquation \textbf{potentielle} avec différents critères de validation. La subjectivité de l'expérimentateur joue sur les deux tableau, et donne à voir dans cette subtile inter-dépendance qui relie le choix des hypothèses et le choix des critères une forme incertitude quand au résultat assez difficile à prévoir et quantifier.

Que se passe-t-il lorsque le potentiel explicatif d'une hypothèse pourtant appuyé par des résultats empirique constaté dans le système observé s'avére invalidé par une analyse de sensibilité ou un critère d'évaluation ? Et cela, alors même que l'experimentateur considère celle-ci comme étant indispensable dans le développement d'une dynamique donné ? Que se passe-t-il au contrare lorsqu'un critère d'évaluation est atteint alors que cela n'était pas attendu au vu de la structure du modèle ? 

%La fonction heuristique de la simulation pouvant s'exprimer tout autant dans cette \enquote{surprise} d'une divergence entre le potentiel investit dans les hypothèses et les critères selectionnés, que dans la surprise suivant l'introduction de nouveaux critères contraignant le modèle, et remettant en cause ce même potentiel de représentation investit dans certaines hypothèses. 

Il y a une divergence nécessaire entre la volonté du modélisateur de rendre compte d'un système observé par un réseau d'hypothèse qui lui parait parcimonieux, nécessaire et cohérent d'un point de vue thématique (le potentiel investit), et la réponse effective apporté par la mise en dynamique de ces causalités lues au travers des critères selectionnés pour en rendre compte. % la possibilité d'infirmer ou d'affirmer de nouvelle connaissances, avec le développement de nouveaux critères, de nouvelles hypothèses ayant jusque là échappé aux raisonnement du modélisateur.

\foreignquote{english}{In developing a game or simulation, the designer is required to be explicit about the nature and relationships between the units in the operating system and their counterparts in the observable universe. He must specify the conditions which cause a relationship to vary. In constructing an operating model a connection between previously unrelated findings may be discovered. Alternatively, a specific gap in knowledge my be pinpointed and hypotheses required by the model my be advanced to provide an explanation.} \autocite[219]{Hermann1967}

La surprise volontaire ou involontairement produite au cours de cette divergence, et qui accompagne généralement l'activité de modélisation, revient sous le nom d'abduction, le terme venant de Charles S. Peirce \autocites{Besse2000, Banos2013, Phan2006, Livet2014} 

Une capacité dont on a déjà vu en citant Hacking \autocites{Hacking1983,Hacking2003, Hacking2006} qu'elle tenait plus d'une propriété inhérente à l'humain, existant de façon préalable à ses créateurs Aristote, ou Peirce. Un modèle de cognition remis au gout du jour ces dernières années, et qui parait correspondre assez bien, est celui du \enquote{cerveau statisticien},  \enquote{cerveau prédictif}, ou \enquote{cerveau bayésien} pour qui \enquote{penser c'est avant tout prédire}. Cette machine à inférer permanente, construisant des logiques qui lui sont propre à partir du peu d'informations qui lui sont donnés directement ou indirectement, quitte à rapeller en urgence d'ancien schéma, fournit comme on pourrait s'en douter plus de mauvaises prédictions que de bonnes. Mais peu importe, ce qui est important ici, c'est sa capacité à apprend rapidement de ces erreurs. 
%Cette logique bayésienne qui consiste à formuler une hypothèse a priori de façon consciente ou inconsciente \Anote{kauffman}, prise de façon rapide ou lente, basé sur nos connaissances passés ou sur notre environnement présent, pour la confronter et la réévaluer au yeux de la réalité de façon itérative correspond assez bien il me semble à ce que l'on pourrait apeller \enquote{abduction}, apellée également \enquote{inférence de la meilleure explication}. 
Ainsi pour Stanislas Dehaene, partisant de ce modèle cognitif, \enquote{Ce que Pierce appelle l'abduction n'est rien d'autre que ce que les sciences cognitives contemporaines nomme l'inférence bayésienne et qui consiste à mener un raisonnement probabiliste en sens inverse afin de remonter aux causes cachées d'une série d'observations.}

Cette théories qui touche à l'ensemble des disciplines oeuvrant dans le champs des sciences cognitives sont mieux décrites par exemple par Stanislas Dehaene dont les cours sont disponibles sur le \href{http://www.college-de-france.fr/site/stanislas-dehaene}{@site} du Collège de France. Toutefois dans l'utilisation des modèles de simulation d'une part ce n'est pas le monde réel qui nous surprend, mais ce qui se passe dans le modèle de simulation, et d'autre part il est plus intéressant d'adopter une démarche active et créative dans la mobilisation des hypothèses, afin de maximiser la surprise plutot que de la miminiser, de dépasser son horizon de connaissance plutot que de s'y conforter. 

% Proof of possibility ? 
Comme le résume bien Banos dans son HDR, \enquote{l’esprit même de l’abduction au sens de Peirce, désignant cette capacité de l’être humain à générer des hypothèses temporaires à partir de l’information incomplète dont il dispose. Appliquée à la démarche scientifique, l’abduction renvoie ainsi à la capacité du scientifique à se mettre en position d’étonnement, à se laisser guider par la recherche de l’inattendu et plus généralement à laisser libre cours à sa créativité.} \autocite{Banos2013}

Comme on pouvait alors si attendre, les motifs développés par les modélisateurs soutenant cette approche sont très loin de ceux attendus pour la prédiction, ou c'est d'abord la robustesse des résultats qui prime, car \enquote{[...] dans le cas d’une modélisation compréhensive, où l’objectif est d’apprendre des propriétés du système que l’on reconstruit \textit{in silico}, le fait d’avoir, pour une partie des conditions expérimentales des comportements très erratiques du système, loin de discréditer le modèle nous apprend au contraire des éléments de son fonctionnement et nous permet même d’anticiper le fonctionnement du phénomène modélisé. Si sous certaines conditions le modèle est très instable c’est une information très enrichissante sur le modèle et sur le phénomène considéré.} \autocite{Amblard2010} 

Force aussi de constater que cette incrémentalité dans la construction d'un modèle de simulation ne suit pas vraiment un modèle linéaire de développement, et s'accompagne, au moins dans les sciences humaines et sociales d'une activité de raisonnement en partie imprévisible. Ainsi il peut paraitre paradoxal pour un modélisateur débutant de voir à quel point il est important de \enquote{malmener} les modèles que l'on a précédemment construit avec raison et parcimonie. L'important ici nous dit \textcite{Amblard2010}, c'est que cela participe à l'élaboration de la compréhension ou de l'explication des phénomènes considérés. Car comme le dit dans son tout premier principe \textcite[65]{Banos2013}, modéliser c'est avant tout apprendre.

D'autre définitions de l'abduction \Anote{abduction_definitions} permettent de mettre en valeur d'autres de ces propriétés, la création en est une, mais avec celle-ci vient aussi la selection. Si on se rapporte aux processus de cognition, la conscience apparaitrait par exemple comme un filtre discrétisant, selectif, d'une pensée inconsciente fluctuante et résolument continue. On peux supposer que lors de la modélisation, ce type de processus est également à l'oeuvre, et il n'est pas rare lorsqu'on construit un modèle de simulation d'avoir à choisir, ou à ne pas choisir, avec plusieurs hypothèses ou implémentation d'hypothèses alternatives. Le deuxième choix expliquant aussi en partie pourquoi les interfaces utilisateurs de nombreux modèles de simulation Netlogo sont aussi riches en boutons de selection.

%Cela soulève la possibilité d'hypothèse explicative concurrente ou inter-dépendante dans l'apparition d'un phénomène, dont certaine échappe forcément au seul modélisateur géographe du fait par exemple de la nature inter-disciplinaire des objets engagés, auquel il faut encore appliquer une selection plus consciente en décidant de mettre plus ou moins en avant des hypothèses susceptible de surprise. 

%La surprise ne vient donc pas seulement du modèle, mais aussi de ce que font les autres manipulant les modèles, surtout dans le contexte inter-disciplinaire ou nous évoluons, les limites de nos connaissances se manifestant assez vite lorsqu'il s'agit d'étudier un phénomène ou un objet partagé, comme les villes par exemple. 


%% A REPLACER

% S'exprime dans une dynamique ? 
\paragraph{Quels critères d'évaluation pour quelle mesure des hypothèses ?}

On a montré que l'objectif d'une adéquation avec le système observé n'avait pas de sens, mais on n'a pas évoqué les modalités de ce rapprochement dans le rapport d'évaluation existant entre hypothèses et critères quantitatif mobilisés dans le modèle. Pourquoi ne pas envisager ici une validation externe basé sur une comparaison empirique et terme à terme des hypothèses constitutive entrant dans la structure causale du modèle  ?

Pour \textcite{Batty2001} et du fait l'\textit{Observational Dilemna}, cette solution n'apparait pas faisable en général. \foreignquote{english}{In principle, each element of this process should be explicit and should be capable of being validated with observed data. In practice, this is rarely if ever the case. The data set would be too large, it would be impossible to collect in its entirety, it may be impossible to even observe and measure. Yet the processes are known to be important. Other criteria must thus be used.} Outre donc la question de la disponibilités des données en sciences humaines et sociales, l'\textit{Observational Dilemna} démontre l'impossibilité de s'abstraire de cette intrication entre cause et effet lorsqu'on observe un phénomène (complexe) en sciences humaines et sociales. De fait, cette caractéristique des systèmes complexes suppose aussi l'impossibilité d'établir l'unicité des hypothèses avancées pour décrire un phénomène, mais aussi par extension celle des critères d'évaluation, qui reste eux aussi des construits formulés sur la base d'une observation du système observé.

On peut ajouter à çà une autre limite concernant les données, on a effet vu que toutes les hypothèses du modèles n'avait aucune raison de se placer au même niveau d'abstraction, ou d'appartenir au même formalisme. Il est par exemple déjà difficile d'accéder à des données dans une fenêtre spatio-temporelle donné, est-il possible d'en avoir sur plusieurs fenetres, voire plusieurs fenetres simultanées ? 

Pour \autocite{Amblard2006}, si cette validation externe basé sur une comparaison quantitative avec les données n'est pas impossible, elle n'en reste pas moins difficile à mettre en oeuvre de façon systématique, pour les raisons évoqué ci-dessus, l'appel à des critères d'évaluation plus qualitatif, dont il faut déjà justifier la construction, restent les plus évidents à mettre en oeuvre. Un point de vue assez logiquement partagé par \textcite{Batty2001} \foreignquote{english}{ [...] complex systems models have multiple causes which display a heterogeneity of processes that are impossible to observe in their entirety. The focus is on more qualitative evaluation of a model’s plausibility in ways that relate to prior analysis of the model’s structure.} 

% A deplacer et a remettre dans le flux au desssus ? 

Une des autres originalité dans l'analyse d'Hermann réside dans les remarques très juste et très précoce qu'il a formulé sur la relativité des hypothèses et des critères mobilisé dans les modèles. Celui-ci est en effet tout à fait conscient qu'il ne s'agit pour l'une comme pour l'autre que d'assertions sur la réalité, comme nous l'avons déjà discuté auparavant. Il propose donc d'essayer de faire au mieux. En adoptant une validation multi-critère \Anote{methode_hermann} intégrant les objectifs ayant guidé la construction du modèle il espère ainsi renforcer le crédit qu'il est possible d'apporter aux simulations. \foreignquote{english}{We have arrived at the position, then, that multiple validity criteria are needed because of the error of measurement and because of the recognition that criteria can be only assertions about \enquote{reality}} 

Comme on pouvait toutefois s'y attendre, cette impossibilité d'admettre l'unicité des critères pour juger la structure causale mobilisé s'insère dans un questionnement plus large. Faut-il effectivement juger la valeur des hypothèses constituantes de cette structure uniquement vis à vis de la réponse à ces critères ?

Si on en croit \textcite[17]{Besse2000}, pas vraiment, car cela serait oublier qu'\enquote{Une hypothèse possède une signification propre, avant même d’avoir été engagée dans l’aventure hautement improbable des programmes de validation. Cela nous conduit à reconnaître dans l’activité scientifique un moment de la production du sens, a coté du mouvement vers l’établissement des vérités.}

L'abduction de part l'argument naturaliste évolutionniste qu'on lui prête,dépasse le simple cadre de la logique dans lequel de toute façon elle posait déjà problème, et n'intervient donc pas comme un moyen de preuve : \enquote{ On nous propose plutot d’envisager l’ensemble des démarches par lesquelles les chercheurs s’orientent vers les hypothèses qui semblent plausibles, en éliminant celles qui ne peuvent etre considérées comme pertinance. } A ce titre, \enquote{La démarche abductive permet un authentique gain de sens, une progression dans l’élucidation}

Un argument supplémentaire pour parler d'évaluation plutôt que de validation \autocite{Amblard2006}, car celle-ci s'inscrit dans un projet parallèle à l'activité de construction du modèle, dont la mise en œuvre implique sinon la construction au moins l'existence préalable d'hypothèses, et d'indicateurs pertinents sur le système observé; une expertise cumulé qui dépasse de loin en durée et en travail le seul projet de construction d'un modèle, et fait souvent intervenir un système de modèle dans une démarche de construction des connaissances de portée beaucoup plus large que cette seule construction de modèles de simulation. Un point que l'on a déjà abordé dans le paragraphe \ref{decorreler_validation} pour justifier d'une décorrélation des problématiques de la Validation vue sous l'angle réducteur de cette seule activité de modélisation multi-agents.

Que cela soit les paramètres, valeur de paramètres, hypothèses mobilisés, choix d'implémentation des hypothèses, critères d'évaluations ( fait stylisés ou données ) construit ou choisi, tout ces étapes ne peuvent être mis en oeuvre si on n'accepte pas de voir l'activité de modélisation pour la simulation comme parti prenante d'une activité de construction des connaissances plus globales. C'est ce qui rend aussi difficile l'évaluation de publication soutenant l'originalité d'un modèle de simulation par un public n'ayant pas connaissance de ce système de modèles et des interactions complexes et réflexives qui relient ceux ci, que cela soit en amont ou en parallèle de la construction des modèles de simulations.

Cette démarche globale a été plusieurs fois théorisé par les géographes \autocites{Besse2000, Sanders2000, Mathian2014}, et une application plus explicite des relations que peut entretenir un modèle de simulation avec d'autres type de modélisation (statistique, spatiales) peut être vu dans la thèse de Clémentine Cottineau \autocite{Cottineau2014a, Cottineau2014b}. 

De plus, il reste difficile donc d'éliminer une hypothèse présente dans le modèle en fonction de sa seule mise en défaut observés à un instant $t$ donné dans la construction d'un modèle, notamment lorsque la présence de celle ci fait sens du point de vue des objectifs qui ont été fixés par le modélisateur. 

D'autant plus qu'il faut aussi prendre en compte cette double dynamique dans lequel opère la construction et la complexification des modèles et des indicateurs pour en rendre compte. Une hypothèse valable à un instant $t$ ne le sera peut etre plus à un instant $t + 1$, ou inversement. 

Peut être n'était-ce simplement pas le moment pour intégrer cette hypothèse au modèle, celui-ci étant encore trop simple ? Peut être manquait-il des interactions pour que sa dynamique soit révélé ? Peut être que l'indicateur devant rendre compte de cette dynamique n'est pas adapté ? Peut être que l'implémentation proposé n'était tout simplement pas la plus adapté à ce moment là ? etc. 

Ce qui a mon sens soulève ici plusieurs remarques : 
- il est vraiment difficile de savoir ce qui va se passer avec l'intégration ou le retrait des hypothèses, ou des critères d'évaluation dans un modèle de simulation si on ne dispose pas d'un outil permettant d'évaluer systématiquement chacune de ces modifications,
- cette évaluation doit être mis en place de façon immédiate, dès que les premières questions sont posés à la structure causale du modèle, afin de ne pas biaisé le raisonnement construit par la prolongation d'une phase de \textit{face validity} pouvant très vite devenir problématique de part les redéveloppements qu'elle suppose dans le futur.
- malgré cela, il faut bien voire qu'une exploration des comportements du modèle, même complète, ne fera pas disparaitre ce problème, qui tient avant tout de l'avancement du raisonnement dans la construction du modèle.

Il reste donc à gérer cette possibilité de réengager les hypothèses et les critères à différents moments dans la construction des modèles, et soutenir une activité de construction cumulative qui ne soit pas \enquote{oublieuse} de cette autre espace temps dans lequel se construise les hypothèses et les différents critères mobilisés. 

Cette variabilité exprimé dans la construction et la paramétrisation des structures causales et des critères associés renvoie à ce phénomène bien connu des modélisateurs en sciences humaines et sociales, à savoir l'équifinalité.

% La question des modes de constructions
%Une solution élégante à été proposé par plusieurs auteurs, sous la forme d'une famille de modèle.

\paragraph{Equifinalité}

On a vu au cours de notre argumentation que la recherche d'un réalisme structurel ne pouvait suffire à vérifier un modèle. Il n'est pas non plus possible d'obtenir, voire même de formuler, des critères quantitatif susceptible de permettre la mise en place d'une telle vérification terme à terme entre hypothèse et critères mobilisés. La mise au jour exhaustive et transparente des dynamiques animant la structure causale de nos systèmes complexes (qui ne serait donc plus complexe) par la calibration ou l'exploration complète des modèles, si elle était possible, n'enleverait également en rien la possibilité de valider les critères avancés, car un tout autre jeu d'hypothèses, d'implémentation d'hypothèses, de paramètres, de valeur de paramètre pourrait très bien conduire au même résultat.


\foreignquote{english}{\textcite{Oreske1994} crisply describe the problem: it is impossible to verify the representational truth of any model of an open system. There  is a many to one relationship between the structure of models and the behaviour they produce, so that many models can account for the same observed outcome. This is the equifinality problem. One common (incorrect) response to the problem is to examine the internal consistency of the model, and to assume that internal consistency guarantees a true representation of reality.} 

Cette équifinalité, quant on la regarde sous cette forme, peut être à la fois considéré comme une limite dans l'établissement de vérité, voire une faille exploitable par les critiques de la simulation. Seulement, comme on a déjà pu le préssentir, la question de la preuve ou de la vérité n'est pas au coeur des préoccupations des chercheurs en sciences humaines et sociales, qui font appel à la simulation pour une tout autre raison. 



La recherche 

Il est en effet impossible de prouver qu'il n'y a pas un tout autre ensemble d'hypothèses et de fait stylisés qui puissent être mobilisé pour rendre compte d'un phénomène. 

L'équifinalité est donc à ce titre une limitation indépassable à la connaissance qui peut être déduite de nos modèles. Pourtant, si on reprend l'objectif avancé par \autocite{Varenne2014},  \enquote{[...] la fécondité propre à la géographie de modélisation contemporaine et à ses différentes formes de manifestation tient en grande partie à sa capacité à affronter cette question de la sous-détermination, à comprendre qu’il ne s’agit plus tant pour elle de chercher des théories que de développer des modèles aux fonctions épistémiques multiples.} 

L’existence de théories alternatives multiples est une constante dans l’histoire des sciences humaines. L'étude de l'objet social est un construit contextuel qui se nourrit d'une multiplicité des point de vues. C'est à ce titre que Jean-Claude Passeron \autocite{Passeron2006} nous met en garde contre une tentative de vérification des modèles qui serait décorrélée de tout contexte historique. Le terme \enquote{vérification} \foreignquote{english}{[...] stands for absolute thruth } \autocites{David2009, Oreskes1994} et se rapporte avant tout ici à la notion d'équifinalité \autocite{OSullivan2004} 

Pour lui le faillibilisme poppérien qui se cache derrière la méthode hypothético-déductive ne peut pas s'appliquer à la construction de théorie dans le cadre des sciences humaines et sociales. 

Toutefois il faut quand même accepter l'existence d'une base commune pour discuter de ces échanges entre la géographes et les autres disciplines, en posant collectivement la question de la \enquote{cumulativité} \Anote{pumain_cumulativité} des connaissances en sciences humaines et sociales. Comme l'indique \textcite{Pumain2005} dans un article dédié à ce sujet, \enquote{La condition indiquée par J.C. Passeron (\enquote{ la sociologie n’a pas et ne peut prendre la forme d’un savoir cumulatif, c’est-à-dire d’un savoir dont un paradigme théorique organiserait les connaissances cumulées }, 1991, p. 364) n’est-elle pas excessivement exigeante ? Les connaissances des sciences dites \enquote{ dures }, expérimentales, sont-elles vraiment organisées dans un même paradigme théorique ? [...] La multiplicité des contextes différents, dans l’espace et dans le temps, est aussi invoquée par J.C. Passeron comme un obstacle rédhibitoire à la comparaison des cas et donc à la cumulativité des connaissances.} 

Toutefois pour Denise Pumain, qui a déjà experimenté avec d'autres géographes la possibilité de ces transferts entre disciplines des sciences humaines et sciences  \autocites{Pumain1989,Sanders1992, Dastes1998}, il ne faudrait donc pas tomber dans un excès de relativisme tel que l'on trouve dans certaines postures postmoderne. Il est possible de travailler à la mise en place de méthodes \Anote{pumain_methode} propre à faire converger ces disciplines vers l'articulation et l'enrichissement de concepts, d'objets au travers de nouvelle grilles de lecture venant supporter la constitution d'un savoir, qui ne sacrifie si possible ni l'originalité, ni la diversité des points de vues engagés. Alors nous dit Denise Pumain, \enquote{Nous pourrions ainsi, tout en produisant des formalismes nouveaux, illustrer la question de la complexité d’une façon bien plus éclairante [...] La complexité d’une notion serait mesurée par la diversité des regards disciplinaires nécessaires à son élaboration, à l’intelligibilité des objets ou des processus étudiés, selon un objectif donné de précision des énoncés et des contextes}

Le modèle de simulation parait être un excellent support pour l'application et la discussion concrete autour de ces hypothèses, nouvelles, pouvant émerger de la mise en place d'un cadre commun. Les projets fortement inter-disciplinaire que sont par exemple Archeomedes, TransMonDyn, Alpage, ou GeoDivercity \autocite{Chapron2014} semblent tous démontrer quelle fertilité en terme de formalismes, de modèles de simulation, et de connaissances produites peut avoir une telle remise à plat.

L'etude de cette problématique de l'équifinalité à l'orée des débats ayant lieu dans une communauté inter-disciplinaire telle que celle gravitant autour du journal JASSS est également intéressante car elle introduit chez les sociologues un cadre pour penser la construction et l'évaluation des modèles, d'origines assez ancienne, qui intègre certains des éléments discutés précédemment : \enquote{les mécanismes générateurs}.

%Débat CONTE / EPSTEIN, et le retour aux mécanismes générateurs

Une entrée par les critiques récentes formulés sur ce cadre historique des \enquote{Science Générative} initialement formulé par Epstein est un bon exemple pour montrer que la prise en compte de l'équifinalité, à elle seule, n'est effectivement pas suffisante pour justifier de la crédibilité des modèles, et peux même dans certains cas fournir une base argumentaire qui permet de réduire la portée explicative des modèles et de décrédibiliser l'utilisation de la simulation en sciences sociales.

C'est la faille emprunté par \textcite{Yanoff2008} qui s'appuie sur le modèle des Anasazi pour proposer une critique générale des \textit{Artificial Societies}, un terme dont il faut dire par avance qu'il est désuet, étant donné la diversité de modèle opérant aujourd'hui dans la simulation en science et sociale.

Le modèles des Anasazi \autocites{Dean2000, Epstein2002} ne représente déjà à cette époque et en SHS qu'un type de modèle de simulation parmis une multitude. Le motto bien connu d'Epstein pour une \textit{generative social science} \foreignquote{english}{If you didn't grow it, you didn't explain its emergence} \autocite{Epstein2006} apparait par contre pour de nombreux modélisateurs comme une source d'inspiration, cela malgré son age et ses défaut, plusieurs fois analysés et cartographiés aux travers d'analyses de la dynamique interne \autocites{Janssen2009, Stonedahl2010, Schmitt2013}[151]{Schmitt2014}. Grunne-Yanof n'ignore probablement pas donc que lorsqu'il s'attaque à ce motto sur ce modèle assez symbolique, il vise en réalité une communauté et un spectre d'application de ce type de modèle beaucoup plus large. 

Ce défi que tacle Grüne-Yanoff sans vraiment le nommer, c'est l'équifinalité,  et plus précisément l'équifinalité telle quel est exprimé dans le cadre de cette science générative définis par Epstein.

Si Chattoe reconnait que l'existence d'un critère unique n'est effectivement pas suffisant pour juger de la qualité des hypothèses du modèle, l'attaque mené par Grüne-Yanoff sur ce point envers les Anasazi reste une attaque \textit{ad-hoc}, dont les conclusions ne peuvent en aucun cas être généralisé à la méthode utilisée pour construire les modèles de simulation en sciences humaines et sociales. Celui-ci ne faisant d'ailleurs dans sa démonstration aucun cas de l'existence d'une telle méthodologie sur lesquels les auteurs du modèle aurait pu se baser pour la construction du modèle \Anote{yanof_equi_a}, or celle-ci existe bel et bien dans les ouvrages de références, une erreur que \textcite{Chattoe2011} juge difficilement pardonnable lorsqu'on s'adresse ainsi à toute une communauté, avec son histoire, ses méthodes, ses codes, ses discussions, ses ouvrages et articles de références. 

La phrase d'\textcite{Epstein1999} \textit{If you didn’t grow it, you didn’t explain it.} est moins ambigue si on regarde le papier de clarification publié par l'auteur en 2006 : 

\foreignquote{english}{The scientific enterprise is, first and foremost, \textbf{explanatory} [...] If you didn’t grow it, you didn’t explain it. It is important to note that we reject the converse claim. Merely to generate is not necessarily to explain (at least not well). A microspecification might generate a macroscopic regularity of interest in a patently absurd—and hence non-explanatory—way. For instance, it might be that Artificial Anasazi [Axtell, et al. (2002)] arrive in the observed (true Anasazi) settlement pattern stumbling around backward and blindfolded. But one would not adopt that picture of individual behavior as explanatory. In summary, \textbf{generative sufficiency is a necessary, but not sufficient condition for explanation.}} \autocite{Epstein2006}

La générativité n'a jamais été pour lui une condition suffisante à l'explication, et l'équifinalité est un concept bien connu de l'auteur, qui renvoie pour cet effort de selection la balle à chacune des disciplines. 

\foreignquote{english}{Of course, in principle, there may be competing microspecifications with equal generative sufficiency, none of which can be ruled out so easily. The mapping from the set of microspecifications to the macroscopic explanandum might be many-to-one. In that case, further work is required to adjudicate among the competitors. [...] In any event, the first point is that the motto is a criterion for explanatory candidacy. There may be multiple candidates and, as in any other science, selection among them will involve further considerations.} \autocite{Epstein2006}

Que faut-il en retenir ? Tant que les modèles publiés ne montre pas plus d'efforts pour décrire à la fois les démarches de modélisations ayant permis la construction des critères et des hypothèses, et l'activité d'évaluation qui autorisent leur présences dans les modèles, le risque de voir ce type de publication se reproduire n'est pas écarté. 

Parmis les autres critiques de cette publication, on citera celle \textcite{Elsenbroich2012}. Celle-ci insiste à la fois sur le fait que les problèmes avancés par Grüne-Yanoff ne sont en rien spécifique à la modélisation multi-agents, mais revient surtout sur la partie explication avancé par ce dernier en lui donnant raison sur un point.

Elle est d'accord pour dire que la simulation multi-agents, pas plus que les sciences sociales, ne peux effectivement fournir de chaine causale complète prise au sens classique de la causalité. Toutefois, il existe selon-elle un autre cadre d'analyse qui permet aujourd'hui de dépasser cette limitation, et de produire quand même une explication, avec le transfert aux sciences sociales et à la modélisation multi-agents des thèses du biologiste Machamer \autocite{Machamer2000}

Avant de rentrer plus dans le détail sur ce cadre d'analyse, il semble que le point de vue d'Elsenbroich rejoigne donc la critique qu'a formulé Conte2007 à l'égard de la théorie d'Epstein lors d'une revue de son livre \autocite{Epstein2007} (faut il voir là une différence ancré dans l'histoire de ces deux courants simultanés, européen et américain ?). 

Par son acceptation des thèses de Machamer, elle rejoint de fait les partisant du courant de modélisateurs portant actuellement le cadre d'analyse dit des \enquote{mécanisme générateurs} \autocites{Hedstrom2010, Conte2007, Manzo2007}, s'opposant à la  \enquote{generative social science} \autocite{Epstein1999}  Pour \textcite[698]{Livet2014} c'est deux visions s'affrontent, mais sur quelle base exactement ? 

\enquote{Si une telle simulation « générative » peut être vue comme une condition nécessaire pour une science sociale computationnelle, elle ne suffit pas à fournir une explication ultime du phénomène. Tout d’abord, aux fonctions de la simulation doit correspondre un processus causal (Conte, 2007). De plus, ce type de modèle permet d’identifier un candidat explicatif pour ce phénomène, sans que ce soit nécessairement la seule explication possible, ni même forcément l’explication pertinente dans tous les cas de figure. La position extrême de Joshua M. Epstein a été critiquée pour la modélisation à base d’agents par Michael W. Macy et Andreas Flache dans leur ouvrage de synthèse sur la sociologie analytique (2009), où l’on préfère la notion plus large de \enquote{mécanismes générateurs}} \autocite{Livet2014}


Conte soulève deux critiques envers le cadre formulé par Epstein, pour elle : 
- il faut associer une théorie des causes à cette émergence au risque sinon d'obtenir une explication fausse ou ad hoc.
- il faut pouvoir reconstituer une chaine d'événément qui vont des causes aux effets, sinon l'explication par génération n'est qu'une simple reproduction de l'effet.


I will suggest that producing causes
and their link to effects must be hypothesized independent of generation: rather than wondering "which are the sufficient conditions to
generate a given effect?", the scientist should ask herself what is a general, convincing explanation, and only afterwards, she should
translate it into a generative explanation.

Ce que nous dit Conte c'est qu'il faut éviter à tout pris une explication ad-hoc; or dans le cadre prévu par Epstein, rien ne semble interdir la formulation d'une seule règle permettant dans son expression dynamique (growing) de reproduire l'explanandum; Ce n'est pas suffisant, il faut pour Conte que les causes avancées ne sont explicatives que si on a réussi à reconstituer la chaine de causalité complète qui va de cette cause à la production de l'événement. Autrement dit il faut éviter de mettre en oeuvre des règles qui n'apporte rien d'autre que la reproduction du phénomène, elle ne sont que des boites noires ou des raccourcis peu informatives.

They look for an informative explanation, which incorporates additional understanding of the level of reality that the phenomena of study belong to. In our example, this means an explanation adding further understanding of social individuals.

Generative explanation requires a theory of the causes from which to grow the effect, otherwise the explanation is irrelevant and a hoc.

Generative explanation requires a theory of the linked chain of events from those causes to effects, otherwise there is no
generative explanation but mere reproduction of the effect.

Il semblerait que ce soit les sociologues qui portent ce cadre d'analyse depuis un certain temps, au travers notamment des travaux de Coleman, mais aussi de Boudon. Quel rapport donc avec la biologie ? 




%- Les problèmes identifiés comme des problèmes de données liés aux ABM dans les Anasazi par Grüne-Yanoff sont applicables en réalité à toutes les sciences humaines : l'absence, l'incomplétude, l'incertitude des données, et l'impossibilité de mesurer des phénomènes empecherai l'obtention d'une chaine de causalité complète, l'inférence abductive et la possibilité d'une explication concurrente renvoie automatiquement à une explication causale partielle, et rend la possibilité d'une chaine de causalité complète impossible.
%- les hypothèses en entrée n'ont jamais été falsifié pour approcher les données en sortie, comme le suppose Grüne-Yanoff.
 

tout comme d'ailleur cela n'avait pas convaincu ...  qui voyait dans cet multiplicité de combinaisons possible un aveux de faiblesse dans la capacité causale de ces modèles. 



La levée de ce point montre bien à quelle point  \textcite{Yanoff2008} avait oublié de préciser que la construction et l'évaluation d'hypothèses de comportement crédible était clairement l'intérét du modèle Anasazi, et non pas juste la réplication \enquote{stupide} d'une régularité macroscopique par tout les moyens. 

Toutefois même évoqué ainsi, ce problème de l'équifinalité ne semble pas satisfaire \textcite{Conte2007}, 







L'équifinalité offre ce support pour confronter nos théories sur un objet social qu'il est impossible de tout façon impossible de voir dans son unicité.


Si on comprend les enjeux d'un tel projet, se pose alors les moyens de sa réalisation; la systématisation des évaluations avait déjà été annoncé comme un outil devant être mobilisé dès la pose des premières hypothèses, mais elle devient absolument nécessaire pour rendre cette fouille de modèles réaliste, et passé peut être à une échelle supérieure, celle de la construction et de l'étude de famille de modèles comme premier élément de réponse intégrateur de la pluralités des points de vues.

A ce titre, le recours au calibrage, et la recherche de cohérence interne dans les dynamiques pourraient passer pour une tentative de mieux définir par ce biais les processus en jeu dans un contexte réel. Pour \autocite{OSullivan2004} cet argument est encore un leurre, car toujours au vu de l'équifinalité, si ces procédures améliorent bien la connaissance du modèle, absolument aucune garantie ne peut être donnée sur l




\subsubsection{Les positions innovantes d'Hermann vis à vis de la validation}
\label{sssec:position_hermann}

\hl{Introduction sur l'apport d'Hermann dans le plan }

\paragraph{Une vision de la validation différente chez les pionniers du mouvement S\&G}

Charles F. Hermann opère dans la branche des simulations appelés à l'époque par Shubik les \textit{Man-Machine Games} \autocite{Shubik1972}. Une catégorie de simulation qui intègre dans son exécution un couplage entre un ou plusieurs systèmes numériques et des humains, qui peuvent être amené à interagir entre eux, ou avec les machines. Ce type de simulation de structure hétérogène est intéressante dans le sens ou elle permet d'intégrer l'arbitraire humain dans une chaîne d'interaction complexe qui n'aurait pas pu être établi autrement, du fait de l'impossibilité de programmer des interactions et des réactions humaines face à des situations précises. Même si ce type de techniques est motivé par une multitude d'usages, ce n'est pas par hasard si elle se développe particulièrement au cours de la guerre froide aux états-unis, toujours sous la direction d'institution militaires. Ce genre de techniques permettant par exemple de simuler et de reproduire des guerres au travers d'inter-relations diplomatiques et/ou économiques \autocite{Hermann1967b}, avec la possibilité de mesurer via des indicateurs adaptés l'importance et l'impact de différents scénarii sur le couple humain/machine.

Ce type de simulation est particulièrement représenté dans des publications qui traitent de la simulation au sens large, comme par exemple le journal \textit{Simulation and Gaming} ou \textit{S\&G} \autocite{Crookall2011}, dont l'activité remonte au début des années 1970. On retrouve parmi les auteurs ayant participé au développement de la discipline des personnalités importantes comme Guetzkow, Shubik, Coleman, etc. \autocite{Crookall2012}. Aujourd'hui, le terme à évolué vers ce que l'on pourrait probablement appeler des jeux sérieux, l'utilisation de l'ordinateur n'étant plus forcément un élément obligatoire dans ce type de simulation. Du coté des objectifs qui sont aujourd'hui susceptibles de motiver l'utilisation de ces techniques, \textcite{Shubik2009} définit une taxonomie en 6 objectifs : \textit{teaching, experimentation, entertainment, therapy and diagnosis, operations, training }

Cette présence d'une dimension humaine dans les simulations introduit une complexité qui touche forcément à plusieurs objets d'études des sciences humaines (psychologie, sociologie, etc.), et il n'est donc pas étonnant que l'on retrouve ce type de publication dès l'apparition des premiers ouvrages inter-disciplinaires sur la simulation, quand elle ne les pilote pas; Harold Guetzkow par exemple est un des personnages importants qui gravitent autour de Herbert Simon au GSIA (Graduate School of Industrial Administration) de Carnegie Tech dans les années 1950-56 \autocite{Guetzkow2004}, et qui a beaucoup oeuvré pour le développement de la simulation dans ces sciences politique et psychologique (Inter-Nation simulation lab) \autocite{Janda2011, Druckman2010}. Celui ci s'inscrit exactement dans la même branche que Hermann, et apparaît deux fois comme premier éditeur dans des recueils de textes pluri-disciplinaires traitant de la simulation au sens large, preuve aussi de son implication dans le développement et la diffusion de ces techniques au delà de sa propre discipline \autocite{Guetzkow1962, Guetzkow1972}

\paragraph{L'apport du contexte dans l'évolution du sens attaché à l'activité de simulation}

Ce qui est intéressant dans ce type de simulations, c'est qu'elle force à penser la validation des modèles sous un angle qui doit nécessairement tenir compte de la variabilité inhérente aux comportements humains, par essence difficilement évaluable et réplicable. C'est de cette contrainte, et parce que \textcite{Hermann1967} s'intéresse au modèle de simulation pour d'autres objectifs que la prédiction (\textit{teaching, training, theory-building}), que celui-ci développe à mon sens une vision de la validation beaucoup plus réaliste pour les sciences sociales que celle proposé à la même période par Naylor.

\foreignquote{english}{First, the validity of an operating system is affected by the purpose or use for which the game or simulation is constructed [...]}\autocite[217]{Hermann1967}

% Plus d'information à ajouter, soit sur la dite boucle (sachant que le conceptual correspond quand meme pas mal à ce que lon fait, voir Sargent2010), Si la boucle définit par les tenants de la \textit{V\&V} n'est pas inintéressante, et de façon générale résume bien le cycle de vie qui correspond à la construction d'une simulation, de nombreuses questions reste en suspens sur le choix et la mise en œuvre des techniques telles qu'elles sont décrites. La construction et la mise en oeuvre des critères en fait partie. Les objectifs sont cités dans la définitions mais on ne rentre pourtant pas dans le détail de la relation entre ces objectifs et la construction du modèle, qui est laissé à l'expertise de l'utilisateur, en cela Hermann ne propose pas mieux dans sa description d'une boucle modélisatrice que les dernières avancées portés par Sargent2010, toutefois sa réflexion est par son orientation, et par sa précocité de réflexion son intéressante il me semble à citer. les moyens technique de la mise en oeuvre par exemple ? 

%Dans l'explication sociologique, la réalité structurelle n'est pas forcément d'intérét pour la construction du modèle. (bulle)

%Cette observation amène Hermann à considérer que la validation des composantes de la structure mérite une attention tout aussi importante que la seule comparaison avec des données de sorties, notamment dans un cadre explicatif.  curl -k -o ~/backups/pinboard-backups/pinboard-$(date +\%y\%m\%d).json 'https://api.pinboard.in/v1/posts/all?&auth_token=username:APItokenhere&format=json'

En s'appuyant sur ce premier argument évoquant l'existence d'une dépendance liant processus de validation et objectif poursuivis par le modélisateur, Hermann semble \textit{de facto} mettre en défaut une définition de la simulation ayant comme première et unique vocation à représenter au mieux le système observé. 

On observe donc ici le tranfert d'une définition de la simulation comme simple \enquote{type de modèle} vers la définition plus générale d'une simulation \enquote{ caractérisée non pas tant par l’unité d’une fonction cognitive qu’elle assurerait toujours sous une forme ou sous une autre que par son fonctionnement interne, fonctionnement qui, bien sûr, mais seulement secondairement, se trouve avoir aussi des conséquences sur sa ou ses fonctions cognitives. Une simulation nous paraît ainsi devoir être prioritairement caractérisée par ce qu’elle est – ou fait – de manière interne plutôt que par ce qu’elle fait au sens d’une fonction cognitive quelconque qu’elle assurerait toujours et qu’on en attendrait prioritairement de l’extérieur : à ce titre, nous proposons de dire qu’\textit{elle est avant tout un traitement spécifique sur des symboles et qui prend toujours la forme d'au moins deux phases distinctes. 1) une phase opératoire [...] 2) une phase d'observation [...]}} \autocite[33-34]{Varenne2013}

La discussion sur l'existence de multiples objectifs de modélisation permet à Hermann de révéler la diversité et l'attachement de la validation à un contexte, et de noter d'une part comment la variation de ce dernier affecte les modalités de cette comparaison entre système simulé et système observé, et d'autre part comment cela affecte la perception du résultat engendré par cette comparaison.

\foreignquote{english}{The first comment is that the validation of an operating model cannot be separated from the purpose for which it is designed and used. [...] The second observation somewhat mediates the first. For the most part the various purposes for conducting games and simulations do not negate the need for criteria we can use to estimate the degree of fidelity with which one system (the operating model) reproduces aspects of another (the reference system). Given some purposes for using games and simulations (such as exploring nonexistent universes), finding appropriate criteria in the referent system is quite difficult. With other objectives, the value of the operating model may remain even if the fit between the model and various criteria representing the observable universe is poor (as in theory building).} \autocite[219]{Hermann1967}

\Anotecontent{naylor_etonnement}{On pourra peut être être étonné de retrouver la démarche de Naylor dans les approches subjectives sachant la description qu'on en a fait au préalable. Mais il y a bien une part de subjectivité dans cette démarche, l'application de chacune des étapes de la multi-stage validation faisant quand même appel à une forme d'expertise pour constituer le jeu des hypothèses que l'on estime valable en vue du test final de comparaison aux données.}

L'acceptation d'un gradient de valeur pour juger de la validation rompt avec la méthode binaire proposé par Naylor, la validation d'un modèle passant à présent par l'acceptation subjective d'un seuil de représentativité relatif à l'objectif poursuivis. Avec pour conséquence qu'une \foreignquote{english}{[...] simulation or game relatively valid for one objective may be not be equally valid for another.}

Si la notion de seuil n'est pas explicitement abordé par Hermann, c'est pourtant sous cette définition que la \textit{V\&V} actuelle va reprendre ce concept. Avec la position absurde suivante, celui de se fixer un seuil de représentativité général à atteindre \textit{a priori}, une contrainte peu réaliste dans le cadre des sciences humaines, ou quantifier un tel seuil n'aurait pas de sens. 

\foreignquote{english}{\textbf{Principle 2: The outcome of simulation model VV\&T should not be considered as a binary variable where the model is absolutely correct or absolutely incorrect } [...] The outcome of model VV\&T should be considered as a degree of credibility on a scale from 0 to 100, where 0 represents absolutely incorrect and 100 represents absolutely correct. 

\textbf{Principle 3: A simulation model is built with respect to the study objectives and its credibility is judged with respect to those objectives } [...]The study objectives dictate how representative the model should be. Sometimes, 60\% representation accuracy may be sufficient; sometimes, 95\% accuracy may be required depending on the importance of the decisions that will be made based on the simulation results. Therefore, model credibility must be judged with respect to the study objectives.}\autocite[15-16]{Balci1998}

La position de \textcite[166]{Sargent2010}, tout en étant relativement similaire, propose une vision plus fine et plus réaliste ou le seuil de précision attendu est attaché aux variables de sorties. Un point important sur lequel nous reviendrons plus longuement dans la suite de cette partie. \hl{ref vers la bonne partie}

\foreignquote{english}{A model should be developed for a specific purpose (or application) and its validity determined with respect to that purpose.[...] A model is considered valid for a set of experimental conditions if the model’s accuracy is within its acceptable range, which is the amount of accuracy required for the model’s intended purpose. This usually requires that the model’s output variables of interest (i.e., the model variables used in answering the questions that the model is being developed to answer) be identified and that their required amount of accuracy be specified. The amount of accuracy required should be specified prior to starting the development of the model or very early in the model development process.}\autocite[166]{Sargent2010}

Ces deux citations permettent de montrer au passage comment la vision de la validation défendu par Hermann a été intégrés dans une forme très approchante par des acteurs de la \textit{V\&V} comme Balci ou Sargent, dont on a vu précédemment les définitions dans la section \ref{ssec:def_generique_validation}. Ces deux derniers sont en réalité les acteurs majeurs d'une synthèse (voir la figure \ref{fig:S_syntheseBalci}) opéré dans les années 1980-1990 \autocite{Nance2002}, dont on peut dire qu'elle est marqué par un retour à une certaine forme de neutralité (voir par exemple le rejet des aspects philosophiques décrits décrits dans la section \ref{ssec:def_generique_ validation}  qui se double d'un jargon technique spécifique à l'établissement d'un processus qualité exploitable pour l'ingénierie) . Des adaptation qui permettent probablement de mieux accepter en son sein des typologies de techniques aussi différentes que celle de Naylor\Anote{naylor_etonnement} ou Hermann. Régulièrement révisées, \textcite{Balci1998} fait ainsi état dans sa dernière taxonomie d'un catalogue de 75 techniques différentes dans lequel peuvent piocher les modélisateurs en fonction de leur besoins. 

\begin{figure}[h]
\begin{sidecaption}[fortoc]{ On remarquera la forte présence des techniques présentés par Hermann dans la synthèse proposé par Balci en 1986 \autocite{Balci1986}}[fig:S_syntheseBalci]
  \centering
 \includegraphics[width=.9\linewidth]{subjective_balci.png}
  \end{sidecaption}
\end{figure}

Avec le constatation de cette variabilité dans l'objectif poursuivies par les modélisateur, dont dépendent la mise en oeuvre et l'évaluation de la validation, Hermann remet déjà clairement en doute l'existence crédible d'un critère de validation unique. Afin de montrer qu'il ne s'agit pas seulement d'une question de disponibilités des données, et pour amener par la suite sa proposition de méthode multi critères, Hermann s'attaque donc en premier lieu à réduire la portée des confirmations apportés sur un système observés par l'emploi de la seule technique de validation basé sur la comparaison de données en sortie des modèles de simulation.

\paragraph{Les limites d'une validation basé sur le critère de la comparaison historique}

Pour montrer qu'il existe des limitations dans la confiance que l'on peut mettre dans la validation lorsqu'il s'agit de comparer des données historiques (dans le cas des simulations de reproduction de guerre, on parle ici plutôt de reproduire des séries d'événements historiques) -cela même si elles sont idéalement toute rendu disponible- aux données en sorties de simulation, \textcite{Hermann1967b} s'appuie sur les travaux de \textcite{Pool1965}.

\foreignquote{english}{This correspondence does not demonstrate that the simulation correctly represents the structure and processes that were operative in the historical occurence. We are speculating on the similarity between the historical and simulated inputs on the basis of the similarity of their outputs. Different relationships among various combination of properties in the simulation conceivably could produces outcomes like those in the historical situation.

A simulation of the 1960 national Presidential election predicted the percentage of the vote for each candidate - the outcome - with considerable success. The designers of that simulation observe, however, that \enquote{it may legitimaly be asked what in the equations accounted for this success, and whether there were parts of the equations in the simulation that contributed nothing or even did harm} Further analysis of the equations in the simulation revealed that the outcome was predicted despite the fact that at least one equation misrepresented aspects of voter turnout. Part of the structure was incorrect, but the simulated result still matched the actual outcome. Despite this difficulty, our confidence that the simulation has captured some aspects of the voting process is greater than it would have been if the simulation had failed to replicate the campaign outcome. Confidence in the simulation would increase further as the operating model demonstrated ability to produce outcomes that corresponded with various elections. In sum, the similarity between simulation and historical events can provide at best only indirect and partial evidence for the correctness of the simulated structures and processes that produced the outcome.}

Ce que nous dit Hermann ici, à la différence de Naylor, c'est que même dans le cas idéal ou toute les données serait présente, ce mode classique de validation ne peut pas être suffisant, cela quelque soit l'objectif poursuivis par le modélisateur. Un constat que nous avions déjà acquis à la lecture des déboires des géographes avec les préceptes de validation néo-positivistes, associant dans une démarche de modélisation instrumentaliste prédiction et explication (section \ref{sssec:realite_neopositiviste}).

Ce constat reste encore valide aujourd'hui et cela indépendamment de la technique utilisé. Ainsi, \textcite[106]{Amblard2006} nous rappelle que dans le cadre des modèles agents, où le modélisateur cherche à évaluer la portée explicative de ces hypothèses, \enquote{[...] la recherche de similitudes avec les données, si elle peut être utile, ne peut absolument pas être un critère unique et définitif de validation}

\hl{Trouver une question ? ou faut il directement poser ici les hypothèses de la validation multi critères, tout en la discutant par la suite ? }

Pour comprendre en quoi la proposition d'Hermann est encore aujourd'hui une base pertinente et originale pour développer une réflexion sur la validation dans les sciences humaines, il est nécessaire d'évoquer et de discuter dans la suite de cette argumentaire les problématiques qui justifient selon lui d'adopter une validation multi-critère pour la validation : \foreignquote{english}{We have arrived at the position, then, that multiple validity criteria are needed because of the error of measurement and because of the recognition that criteria can be only assertions about \enquote{reality}}. 

Une reflexion qui se niche dans l'observation des modalités de construction des deux systèmes modélisé et observé, la question de la \enquote{représentativité} se présentant comme le résultat d'une confrontation entre ces deux objets, et cela au regard de l'objectif poursuivi par le modélisateur.

Les débats évoqués par la suite autour de cette problématique sont mis en parallèle des innovations apparu dans les années 1990, avec l'apparition et la diffusion des systèmes multi-agents (SMA) et des Automates Cellulaires (AC) comme nouveaux outils pour la représentation de dynamiques spatialisés complexes en sciences humaines et sociales (la quatrième vague d'innovation selon \autocite{Banos2013a}). Si cette technologie a indégnablement permis la levée de certaines barrières théorique et techniques en permettant l'intégration de l'hétérogène dans les modèles, tant du point de vue des échelles que des formalismes mobilisable pour la représentation des hypothèses \Anote{lena_bottomUp}, elle a aussi de fait participé à la complexification de cette question de la validation. \autocite[38-41]{Varenne2013} \hl{A détailler plus si j'ai le temps ... }

De cette variation dans la formalisation des hypothèses découlent des différences importantes dans les résultats, comme le prouve de nombreux travaux et publications étudiant ces transferts d'un formalisme à un autre tout en minimisant l'écart aux hypothèses. C'est une question qui s'est rapidement posé comme importante dans le cadre de la validation, l'\enquote{alignement de modèle} visant à établir quelle variabilité pouvait être imputable non pas aux hypothèses, mais à leur différece de support informatique. \hl{ref epstein}

Malgré cela, il me semble que les questions opérés en amont de la selection, de l'introduction et de l'organisation des hypothèses dans un réseau de causalité en partie support de l'explication reste quand à elles relativement indépendante de la technologie sous jacente. Ainsi le mode opératoire décrit par les pionniers réalisant le modèle A.M.O.R.A.L basé sur l'utilisation des systèmes dynamiques sont confrontés au même dilemme quand à la selection des hypothèses représentative qu'un modélisateur qui voudrait réaliser ce même modèle usant du méta-formalisme agent. \hl{A voir pour le muscler avec la boucle données -> modele -> données)}

\subsection{Le problème de la validation ramené à une confrontation des représentations entre système modélisé et système observé}
\label{ssec:confrontation_sysmodelise_sysobserve}


Mougenot2006 p100 pour une analyse de Machamer sur la diversité mécanismes générateurs

Busino2003 sur l'équifinalité et passeron

Intégrer l'observational dilemna comme une richesse (cf equifinalité) Busino2003 p60

Groupe de personne qui tente de remettre au coeur du débat la confrontation avec l'empirie et la nécessité de questionner et dessiner les types d'inférence rendu possible par la simulation : Boero2005, Manzo2005, etc.

\foreignquote{english}{A simulation or game is the partial representation of some independent system. Usually we are interested in simulation as a means for increasing our understanding of the system it is intended to copy. Therefore, the representativeness of a simulation or game becomes extremely important in assessing its value. The process of determining how well one system replicates properties of some other system is called validation.[...] In the present analysis however, validation will be defined more broadly as any comparison between the representation of a system and specified criteria} \autocite[216]{Hermann1967}

\hl{repetition ?}
La question de la représentativité d'une simulation est un sujet délicat à traiter car sa valeur se dessine à l'intersection d'au moins deux activités, la construction d'un modèle opérationel et la construction d'une grille d'évaluation, deux activités dont on s'apercoit par la suite qu'elles sont en réalité étroitement liées. 

\subsubsection{La construction d'un modèle de simulation opérationel satisfaisant : Quelles hypothèses pour quelle représentativité ?}
\label{sssec:hypothese_representativite}

La V\&V a toujours mis en avant le fait que la modélisation soit un processus incrémental tout à fait nécessaire pour obtenir un modèle de simulation satisfaisant, que cela soit dans les analyses de Naylor, ou d'Hermann. Ce dernier se réfère en 1967 au principe de parcimonie, une méthode qui implique une abstraction, une simplification du système à représenter, et qui pour lui met logiquement et automatiquement en péril la représentativité. \Anote{Herman_parcimonie} 

\hl{INTEGRER ICI LA REFLEXION HERBERT SIMON sur Hypothèse simplifié, mais également la réflexion d'arnaud et cie sur l'activité de construction => PAS SUR EN FAIT}

Une parcimonie hérité du principe d'Ockham dont on sait qu'elle n'est en aucun cas un synonyme de simplicité dans sa mise en oeuvre, celle-ci nécessitant au contraire un effort intellectuel important pour déterminer quelles sont les hypothèses réellement représentatives du problèmes à analyser. Sur le plan de complexité, Poincarré ou le prix nobel d'économie Herbert Simon à fait état plusieurs fois des capacités d'expression du complexe rendu possible par l'usage de la simulation .\autocite{Banos2013a}


Seulement, et compte tenu de ce qui a été dit auparavant sur l'importance de l'objectif dans la perception de cette \enquote{représentativité}, Hermann est aussi d'accord pour dire que cette dernière ne fait pas systématiquement la valeur du modèle - tant soit peu qu'on arrive à fixer une valeur -, comme dans le cas de l'explication, ou elle n'intervient que partiellement, tout en restant selon lui nécessaire (voir l'objectif \textit{hypothesis and theory construction}).

Sur ce point Hermann et par la suite les tenants de la V\&V ne nous en diront pas beaucoup plus, en partie du fait de la spécificité de l'activité de modélisation en science humaine et sociale, et en géographie, sur lequel nous allons nous pencher dans la suite de cette argumentation.

\textit{Que faut il entendre ici par partiellement ? Pourquoi doit on conserver une accroche avec le réel ? Quels sont les leviers permettant au géographe de compenser cette perte de représentativité par un gain en compréhension sur le système à étudier ? }

Il semble exister dans la volonté de construire un modèle explicatif deux attracteur possible et apparemment opposé, avec d'une part la volonté de se rattacher à une forme de réalisme au travers de l'injection d'une part maitrisé de réalité tout au long du processus de construction \Anote{durand_observation}, et de l'autre une force qui nous pousse à se détacher de cette même empirie en faisant un certain nombre de choix sur la nature et les interactions entre hypothèses constituantes du modèle.

La sociologue et épistémologue \textcite{Bulle2005} a bien formalisé ce dilemme dans la nécessité pour tout modélisateur de positionner son modèle sur un gradient opposant le réalisme des causes des modèles explicatif\Anote{bulle_modele_explicatif}, au réalisme des effets des modèles descriptif. 

Pour mieux comprendre comment se déduit l'explication d'un tel positionnement sur ce gradient, le mieux est encore de commencer par évoquer un des extremes, en invoquant par exemple le modèle universellement connu de Schelling. De par sa portée d'application extremement générale et la nature très abstraite de ces paramètres celui-ci constitue en soit un extreme intéressant pour comprendre ou se situe encore l'explication lorsque le détachement de la réalité est tel. Sur ce point, les analyses de \textcite{Bulle2005} et \textcite{Phan2008} se réfèrent principalement à l'essai de \textcite{Sugden2002} pour évoquer quel type de relation entre les deux mondes peut on attendre de ce type de modèle épuré. 

Les résultats qui dérivent de la mise en dynamique des règles dans le modèle de Schelling sont d'une telle universalité, d'une telle robustesse qu'il n'est même plus question de confronter ces résultats à une réalité. A cet égard le potentiel explicatif de ce type de modèle s'oppose selon \textcite{Bulle2005} à tout réalisme empirique. \hl{transition?}

De ce point de vue, \enquote{ le modèle n'est pas tant une abstraction de la réalité qu’une réalité parallèle [...] bien que le monde du modèle soit plus simple que le monde réel, celui-ci n'est pas une simplification de l'autre. Le modèle est réaliste dans le même sens qu'un roman peut être appelé réaliste [...] les personnages et les lieux sont imaginaires, mais l'auteur doit nous convaincre qu'ils sont crédibles } \autocites[131]{Sugden2002}[10]{Phan2008} 

L'effet d'une telle recombinaison d'hypothèses revient à mettre en oeuvre un \enquote{monde crédible} où l'inférence inductive est mobilisé pour identifier des similitudes significatives entre les deux mondes. \autocites{Livet2006, Phan2008}. Tout le travail réside donc dans l'interpretation prudente qui peut être faite entre ces résultats d'un monde factice et d'une réalité.

Un processus commun utilisé dans toute oeuvre de fiction pour piquer la curiosité du spectateur/lecteur, la mise en exergue volontaire de tendance du monde réel dans un monde imaginaire permettant d'entamer une réflexion sur l'existence, la portée, la nature de cette même tendance dans le monde réel. Les villes ou les sociétés mis en avant dans des oeuvres de fiction cinéma ou dans la littérature ne sont jamais que des mondes plus ou moins crédibles (Gotham City, 1984, Matrix, la série Black Mirror, etc. car la liste est longue ...)  pour mettre en avant un discours, ou des tendances du monde réel sur lequel doit porter le questionnement; (http://www.influxpress.com/imaginary-cities/ ,  \href{http://cybergeo.revues.org/1170#tocto1n9?}{cybergeo})

Si le discours scientifique n'a clairement pas cette obligation ludique, il n'en reste pas moins que ce processus de reconstruction crédible est un outil formidable pour questionner les processus à l'oeuvre dans le monde réel \Anote{ruffat_samuel_ville}. Mais cette ambiguité de lecture a déjà mené à de nombreux malentendu, d'une part envers le grand public (Voir forrester, mais également \Anote{deffuant_debat}) qui pourrait prendre des résultat de simulation pour la réalité avec tout les conséquences que cela suppose, mais également parfois entre scientifique provenant de divers horizons. Ainsi après la lecture de la critique par \textcite{Chattoe2011} de l'article de \textcite{Yanoff2009}, il ressort toute la difficulté d'évaluer la méthodologie et le travail réalisé autour d'un modèle au travers d'une seule publication, nottament lorsque la fonction cognitive recherché n'est pas vraiment exprimé, ce qui provoque aussi ce décalage entre attente du lecteur et processus réel de recherche qui sous tend la construction du modèle. \hl{Bof à reformuler}


\textit{Mais doit on se contenter d'aussi peu de certitude ?? Comment peut on renforcer la confiance dans la capacité explicative des hypothèses ainsi mobilisé ? }

\textcite{Bulle2005} evoque bien l'existence de modèle à cheval entre potentialité explicative et potentialité descriptive. Ainsi \enquote{appliquée aux processus sociaux réels, la simulation peut allier au potentiel descriptif offert par l’imitation d’effets empiriquement observables, le potentiel explicatif que lui confère la mise en œuvre de relations causales effectives. }

\hl{+ détail ici sur ce que cela veut vraiment dire ? }

La reintroduction de l'empirie dans les modèle de simulation autorise la mise en route d'un processus de validation, mais celui-ci se heurte rapidement à la différence de nature entre les résultats produits par des hypothèses \textit{reconstruite} et le monde réel. 

Car le résultat produit par cette dynamique est artificiel, et met en évidence l'apparition d'un nouveau niveau d'empirie comme lieu parallèle d'expérimentation, opérant dans un monde -in silico- en dehors de la réalité; ce qui amènent les épistémologues comme Varenne à parler ici d'\enquote{expérience concretes du second genre} faisant de la simulation une \enquote{quasi-expérimentation} \autocites{Phan2008, Varenne2007}

On en déduit que quelque soit notre placement sur ce gradient, il est vain de chercher à valider un \enquote{seuil de suffisance} caractérisant \enquote{l'injection de réalisme à atteindre qui autoriserai une inférence certaine sur le monde réel}, puisque de toute façon cette inférence s'appuie sur un résultat \enquote{artificiel} forcément discutable. \Anote{bulle_modele_autonome} \Anote{phan_livet_modele}   \hl{Equifinalité => proof of impossibility plus forte que proof of possibility ?}

Ce constat vient considérablement affaiblir la pratique de validation se basant uniquement sur la comparaison de résultat en sortie du système simulé et du système observé, et colle jusqu'à présent assez bien avec les remarques formulés par Hermann.

\hl{Note sur le laboratoire virtuel peut etre casé ici} ?? 

La confiance envers les capacité explicatives des hypothèses ne se jugent pas tant dans la comparaison des résultats avec le réel observé, mais dans l'exploration de ce monde simulé en fonction de critères observés dans le réel, dans l'espoir d'en dégager une connaissance qui doit encore être vérifié. Le problème est en quelque sorte inversé, ce n'est plus le réel qui est directement visé dans le modèle, mais le modèle qui est visé par notre compréhension du réel au travers de critères, de proxy, qui viennent questionner ce monde virtuel en lui imposant de nouvelle contraintes issue du monde réel, révélant par là même les forces et les faiblesse de nos hypothèses initiales. \hl{schéma?}

\hl{ faire remonter la question des critères et de leur représentativité ici }

A ce titre, et en s'inspirant de la remarque faites par \textcite{Bulle2005} à ce sujet, il sera toujours nécessaire et légitime de questionner la pertinence des rapports mesurés entre les liens causaux proposés et le ou les critères qui sont censés en rendre compte. \hl{A mettre en note} Cette remarque a un effet de bord intéressant, l'impossibilité de trouver des critères empiriques satisfaisant pour endosser tout ou part de la dynamiques exprimé par les hypothèses est un bon marqueur pour désigner la faible emprise du modèle sur la réalité. Cet effet déjà constaté par Hermann est très bien reproduit par l'observation du modèle de Schelling, dont la dynamique se prête très mal à une quelconque comparaison avec des critères empirique.

\subparagraph{Quel statut pour les hypothèses mobilisés ? } 

La coincidence avec des données historiques devient un critère parmis tant d'autres, et doit absolument être désacralisé du fait de sa faiblesse explicative lorsqu'il est mobilisé seul, au risque de voir émerger des polémiques comme celle opposant encore récemment \textcite{Yannoff2009} à \textcite{Chattoe2011} et \textcite{Elsenbroich2012} sur le très connu modèle de simulation des Anasazi.

Ainsi encore très récemment des auteurs comme Grune-Yannoff \autocite{Yannoff2009} ont tenté d'affaiblir la portée des explications possible par les modèle de simulation en science sociales en se basant uniquement sur la critique de modèle de simulation dites des \textit{Artifical Societies}, un terme inventé plus ou moins à la même période et de façon indépendante \Anote{wikipedia_convergence} par \textcite{Epstein1996} \Anote{epstein_artificial} et \textcite{Gilbert1995a} selon \textcite{Gilbert2000}. 

Utilisé principalement durant les années 1990 pour désigner les premiers modèles de  simulations orienté agent, un terme à présent beaucoup moins utilisé \autocite{Chattoe2011}

Le modèles des Anasazi \autocites{Dean2000, Epstein2002} ne représente, même en 2009, qu'un modèle de simulation dont l'objectif n'est qu'un parmis une multitude d'autre possible en SHS, nottamment si on se réfère à sa position \autocite{Schmitt2013} dans des classifications comme celle de \autocites{Banos2012, Banos2013}; d'autre part ce modèle, malgré ses défauts connus \autocites{Janssen2009, Schmitt2013}, continue d'être le support de discussions autour des mécanismes et de démarches innovantes visant à mieux cartographier sa dynamique interne, preuve que la tâche même de compréhension factuelle du modèle n'est ni facile, ni abandonné \autocites{Stonedahl2010, Janssen2009}

Les modélisateurs des sciences sociales se réfère aujourd'hui plus volontier à la devise sous jacente qui apparait déjà à la lecture du modèle Anasazi, le moto bien connu d'Epstein pour une \textit{generative social science} \foreignquote{english}{If you didn't grow it, you didn't explain its emergence} \autocite{Epstein2006}. Grunne Yanof ne peut donc pas ignorer que sa critique au modèle specifique des Artificial Societies vise en réalité un spectre beaucoup plus large de modèle mettant en oeuvre ce motto.

Pour comprendre cette polémique que l'on juge importante dans la défense de l'explication par la simulation en SHS il nous faut développer plus en avant les récentes réflexions autour de la nature des hypothèses, en replacant celle ci dans un contexte historique plus récent. 



1) Sans compter que l'injection de réalisme n'a pas vocation à être homogène, au contraire, et c'est bien l'avantage de ces mondes crédibles, c'est qu'ils autorisent la mise en oeuvre de raccourci autorisant la mise en oeuvre de dynamique intéressante tout en minimisant les perturbations. \hl{(pluriformalisation), a voir si ca reste ici}



DEUXIEME ASPECT : NECESSITÉ DE CRITÈRES, REPRESENTATIVITÉ DES CRITÈRES
%Plusieurs types de critères, des mesures, des seuils, mais également des patterns ou forme stylisés.

Conscient de ce problème Hermann propose de compenser l

pour lui tient dans donc la multiplication des points de vues sur le modèle, et la mobilisation d'un ensemble de critères venant de façon incrémentale contraindre positivement ou négativement la dynamique du modèle, nous éclairant en retour sur la pertinence des hypothèses en jeu.

Dès lors, le point de vue dynamique devient inévitable pour comprendre un jeu d'alternance jusque là éludé, l'alternance entre l'exploration et la conception du modèle. (Cf ce qu'a déjà théorisé Amblard)






TROISIEME ASPECT, DEPENDACE + INTERSECTION de la dynamique de construction, et de celle de la validation + spécialité de la géographie, et nécessité d'un nouvel axe, particulier et général

Les analyses comme celle tenu par \enquote{Bulle2005} donnent la clef d'une explication statique, or l'activité de modélisation tel qu'on a déjà pu l'aborder dans la section \ref{p:autre_def_modele} s'ancre dans une dynamique où le modèle n'est qu'un instantané de la reflexion.



dans la vie   comme d'un résultat d'une dynamique. Or,  sans indiquer comment ces modèles se construisent réellement.... 

=> Partir d'un modèle explicatif généraliste, mais peu intéressant d'un point de vue de l'inférence, pour le complexifier selon un scenario orienté vers un gain explicatif.



\hl{Mécanismes générateurs, je met ca ou ? }

Tout dépend de ce que l'on entend par explication, une des fonction première de la simulation résultant dans l'expression et la formalisation d'hypothèses qui amènent le modélisateur à formuler les zone d'ombres potentielles.

Epstein revient lui même sur une controverse quand à son motto \autocite{Epstein2006}, indiquant qu'il n'a jamais été question d'expliquer avec la mise en avant de la seule \enquote{générativité} des processus à l'oeuvre dans un modèle. 

la méconnaissance d ar les hypothèses mobilisés et les causalité mise en jeu, même maximisant une certaine forme de réalisme, ne seront d'une part jamais vérifiables (absence d'empirie, irréalisme), ni même justifiable (équifinalité), invalidant de fait la comparaison des séries de données produites par le modèle à des séries de données réelles. 

Encore une fois, c'est oublier que la validation des sciences humaines n'a la plupart du temps pas du tout pour objectif de privilégier le réalisme des effets au réalisme des causes, par exemple en biaisant des hypothèses pour arriver à ses fins, comme cela pourrait être le cas dans un modèle météorologique \autocite{Kuppers2005}. 

Il me semble que la lecture de deux débats permettent de situer l


Clairement la simulation ne saurait se résumer à l'émergence. 



=> Importance du protocole de construction (Varenne, Phan)

----

Sans compter en réalité 

Autrement dit, l’adéquation aux données de l’observation ne permet pas de juger la pertinence explicative des modèles, tandis que le réalisme causal des hypothèses tend à contrarier leurs potentialités descriptives.

Cela parce que le modèle est mobilisé gagne



>> Trajectoire ? Nécessaire si on veut evoquer l'évolution dans une dynamique. Hum ou je peux caser ca ...







Voir schéma de clémentine. ?

Se pose donc la question suivante pour le modélisateur, comment juge t il de la validité de son modèle une fois ses hypothèses arrachées à la réalité et transposé dans un monde reconstruit et abstrait ou l'objectif n'est plus la similitude avec le système observé ?

% arrrrrrrrrrrrrrr


De fait \enquote{ La question est alors de savoir s’il existe un critère permettant d’apprécier la justesse des relations causales mises en œuvre.} \autocite{Bulle2005}

% XAAAAAAAAAAAAAAAAAAAAAA

%Une des réponses tient dans l'analyse de \textcite{Phan2008} citant Sugden sur l'utilisation de la simulation pour construire des mondes crédibles.






Une façon de réaffirmer la spécificité de la géographie comme science spatiale et historisé; les hypothèses mobilisés s'inscrivant le plus souvent dans un espace et dans une temporalité donné, le choix de se libérer d'une dimension ou d'une autre étant lourd de sens quand a la représentativité du système modélisé face au système observé.

Un choix pourtant parfois nécessaire dès lors que l'on veut maitriser un scénario . 

Cela serait également oublier que l'activité de modélisation impose de suivre une trajectoire dans un espace où les hypothèses ne sont pas les seules amener à varier, les critères de l'évaluation devant également évoluer, au risque sinon de voir un décalage s'installer et la crédibilité du modèle s'effondrer.



Un décalage que Varenne expose dans son papier ... (à retrouver)


L'analyse \textcite{Bulle2005} sur cette thématique en sociologie est très éclairante. 

%Parmi les facteurs causaux mis en jeu par le modèle, certains représentent des causes réelles.

. même indépendante et donc tout devrait reposer sur les seules capacités d'inférences de l'observateur, comme le présente Livet ? Ou peut on aller un plus loin, et définir une grille de critères pour jauger de notre capacité à inférer, comme le suppose Hermann lorsqu'il met en relation  ? 

, , ne peux être jugé que partiellement par les seules hypothèses qu'il contient, et doit pouvoir s'évaluer en fonction de sa réponse à des critères d'évaluation, eux même représentatif du système observé.

Les hypothèses n'étant pas la réalité, et ne voulant pas être la réalité, alors la validation d'un modèle n'a de sens que face une grille d'évaluation adapté, qui permet la mise en confrontation toute relative des hypothèses avec d'autres hypothèses. (bulle ? )

\subparagraph{Le mystère de la V\&V sur cette question}

Dans la littérature de la V\&V, la selection des hypothèses pertinentes vis à vis du système observé est bien souvent cristalisé dans la notion de modèle conceptuel. (défaut à détailler) 

=> Pas valide dans le cas d'une complexification, on préfère partir d'une hypothèse nulle, et non pas d'un modèles aux hypothèses déjà pré-établi, dont on ne peux déjà plus dire grand chose. bof.


L'originalité d'Hermann réside dans ses remarques faites sur la relativité des critères, dont il est déjà conscient qu'elle ne sont que des assertions sur la réalité.  C'est sur ce point particulier que l'on va insister par la suite;


IDEE : PASSAGE DE MULTI CRITERE ON SUIT ENSUTE LE PLAN POUR ALLER VERS LIMITATION IMPOSÉ PAR LA SOUS DETERMINATION, ET L'ARGUMENTATION DE SULLIVAN

\subparagraph{Quelles critères pour quelles réalités d'un phénomène ? }

Cette notion de critères appelle de suite à poser la question de la mesure des phénomènes à comparer, et de leur traduction dans des critères représentatif du système observé.

\textcite{Hermann1967, Hermann1967b} propose d'établir non pas une méthode, mais une série de méthodes complémentaires, dont il détaille pour chacune d'elle les qualités et les faiblesses pour la comparaison entre système modélisé et système de référence. Chaque méthode constitue ce qu'il appelle un \textit{validation criteria} \Anote{methode_hermann}, un type de critère de validation générique dont le choix et la mise en œuvre effective est déterminé par le modélisateur en fonction des objectifs poursuivis.

% S'exprime dans une dynamique ? 
\subparagraph{La dépendance des hypothèses aux critères de validation}

%Ce degré de représentativité tel que définit par Hermann étant la mesure à un instant \textit{t} de la construction d'un modèle, de la réponse d'un jeu d'hypothèses selectionnés pour leurs potentialité supposés à satisfaire à un ensemble de critères. Des critères également selectionnés pour représenter au mieux un ou plusieurs aspects du système observé. \hl{Bof, à reformuler}

La présence d'une hypothèse dans le modèle est justifié tout à la fois par l'expertise du modélisateur que par son adéquation potentielle avec différents critères de validation. 

Adéquation potentielle car elle resulte de l'extraction du monde observé par les yeux de données modélisés, et d'une chaîne causale supposé. 


La subjectivité de l'expérimentateur joue sur les deux tableau, et donne à voir dans cette subtile inter-dépendance qui relie le choix des hypothèses et le choix des critères une forme incertitude quand au résultat assez difficile à prévoir et quantifier.

La fonction heuristique de la simulation pouvant s'exprimer tout autant dans cette \enquote{surprise} d'une divergence entre le potentiel investit dans les hypothèses et les critères selectionnés, que dans l'introduction de nouveaux critères remettant en cause ce même potentiel de représentation investit dans certaines hypothèses.

Pour donner un exemple plus parlant de surprise, le potentiel explicatif d'une hypothèse pourtant appuyé par des résultats empirique constaté dans le système observé pourrait tout à fait s'avérer invalidé par une analyse de sensibilité, alors même que l'experimentateur considère celle ci comme étant indispensable dans le développement d'une dynamique donné. \hl{peu clair}

Il y a une rupture opéré entre la volonté du modélisateur de rendre compte d'un système observé par un jeu d'hypothèse qui lui parait parcimonieux, nécessaire et cohérent d'un point de vue thématique, et la réponse effective apporté par la mise en dynamique d'un ensemble de causalités opérant dans un cadre fermé limité. Parmis les causes possibles de cette divergence surgit alors la possibilité d'affirmer de nouvelle connaissances, avec le développement de nouveaux critères, de nouvelles hypothèses ayant jusque là échappé à l'oeil du chercheur.

\foreignquote{english}{In all probability some distributions of events or some kinds of hypotheses will produce results with unacceptable divergence between the operating model and the observable universe. Although these incongruous may not pinpoint the inadequacy in the model, they should provide a diagnosis of the general area which seems unrepresentative.}



Hors si il est courant d'établir un modèle conceptuel pour cristaliser un jeu d'hypothèse à mobiliser dans une simulation, l'établissement d'un programme 


Pour ne rien simplifier, l'apport de formalisme hétérogène permis par les dernières techniques rend d'autant plus complexe l'évaluation d'un jeu hypothèses décrits dans des formalismes et des niveaux de généralités divers, et d'échelles variable et quelque fois dépendante.


%(1) The validation of a simulation or game is always a matter of degree. Moreover, a given operating model may be relatively more valid by some criteria than by others. 

On retombe sur les problématiques levés dans la section définissant le dilemme touchant l'approche de Forrester, la question de la représentativité des hypothèses devenant centrale dans ce questionnement.


% La question des modes de constructions



XXXXX



%PEU CLAIR : Second, model validation can be expected to vary according to the type of validity criteria used.C'est cette notion qui est appelés par Hermann dans le deuxième point de sa définition pour la validation, et dont on peut trouver ci dessous une expression qui établit le lien avec la problématique de la représentativité.


Si on en revient à la nécessité d'établir des critères 
%Second, model validation can be expected to vary according to the type of validity criteria used.




Comment qualifier alors la validité d'un modèle, et plus particulièrement les hypothèses que contient ce modèle ?

Hermann s'interroge à ce titre sur la nature et la crédibilité de la relation qui peut être tissé entre un système de référence (plus ou moins accessible) et les briques mobilisé dans le modèle lorsqu'il s'agit par exemple de développer avec un même modèle des scenario alternatifs tout aussi crédible les uns par rapport aux autres.


%Après discussion avec Clémentine il y a aussi le fait que les critères ne sont pas forcément connus à l'avance, et viennent contraindre le modèle au fur et à mesure de sa construction.

\autocite{Cottineau2014a}


ont on ne peux savoir si elle est lié à un différentiel de niveau d'abstraction, à un défaut d'implémentation, à un défaut de paramétrage.

Il ne s'agit pas forcément ici de porter un jugement de valeur sur les hypothèses, ou sur la pertinence de leur mobilisation compte tenu de la question posés


Du point de vue du modélisateur, quels sont les incertitudes révélés dans l'activité de construction ? Et comment peut on jugé de la validité des hypothèses dans l'encadrement L'établissement de la valeur d'une hypothèses face à un ensemble de critère d'évaluations, mais également celle qui juge de l'évolution


\paragraph{Un modèle de critère}

--- \hl{en cours de construction} ---

Cette logique soulève au final plusieurs questions :
> Comment jauger la valeur d'une hypothèse ? Avec des indicateurs, oui mais quelles indicateurs ?
> Comment

% Critères agissent comme une contrainte sur le domaine de validité exprimé, et pousse dans un premier temps non pas tant le modélisateur vers un degré de réalisme plus important, mais vers la découverte de zones de comportements qui permettent le retravail des hypothèses


% Multiplication critères va de paire avec le scenario poursuivie par le modélisateur,
% De la valeur des hypothèses mises en jeu ? Equifinalité, exploration ?
% scenario : complexification, simplification ?

% A developper ou pas ?

%On retrouve ainsi de façon implicite à son argumentation l'expression de cette difficulté pour le modélisateur d'atteindre cette mise en relation du modèle opérationnel (plus ou moins simplifié fonction de l'objectif poursuivis) et d'une réalité au travers l'établissement de critères objectifs, réalité dont on sais par ailleurs qu'elle est déjà déformé à la fois par la vision localisé de l'expérimentateur sur un phénomène et par celle du choix de la mesure, de la structure mobilisé pour le capturer.

Avec pour conséquence directe la nécessité d'une remise en cause légitime et permanente des inférences qu'il est possible de réaliser du modèle vers la réalité.

% Ouverture sur les patterns ?

\Anotecontent{hermann_doute}{Malgré le développement de ces différentes techniques, Hermann reste très prudent sur la possibilité d'inférer des conclusions à partir des simulations dans son propre domaine d'étude : \foreignquote{english}{Until more validation exercices are conducted, it is premature to accept or reject simulation as an important new tool for studying political phenomenon} \autocite{Hermann1967b}}

Sachant toutes ces limitations et la perfection de toute façon impossible, Hermann entretient toutefois l'espoir\Anote{hermann_doute}, par la mise en œuvre répétés de ces multiples méthodes qui guident et interrogent la construction du modèle au travers de perspectives différentes, de dessiner une carte relative de la confiance que l'on peut accorder à un modèle; cela toujours en gardant à l'esprit que ce résultat n'est pas généralisable, et reste lié aux objectifs ayant motivé la construction du modèle, comme le résume bien sa conclusion :

\foreignquote{english}{(1) The validation of a simulation or game is always a matter of degree. Moreover, a given operating model may be relatively more valid by some criteria than by others. (2) The validation of an operating model cannot be separated from the purpose for which it is designed and conducted. Therefore, a simulation or game relatively valid for one objective may be not be equally valid for another. (3) Given multiple validity strategies, several of the broadly applicable criteria may be reasonably applied in a particular sequence. [...] (4) The use of human participants in games significantly alters the required validation procedures. Although some major problems are reduced by this introduction of real properties, the net result would appear to make the estimation of validity more complex.} \textcite{Hermann1967}


\hl{ avec retour à la neutralité car on propose des techniques, comme analyse de sensibilité, et on affirme la aussi la dépendance du modèle à l'objectif poursuivi, mais on ne sais toujours pas quel est la valeur des hypothèses ... une telle approche se rapproche de la conclusion qu'on a pu tenir au début du chapitre, il n'y a pas vraiment de manuel autre que des bons conseils, bref, ici aussi on botte en touche conscient des limitations de chacune des techniques. (permet d'apporter la question de la sous détermination gentiment, en la présentant comme une richesse en science humaine)}

\hl{------------------------ en cours ------------------------}

Ici deux aspects important ne sont toujours pas traités, la construction du modèle comme processus historique lui aussi validable, la gestion de la sous détermination données / théories.


\paragraph{Le retour à la neutralité de la V\&V}


% Un constat effectif avec AMORAL + REMARQUE DE DENISE SUR FORRESTER
% + REPONSE A UN DES SEVEN SINS QUI ÉTAIT LE MODELE BLACKBOX

La naissance des systèmes dynamiques de Forrester allant de pair avec cette nouvelle méthode de construction des modèles autorisant la construction de structure causale beaucoup plus complexe que les précédentes techniques de simulation.

LeBerre1987 = Graphe causal ?

%Une critique qui tient à la structuration des modèles , notamment lorsqu'ils sont construit comme des systèmes faisant interagir des chaînes complexes de causalités, comme c'est le cas dans le cadre des systèmes dynamique ou des modèles multi-agents, dont le support conceptuel et formel est plutôt à trouver dans les outils du paradigme systémique.


=> Une des solutions on la vu poursuivis par les auteurs à été de se détacher de cette subjectivité sans toutefois la nier, en proposant une démarche théorique de construction de modèle qui délègue cette responsabilité au constructeur.

C'est du fait de cette contiguïté entre approche philosophique, et les approches pratiques de la validation qu'opèrent une relecture ou une appropriation des termes responsable de la plupart des ambiguïtés qui conduisent encore aujourd'hui à des débats terminologiques sans fin. \autocite{David2009}

Ces définitions apparaissent dans de nombreuses publications, toute disciplines confondues, y compris en géographie. Elles sont supposés offrir un cadre structurant et relativement neutre pour penser le processus de construction des modèles en général, et propose une terminologie suffisamment claire pour la mise en œuvre de pratiques standardisées.

Si l'approche plus récente de Sargent a certes permis de définir une démarche générique, elle exclue volontairement du débat le contexte subjectif de leur utilisation, et renvoie chaque discipline à l'explicitation de ses propres usages guidant l'avancement dans le processus incrémental de validation. \hl{Il en est de même pour la plupart des guides existant ...}

Mais cette approche de délégation, si elle a le mérite d'offrir un cadre structurant et neutre, qui est largement repris dans différentes disciplines, ne suffit pas. Car comme le disent bien ces auteurs, la validation est une étape incrémentale, qui s'effectue dès les premières itérations, ce qui renvoie dès les premiers instants le modélisateur à sa propre débrouillardise avec les outils, et laisse irrésolu tout les problème périphériques à cette mise en oeuvre... (cf faire plutot un rappel à la première partie sur la validation)

Il y a donc en permanence dans l'activité du modélisateur l'illustration de multiples tensions qui font de celle ci une expérience parmis d'autres, et nous rapproche déjà d'un point de vue plus proche d'une vision relativiste qu'objectiviste. L'historique d'un modèle se lisant tout autant au travers des choix d'hypothèses exercés par le modélisateur tout au long de son expérience de modélisation, que dans la lecture de l'objet finalisé. Une tension entre d'un coté la volonté d'expliquer des données par un ensemble d'hypothèses explicatives respectant un critère de parcimonie, et de l'autre coté cette volonté naturelle du modélisateur à tenter d'expliciter un maximum de cette variabilité vis à vis de la séries de données dont on dispose, et dont on sais par ailleurs que celle ci est déjà loin d'être neutre, exhaustive ou exempt d'erreurs.

=> Clementine avait une phrase bien pour ca ! (voir fiche)

Dialogue avec les outils
Dialogue avec les chercheurs
Dialogue avec l'extérieur
?

Ainsi dans le cadre de notre étude, le terme \enquote{vérification}  \foreignquote{english}{[...] stands for absolute thruth } \autocite{David2009} \autocite{Oreskes1994} et se rapporte avant tout ici à la notion d'équifinalité \autocite{OSullivan2004} En dehors de toute considération technique, cette équifinalité qui décrit le fait que m-modèles créés par les scientifiques peuvent représenter la même réalité ( ou modèle de la réalité ), est tout à la fois un moteur et une limitation dans notre capacité de construction des connaissances.


\paragraph{La limitation des approche en ingénierie pour la validation en science sociale}

= Si depuis les auteurs comme Sargent et Balci ont largement revu leur cadre d'analyses afin d'y intégrer d'autres techniques de validation,

Toutefois, et c'est sûrement là le prix à payer d'une telle généricité dans les termes, cette définition ne prend pas en compte le contexte d'application où opère cette validation, vérification.

Si ce qui compte avant tout c'est le contenant du modèle, alors il faut prendre en compte plusieurs limitations. La pluri-formalisation des modèles, la multiplicité des niveaux de généralités.

L'incrémentalité de la démarche ? (présente dans les définitions, mais se rapporte à un catalogue de test, voilà tout.)

Sans se raccrocher non plus à l'étiquette de relativiste, qui nous obligerai à nous couper de tout discours scientifique, la position défendue par Naylor parait encore plus intenable pour une application dans les sciences humaines et sociales.

Quand à la vision poppérienne, qui assimilerai le processus de validation des modèles à une démarche de falsification, même si elle est intéressante, nous parait la aussi incompatible avec l'acceptation de la pluralité des points de vues qui fondent le débat dans les sciences humaines.


mais également de façon générale en sciences humaines et sociales, dont on a bien du mal à imaginer qu'elle supporte un tel transfert de ces concepts d’ingénierie sans aucune transformation, un point détaillé par la suite.




une notion difficile à saisir du fait de son rattachement à un débat philosophique, nécessaire dès lors qu'il s'agit d'évaluer la connaissance produite par les modèles.

Ce rapport entre

En effet, la question de la \enquote{Vérification} des modèles, au sens philosophique du terme (valeur de vérité), reste indépassable du fait des multiples biais amenant l'observateur à toujours questionner la valeur de cette connaissance qui résulte d'un transfert entre les résultats d'un modèle volontairement imparfait (\enquote{simplifié}, donc réducteur par définition), et la \enquote{réalité} dans toute sa complexité \autocite{OSullivan2004}.

%ATTENTION, EXISTE AUSSI DANS LA PARTIE  1 EN C/C
L’existence de théories alternatives multiples est une constante dans l’histoire des sciences humaines. L'étude de l'objet social est un construit contextuel qui se nourrit d'une multiplicité des point de vues. C'est à ce titre que Jean-Claude Passeron \autocite{Passeron2006} nous met en garde contre une tentative de vérification des modèles qui serait décorrélée de tout contexte historique. Pour lui le faillibilisme poppérien qui se cache derrière la méthode hypothético déductive ne peut pas s'appliquer à la construction de théorie dans le cadre des sciences humaines et sociales. L'équifinalité est à ce titre un moteur permettant de confronter nos théories sur un objet social  qu'il est impossible de tout façon impossible de voir dans son unicité.

Le processus de modélisation apporte une dimension supplémentaire à l'analyse de chacun de ces points de vue.Car il est hélas impossible de prouver par les modèles qu'il n'y a pas un tout autre ensemble de fait stylisés ou d'interactions qui soit capable d'arriver à la même observation, enlevant de fait toute unicité d’une explication \enquote{scientifique} au point de vue représenté par le modèle. L'équifinalité est donc à ce titre une limitation indépassable à la connaissance qui peut être déduite de nos modèles.

espace paramètres !

Le terme \enquote{validation} quant à lui est souvent entendu pour définir un état qualifiant la correspondance entre des observations empiriques et les sorties de la simulation. Compte tenu de la notion d'équifinalité, cet état de correspondance ne suffit pas à prouver que le modèle représente bien la \enquote{réalité}, dans la mesure où l’unicité de cette adéquation peut être remise en cause par le jeu de nouvelles hypothèses.

\paragraph{Limitation ancienne}
Exemple de citation dans \textcite[192]{Sheps1971}, pumain82 qquepart, archéologue voir temps.txt et Lake2013,

De façon plus générique la percolation du concept d'auto-organisation dans les sciences sociales et en géographie permet il me semble de donner une définition plus générale de ce type de sous détermination comme résultat de l'étude d'un processus à l'équilibre (On parle ici d'équilibre d'état, mais éloigné de l'équilibre thermodynamique, dans un système ouvert, cf. \textit{steady state} de Prigogine) sachant que tout \textquote[Pouvreau2013, 114]{[...] processus d’équilibre peut être formulé téléologiquement [autrement dit] Toutes les lois systémiques ont la particularité que ce qui apparaît pour l’ensemble du système comme un processus causal d’équilibre peut être formulé téléologiquement pour les parties. Ce qui correspond à un processus causal d’équilibre apparaît pour la partie comme un événement téléologique, en ce que l’action de cette dernière semble dirigée vers le \enquote{but} consistant à prendre sa place \enquote{convenable} dans le tout}.

Peu importe donc l'étude de cette loi en tant que telle, puisque celle ci apparaît comme phénomène observable universel, ce qui intéresse le scientifique, ce sont les faisceaux d'hypothèses plausibles permettant d'approcher (ou pas, comme on l'oublie souvent, la négation est aussi explication !!) cette loi. La particularité de la géographie à ce niveau résidant avant tout dans sa capacité à maintenir ce faisceau d'hypothèse cohérent dans une diversités d'échelle et de temps, plus difficile à mobiliser dans d'autres disciplines.

Si on reprend l'objectif avancé par \autocite{Varenne2014}, \enquote{[...] la fécondité propre à la géographie de modélisation contemporaine et à ses différentes formes de manifestation tient en grande partie à sa capacité à affronter cette question de la sous-détermination, à comprendre qu’il ne s’agit plus tant pour elle de chercher des théories que de développer des modèles aux fonctions épistémiques multiples.} Si on comprend les enjeux d'un tel projet, se pose alors les moyens de sa réalisation; la systématisation des évaluations devient un outil au cœur de la construction des modèles, absolument nécessaire pour rendre cette fouille de modèles réaliste, et passé peut être à une échelle supérieure, celle de la construction et de l'étude de famille de modèles comme premier élément de réponse intégrateur de la pluralités des points de vues.

La notion de \enquote{laboratoire virtuel} traditionnellement limité à l'expérimentation du modèle mute, et se pare aujourd'hui d'une acception légèrement différente. Des chercheurs \autocite{Schmitt2014} \autocite{Amblard2003} ont voulu étendre cette notion pour y inclure également l'ensemble des méthodes et outils jugé nécessaire à l'étude de ce premier niveau d'expérimentation que représente la construction d'un modèle de simulation (la variation des hypothèses dans le modèle), désignant par ce fait un niveau supplémentaire d’expérimentation (la variation des outils et méthodes pour construire et étudier le modèle).

%\begin{quotation} In fact, utility of simulation is sometimes confused with validity. The one refers to its usefulness for some purposes, whereas the other refers to its degree of correspondence with the real world. Since utility requires some degree of validity, some authors speak of a model as having been \enquote{validated} by some use to which it has been put. Validity of a model, however, is not and end in itself but merely a means of enhancing the utility of the model – and usually only up to a point. Both validity and utility are commonly matters of degree. […] While validity is the ultimate test of a theory, the ultimate test of a model is its utility.  \\ \sourceatright{ \autocite{Guetzkow1972}}\end{quotation}

%Comme \autocite{Amblard2006} le propose, nous remplacerons donc le terme de \enquote{Validation}, qui prête à confusion, par celui d’\enquote{évaluation}, qui n'est pas sans rappeler la notion d'utilité telle que définie dans la citation ci dessus.

\subsection{La validation, l'expérimentation et le laboratoire}

\paragraph{Quelle validité pour l'analogie du laboratoire ?}

Dans le cadre de cette thèse, nous défendrons une \enquote{évaluation} de modèle qui se confond presque complètement avec la méthodologie de construction qui la soutient. Cette \enquote{ validation interne } doit selon nous être systématisée au regard de la \enquote{ validation externe } qui mesure classiquement la correspondance entre données simulées et observées face à la question posée. C’est en cela que la démarche que nous proposons est \enquote{ systématique }. Les opérations nécessaires à la \enquote{ validation interne } telles que l'introduction, la modification, ou la suppression d'hypothèses, s’effectuent donc à la mesure de leur apport qualitatif et quantitatif dans l'explication de la dynamique globale sur laquelle se fonde la \enquote{ validation externe }. Autrement dit, c'est la recherche d'une cohérence qualitative autant que quantitative de la dynamique interne qui nous guide dans notre recherche de correspondance avec les données observées.

A ce titre, le recours au calibrage, et la recherche de cohérence interne dans les dynamiques pourraient passer pour une tentative de mieux définir par ce biais les processus en jeu dans un contexte réel. Pour \autocite{OSullivan2004} cet argument est encore un leurre, car toujours au vu de l'équifinalité, si ces procédures améliorent bien la connaissance du modèle, absolument aucune garantie ne peut être donnée sur la qualité et la transférabilité de cette connaissance pour l'étude de processus réel. Cela est d'autant plus vrai lorsqu'il s'agit de système complexes, dont la nature même empêche toute  mesure des dynamiques à l'oeuvre lors des processus d'émergence, et rend donc discutable toute comparaison possible avec des dynamiques simulées.

\begin{quotation} It is clear that assessment of the accuracy of a model as a representation must rest on argument about how competing theories are represented in its workings, with calibration and fitting procedures acting as a check on reasoning. So, while we must surely question the adequacy of a model that is incapable of generating results resembling observational data, we can only make broad comparisons between competing models that each provide ‘reasonable’ fits to observations. Furthermore, critical argument and engagement with underlying theories about the processes represented in models is essential: no purely technical procedure can do better than this.  \\ \sourceatright{ \autocite{OSullivan2004}} \end{quotation}

% Un point de vue partagé par {Batty2001} ce qui permettrai d'introduire la notion de système complexe également !


\paragraph{Cout de l'évaluation}


\paragraph{Ouverture sur le collectif}

Ainsi plus que les solutions techniques, c'est dans le processus de discussion et d'échange autour des hypothèses admises dans les modèles que notre connaissance sur les phénomènes réels est amenée à progresser. Par la mobilisation, l'hybridation, la confrontation de modèles ou briques de modèle issues d'angles de vues inter-disciplinaires,  on met en œuvre une grande discussion à même d'éclairer cette dynamique globale qui serait de toute façon insaisissable dans sa globalité. {cf transcidisciplinarité de morin ?}

\autocite{Rouchier2013} s'appuyant sur une définition de \todo{Gilbert et Artweiler} décrit cette forme de validation basée sur la réutilisation et l'enrichissement collectif des modèles comme étant post-moderne, \enquote{ dans la mesure ou elle base la valeur d'un modèle au regard de son usage par une communauté d'usagers }. Il y a donc dans le processus d'évaluation des modèles de simulation une dimension collective qui ne peut plus être niés dans l'établissement d'outil et de méthodologie . De façon plus générale, \autocite{Rouchier2013} évoque et décrit bien dans un article récent \enquote{  Construire la discipline \enquote{ Simulation Agent }} la nature de ce mouvement structurant qui œuvre dans la construction de communauté scientifique. Celui ci prend forme autour de revues revendiquant une large ouverture inter-disciplinaire, tel que JASSS, qui font alors office de catalyseur en supportant, relayant ces discussions de fond, à la fois sur le plan méthodologique et technique.

Pour pousser l'analogie du \enquote{laboratoire virtuel} encore plus loin, il s'agirait alors d'ouvrir ce laboratoire aux autres scientifiques, d'en faire \enquote{place publique} afin de montrer l'histoire de nos protocoles, de nos modèles, de nos résultats \foreignquote{latin}{in vivo}, en assumant au passage toutes les contraintes que cela suppose. Dès lors, comment ne pas mettre en relation la complexification de cette représentation avec une épistémologie des pratiques du laboratoire tel que développés par Ian Hacking, ou Bruno Latour , et d'évaluer nos experimentation au regard d'un réseau de résultat cohérent, et non plus de théories dont on ne peut pas plus donner au final de réalité qu'à celle donnés à nos expérimentation ?

Si les débats sur le plan de l'analogie entre expérimentation réelles et virtuelles sont encores brûlant, un certain nombre de différence et de points communs ont déjà été assurés, et permettent de manipuler cette analogie avec prudence. Et nombreux sont les chercheurs ayant déjà suivis une voie similaire, replacant l'abduction et ses différents supports dans la construction et l'évaluation des modèles, et en acceptant au préalable les préceptes d'Epstein, dans son fameux if you didn't grow it you didn't explain it ... %% A developper.

Il s'agit maintenant d'explorer cette épistémologie qui remet au premier plan la démarche exploratoire et les outils qui la supportent, semblable en plusieurs points aux

Faisant cela, l'autonomie du modèle se diffuse à l'autonomie des démarches, des outils qui la composent, et des personnes qui les manipulent.

Une trajectoire des modèles déjà constaté dans nos pratiques de modélisation \autocite{Banos2013}, l'inter-disciplinarité inhérentes aux systèmes complexes cautionnant ces migrations pour éclairer des objets complexes à l'aube de cette diversité de points de vues, par l'emploi de nouvelle théories, de nouvelles échelles de temps et d'espace, et impliquant la transformation, au delà du modèle, de la démarche accompagnante qui permet son évaluation.

Quelques auteurs progressent sur cette voie en sciences humaines et sociales, mais cela reste des cas relativement isolés \autocite{Ngo2012} \autocite{Schmitt2014} \autocite{Heppenstall2007} \autocite{Stonedahl2011a} entre autres.

Dans sa conclusion \autocite{Rouchier2013} mise sur le développement de la crédibilité de cette discipline dans les années à venir, grâce aux revues, aux règles de conduites édictées, et aux modèles repris et discutés au cœur de cette communauté \autocite{Hales2003}.

%penser a faire un schema sous forme d'arbre a différentes racine, plutot vertical donc ....

%Au moins deux entrées epistémo pr repenser la pratique de l'évaluation :
%a) epistémo expérimenation interressante a aborder, car permet d'intégrer certains notions intéressante, comme l'autonomie des modeles, la reintroduction de l'experience face a la théorie, les style de pensée cumulatif qui rendent  compatible différente démarches, etc...
%b)la piste des mécanismes , avec filiation en biologie, refus de lhypthetico deuctivisme et l'absence de loi deductive, pont entamé par manzo, avec etude mot mécanisme qui peut etre prolongé par le papier quui différencie deux type demecanisme, et raccroche a la vision de la nouvelle biologie systémique en certain aspect ... introduire machamer et elseinbroch egalement ....
%=> Dimension collective supplémentaire a ces approches qui a elle seule ne font que définir une démarche de construction, qu'il faut rendre collective,  ce qui apporte contrainte supplémentaire ? (pas sur que ca soit au meme niveau en fait)


%Même si il est bon de garder une vision du futur optimiste du fait des avancés qui ont émergé des discussions ces dernières années, les problématiques que l'on rencontrent encore aujourd'hui dans le cadre de la simulation de modèles agents en géographie continue de faire écho à celles déjà mainte fois relayées par diverses publications ces dernières décennies\todo{ref JASS} \autocite{Squazzoni2010}  \autocite{Richiardi2006} \autocite{Windrum2007}. Sachant cela, il est difficile alors de ne pas sentir naître un sentiment plus mitigé sur cet avenir, car si la communauté n'arrive pas à dépasser tout ou partie des problèmes qui enrayent la diffusion des pratiques de simulation, comme cela semble être le cas, alors c'est toute la reconnaissance de ce champ comme une discipline scientifique à part entière qui reste limité.


\input{positionRecherche}

\printbibliography[heading=subbibliography]

\textbf{Plan}

Historique et Revue des pratiques existantes (chapitre 1)

Les fonctionnalités d’un laboratoire virtuel étendu (construction des modèles, exploration, visualisation) (chap 2)

SimpopLocal (calibrage) (chapitre 3)

MicMac (analyse sensibilité) (chapitre 4)

Conclusion

\appendix

\chapter{Historique du paradigme systémique}

\subsection{Retour sur la fondation et les apports du \enquote{paradigme systémique} au début du XXème siècle}
\label{ssec:systemique}

De la même façon que les épistémologues des sciences comme ici Olivier Orain \autocite{Orain2001}, l'auteur ne détaillera pas ici une approche inter-disciplinaire de la notion \footnote{Au sens donné par Piaget, voir note de bas de page \autocite {Orain2001}} de \enquote{système}, difficile à envisager dans un cadre global car sa diversité d'acceptation est fonction, d'une part de la rapide évolution de cette notion depuis les années 1940, et d'autre part la règle définissant l'acceptation de cette \textit{notion} dépend non seulement de la variabilité inter-disciplinaire, mais aussi intra-disciplinaire. Le terme \enquote{approche systémique} est alors proposé par \autocite{Orain2001} pour incarner cette diversité d'intégration par les disciplines des sciences sociales de la \enquote{théorie systémique} ou \enquote{systémique}.

La complexité d'approche caractéristique de cette notion est pour Jean Louis Lemoigne grandement lié à la reconstruction épistémologique \textit{a posteriori} de ce qu'il appelle \enquote{paradigme systémique}. Une acceptation qui parait d'autant plus justifié tant l'étude exhaustive de la ramification qui découle du concept est impossible, et sans rentrer dans les détails de querelles entre les différentes chapelles, il est acceptable de voir cette construction comme un processus de raffinement cumulatif. \hl{a dire mieux}

\subsubsection{La Cybernétique}
\label{ssubsec:cybernetic}

\paragraph{Des outils pour penser une nouvelle causalité}

Une des branches communément admises comme fondatrice du mouvement tient dans l'organisation des conférences de Macy entre 1942 à 1953. Celle ci sont considérés comme un des tout premier regroupement interdisciplinaire et marque une période de changement profond dans l'histoire des sciences en général, et particulièrement en science sociale. Celles ci vont réunir pendant plusieurs années autour d'une même table des acteurs majeurs des sciences physiques et sociales pour discuter autour de régularités communément observés, avec pour idée la construction d'un savoir commun que l'on pourra alors qualifier de trans-disciplinaire.

Les conférences naissent suite à la rencontre entre un mathématicien réputé au MIT N. Wiener, un neurobiologiste A. Rosenbluch, et un ingénieur électronicien J.Bigelow qui vont opérer un rapprochement entre l'homme et la machine entre 1942 et 1946 (pour rappel le premier ordinateur ENIAC est opérationel en 1946) par le biais de groupes inter-disciplinaires chargés d'explorer ce \textit{no man's land} à l'interface des deux disciplines.

Plusieurs \enquote{outils} dérivent de ces premiers séminaires organisés dès 1942 à la Josiah Macy, Jr. Foundation : la notion de \enquote{boite noire} ou système téléologique fonctionel, et la notion de \textit{feedback} ou causalité circulaire, avec pour objectif principal l'étude de l'homéostasie introduite auparavant par les travaux pionniers du physiologiste Walter Cannon en 1926.

Si la notion d'homéostasie pour des organismes vivants apparaît pour la première fois cité par Claude Bernard 1865, celle ci est reprise et étendue par Walter Cannon en 1932 dans le livre \textit{The Wisdom of the Body} \autocite{Cannon1932} comme « l’ensemble des processus organiques qui agissent pour maintenir l’état stationnaire de l’organisme, dans sa morphologie et dans ses conditions intérieures, en dépit de perturbations extérieures ». Ainsi dans le cadre de son application biologique cette rétro-action permet de décrire un certain nombre de mécanisme à l'oeuvre dans une cellule en interaction avec son environnement qui tente de maintenir de façon stable dans son milieu la concentration d'éléments comme les ions, la glycémie, etc.

L'attention des discutants dans ces premier séminaire porte donc avant tout sur l'ubiquité du concept et la pertinence de son transfert hors des systèmes biologiques. Wiener fait alors un rapprochement décisif entre les problématiques de calcul de trajectoire en balistique et des maladies nerveuses ayant pour symptôme l'ataxie. De ces discussions émergent alors un même schéma explicatif qui semble à la fois convenir à ces problématiques, la \enquote{causalité circulaire}. \autocite[774]{Pouvreau2013, Rosnay1975}

L'approche néo-béhavioriste retenue par les discutants \enquote{consiste à étudier un objet comme une \enquote{boite noire}, par l'examen de l'extrant de l'objet [i.e tout changement produit dans son environnement] et des relations entre cet extrant et l'intrant [i.e tout événement externe qui modifie l'objet]} \autocite{Pouvreau2013} En adoptant cette approche, le \enquote{comportement} d'une entité est perçu \enquote{comme tout changement extérieur détectable de cette entité par rapport à son environnement} , et par téléologique il faut entendre un comportement \enquote{finalisé} c'est à dire déterminé par un mécanisme de \enquote{rétroaction} négative. De la connaissance de ces entrants et de ces sortants, on peut en déduire qu'il existe une retro-action négative ou positive, ou \textit{feedback} permettant de décrire progressivement le système de commande de la boite noire.

L'introduction de cette \enquote{causalité circulaire} est pour l'époque loin d'être anodine car elle remet en cause le schéma classique linéaire cause \textrightarrow conséquence, qui se traduit dans le temps par la relation avant \textrightarrow après, la cause étant irrémédiablement suivi d'une conséquence. La possibilité de causalité circulaire, positive ou négative, brise ce schéma, et ne permet plus d'isoler un ordre entre cause et conséquence, c'est le problème de \enquote{la poule et de l'oeuf}. En réintroduisant la poursuite d'un but, on injecte une autonomie, une spontanéité, une dynamique entre objets qui était jusque là absente de la causalité linéaire déterministe.

Appliqué à un système servo-mécanique, la stabilité de celui-ci suppose la capacité à anticiper et à annuler les agressions extérieures par une capacité de régulation (flexibilité) qui repose plus alors sur la dynamique des interactions que sur la structure physique en place (rigidité), un mode de fonctionnement impossible si on se place dans le cadre de la \enquote{pensée classique} de l'époque.

%Dans "Behavior, Purpose and Teleology", le terme téléologie est à ce titre utilisé comme un synonyme de "l'objectif controllé par la rétroaction".\footnote{wikipedia}

\paragraph{La réintroduction du concept de \enquote{téléologie}}

Avec la mise en place d'une classification de ces comportements, et en prenant distance du concept de \enquote{causalité finale} qui lui était rattaché, les auteurs espèrent ainsi redorer le concept de téléologie, renouant avec la reconnaissance de l'\enquote{importance du but} qui avait disparu avec la mise au ban de ce concept. Reprenant les explications de \autocite[776]{Pouvreau2013}, celui-ci cite \autocite[23-24]{Rosenblueth1943} \enquote{[...] Puisque nous considérons la finalisation comme un concept nécessaire afin de comprendre certains modes de comportement, nous suggérons qu'une étude téléologique est utile si elle évite les problèmes de causalité et se limite à s'attacher à l'étude du but [...] Le comportement téléologique devient synonyme de comportement contrôlé par une rétroaction négative et gagne donc en précision par une connotation suffisamment restreinte.} La finalité est reintroduite via le concept de \enquote{téléologie}, mais elle est libéré de la notion de \enquote{causalité} qui lui était autrefois associé. Elle redevient l'étude des comportement associé à un but, dont l'importance ne peut plus être nié, et redevient compatible avec le concept autrefois opposé de déterminisme.\footnote{Pour donner un exemple peut-être plus parlant, l'étude en biologie des comportement oeuvrant dans la formation d'un organisme par une méthode téléologique n'empêche pas l'usage d'un cadre de pensée déterministe  correspondant à la formation d'un même organisme à partir d'un même code initial (un déterminisme largement remis en cause depuis, voir par exemple \href{http://www.nytimes.com/2014/01/21/science/seeing-x-chromosomes-in-a-new-light.html?ref=science&_r=0}{New York Times} )}

De ces discussions deux articles fondateurs à la fois des sciences cognitives \autocite[23]{Dupuy2000} et de la cybernétique vont être publiés : \textit{Behavior, Purpose and Teleology}ou Rosenblueth, Wiener, et Bigelow \enquote{ propose de déconstruire la distinction entre action volontaire et acte réflexe, en assimilant la volonté à un mécanisme de rétro-action (\textit{feedback})}; et \textit{A logical calculus of the ideas immanent in nervous activity} où McMulloch et Pitts donne \enquote{une base purement neuroanatomique et neurophysiologique au jugement synthétique \textit{à priori}, et de donner ainsi une neurologie de l'esprit}

\paragraph{ Les limites du transfert des concepts aux sciences sociales}

\subparagraph{Introduction aux sciences sociales}
Parmis les auteurs de ces premiers séminaires organisés entre 1942 et 1944 figurent deux représentant des sciences sociales, Gregory Bateson et Margaret Mead. Enthousiastes, il vont rapidement trouver dans l'étude des concepts développés dans ce premier séminaire (1942) un écho à leur propre travaux sur la dynamique sociale, la notion d'homéostasie n'étant qu'un nouveau mot permettant de rassembler des travaux existants déjà au fait de ces phénomènes. Cette mise au jour de problématiques commune entre le biologique et le mécanique permet d'envisager la construction d'un référentiel lui aussi commun; une prise de conscience qui va amener les auteurs du cercle de réflexion initial à envisager rapidement l'élargissement de celui-çi à l'ensemble des acteurs des sciences sociales.

La suite des conférences de Macy (1946-1952) sera organisés par Arturo Rosenbluch et son ami Warren McCulloch, un autre neurobiologiste. Cette ouverture vers les sciences sociales est timide dans un premier temps, et ce n'est qu'à la 2ème conférence en octobre 1946 sur une suggestion de Lazarsfeld que les conférences concrétise cette ouverture dans le cadre d'un sous séminaire intitulé \textit{Téléogical Mechanisms in Society}. La 4ème conférence acte cette ouverture et introduit pour la troisième fois de suite une modification de l'intitulé, avec cette fois ci l'adjonction d'une dimension sociale à un objet d'étude, qui apparaît encore à cette date difficile à définir : \enquote{la causalité circulaire et des mécanismes de \textit{feedback} dans les systèmes biologiques et sociaux}. Le terme \textit{Cybernetics} est pour la première fois introduit dans les séminaires par Wiener en 1946. Il faut toutefois attendre 1949 et la septième conférence pour que sous l'influence d'un nouveau participant nommé H. Von Foerster, ce terme chapeaute de façon définitive les prochains intitulés de séminaires. Au final, ces dix séminaires vont participer de l'émergence de la \enquote{science cybernétique} en \enquote{permettant l'échange effectif de savoir et d'experiences, tant entre les disciplines qu'entre les sciences et la société}, réalisant par là un des objectifs annoncé par Wiener et Rosenbluch dans leur classification, faisant de la cybernétique une \enquote{[...] science générale des systèmes à comportement finalisé ayant principalement pour objet ceux dont le comportements est \enquote{téléologique} } \autocite{Pouvreau2013}

\subparagraph{Des biais mécanisistes mettent en échec ce premier transfert}

Wiener mais aussi d'autre acteurs de la cybernétique ont vus assez tôt tout l'intérêt que pourrait apporter l'utilisation et le transfert d'outils comme \enquote{la boite noire}, ou le principe de régulation par \enquote{rétro-action} une fois appliqué à l'étude des interactions dans les systèmes sociaux. Mais les difficultés d'applications et les critiques ont rapidement mis à mal cet objectif trans-disciplinaire, pour plusieurs raisons qui tiennent : d'une part à l'existence de restriction mathématiques remettant en cause la scientificité des résultats obtenus : (a) les statistiques sur le long terme étant difficile à obtenir (b) la difficulté à minimiser la distance entre observateur et phénomène observés, et donc le biais qui s'applique aux données dans un tel cadre; et d'autres part au réductionnisme et le biais mécanicistes touchant la vision de certains acteurs des conférences de Macy  : \enquote{[...] la vie était pensée comme un dispositif de réduction d'entropie ; les organismes et leur associations, en particulier les hommes et leurs sociétés, l'étaient comme des servomécanismes ; et le cerveau comme un ordinateur} \autocite[784]{Pouvreau2013}

\autocite[782]{Pouvreau2013} explique très bien les limitations qui font  de l'extension de la cybernétique au sciences humaines une simple \enquote{[...] ressemblance superficielle au niveau du formalisme. Ne serait-ce que parce que dans un système tel que conçu par la \enquote{première} cybernétique, par définition fermé à l'information, la téléologie ne peut qu'être confinée au cercle d'un but déterminé; et que pour cette raison, ce modèle ne permet pas de comprendre de quelle manière un système peut être amené à redéfinir ses buts à partir de ses interactions avec son environnement, la pertinence d'une téléologie relative à des buts \textit{intentionels} restant donc intacte en sciences humaines}.

\subsubsection{La GST ou la théorie des \enquote{systèmes ouverts}}
\label{ssubsec:gst}

Cette incapacité de la première cybernétique à coller aux problématique des systèmes sociaux va trouver un écho plus positif dans un courant qui se développe en parallèle du mouvement cybernétique. Ce mouvement fondé par le biologiste Ludwig Von Bertalanffy en 1937 peut être considéré comme la deuxième branche venant enrichir le paradigme systémique. Tout en apportant de nouveaux concepts, celui ci va se positionner de façon critique par rapport à la \enquote{première cybernétique} tout en englobant par la suite les autres innovations qui proviendront de ce courant, Asbhy jouant le rôle important de médiateur entre ces deux courants.\autocite[]{Pouvreau2013} De cette prise de position va peu à peu découler la construction d'une théorie établissant une méthodologie logico-mathématique à vocation unifiante, accessible à n'importe quel champs disciplinaire pour décrire les lois de structure similaires (isomorphe). \autocite{LeMoigne2006a}.

Ainsi rapporté par LeMoigne en 1977, cette \enquote{vision stupéfiante est celle d'une une théorie générale de l'univers, du système universel} \autocite[59]{Lemoigne1977}. Le mot \enquote{Vision} est ici quasi synonyme de \enquote{Révélation}, car elle amène à voir une tout autre approche du réel pour qui s'en rapporte. Ainsi selon les mots même de Bertalanffy, \enquote{De tout ce qui précède se dégage une vision stupéfiante, la perspective d'une conception unitaire du monde jusque-là insoupçonnée. Que l'on ait affaire aux objets inanimés, aux organismes, aux processus mentaux ou aux groupes sociaux, partout des principes généraux semblables émergent} \autocite[59]{Lemoigne1977} \autocite[220]{Bertalanffy1949}. Une idée déjà existante dans la maxime célèbre de Claude Bernard en 1885, remise au gout du jour par \autocite{Lemoigne1977}, celle-ci résume toute la souplesse offerte par cette notion d'un point de vue de la modélisation :  \enquote{Les systèmes ne sont pas dans la nature mais dans l'esprit des hommes}

Cette théorie nommé \textit{General System Theory} (GST) est évoqué pour la première fois en public en 1937-38 par Bertalanffy, s'ensuit alors la rédaction d'une première ébauche en 1950, et il faudra attendre 1968 pour qu'un ouvrage titré \textit{General System theory: Foundations, Development, Applications} proposent une synthèse de toutes les avancées. La durée de développement de cette théorie n'est pas anodine, et si on en croit Pouvreau \autocite{Pouvreau2013} qui a analysé en détail la très vaste littérature associé à cette thématique, cette théorie n'en est pas vraiment une en réalité. En effet l'état inachevé du projet de Bertanlanfy laisse plus à penser qu'il s'agit là d'un \enquote{projet}, et c'est à ce titre que Pouvreau préfère employer le terme de \enquote{systémologie générale} pour désigner ce qu'il définit alors comme \enquote{le \textit{projet} d'une \textit{science de l'interprétation systémique} du \enquote{réel} } \autocite[9]{Pouvreau2013}. L'hypothèse défendu par Pouvreau étant que cette \enquote{[...]science de l'interprétation systémique du \enquote{réel} se caractérise en fin de compte comme une herméneutique, au sens où elle a pour vocation d'élaborer à la fois les moyens de construire des interprétations systémiques d'aspects particulier du \enquote{réel} sous la forme de modèles théoriques spécifiques et les moyens d'interpréter à leur tour de tels modèles comme des déclinaisons de modèles systémiques théoriques d'un degré de généralité supérieur.}\autocite[9-10]{Pouvreau2013}

Mais avant de même de fonder ce projet unifiant qui par la suite va rayonner et être absorbé (non pas sans déformation ..) dans un grand nombre de disciplines, dont la géographie, il est intéressant de rappeler comment la théorie biologique de Bertalanffy a participé de la formation de grandes notions comme l'\enquote{équifinalité} ou l'\enquote{auto-organisation}, des notions aujourd'hui communément admises comme fondatrice du paradigme actuel de la \enquote{complexité}.

Bertalanffy poursuivant depuis 1937 avant tout cet objectif de dépasser la compréhension des systèmes biologiques  englué jusque alors dans une dualité opposant les \enquote{vitalistes} et \enquote{mécanistes}. La synthèse de ces travaux est organisé dans une \enquote{biologie organismique} qui fonde une troisième voie visant d'une certaine manière la réconciliation entre les deux approches \autocite[55-56]{Lemoigne1977} \autocite[258]{Bertalanffy1949}. Avec cette nouvelle biologie théorique il s'agissait donc d'incarner \enquote{l'avenir de la biologie" en établissant via la mobilisation de moyen scientifique (analyse et analogies physico-chimique et mathématique du vivant) écartant la métaphysique/psychiques, un programme de recherche des \enquote{loi systémiques ou d'organisation à tous les niveaux de la nature vivante} entendues comme \enquote{l'explication de l'harmonie et de la coordination des processus à partir de la dynamiques des forces qui leur sont immanentes}}\autocite[456]{Pouvreau2013}. Principalement \enquote{ordonnées en direction de la conservation de la totalité}\autocite[440-458]{Pouvreau2013} dans une \enquote{tendance à une complication croissante}, cette \enquote{Gestalt organique} de la théorie \enquote{organismique} de Bertalanffy place \enquote{l'Organisation} des processus comme une véritable problématique de recherche, et met de coté la question de la \enquote{finalité} du vivant.\autocite[455-457]{Pouvreau2013}

Déjà tout à fait conscient que \enquote{le tout est plus que la somme des parties} Bertalanffy admet que l'étude des mécanismes physico-chimiques des processus vitaux tient plus d'une heuristique de recherche, une \enquote{méthode téléologique qui permet \enquote{d'examiner jusqu’à quel point le caractère de conservation de la totalité se manifeste dans les processus qui se déroulent en eux}} sans jamais arriver à en donner une complète description.\autocite[464]{Pouvreau2013}

Cette \enquote{biologie théorique organismique} (également appelé de façon synonyme par Bertalanffy \enquote{théorie systémique du vivant}) montre en bien des points toutes les prémisses d'une pensée systémiste et non réductionniste qui dépasse déjà largement le cadre seul de la biologie, et cela même avant 1937 et l'introduction de \enquote{systèmes ouvert} \autocite[499]{Pouvreau2013} qui ont fait la renommée de l'auteur.  Cette \enquote{biologie organismique} de Bertalanffy, bien évidemment construite sur les acquis et l'aide de bien d'autres de ces contemporains (voir \autocite{Pouvreau2013}, arrive à maturité en 1937 \autocite[14]{Pouvreau2013}, et présente déjà à ce stade tout les traits d'une première \enquote{systémologie restreinte}, qui va servir d'\enquote{antichambre} à la formation de la future \enquote{systémologie générale} (la première évocation publique date de 1945, mais des traces indirectes de ses premiers discours semblent remonter à 1937).\autocite[670]{Pouvreau2013} de Bertalanffy.

% D'abord on fait le point sur les principes (ce qui suppose de faire une grosse parenthèse avec tout ce que l'on a décrit sur la thermodynamique) et ensuite on peut passer à la critique, évoquant l'équifinalité et la hierarchisation de processus qui permet de recentrer aussi l'étude des boites noires.

L'articulation entre les deux \enquote{principes organismiques} qui fondent sa théorie apparaît de façon très claire dans une première définition du vivant en 1932, ici cité dans sa version telle que raffinée par Bertalanffy en 1937, date à laquelle selon

%Définition des deux principes organismiques !?

Le premier principe théorique \enquote{organismique} de Bertallanfy s'appuie sur le principe biologique fondamental qu'il a énoncé dès 1929 avec la \enquote{conservation du système organique en équilibre dynamique}. Un équilibre qui parait statique d'un point de vue extérieur, mais qui est en réalité dynamique car son existence même est basé sur la remise en jeu permanente d'une partie du travail effectué par la cellule pour maintenir le système organique loin de l'équilibre \enquote{vrai} (physique, c'est à dire celui qui correspond à une mort thermique, ou chimique qui ne peut pas produire non plus de travail à l'équilibre). Un \enquote{équilibre de flux} qui ne peut être réalisé que parce que l'organisme n'est ni un système fermé, ni un système statique, mais un système dont l'ordre et l'organisation (def à valider ici) est fondé sur un travail issue d'un \enquote{flux} de matière et d'énergie résultat d'une transaction à double sens avec son environnement. \autocite[472]{Pouvreau2013} Je me permettrai de citer ici Morin, qui reprenant Héraclite, évoque très bien cet antagonisme à l'oeuvre dans les systèmes organiques, mais aussi par extension sociaux \enquote{Vivre de mort, mourir de vie} : \enquote{ ne vivons-nous pas de la mort de nos cellules qui vieillissent et se décomposent pour laisser la place à des cellules jeunes ? [...] La vie et la mort sont certes deux ennemies fondamentales, mais la vie lutte contre la mort en utilisant la mort. Néanmoins, il est tuant de se régénérer en permanence. C’est épuisant. Finalement, on mveurt à force de rajeunir. On meurt de vie. } \autocite{MorinXX}

% Critique cybernétique
Le principe d'\enquote{équilibre des flux}, même si il peut être rapproché du concept d'\enquote{homéostasie} définit par les tenants de la \enquote{première Cybernétique} (en analogie avec les systèmes mécaniques) comme la \enquote{conjonction des processus par lesquels, nous autres, être vivants, résistons au courant général de corruption et de dégénérescence} est trop généraliste pour application en tant que tel à toute les notions de régulations organiques. \autocite[194]{Morin1977} \autocite{Wiener1950}. L'\enquote{homéostasie} tel que définit par Wiener dans le cadre de la Cybernétique s'avère en réalité être un mécanisme de régulation organique parmi tant d'autres, tous n'étant pas basé sur le schème de rétro-action. A ce titre, la notion d'\enquote{homéostasie} pourtant quasi semblable dans sa définition à l'équilibre de flux dans un système ouvert, mobilise en réalité un tout autre fonctionnement que le schème de rétro-action Cybernétique, et tient plus de l'extension aux systèmes ouverts du principe dit de \enquote{Le Chatelier}. De la même façon la régulation intervenant dans le processus de croissance des organismes qui nécessite la régénération, et l'évolution des structures dans le temps n'est pas compatible avec l'ordre structural pré-établi des machines et le scheme de rétro-action promis par la Cybernétique. La vision \enquote{machinaliste} limité/biaisé des premiers cybernéticiens n'est donc pas satisfaisante pour une application aux systèmes organiques, dès lors qu'il faut accepter la constance non pas des structures mais des interactions entre les structures. Bertalanffy développe une classification plus complète de ces régulations qu'il considère selon le type de leur téléologie, et introduit le concept d'\enquote{équifinalité} comme téléologie dynamique moteur dans la construction et le maintien des systèmes organiques. Dans ce contexte, le principe d'équifinalité \autocite[131]{Pouvreau2013}, est ainsi évoqué pour la première fois comme la possibilité d'atteindre le même état finalisé à partir de trajectoires quelconques, un processus impossible dans le cadre de système fermé où les condition initiales définissent par avance l'état final. Ce faisant, Bertalanffy introduit la primauté de l'ordre dynamique sur l'ordre structurel et fait de l'équifinalité un concept qui dérive de l'ouverture des systèmes. \autocite[489]{Pouvreau2013} \autocite[647]{Pouvreau2013} Un exemple illustrant les effets de l'équifinalité dans les organismes vivants peut être montré avec le processus de division embryonnaire. Ainsi un organisme a qui ont impose la fragmentation, la régénération, ou des blessures d'unités biologiques élémentaires comme les gènes ou les chromosomes va de façon constante s'organiser suivant un plan pré-établi menant à la \enquote{constitution d'un tout}, autrement dit un organisme complet.

%Il nécessite un autre mode d'explication de processus téléologique, celui de la cybernétique s'avérant incompétent au regard du principe d'équifinalité observé dans les systèmes organiques.

% Bertalanffy s'appuie dans sa critique à raffiner sa classification des téléologies, ce qui lui permet d'introduire le concept d'équifinalité comme sous-type de téléologie dynamique, un type de processus de régulation qui selon lui ne peut pas être expliqué par les schèmes cybernétique initiaux, seulement capable de mobiliser le concept de finalité en regard d'une explication basé sur un arrangement structural pré-établi (une machine faites de composants) et non pas l'ordre  dynamique propres au système en équilibre de flux.

La combinaison des deux principes \enquote{organismique} menant à la théorie des \enquote{système ouvert en équilibre de flux} deux heuristiques de recherches \autocite[481]{Pouvreau2013}:
\begin{itemize}
\item La subordination du \enquote{principe de hierarchisation} à celui du \enquote{système ouvert en équilibre de flux}, autrement dit la genèse et le maintien de l’ordre hiérarchique d’un \enquote{système organique} est conditionné par l'existence d'un \enquote{système ouvert en équilibre de flux}
\item  La relation précédente est un principe ubiquitaire s’appliquant à tous ses niveaux
\end{itemize}

Cet idée sera particulièrement fructueuses une fois articulé avec le principe d'un enboitement des systèmes, l'accroissement du degré de liberté dans un système résultant de l'équifinalité.
 \autocite[38]{Bertalanffy1973} \autocite[786-788]{Pouvreau2013}

%Developpement rendu possible uniquement par l'apport des théories de la thermodynamique ... l'expression d'une trajectoire indépendamment de l'état final, celui ci n'est qu'un processus de régulation parmis d'autres, car ce même système organique est non seulement capable de maintenir son état mais choses plus importante, il permet surtout de produire de l'organisation, de la complexification.

% Relation avec science sociale ??
% => entéléchie /
Cette notion d'équifinalité reliant un niveau micro à un niveau macro pourra par la suite être transposé dans les système sociaux, le parallèle de l'individu comme acteur réflexif dans la société sera mobilisé par ?

De ce fait la Cybernétique n'est pour Bertalanffy qu'un cas particulier dans une systémologie dont il pense qu'elle peut être beaucoup plus universelle... ++ Homéostasie avec Ashby ? ++

Tel que définie, cette notion d'équilibre dynamique de Bertalanffy est bien différente de celle produites en physique et en chimie, qui se caractérise justement par l'absence de travail disponible, l'énergie disponible étant minimale. Pour que la permanence d'un ordre puisse être effective dans la théorie organismique, il faut qu'il y ai un échange, un flux d'énergie mais aussi de matière possible avec l'environnement; une différenciation qui amène Bertalanffy à développer dès 1937 une théorie des \enquote{systèmes ouverts}, la seule capable de s'appliquer également à des systèmes sociaux par la suite.

% Sur l'ouverture des systèmes
Pour mieux comprendre en quoi cette ouverture est importante pour l'application du paradigme systémique aux sciences sociales, il faut revenir quelques décennies en arrière pour définir les limitations des premier systèmes issue de la thermodynamiques, limitations qui par la suite ont irrigués les réflexions initiales des cybernéticiens tout autant que les motivations de Bertalanffy pour les dépasser dans le cadre de sa théorie \enquote{organismique}

La seconde loi de la thermodynamique esquissé par Carnot et formulé par Clausius en 1850 montre que l'energie calorifique ne peut se reconvertir, elle se \textit{dégrade} et perd son aptitude à effectuer un \textit{travail}. Clausius nomme \enquote{entropie} cette diminution irréversible de l'aptitude à se transformer et à effectuer un travail, propre à la chaleur.\autocite[35]{Morin1977} Prigogine dans la \textit{fin des certitudes} écrit à propos de l'entropie qu'elle \enquote{[...] est l’élément essentiel introduit par la thermodynamique, la science des processus irréversibles, c’est-à-dire orientés dans le temps.}

C'est Boltzmann, Gibbs et Planck qui vont par la suite faire le lien entre le niveau micro des particules et la notion de chaleur. Parce que la chaleur est caractérisé par l'agitation désordonné des molécules dans un systèmes, l'entropie devient plus qu'une simple réduction du travail, c'est aussi l'ordre et le désordre des molécules qui en est la cause. Cette transformation s'effectue avec création d'entropie, une \enquote{quantité de désordre} qui ne peut que croître dans le temps et cela jusqu'à atteindre une valeur maximale équivalente à ce nouvel état d'équilibre. De ce fait et de façon générale celle-ci définit comme évolution irréversible toute transformation réelle dans un système isolé (Système où la frontière est totalement imperméable : l'Univers est par définition un tout englobant) ou fermé (Système ou la frontière est perméable aux flux entrant ou sortant d'énergie mais imperméable aux échanges de matière : La Terre reçoit de l'énergie du soleil) partant d'un état non stable et se dirigeant vers un nouvel état stable.  Ainsi si on considère l'univers comme un méta-système isolé englobant tout les autres, alors ce second principe a pour corollaire que l'entropie de l'univers augmente vers un état de désordre maximal qui se traduit en définitive par une mort thermique.

L'intuition de cette possible analogie entre loi gouvernant systèmes physiques et biologiques est issues des réflexions menés par Boltzman, qui comme ces contemporains du XIX siècle est admiratif pour la récente théorie évolutive de Darwin \autocite[27]{Prigogine1996}. Celui ci tente alors un parallèle avec ses propres travaux sur la seconde loi de thermodynamique, que l'on retrouve dans une des fameuses citations présente dans son livre \enquote{second law of thermodynamic} : \enquote{ The general struggle for existence of living beings is therefore not a struggle for raw materials — the raw materials of all organisms in the air, water and soil are in abundance there — nor about energy, which in the form of heat, unfortunately, is contained abundantly [but unfortunately] [in]convertible in each body, but a struggle for entropy, which is available [disposable] by the transfer of energy from the hot sun to the cold earth.}

% Le sys ouvert/fermé , de la thermodynamique à la biologie ?
Le point de vue de Boltzmann est repris et théorisé par Alfred J. Lotka, un mathématicien, chimiste et statisticien qui va largement influencé par la suite Bertalanffy dans la formation de sa \enquote{systèmologie générale} \autocite[178]{Pouvreau2013} par ces études de la démographie des populations et des flux de matières dans le monde biologiques \autocite[545-546]{Pouvreau2013} , toutes deux usant largement des équations différentielles (un premisse d'isomorphisme mathématique applicable à diverses disciplines pour qui quiconque tente de rentrer dans le formalisme de Lotka, et par la suite Lotka et Volterra \autocite[550]{Pouvreau2013}). De la même façon que Bertalanffy par la suite, celui ci ignore sciemment les débats entre \enquote{vitalistes} et \enquote{mécanicistes}, et adopte un point de vue unificateur qui vise la réconciliation entre système physique et système biologique, et part à la recherche d'isomorphisme en s'appuyant sur le processus d'irréversibilité commun aux deux paradigmes : \enquote{[...] the law of evolution is the law of irreversible transformation; that the \textit{direction} of evolution [...] is the direction of irreversible transformations. And this direction the physicist can define or describe in exact terms. For an isolated system, it is the direction of increasing entropy.  The law of evolution is, in this sense, the second law of thermodynamics} \autocite[26]{Lotka1925}.

Dès 1922 \autocite{Lotka1922a} \autocite{Lotka1922b} Lotka une nouvelle théorie qui acte la capacité de capturer de l'énergie comme un optimum à atteindre guidant la sélection tel quel est décrite par l'évolution Darwinienne. Il est également l'un des premier à percevoir les limites des lois actuelle de la thermodynamiques pour expliquer les processus du vivants, ainsi \enquote{Tenant pour légitime de traiter les êtres vivants et leurs associations comme des systèmes physiques, Lotka insistait toutefois sur le fait qu’il s’agit de « systèmes ouverts » aux flux de matière et d’énergie (ainsi que Raymond Defay (en 1929) et Bertalanffy (en 1932) les qualifièrent plus tard), capables d’échapper à l’équilibre thermodynamique défini par un maximum d’entropie promis aux systèmes fermés par le Second Principe, et d’évoluer vers une structuration croissante.} \autocite[179]{Pouvreau2013}

En effet pour un système vivant, l'état d'équilibre tel que décrit pour des systèmes clos ou isolé, correspond à un état de mort cellulaire. Hors, il est prouvé empiriquement à cette période que les systèmes vivants évolue dans un environnement chimique en perpétuel évolution loin de l'équilibre, et sont de fait capable de maintenir un haut niveau d'organisation par l'échange d'énergie et de matière avec l'environnement. Autrement dit, il n'est pas possible de concevoir l'équilibration permanente des systèmes vivants comme le résultat d'une évolution entropique croissante \autocite[248]{Lemoigne1977}. Des résultats énoncés sous forme de loi en 1929 par Bertallanfy, qui fait de \enquote{la conservation de système organique en équilibre dynamique} un \enquote{principe biologique fondamental}, et qui deviendra plus tard dans sa théorie \enquote{organismique}, le premier principe de  \enquote{système ouvert} en \enquote{équilibre de flux}. \autocite[492]{Pouvreau2013}

Mais en voulant faire l'analogie entre ces deux systèmes, une question va rapidement se poser aux scientifiques. \enquote{Comment la progression irréversible du désordre pouvait elle être compatible avec le développement organisateur de l'univers matériel, puis de la vie, qui conduit à homo sapiens ?}, une question qui va engendrer la problématisation et un changement de point de vue radical. Comme le résume bien \textit{a posteriori} Morin dans son premier tome de \textit{La Méthode}, \enquote{A partir du moment où il est posé que les états d'ordre et d'organisation sont non seulement dégradables, mais improbables, l'évidence ontologique de l'ordre et de l'organisation se trouve renversée. Le problème n'est plus : pourquoi y a-t-il du désordre dans l'univers bien qu'il y règne l'ordre universel ? C'est : pourquoi y a-t-il de l'ordre et de l'organisation dans l'univers ? } \autocite[37]{Morin1977}

Avec de tel propos se pose alors rapidement la question des mécanismes à l'oeuvre dans le vivant qui permettrait en quelque sorte de rétablir l'universalité de la seconde loi thermodynamique. Bien qu'intuité par de nombreux chercheur comme Lotka ou Bertalanffy, il faudra attendre les années 1940 pour que s'amorce plus concrétement ce rapprochement entre paradigme évolutionniste et domaine de la thermodynamique, concrétisé par le partage des théories entre biologistes et physiciens, qui va se réaliser notamment sous le couvert des récents progrès de ce dernier, permettant l'émission de nouvelle hypothèses.

Reprenant l'acceptation d'un système ouvert, c'est le livre \textit{What is Life} de Schrödinger (1944) qui va marquer le plus les esprits, et soulève le mieux ce paradoxe à la croisée des deux théories. Deux choses au moins fascine celui-ci \autocite{Foerster1959}, d'une part l'existence d'un code héréditaire qui définit au niveau micro la formation, l'organisation d'organisme au niveau macro (le principe \enquote{order-from-order}), d'autre part l'étonnante stabilité de ce code héréditaire immergé à 310 Kelvin \autocite[47]{Schrodinger1944}, et qui ne répond donc pas au fameux principe statistique \enquote{order-from-disorder} établit précédemment par Boltzmann.

En inscrivant comme nécessaire l'existence d'un code génétique comme un plan guidant l'évolution (tout comme Bertalanffy qui développe des théories similaires à la même époque), il introduit avec son concept de d'"entropie négative" un principe qui rend de nouveau compatible la seconde loi de thermodynamique avec l'évolution des systèmes biologiques : \enquote{le physicien attribuait le maintien de l’organisme dans un état \enquote{ stationnaire } éloigné de l’équilibre vrai à sa capacité de se \enquote{ nourrir } d’\enquote{ entropie négative } grâce à son ouverture sur son environnement. Une \enquote{ néguentropie } interprétée comme une \enquote{ création d’ordre à partir d’ordre } -- l’organisme créant un ordre spécifique à partir de la matière déjà ordonnée, structurée d’une manière déterminée mais devant être transformée pour ses besoins énergétiques, qu’il trouve dans son environnement} \autocite[502]{Pouvreau2013} Autrement dit, le maintien de l'organisation est un équilibre dynamique, un jeu à somme nulle où la création d'entropie est annulé par la capacité des organismes à transformer l'énergie, l'ordre puisé dans l'environnement pour maintenir ce degré d'organisation, un processus qualifié de néguentropique. Ce concept, déjà difficile à accepter tel quel dans sa généralité \autocite[225]{Lemoigne1977} va par la suite être raccroché à théorie de l'information de Shannon après son introduction en 1948 dans le microcosme Cybernétique. L'introduction de cette théorie étant un autre moment fort (avec la thermodynamique) ayant inspiré de nombreux développement dans la cybernétique. Mais les tentatives d'unification entre les deux théories débouche sur deux rapprochement possible, avec d'une part la qualification d'une \enquote{information pensé comme quantité physique} ou d'autre part l'expression des \enquote{quantité physique pensé comme de l'information}, selon que l'on adopte le point de vue de Wiener ou de Brilloin 1956 (auteur de la néguentropie qui associe qui associe \enquote{information} et principe de négentropie ). Ces points de vues font encore à l'heure actuelle l'objet de nombreux débats, certains voyant la physique de l'information comme un point de départ à creuser pour appeller une théorie de l'"organisation" \autocite[37-38]{Morin2005}, alors que d'autres n'y voient qu'un concept flou seulement basé sur la similitude des deux formules.  Autant de ramifications naissent de ces positions, et leur présentation dépassent de loin le seul cadre d'étude de cette thèse, mais le lecteur pourra se référer au travail de \autocite{Triclot2007} pour mieux comprendre le point de départ d'un malentendu qui dure toujours /footnote{Voir par exemple la différence de ton qui existe entre le site http://www.eoht.info/page/Information+theory, mais aussi les notes de bas de pages de \autocite[277]{Lemoigne1977} }.

\autocite[482]{Pouvreau2013} Mais finalement plus que les idées développés par Shrödinger, la plupart étant déjà largement sous entendu dans les travaux des biologistes de l'époque, il semblerait plutôt que cela soit avant tout ce nouvel éclairage physiciste apporté à la biologie {REF}, et l'espoir déguisé (finalement non réalisé) de trouver de nouvelles lois physique à l'oeuvre dans la construction du vivant associé à la grande diffusion du petit livre dans le grand public qui amèna peut être de nombreux physiciens à ne plus ignorer les avancés dans ce domaine, notamment durant les années 1940 / 50, tel que Prigogine \autocite[77]{Prigogine1996}, Von Foerster, etc. \autocite[73]{Lemoigne1977}

Mais conscient des manquements et des reproches faites à son approche, alors incomplète, car focalisé sur la cinétique, celle ci n'est pas relié à une théorie plus explicatives sur les mécanismes energétiques à l'oeuvre justifiant l'existence de ces propriétés des systèmes vivants dans le cadre des systèmes ouverts. C'est les récents développements sur la \enquote{Thermodynamique des processus irréversibles} qui va introduire a posteriori la possibilité d'une thermodynamique des systèmes ouverts compatible avec l'approche de Bertlanffy. Des physiciens ayant participé à ces travaux sur la thermodynamique des systèmes ouverts loin de l'équilibre (Osanger, etc.) c'est les travaux de Prigogine  en 1946 \autocite{Prigogine1946} qui vont le plus attirer l'attention de Bertalanffy. Lorsque celui ci découvre vers 1948 ces récentes avancées qui semble faire parfaitement écho à ces travaux ( Prigogine n'hésitant pas à citer Bertalanffy comme un de ses modèles d'inspiration \autocite{Prigogine1996}), le rapprochement se fait assez rapidement et Bertalanffy n'hésite pas à promouvoir cette nouvelle thermodynamique comme le parfait support physique justifiant des principes qu'il a établi dans sa propre théorie des système ouvert en équilibre des flux ! \autocite[653-658]{Pouvreau2013}

Pas étonnant donc de voir Bertallanffy s'appuie sur les écrits de Schrödinger pour re-formuler et préciser ses premières intuitions,
+

Malheureusement le \enquote{théorème de Prigogine} de \enquote{minimum de production d'entropie} ne s'exprime que dans des conditions semblent il très drastiques \autocite[53]{Lebon2008} et limité à des systèmes très proche d'un état d'équilibre tel que le prouve les travaux de Denbigh : \enquote{ It is possible that certain reactions in biological systems may be sufficiently close to equilibrium for the rate of entropy production due to them to be very small. But in general it seems that the notion of minimum entropy production has no real significance as applied to chemical reaction in open systems [...] it is incorrect to regard the tendency of an open system to approach a stationary state as being determined by thermodynamic factors. The stationary state may or may not coincide with a state of minimum entropy production, according to whether the rates of the individual processes are linear functions of thermodynamic variables. In the above we have assumed this to be the case for diffusion (eqn. (ll)), but it is known not to be true for chemical reaction.} \autocite{Denbigh1952}

Hors l'état des systèmes biologiques est semble t il loin d'être proche d'un état d'équilibre thermodynamique.. Bertalanffy qui jusqu'à présent se contentait de relier les résultats à son programme organismique ne cache alors plus sa déception lorsque en 1953 il écrit \enquote{Un minimun de production d'entropie ne caractérise donc pas l'équilibre des flux dans les systèmes ouverts [...]}; autrement dit \enquote{la thermodynamique [...] ne nous dit jamais ce qui peut se passer dans un système, ce qui est permis [...] Et le problème de l'organisation progressive, la tendance néguentropique de l'évolution des organismes simples aux organismes compliqués, reste à présent non résolu.} Bien qu'ils n'abandonne pas l'idée de voir expliquer un jour sa théorie organismique par une théorie thermodynamique adapté, il abandonne en 1953 l'étude de la biophysique des systèmes ouverts et se consacre par la suite uniquement à la construction de sa théorie du système général.

Le fait est qu'il y a réduction d'entropie dans les systèmes en équilibre de flux, et qu'il y a maintient et augmentation du niveau d'organisation, sans que l'on sache pourquoi pour le moment dans le monde du vivant. Si l'analogie et le pont entre tissé entre physique et biologie semble donc encore soumis à questionnement, les travaux de Prigogine sur la thermodynamique des systèmes ouverts va continuer quand a elle à ouvrir bien d'autres perspectives, notamment dans les systèmes sociaux.

%paragraphe dimension reflexive auto-orga ...
Elle dépasser largement ce cadre, et appuie sur des bases physiques le concept d'"auto-organisation", une notion déjà introduite dans le mouvement cybernétique par Ashby, un homme clef dans la convergence des idées entre Cybernétique et GST.

Ashby, tout comme Von Foerster interviennent dans la création de la seconde cybernétique, et introduise une dimension réflexive aux débats.

Inspiré par Von Foerster, vont alors introduire un autre concept \enquote{d'order from noise}, totalement différent du \enquote{order-from-disorder} de Schrodinger.

TODO : Partie plus axé sur les changements de causalité ? (vient avant ou apres ici ?)

L'équifinalité

Un autre concept important est introduit par Ashby dans le mouvement Cybernétique, le concept d'auto-organisation, l'introduction du mot \enquote{auto} amorcant ainsi un virage réflexif qui annonce la seconde Cybernétique, piloté par Von Foerster.


%Des auteurs comme Prigogine en 1947 >> clairement inspiré par bertalanffy/ Schrodinger...  cf Pouvreau et internet
%Il fait le lien avec processus physique =>
%http://www.informationphilosopher.com/solutions/scientists/prigogine/
%http://www.informationphilosopher.com/solutions/scientists/schrodinger/

%http://en.wikipedia.org/wiki/Entropy_%28information_theory%29#Relationship_to_thermodynamic_entropy

C'est également à cette époque, que relayant les premiers travaux de Prigogine sur les systèmes dissipatifs, Bertalanffy va catalyser ainsi ces idées dans sa GST.

Ce procédé sera transféré au réel par Ashby, un autre cybernéticien qui travaillera dès 1946 à la mise au point d'une machine expérimentale capable de reproduire de façon mécanique cette dynamique de stabilisation face aux variations de son environnements. Nommé \enquote{homéostat} celle çi sera construite en 1948, et présenté aux conférences de Macy en 1952.

WIkipedia => L'implication de la cybernétique dans la systémique est historiquement plus liée au « deuxième mouvement cybernétique ». En effet, si selon Norbert Wiener la cybernétique étudie exclusivement les échanges d'information (car c'est « ce qui dirige » les logiques des éléments communicants d'où le mot cybernétique), dans son évolution qui engendrera la systémique, on réintègre les caractéristiques des composantes du système, et on reconsidère les échanges d'énergie et de matière indépendamment des échanges d'information.

La dégradation de l'énergie nécessaire pour maintenir une organisation implique l'irréversibilité des transformations.


The history of an open system is part of its structure, and Prigogine links open systems to irreversibility. Prigogine calls open systems dissipative. Put more simply, this means that matter does not tend to organise itself in a particular location unless there is some external energy source powering it. Evolution can be seen as matter organising itself.


The term \enquote{self-organizing} was introduced to contemporary science in 1947 by the psychiatrist and engineer W. Ross Ashby.[9] It was taken up by the cyberneticians Heinz von Foerster, Gordon Pask, Stafford Beer and Norbert Wiener himself in the second edition of his \enquote{Cybernetics: or Control and Communication in the Animal and the Machine} (MIT Press 1961).

Self-organization as a word and concept was used by those associated with general systems theory in the 1960s, but did not become commonplace in the scientific literature until its adoption by physicists and researchers in the field of complex systems in the 1970s and 1980s.[10] After Ilya Prigogine's 1977 Nobel Prize, the thermodynamic concept of self-organization received some attention of the public, and scientific researchers started to migrate from the cybernetic view to the thermodynamic view. WIKIPEDIA


Malgré les critiques soulevés de part et d'autres, du faite entre autre d'un objectif peut être un peu sur-évalué voire immodeste, celle ci aura un large écho auprès des sciences humaines, et notamment en géographie; d'abord anglo-saxonne \autocite{Haggett1965, Chorley1962}, puis par diffusion en France \autocite{Raymond}.



L'avénement de la deuxième cybernétique :
La régulation apparaît en effet comme un phénomène majeur chez les organismes vivants, puisqu’elle « retarde la dégradation de l’énergie et donc l’augmentation de l’entropie » (p 129), et associée au retard d’entropie et à la computation, elles forment l’essence même de la cybernétique


\printbibliography[heading=subbibliography]

\stopcontents[chapters]
