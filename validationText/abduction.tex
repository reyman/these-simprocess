
La présence d'une hypothèse dans le modèle se justifie donc autant par l'expertise du modélisateur que par son adéquation, ou sa non-adéquation \textbf{potentielle} avec différents critères de validation. La subjectivité de l'expérimentateur joue sur les deux tableaux, et donne à voir dans cette subtile inter-dépendance qui relie le choix des hypothèses et le choix des critères une forme d'incertitude quant au résultat assez difficile à prévoir et quantifier.

Que se passe-t-il en effet lorsque le potentiel explicatif d'une hypothèse pourtant appuyé par des résultats empiriques constatés dans le système observé s'avère invalidée par une analyse de sensibilité ou un critère d'évaluation ? Et cela, alors même que l'expérimentateur considère celle-ci comme étant indispensable au développement d'une dynamique donnée ? Que se passe-t-il au contraire lorsqu'un critère d'évaluation est atteint alors que cela n'était pas attendu au vu de la structure actuelle du modèle ?

%La fonction heuristique de la simulation pouvant s'exprimer tout autant dans cette \enquote{surprise} d'une divergence entre le potentiel investit dans les hypothèses et les critères selectionnés, que dans la surprise suivant l'introduction de nouveaux critères contraignant le modèle, et remettant en cause ce même potentiel de représentation investit dans certaines hypothèses.

Il y a une divergence nécessaire entre la volonté du modélisateur de rendre compte d'un système observé par un réseau d'hypothèses qui lui paraît parcimonieux, nécessaire et cohérent d'un point de vue thématique (le potentiel investi), et la réponse effective apportée par la mise en dynamique de ces causalités lues au travers des critères selectionnés pour en rendre compte. % la possibilité d'infirmer ou d'affirmer de nouvelle connaissances, avec le développement de nouveaux critères, de nouvelles hypothèses ayant jusque là échappé aux raisonnement du modélisateur.

\foreignblockquote{english}[{\cite[219]{Hermann1967}}]{In developing a game or simulation, the designer is required to be explicit about the nature and relationships between the units in the operating system and their counterparts in the observable universe. He must specify the conditions which cause a relationship to vary. In constructing an operating model a connection between previously unrelated findings may be discovered. Alternatively, a specific gap in knowledge my be pinpointed and hypotheses required by the model my be advanced to provide an explanation.}

La surprise volontaire ou involontairement produite au cours de cette divergence, et qui accompagne généralement l'activité de modélisation, revient sous le nom d'abduction, le terme venant de Charles S. Peirce \autocites{Besse2000, Banos2013, Phan2006, Livet2014}.

On a déjà précisé en s'appuyant sur les propos de Ian Hacking \autocites{Hacking1989,Hacking2003, Hacking2006} que ce phénomène d'abduction n'était pas un héritage du seul cadre logique, mais une propriété inhérente à l'humain \Anote{stanislas}, existante de façon préalable à ses créateurs Aristote, ou Peirce.
%Cette logique bayésienne qui consiste à formuler une hypothèse a priori de façon consciente ou inconsciente \Anote{kauffman}, prise de façon rapide ou lente, basé sur nos connaissances passés ou sur notre environnement présent, pour la confronter et la réévaluer au yeux de la réalité de façon itérative correspond assez bien il me semble à ce que l'on pourrait apeller \enquote{abduction}, apellée également \enquote{inférence de la meilleure explication}.

Dans l'utilisation des modèles de simulation d'une part, ce n'est pas le monde réel qui nous surprend, mais ce qui se passe dans le modèle de simulation, et d'autre part il est plus intéressant d'adopter une démarche active et créative dans la mobilisation des hypothèses, afin de maximiser la surprise plutôt que de la miminiser, de dépasser son horizon de connaissance plutôt que de s'y conforter. Comme le résume bien \textcite{Banos2013} dans son HDR, \enquote{l’esprit même de l’abduction au sens de Peirce, désignant cette capacité de l’être humain à générer des hypothèses temporaires à partir de l’information incomplète dont il dispose. Appliquée à la démarche scientifique, l’abduction renvoie ainsi à la capacité du scientifique à se mettre en position d’étonnement, à se laisser guider par la recherche de l’inattendu et plus généralement à laisser libre cours à sa créativité.}

Comme on pouvait alors s'y attendre, les motifs développés par les modélisateurs soutenant cette approche sont très loin de ceux attendus pour la prédiction, ou c'est d'abord la robustesse des résultats qui prime, car \enquote{[...] dans le cas d’une modélisation compréhensive, où l’objectif est d’apprendre des propriétés du système que l’on reconstruit \textit{in silico}, le fait d’avoir, pour une partie des conditions expérimentales des comportements très erratiques du système, loin de discréditer le modèle nous apprend au contraire des éléments de son fonctionnement et nous permet même d’anticiper le fonctionnement du phénomène modélisé. Si sous certaines conditions le modèle est très instable c’est une information très enrichissante sur le modèle et sur le phénomène considéré.} \autocite{Amblard2010}.

Force aussi de constater que cette incrémentalité dans la construction d'un modèle de simulation ne suit pas vraiment un modèle linéaire de développement, et s'accompagne, au moins dans les sciences humaines et sociales d'une activité de raisonnement en partie imprévisible. Ainsi, il peut paraître paradoxal pour un modélisateur débutant de voir à quel point il est important de \enquote{malmener} les modèles que l'on a précédemment construits avec raison et parcimonie. L'important ici nous dit \textcite{Amblard2010}, c'est que cela participe à l'élaboration de la compréhension ou de l'explication des phénomènes considérés. Car, comme le dit \textcite[65]{Banos2013} dans son tout premier principe, modéliser c'est avant tout apprendre.

D'autres définitions de l'abduction \Anote{abduction_definitions} permettent de mettre en valeur d'autres de ces propriétés, la création en est une, mais avec celle-ci vient aussi la sélection. Si l'on se rapporte aux processus de cognition, la conscience apparaitrait par exemple comme un filtre discrétisant, sélectif, d'une pensée inconsciente fluctuante et résolument continue. On peut supposer que lors de la modélisation, ce type de processus est également à l'oeuvre, et il n'est pas rare lorsqu'on construit un modèle de simulation d'avoir à choisir, ou à ne pas choisir, avec plusieurs hypothèses ou implémentations d'hypothèses alternatives. Le deuxième choix expliquant aussi en partie pourquoi les interfaces utilisateurs de nombreux modèles de simulation Netlogo sont aussi riches en boutons de sélection...


%Cela soulève la possibilité d'hypothèse explicative concurrente ou inter-dépendante dans l'apparition d'un phénomène, dont certaine échappe forcément au seul modélisateur géographe du fait par exemple de la nature inter-disciplinaire des objets engagés, auquel il faut encore appliquer une selection plus consciente en décidant de mettre plus ou moins en avant des hypothèses susceptible de surprise.

%La surprise ne vient donc pas seulement du modèle, mais aussi de ce que font les autres manipulant les modèles, surtout dans le contexte inter-disciplinaire ou nous évoluons, les limites de nos connaissances se manifestant assez vite lorsqu'il s'agit d'étudier un phénomène ou un objet partagé, comme les villes par exemple.
