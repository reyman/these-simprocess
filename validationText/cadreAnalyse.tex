
%L'etude de cette problématique de l'équifinalité à l'orée des débats ayant lieu dans une communauté inter-disciplinaire telle que celle gravitant autour du journal JASSS est intéressante car elle introduit chez les sociologues un cadre pour penser la construction et l'évaluation des modèles, d'origines assez ancienne, qui intègre certains des éléments discutés précédemment : \enquote{les mécanismes générateurs}.

La critique de la simulation due à l’équifinalité par \textcite{Yanoff2008}  s’appuie ainsi sur le modèle des Anasazi \autocites{Dean2000, Axtell2002} pour proposer une critique générale des \textit{Artificial Societies} \Anote{desuet_as}.

\blockquote[{\cite[326-327]{Schmitt2014}}]{\textit{The Artificial Anasazi Project} a pour but de comprendre la trajectoire démographique des Anasazis, une société amérindienne ayant vécu dans la Long House Valley, au sud- ouest des États-Unis, durant la préhistoire (200 – 1400 ap. J.C.). Ce projet utilise un modèle multi-agents pour simuler le développement d’une société artificielle constituée de ménages amérindiens dans un environnement évolutif, caractéristique de la zone et de l’époque. Les sorties de ces simulations sont ensuite comparées à des bases de données empiriques archéologiques et paléo-environnementales.[...] Il existe deux grands types de dynamiques dans le modèle : celles de l’environnement et celles des agents. Les dynamiques de l’environnement sont simulées grâce aux inputs des données environnementales à chaque itération (représentant une année) de la simulation [...] Les agents représentent des ménages caractérisés par leur âge (à partir du moment de leur formation), leur taille (en nombre de personnes), leur consommation annuelle en maïs, leur capacité de réserve de maïs, leur localisation, et des indicateurs liés aux possibilités de mariage des femmes du ménage. Les règles d’interactions entre agents et avec l’environnement explicitent le comportement de choix de localisation de l’habitat et des champs cultivés par les agents. Une itération représente une année durant laquelle les agents cultivent leur parcelle. À la fin de cette période, les agents peuvent décider ou non de relocaliser soit leur habitat, soit leur parcelle de culture ou les deux, en fonction de leur capacité à couvrir leurs besoins nutritionnels cette année-là. La localisation simulée des habitats et des cultures est cartographiée année après année et comparée aux données.}

Ce modèle et sa dynamique ont été tant de fois répliqués, analysés et cartographiés que leurs défauts sont bien connus des modélisateurs. Malgré cela, ce modèle possède toujours une aura et une symbolique très forte, car il a ouvert la voie à d'autres modèles de simulation en archéologie, et en sciences sociales \autocites{Janssen2009, Stonedahl2010, Schmitt2013}[151]{Schmitt2014}.
Le \enquote{motto} bien connu d'Epstein pour une \textit{generative social science} \foreignquote{english}{If you didn't grow it, you didn't explain its emergence} \autocites{Epstein1996} reste aussi une source d'inspiration pour de très nombreux modélisateurs, malgré les critiques. Grüne-Yanoff n'ignore probablement pas que lorsqu'il s'attaque à ce modèle assez symbolique, il s'attaque aux pratiques de toute une communauté dont il ne fait \textit{a priori} pas partie. Une remarque de \textcite{Chattoe2011} qui peut paraître clanique au premier abord, et dont il faut aussi nuancer la portée, car on peut aussi voir dans cette critique la possibilité pour les modélisateurs d'améliorer les formes prises par cette communication autour des modèles, et des méthodologies associées. \textcite{Elsenbroich2012} résume le problème ainsi :

\foreignblockquote{english}[\cite{Elsenbroich2012}]{Grüne-Yanoff argues that this simulation does not give a causal explanation of the population dynamics of the Anazasi. The argument is not that the simulation is not an explanation of the Anazasi population but that it is not a causal explanation. The simulation does not tell a full causal history of the population dynamics, for example, the total exodus is not an outcome of any run of the simulation, thus leaving the explanation partial. Grüne-Yanoff argues that there is no such thing as a partial causal explanation. A causal explanation has to give a full account of all interactions leading to a phenomenon as there is no formal criterion to distinguish bad partial causal explanations from good partial causal explanations.}

Ce problème fondamental que tacle Grüne-Yanoff sans vraiment le nommer dans son texte, c'est l'équifinalité, et plus précisément l'équifinalité telle qu'on peut la comprendre dans ce cadre d'analyse qu'est la science générative définie par Epstein. Effectivement, même \textcite{Chattoe2011} qui critique de façon très virulente les propos de \textcite{Yanoff2008} doit bien admettre, avec d'autres, que l'existence d'un critère unique - qui plus est quantitatif - n'est effectivement pas suffisant pour juger de la qualité des hypothèses du modèle. Toutefois, comme il précise ensuite, l'attaque menée par Grüne-Yanoff sur ce point envers les Anasazi reste une attaque elle aussi \textit{ad-hoc}, dont les conclusions ne peuvent en aucun cas être généralisées à l'ensemble des modèles de simulation en sciences humaines et sociales. Celui-ci ne faisant d'ailleurs dans sa démonstration aucun cas de l'existence d'une méthodologie sur laquelle les auteurs du modèle auraient pu se baser pour la construction du modèle \Anote{yanof_equi_a}, alors que celle-ci existe bel et bien dans les ouvrages de référence \autocites{Gilbert1995a, Gilbert2005}. Une erreur que \textcite{Chattoe2011} juge difficilement pardonnable lorsqu'on s'adresse ainsi à toute une communauté, avec son histoire, ses méthodes, ses codes, ses discussions, ses ouvrages et articles de références.

Il suffit d'ailleurs d'évoquer cette question pour trouver dans \textcite{Epstein2006} une longue clarification de l'auteur au sujet de sa phrase si souvent reprise d'\textcite{Epstein1999} \textit{If you didn’t grow it, you didn’t explain it.}. Le constat qui en ressort semble accablant pour Grüne-Yanoff :

\foreignblockquote{english}[\cite{Epstein2006}]{The scientific enterprise is, first and foremost, \textbf{explanatory} [...] If you didn’t grow it, you didn’t explain it. It is important to note that we reject the converse claim. Merely to generate is not necessarily to explain (at least not well). A microspecification might generate a macroscopic regularity of interest in a patently absurd—and hence non-explanatory—way. For instance, it might be that Artificial Anasazi [Axtell, et al. (2002)] arrive in the observed (true Anasazi) settlement pattern stumbling around backward and blindfolded. But one would not adopt that picture of individual behavior as explanatory. In summary, \textbf{generative sufficiency is a necessary, but not sufficient condition for explanation.}}

\textbf{La générativité n'a jamais été pour lui une condition suffisante à l'explication}, et l'équifinalité est un concept bien connu de l'auteur qui renvoie pour cet effort de sélection entre les hypothèses la balle à chacune des disciplines.

\foreignblockquote{english}[\cite{Epstein2006}]{Of course, in principle, there may be competing microspecifications with equal generative sufficiency, none of which can be ruled out so easily. The mapping from the set of microspecifications to the macroscopic explanandum might be many-to-one. In that case, further work is required to adjudicate among the competitors. [...] In any event, the first point is that the motto is a criterion for explanatory candidacy. There may be multiple candidates and, as in any other science, selection among them will involve further considerations.}

Au regard de cette clarification, la formule de Macy est reprise parfois de façon un peu rapide comme ici :

\foreignblockquote{english}[\cite{Marchionni2013}]{Thus, Macy and Flache (2009, 263) are right when they challenge J. M. Epstein’s slogan by claiming, \enquote{If you don’t know how you grew it, you didn’t explain it.} The idea of generation is necessary for explanatory understanding, but it is not sufficient.}

La levée de ce point montre bien à quel point \textcite{Yanoff2008} avait oublié de préciser que la construction et l'évaluation d'hypothèses de comportement crédibles était clairement l'intérêt du modèle Anasazi, et non pas juste la réplication \enquote{aveugle} d'une régularité macroscopique par tous les moyens. Il n'a par exemple jamais été question de biaiser les hypothèses de ce modèle comme le laisse supposer \textcite{Yanoff2008} dans la comparaison que celui-ci fait avec le modèle de météorologie présenté par \textcite{Kuppers2005}, volontairement faussé pour prédire correctement.

\textit{Que faut-il retenir d'une telle passe d'arme ?} Tant que les modèles publiés ne montrent pas plus d'efforts pour décrire à la fois les démarches de modélisations ayant permis la construction des critères et des hypothèses, et l'activité d'évaluation qui autorise leur présence dans les modèles, le risque de voir ce type de publication se reproduire n'est pas écarté.

Parmi les autres critiques de \textcite{Yanoff2008}, on citera celle de \textcite{Elsenbroich2012}. Comme \textcite{Chattoe2011}, celle-ci insiste sur le fait que les problèmes avancés par Grüne-Yanoff ne sont en rien spécifiques à la modélisation multi-agents, et se rapportent à l'ensemble des sciences sociales. C'est le cas par exemple la question de l'incertitude ou de l'absence des données mobilisées, de l'équifinalité. Lorsqu'elle aborde la partie \enquote{explication} dans sa critique de Grüne-Yanoff, Elsenbroich est toutefois d'accord pour dire que la simulation multi-agents, pas plus que les sciences sociales, ne peut effectivement ni fournir de chaîne causale complète ou même d'explication potentielle \Anote{explication_potentielle} des phénomènes. Toutefois, et c'est là le plus important, \textcite{Elsenbroich2012} réfute la proposition de \textit{Functionnal Explanation} de Cummins proposée par Grüne-Yanoff. Nous n'avons pas la place de citer tous ses arguments, mais un seul devrait suffire à relever l'impossibilité d'adhérer à cette proposition (\textit{limit-problem}). En effet, pour \textcite{Elsenbroich2012} ce cadre explicatif ne permet tout simplement pas d'avancer des explications partielles :

\foreignblockquote{english}[\cite{Elsenbroich2012}]{Grüne-Yanoff states that a full functional explanation is a causal explanation. This uniqueness is explicitly used in Cumins' Individuation criterion (below). The problem is that we will never have a causal explanation as we can never really be sure we have identified all the real entities and mechanisms at work. The theory of functional explanation does not allow a part causal part functional explanation with the possibility of an increasing causal part in the face of additional information. It also does not allow for isolating causes for phenomena. This would mean that the Schelling model of segregation is not a causal explanation of segregation at all. It is clear that the Shelling model does not tell the whole story of segregation but it shows that segregation can be caused by the possibility of movement even at very high tolerance thresholds.}

Or, si par exemple les mécanismes opérant dans Schelling ne peuvent en aucun cas être considérés comme une explication suffisante pour la mise en lumière de phénomène ségrégatif, cela ne veut pas dire non plus que ces mécanismes n'interviennent pas du tout dans l'émergence de ce phénomène ! Elsenbroich donne ensuite un autre exemple pour justifier d'un nouveau cadre pour penser la causalité. Dans le cas de la récente crise financière aux Etats-Unis, il est difficile voire impossible de décrire l'histoire causale complète, et pourtant on peut l'expliquer en partie en reliant certaines entités ou facteurs déclencheurs, comme les \textit{sub-prime mortgage lending}.

Il existe selon elle un autre cadre d'analyse qui permet aujourd'hui de préserver, et de produire quand même une explication causale, à condition d'accepter une autre forme de causalité non basée sur la régularité \Anote{besse_comprendre}. Elle propose avec \textcite{Hedstrom2010} le transfert aux sciences sociales d'une variante de la notion de \enquote{mécanismes}, comme celle avancée par exemple par le biologiste Machamer \textcite{Machamer2000}. Ce dernier met ainsi en avant une propriété intéressante de cette notion qui permet de réfuter \textbf{en biologie} l'observation de régularité comme une condition nécessaire de l'explication \Anote{regularite_machamer}. \textcite{Elsenbroich2012} résume cet avantage ainsi : \foreignquote{english}{The causal powers are proposed to be directly connected to the entities involved in the event (e.g. via \enquote{capacities} or \enquote{activities}). No full causal story needs to be told but the entities must be shown to be \enquote{at work}}

Par son acceptation des thèses de Machamer, elle rejoint de fait le courant de modélisateurs portant actuellement ce cadre d'analyse dit des \enquote{mécanismes générateurs} \autocites{Hedstrom2010, Manzo2007} s'opposant à la \enquote{generative social science} \autocite{Epstein1999}. Pour \textcite[698]{Livet2014} ces deux visions s'affrontent, mais sur quelles bases exactement ?

\blockquote[\cite{Livet2014}]{Si une telle simulation \enquote{générative} peut être vue comme une condition nécessaire pour une science sociale computationnelle, elle ne suffit pas à fournir une explication ultime du phénomène. Tout d’abord, aux fonctions de la simulation doit correspondre un processus causal (Conte, 2007). De plus, ce type de modèle permet d’identifier un candidat explicatif pour ce phénomène, sans que ce soit nécessairement la seule explication possible, ni même forcément l’explication pertinente dans tous les cas de figure. La position extrême de Joshua M. Epstein a été critiquée pour la modélisation à base d’agents par Michael W. Macy et Andreas Flache dans leur ouvrage de synthèse sur la sociologie analytique (2009), où l’on préfère la notion plus large de \enquote{mécanismes générateurs}}

Avant de donner plus de détails sur ce cadre d'analyse, il semble que le point de vue d'Elsenbroich rejoigne donc la critique qu'a formulée \textcite{Conte2007} à l'égard de la théorie d'Epstein lors d'une revue de son livre \autocite{Epstein2007d}. Une similarité des critiques qui expliquerait aussi pourquoi le cadre d'analyse d'Epstein est de toute façon devenu insuffisant pour accompagner la valorisation des raisonnements sous-jacents à l'activité de modélisation.

Comme le laisse supposer la citation de Livet ci-dessus, l'article de \textcites{Conte2007, Conte2012} s'attaque exactement comme \textcite{Yanoff2008} à la problématique de l'équifinalité, traitée pour elle de façon beaucoup trop légère par Epstein.

\foreignblockquote{english}[{\cite[340]{Conte2012}}]{As already observed [96], one criterion has often been used, i.e., choose the conditions that are sufficient to generate a given effect. However, this leads to a great deal of alternative options, all of which are to some extent arbitrary. The construction of plausible generative models is a challenge for the new computational social science.}

Même si ce n'est donc pas tout à fait ce qu'a voulu dire Epstein qui, comme on l'a vu dans sa clarification de 2006, n'accepte pas la proposition inverse de son motto (générer n'est pas nécessairement expliquer), le cadre proposé par Epstein reste volontairement ambiguë sur la façon dont on est supposé construire et traiter les bonnes règles des mauvaises règles produisant des explications \textit{ad hoc}. Mais il y a un autre problème plus important que soulève Conte.

\textcite{Conte2007} propose un ancrage théorique initial plus prononcé pour les hypothèses mobilisées, ce qui suppose aussi de décorréler ces dernières de leur effets escomptés. Car non seulement la recherche de cause dans la production d'un phénomène n'implique pas forcément la seule notion de générativité \Anote{conte_bystander}, mais en plus appeler sa réalisation est selon elle une incitation à la construction de règles \textit{ad hoc} \Anote{conte_deccorele}.

Il en résulte deux objections générales vis-à-vis du cadre proposé par celui-ci :
\begin{itemize}
\item il faut associer une \enquote{théorie des causes} à cette émergence au risque sinon d'obtenir une \textit{generative explanation} sans importance ou ad hoc.
\item il faut une théorie pour justifier d'une chaîne d'événements qui vont des causes aux effets, sinon il n'y a pas de \textit{generative explanation} mais une simple reproduction de l'effet.
\end{itemize}

%Ce que nous dit Conte c'est qu'il faut éviter à tout pris une explication ad-hoc; or dans le cadre prévu par Epstein, rien ne semble interdir la formulation d'une seule règle permettant dans son expression dynamique (growing) de reproduire l'explanandum; Ce n'est pas suffisant, il faut pour Conte que les causes avancées ne sont explicatives que si on a réussi à reconstituer la chaine de causalité complète qui va de cette cause à la production de l'événement. Autrement dit il faut éviter de mettre en oeuvre des règles qui n'apporte rien d'autre que la reproduction du phénomène, elle ne sont que des boites noires ou des raccourcis peu informatives.

Remarque intéressante, \textcite{Conte2007} ne nous dit pas vraiment vers quelles théories il faudrait nous tourner. On trouve toutefois un début de réponse dans les travaux de \textcites{Hedstrom2010, Manzo2007, Elsenbroich2012} qui proposent au travers des \enquote{mécanismes générateurs} le retour d'un cadre d'analyse. Celui-ci rencontre aujourd'hui un certain regain d'intérêt en sociologie si on en croit \textcites{Berger2010, Hedstrom2010}. Un phénomène que l'on peut aussi observer de plus près avec la parution récente et inédite \Anote{manzo_journal} d'un numéro de la \textit{Revue Sociologique Francaise} dédié à la simulation, éditée par Manzo, un des sociologues porteurs depuis quelques années déjà de ce cadre d'analyse en France \autocite{Manzo2005, Manzo2007}.

Les états de l'art sur cette question réalisés par \textcites{Manzo2005,Manzo2007, Hedstrom1998, Berger2010} montrent que ce cadre d'analyse s'inscrit dans une tradition de sociologie analytique et mathématique qui remonte aux tous débuts de la simulation. On a d'ailleurs déjà évoqué certains de ces auteurs pionniers comme Boudon, Coleman, Merton dans les sections \ref{ssec:engouement_sciencesociale} et \ref{sssec:progressive_systemique}. Comme ils le font remarquer, la notion de mécanismes générateurs pour l'explication est non seulement ancienne, mais également transversale à de multiples disciplines, ce qui rend sa manipulation assez complexe si l'on ne prend pas soin d'en faire l'analyse préalable. \textcite{Hedstrom2010} mais également \textcite{Manzo2005} ont proposé des synthèses après avoir analysé les définitions existantes à la recherche des caractéristiques communes. Ce travail a permis un transfert de certaines propriétés intéressantes à d'autres définitions, comme celle par exemple du biologiste \textcite{Machamer2000} : \foreignquote{english}{Mechanisms consist of entities (with their properties) and the activities that these entities engage in, either by themselves or in concert with other entities. These activities bring about change, and the type of change brought about depends on the properties of the entities and how the entities are organized spatially and temporally.}

La notion de \foreignquote{english}{mechanism} intègre le cadre proposé par les sociologues en s'appuyant sur une théorie de l'action \enquote{Puisque seuls les acteurs ont le pouvoir de relier, de transformer, de construire ou de détruire des aspects de la réalité sociale (Abell 2004 : 293), l’idée de  \enquote{ générativité } serait en effet vidée de son sens en l’absence d’une référence à l’action individuelle (Bunge 1997 : 447 ; Fararo 1989 : 146).}

De ce fait, \enquote{la forme élémentaire d’un \enquote{ modèle générateur } renvoie à une variante spécifique de l’individualisme méthodologique [...] Il s’agit d’une forme d’individualisme méthodologique car un  \enquote{ modèle générateur } repose sur l’admission du primat analytique de l’acteur, de ses raisons et de ses actions.} \autocite{Manzo2007}. Pour ne pas être accusé d'un réductionnisme ou d'un atomisme lié à une seule analyse de la société par le primat de l'individu, les sociologues se rattachent à l'existence en sociologie de théories introduisant une liaison entre \enquote{structure} et \enquote{action}, comme par exemple la notion d'individualisme méthodologique complexe de Jean-Pierre Dupuy \autocite{Dupuy2004} qui \enquote{ s’oppose tant à l’individualisme méthodologique simple qu’au holisme. [...] l’individualisme méthodologique complexe insiste sur la boucle qui unit récursivement les niveaux individuel et collectif } \autocite[9]{Manzo2007}.

L'instrumentalisme est par ailleurs réfuté \Anote{instrumentalisme_refute} \autocite{Hedstrom2010}, et c'est l'explication qui prime ici, car les sociologues s'attache depuis le précurseur \textcite{Harre1972} à comprendre le \enquote{mode de production des phénomènes} \Anote{manzo_mecanisme}. Par conséquence également, la Théorie du Comportement Rationnel (TCR) (voir aussi les débats à ce sujet à la section \ref{p:passeur_systemique}) est considérée comme peu compatible avec \enquote{les mécanismes générateurs} et ces sociologues préfèrent situer, comme \textcite{Conte2007} le laisse aussi supposer, cette réflexion sur les motivation des individus au niveau cognitif et psychologique.

Il ne sera pas possible de rentrer dans tous les détails de cette grille d'analyse proposée par ces sociologues, mais on peut toutefois pointer dans celle-ci les éléments qui montrent une certaine similarité avec les différentes problématiques concernant la Validation évoquées précédemment.

Sur l'activité de formulation des hypothèses (niveau d'abstraction, échelle de représentation, hétérogéneité des formalismes, etc.) dont on a vu qu'elle était liée à la \enquote{facilitation} (ex-simplification) qui guidait les modélisateurs durant l'activité de modélisation, \textcite{Hedstrom2010} formulent des conclusions similaires. Ils se placent dans une tradition philosophique ancienne (Hempel 1965, Salomon 1998, Woodward 2003) et assument le fait que \foreignquote{english}{explanations are answers to question} pour dire : \foreignquote{english}{The why or how question one is adressing determines what the representation of the mechanism should include in order to be explanatory. Only by knowing the nature of the explanatory task at hand can one determine which details of a mechanism are relevant to include and the appropriate degree of abstraction (Ylikoski 2010)}.

L'équifinalité apparait également comme une problématique à gérer dans ce cadre d'analyse. Le modélisateur est appelé, en tenant compte de sa capacité à sélectionner le bon niveau d'abstraction pour une hypothèse/un critère, à inscrire sa démarche de construction dans une relation avec l'empirie pour mobiliser et/ou justifier les mécanismes explicatifs qu'il a sélectionnés pour son modèle.

\foreignblockquote{english}[\cite{Hedstrom2010}]{As it is possible that similar effects can be produced by a number of different (know or unknow) mechanisms, a crucial element in any mechanism-based explanation of empirical facts is the collection of empirical evidence about the assumed entities, activities, etc. The empirical evidence turns a possible mechanism into a plausible mechanism and may eventually lead to the identification of the actual mechanism. By presenting evidence in support of the assumptions of one mechanism and showing the absence for the assumptions of competing mechanisms, we increase the plausibility of the explanatory hypothesis [...] What separates proper mechanism-based explanations from mere mechanism-based storytelling is this king of rigorous checking of the assumptions upon which the mechanism schemes rest.}

\foreignblockquote{english}[\cite{Hedstrom2010}]{The problem often is not the absence of possible mechanisms, but how to discriminate between a number of potential mechanisms. To avoid lazy mechanism-based storytelling, the mechanism scheme must be made explicit and detailled, and its assumptions must be supported by relevant empirical evidence.\\
A mechanism-based explanation describes the causal process selectively. It does not aim at an exhaustive account of all details but seeks to capture the crucial elements of the process by abstracting away the irrelevant details. The relevance of entities, their properties, and their interactions is determined by their ability to make a relevant difference to the outcome of interest.}

Ce qui fait la différence entre un bon modèle et un mauvais modèle se trouve dans sa capacité à justifier des hypothèses par l'empirie, ce qui fait reposer toute la crédibilité du modèle de simulation présenté sur le sérieux et la qualité de la démarche de construction finale proposée.

L’équifinalité, l’explication et l’évaluation sont ainsi intimement liées au processus de simulation. A titre d’exemple on retrouve également chez Hedström et Ylikoski la question primordiale des critères permettant la sélection et l’évaluation des mécanismes explicatifs : 

\foreignblockquote{english}[\cite{Hedstrom2010}]{Roughly, mechanism-based explanations have two kinds of explananda. First, they might address particular empirical facts. In such cases, the description of the mechanism is often a modified adaptation and combination of more general mechanism schemes. Second, they might address stylized facts. Although the explanation of particular empirical facts is the ultimate goal of mechanism-based theory de velopment, most of the time theorists are addressing highly stylized theoretical explananda that do not necessarily have close resemblance to any particular empirical fact. }

En soi, en dehors de l'explication causale qu'il soutient, ce cadre d'analyse ne semble pas différent dans ses mises en garde de ce que l'on peut déjà lire dans la littérature sur la Validation. %Une des différences réside surement dans le fait qu'il existe en sociologie une base plus importante de mécanismes identifié mobilisable que dans d'autre discipline, vu l'historique important de cette notion.

%\enquote{Aussi sophistiqué soit-il, les paramètres d’un « modèle statistique » n’expriment que l’intensité, le signe et, éventuellement, la forme du lien entre deux ou plusieurs aspects du réel, opérationnalisés sous forme de variables. Rien n’est en revanche dit, dans de tels modèles, sur les sources de production de ce lien (Boudon 1976 ; Bunge 1997 ; Hedstrom 2005 : 31-36, chap. 5 ; Sorensen 1998). En l’absence d’une modélisation des mécanismes sous-jacents aux relations observées, l’explication est creuse. Dans ce cas, aucun caractère de causalité ne peut être attribué à l’action de X sur Y. Ni l'antériorité temporelle (ou logique) de X par rapport à Y, ni la résistance de leur lien à l’introduction d’une suite de variables tierces W ne peuvent justifier l’attribution de la causalité au lien observé.}

%Ce dialogue entre les données, les fait stylisée intégré dans le vocabulaire des mécanismes semble faire écho à des pratiques de construction de modèles maintes fois appliqués par exemple au laboratoire Géographie-cités.

Plusieurs concepts évoqués dans le cadre de ces mécanismes générateurs sont intéressants, mais ils doivent encore trouver substance dans une véritable mise en pratique, car les propos d'Hedström et Ylikoski restent pour le moment très théoriques. La nature, la forme et les modalités d'évaluation appelées pour justifier de ces mécanismes par l'empirie restent assez mystérieuses car le passage de la biologie où cette causalité partielle effective était vérifiée dans des mécanismes biologiques bien identifiés (Machamer prend l'exemple du fonctionnement des synapses par exemple) à des mécanismes sociaux n'a rien d'évident \autocite{Varenne2014b}. On peut faire une autre remarque dans la lecture croisée de Conte et du couple Manzo/Hedström. D'un côté, Conte nous dit que la reconstruction d'une chaîne causale n'appelle pas forcément à penser les hypothèses pour leur capacité générative au sein d'une structure causale, alors que cette générativité (cf. un mécanisme est identifié par le type d'effet ou de phénomène qu'il produit, et il est toujours un mécanisme pour quelque chose.) est justement considéré comme un fondement dans \enquote{les mécanismes générateurs}.

De façon plus générale, il n'est pas certain que ce cadre des \enquote{mécanismes générateurs} soit dans son acceptation actuelle, suffisant ou adapté pour la modélisation des phénomènes en géographie. La particularité de la géographie à ce niveau réside tout autant dans sa capacité à maintenir ce faisceau d'hypothèses cohérent dans une diversités d'objets, d'échelles et de temps, ou alors à l'inverse à s'en éloigner volontairement pour rechercher et mettre en avant les spécificités propres à l'articulation de ces échelles \autocite{Sanders2001}. Or, l'inscription spatiale des objets et des interactions entre ces objets, ou même les questions/possibilités que peuvent soulever d'un point de vue théorique cette possibilité d'intégrer dans les modèles agent d'autres niveaux de représentation que l'individu (les villes \autocite{Sanders2006} ?) ne semblent être que peu abordées dans cette littérature théorique.

Le point de rapprochement le plus évident entre les deux démarches se trouve probablement dans les rapports complexes envisagés par chacune des disciplines entre ses modèles statistiques et ses modèles de simulation. Il y a une véritable volonté de la part de ces sociologues de réancrer \Anote{manzo_empirie} la modélisation multi-agents dans l'empirie pour justifier des hypothèses mobilisées dans les modèles \autocites{Manzo2005}[23]{Manzo2007}{Hedstrom2015}. Les sociologues tels que \textcite{Goldthorpe2001} voient même la mise en place d'un couplage vertueux entre outils statistiques levant les régularités empiriques, et modélisation des mécanismes générateurs expliquant l'émergence de ces régularités \autocite[50]{Manzo2005}. Les sociologues comme les géographes s'appuient sur la même matière statistique (en la traitant évidemment de façon différente) pour extraire (inductif) ou construire (hypothético-déductif) leurs hypothèses initiales.

Ce principe d'un ancrage dans l'empirie se retrouve également chez les géographes modélisateurs \autocite{Banos2013}. Ces questions bénéficient même d'un certain recul historique, car elles ont déjà été investies et formalisées en géographie à de multiples reprises et cela pour chacune des innovations susceptibles d'amener un nouvel éclairage sur nos problématiques géographiques \autocite{Sanders2000, Mathian2014}. Une façon aussi de voir au delà et de ne pas se focaliser sur la seule qualité des explications (explication causale, semi-causale, etc.) car la production et le recoupement de résultats provenant d'outils, de méthodes, de modèles, d'échelles, de regards diversifiés est aussi une façon de produire de l'explication par accumulation.

La confrontation des hypothèses intégrées dans les modèles existants \autocite{Pumain1983, Sanders1984} et la construction de ces hypothèses appuyées \autocite{AMORAL1983} sont deux activités rattachées à l'empirie qui se sont imposées comme un objectif initial dès les premières expérimentations. Le dialogue complexe qu'il est possible de mobiliser entre les modèles statistiques, les modèles d'analyses spatiales, et les modèles de simulation a donc été maintes fois expérimenté dans la construction de modèles de simulation.

En géographie, où l'on met en jeu une hétérogénéité dans la nature et les échelles de temporalité et des individus représentés (personne, ville, réseau, gouvernance, etc.), on a vu que cette problématique de l'\textit{observational dilemna} rendait tout à fait illusoire une confrontation systématique des hypothèses mobilisées avec de seuls critères quantitatifs, les \enquote{faits stylisés} étant bien plus souvent invoqués, cela même si la vocation finale est effectivement d'ancrer les modèles dans l'empirie.

On trouvera par exemple un récit complet et récent de ce dialogue fructueux opéré entre modèles statistiques et modèles de simulation dans les travaux de Clémentine Cottineau \autocites{Cottineau2014a, Cottineau2014b}.

Il ne faut toutefois pas dénigrer ce cadre et continuer de s'y intéresser. La mise en avant de manifeste \textcite{Conte2012} ou de cadres d'analyse formalisés comme celui des \enquote{mécanismes générateurs} pour justifier du sérieux de la simulation en sciences sociales ne peut de toute façon être que bénéfique. Tourné vers l'extérieur, il permet de développer et valoriser les enjeux associés à la simulation en sciences humaines et sociales, mais il est aussi utile tourné vers l'intérieur dans l'enrichissement des discussions qui peuvent s'appuyer sur la construction ou la critique d'un objet commun compris de/par tous (comme le \textit{motto} d'Epstein par exemple).

L'assise épistémologique qui accompagne sa formulation apporte également des éléments de défense soutenant une démarche de construction raisonnée que nous n'avions pas forcément pensé à mobiliser de façon aussi explicite dans notre discipline. La question d'un cadre épistémologique pour justifier des explications apportées par une chaîne de causalité - même incomplète - dans nos modèles de simulations n'a été que peu évoquée jusqu'à présent, du moins pas sous cette forme de rapprochement interdisciplinaire autour de la notion de mécanismes.
