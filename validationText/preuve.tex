
L’existence de théories alternatives multiples est une constante dans l’histoire des sciences humaines. L'étude de l'objet social est un construit contextuel qui se nourrit d'une multiplicité des point de vues. C'est à ce titre que Jean-Claude Passeron \autocite{Passeron2006} nous met en garde contre une tentative de vérification des modèles qui serait décorrélée de tout contexte. Le terme \enquote{vérification} \foreignquote{english}{[...] stands for absolute thruth } \autocites{David2009, Oreskes1994} et se rapporte avant tout ici à la notion d'équifinalité \autocite{OSullivan2004} 

Pour lui le faillibilisme poppérien qui se cache derrière la méthode Hypothético-Déductive Nomologique-Déductif (HD-ND) ne peut pas s'appliquer à la construction de théorie dans le cadre des sciences humaines et sociales. Une relecture du fameux \enquote{exemple du cygne} introduisant \enquote{la problématique de l'induction} est introduit par \cite{Allard2000} pour illustrer la spécificité de l'epistémologie de Passeron.

\blockforeignquote[\cite{Allard2000}]{Dans une science nomologique, des énoncés illustratifs découlant d’une loi générale (du type: tous les cygnes sont blancs) n’ont aucune valeur, car ils imposeraient un nombre écrasant (infini) de vérifications : il faudrait aller voir si, à tous les endroits $k$, ou bien il n’y a pas de cygne, ou bien il y a un cygne blanc. Mais dans les sciences sociales, les lieux $k_1 ... k_n$ intéressants pour une étude donnée ne sont pas distribués au hasard. Si je recense les endroits où il est le plus probable que je rencontre des cygnes (zoos, niches écologiques...), et si je rencontre toujours des cygnes blancs; alors j’organise un protocole de vérification empirique qui donne plus ou moins de valeur à la présomption associée à la proposition générale : tous les cygnes sont blancs. Dans les sciences historiques, deux moyens permettent d’accroître cette présomption: il faut à la fois \enquote{multiplier et conjoindre sémantiquement des actes d’exemplification}. Les théories qui ne remplissent pas la première condition restent métaphysiques ; les recherches qui oublient la seconde ne peuvent être que des \enquote{entreprises sociographiques}, puisque associer des énoncés qui ne s’inscrivent pas dans un même langage théorique a peu de sens. Et la valeur d’une théorie sociologique se mesure à sa capacité à faire surgir et à rendre intelligibles des faits ou des relations dont la pertinence ne préexistait pas à cette théorie : c’est alors la \enquote{grille conceptuelle de description du monde} qui permet de multiplier les constats. Par conséquent, on est conduit à distinguer \textit{vérité} (qui s’oppose à la fausseté possible comme falsifiabilité) et \textit{véridicité}, qui correspond au type de connaissance auquel les sciences sociales peuvent prétendre. C’est en raison de cette exemplification rigoureuse que les sciences historiques, sciences interprétatives, n’ont rien à voir avec ce que Passeron appelle \enquote{herméneutique inspirée} ou \enquote{interprétation libre}: cette dernière intervient après une observation historique, ne faisant que paraphraser son sens intrinsèque, ou bien lui adjoignant des significations extrinsèques, c’est-à-dire extra-empiriques. En revanche, la théorie interprétative dans les sciences sociales fait apparaître de nouvelles relations par la comparaison avec d’autres descriptions empiriques.}

Toutefois il semble important de relativiser les propos de Passeron en admettant l'existence d'une cadre commun permettant de penser les échanges et la \enquote{cumulativité} \Anote{pumain_cumulativité} intra et inter disciplines des sciences humaines et sociales. Comme l'indique \textcite{Pumain2005} dans un article dédié à ce sujet, \enquote{La condition indiquée par J.C. Passeron (\enquote{ la sociologie n’a pas et ne peut prendre la forme d’un savoir cumulatif, c’est-à-dire d’un savoir dont un paradigme théorique organiserait les connaissances cumulées }, 1991, p. 364) n’est-elle pas excessivement exigeante ? Les connaissances des sciences dites \enquote{ dures }, expérimentales, sont-elles vraiment organisées dans un même paradigme théorique ? [...] La multiplicité des contextes différents, dans l’espace et dans le temps, est aussi invoquée par J.C. Passeron comme un obstacle rédhibitoire à la comparaison des cas et donc à la cumulativité des connaissances.} 

Pour Denise Pumain, qui a déjà experimenté avec d'autres géographes la possibilité de ces transferts bénéfiques (et prudent) entres disciplines parfois éloignés (physique, chimie, biologie) au travers du cadre systémique \autocites{Pumain1989,Sanders1992, Dastes1998}, il ne faudrait donc pas tomber dans un excès de relativisme tel que l'on trouve dans certaines postures postmoderne. Il est possible de travailler à la mise en place de méthodes \Anote{pumain_methode} propre à faire converger ces disciplines vers l'articulation et l'enrichissement de concepts, d'objets au travers de nouvelle grilles de lecture venant supporter la constitution d'un savoir, qui ne sacrifie si possible ni l'originalité, ni la diversité des points de vues engagés. Alors nous dit Denise Pumain, \enquote{Nous pourrions ainsi, tout en produisant des formalismes nouveaux, illustrer la question de la complexité d’une façon bien plus éclairante [...] La complexité d’une notion serait mesurée par la diversité des regards disciplinaires nécessaires à son élaboration, à l’intelligibilité des objets ou des processus étudiés, selon un objectif donné de précision des énoncés et des contextes}

Le modèle de simulation parait être un excellent support pour l'application et la discussion concrete autour de ces hypothèses, nouvelles, pouvant émerger de la mise en place d'un cadre commun. Les projets fortement interdisciplinaire que sont par exemple Archeomedes, TransMonDyn, Alpage, ou GeoDivercity \autocite{Chapron2014} semblent tous démontrer quelle fertilité en terme de formalismes, de modèles de simulation, et de connaissances produites peut avoir une reconstruction commune à partir d'une telle remise à plat initiale. 

L'équifinalité donc, un argument souvent utilisé par les detracteurs de la simulation qui voit dans l'introduction de cette variabilité l'échec prévisible de toute explication, est ici vue d'une façon positive car elle permet au contraire d'assumer tout autant la réalité d'une multiplicité de facteur explicatif propre à la diversité des points de vue en SHS et en géographie, que sa capture dans un cadre commun susceptible de mettre en valeur cette richesse, ou elle devient le support à la discussion interdisciplinaire. 

Rejoignant par là l'analyse de Passeron et de Pumain, \textcite{Besse2000} expose une version de l'\enquote{explication} qui prend une certaine distance avec l'hypothético-deductivisme et l'explication nomologique dans sa forme logique et rigide, clairement inadapté aux formes sociales (HD-ND). Ce dernier cadre explicatif a déjà reçu un certain nombre d'objection au cours de cette thèse, bien souvent appuyé par l'analyse très juste réalisé par \textcite{Besse2000}. Ces arguments ne sont pas tous remobilisé ici, et on peut se reporter à la section \hl{Section!}. Mais cette volonté nomologique restant au coeur de l'entreprise scientifique de la géographie, cette critique ne serait pas juste si par ailleurs elle ne mobilisait pas une description assouplie, sinon alternative à celle d'Hempel. Celle de Passeron en est une pour la sociologie, Besse propose également une vision assouplie pour la géographie.

\textcite{Besse2000} d'aborder \enquote{l'explication nomologique comme une forme particulière de la \enquote{mise en discours} de la science, et plus précisément encore comme un \textit{moment} au sein de la diversité des actes du savoir scientifique. La question pourrait être posée de manière suivante: faut-il réduire la discursivité scientifique au seul mouvement de la déduction, la science se réduit-elle au seul raisonnement hypothético-déductif, qui en constituerait pour ainsi dire la norme logique ? }
 
L'abduction, comprise non pas comme cadre logique mais comme un argument naturaliste basé sur la prise en compte des aspects cognitif de raisonnement, permet d'aller au delà du cadre logiciste pour penser l'explication en sciences humaines et sociales. Pour Besse \enquote{la démarche abductive permet un authentique gain de sens, une progression dans l'élucidation. Elle indique l'émergence d'un niveau \textit{sémantique} par rapport à un niveau formel dans l'activité scientifique, qui nous conduit à envisager celle-ci dans la perspective d'une dynamique de problématisations ouvertes, plutôt que celle d'une normativité déductive.}. Elle apporte au delà donc d'une \enquote{logique de preuve}, une \enquote{logique de la recherche} comprise comme \enquote{une \textit{logique du sens}, une logique de la progression du sens}

Les implications de cette remarque sont importantes car elle redonne un véritable poids au contexte dans lequel opère cette activité de modélisation pour la simulation. Elle soulève aussi la dissonance qui existe entre notre volonté de rendre compte d'une démarche de raisonnement immergé dans ce contexte, et l'image auto-suffisante sur le plan de l'explication qui en est donné au lecteur dans la plupart des publications de modèles de simulation. 


Comme on pouvait toutefois s'y attendre, cette impossibilité d'admettre l'unicité des critères pour juger la structure causale mobilisé s'insère dans un questionnement plus large. Faut-il effectivement juger la valeur des hypothèses constituantes de cette structure uniquement vis à vis de la réponse à ces critères ?

Un des arguments qui ressort toutefois pour expliquer la mobilisation des hypothèses ou des critères dans les modèles de simulation apparait dans cette prise en compte nécessaire du contexte global, dans le positionnement du modèle dans une démarche plus globale de construction des connaissances en géographie qui limite aussi la portée des preuves amenés par le modèle.


Si on en croit \textcite[17]{Besse2000}, pas vraiment, car cela serait oublier qu'\enquote{Une hypothèse possède une signification propre, avant même d’avoir été engagée dans l’aventure hautement improbable des programmes de validation. Cela nous conduit à reconnaître dans l’activité scientifique un moment de la production du sens, a coté du mouvement vers l’établissement des vérités.}


Une telle issue été déjà préssenti à la lecture des modes de constructions mobilisé dans cette mise en tension, avec la mise en avant d'un processus abductif pour dégager des connaissances, et l'évocation de cette \enquote{facilitation} à l'oeuvre dans la formation des hypothèses. 

Dès lors ce débat sur l'équifinalité apparait comme un malentendu pourvoyé par la communication des modèles, dont toute la partie contextuelle et temporelle a souvent été évincé.

en se tournant vers l'application d'un raisonnement abductif/retroductif.

















----------------

seule n'intervient qu'en partie dans cette question, car il s'intègre, au titre de nombreux autres outils, dans une démarche méthodologique plus générale et englobante \autocite{}. Aborder la validation du seul point de vue de l'outil est à mon sens une erreur, car comme l'ont déjà noté Hélène Mathian et Léna Sanders , les géographes concoivent la démarche, ou plutot les démarches de construction de connaissance sur les objets spatiaux en mobilisant non pas un, mais des modèles parmis un ensembles à leur dispositions. 

Tous sont construction, et donc tous sont faux à des degrés divers. La question de la validation prise hors de son contexte thématique n'a pas de sens dans un cadre pratique de construction de connaissances, mais tout cela à déjà été dit, et c'est pour cela que nous préférons dans la pratique nous concentrer en tant que modélisateur, sur l'évaluation des modèles de simulations, plutot que sur leur validation. \autocite{Amblard2006}


La modélisation comme processus abductif renvoie en alternance aux données et aux modèle de données, qui elle même renvoie aux hypothèses et aux implémentations de ces hypothèses, aux indicateurs et à la façon dont on les a construit, etc. A l'image des autres outils, il peut être mobilisé de façon hypothético-déductive, ou de façon 

---









%%%%%%%%%%%%%%%%%%%%%%%%%%%%%%%%%%%%%%%%%%%%%%%%%%%%
%%%%%%%%%%%%%%%%%%%%%%%%%%%%%%%%%%%%%%%%%%%%%%%%%%%%
%%%%%%%%%%%%%%%%%%%%%%%%%%%%%%%%%%%%%%%%%%%%%%%%%%%%

L’existence de théories alternatives multiples est une constante dans l’histoire des sciences humaines. L'étude de l'objet social est un construit contextuel qui se nourrit d'une multiplicité des point de vues. C'est à ce titre que Jean-Claude Passeron \autocite{Passeron2006} nous met en garde contre une tentative de vérification des modèles qui serait décorrélée de tout contexte historique. Pour lui le faillibilisme poppérien qui se cache derrière la méthode hypothético déductive ne peut pas s'appliquer à la construction de théorie dans le cadre des sciences humaines et sociales. 
Le processus de modélisation apporte une dimension supplémentaire à l'analyse de chacun de ces points de vue.Car il est hélas impossible de prouver par les modèles qu'il n'y a pas un tout autre ensemble de fait stylisés ou d'interactions qui soit capable d'arriver à la même observation, enlevant de fait toute unicité d’une explication \enquote{scientifique} au point de vue représenté par le modèle. L'équifinalité est donc à ce titre une limitation indépassable à la connaissance qui peut être déduite de nos modèles.

%%%%%%%%%%%%%%%%%%%%%%%%%%%%%%%%%%%%%%%%%%%%%%%%%%%%
%%%%%%%%%%%%%%%%%%%%%%%%%%%%%%%%%%%%%%%%%%%%%%%%%%%%
%%%%%%%%%%%%%%%%%%%%%%%%%%%%%%%%%%%%%%%%%%%%%%%%%%%%

Il faut donc ajouter à cela, La mise au jour exhaustive et transparente des dynamiques animant la structure causale de nos systèmes complexes (qui ne serait donc plus complexe) par la calibration ou l'exploration complète des modèles, si elle était possible, ne garantirai en rien sa validité, car un tout autre jeu d'hypothèses, d'implémentation d'hypothèses, de paramètres, de valeur de paramètre pourrait très bien conduire au même résultat.

Cette équifinalité, quant on la regarde sous cette forme, peut être à la fois considéré comme une limite dans l'établissement de vérité, voire une faille exploitable par les critiques de la simulation. Seulement, comme on a déjà pu le préssentir, la question de la preuve ou de la vérité n'est pas au coeur des préoccupations des chercheurs en sciences humaines et sociales, qui font appel à la simulation pour une tout autre raison. 

Ainsi pour \autocite{Varenne2014},  \enquote{[...] la fécondité propre à la géographie de modélisation contemporaine et à ses différentes formes de manifestation tient en grande partie à sa capacité à affronter cette question de la sous-détermination, à comprendre qu’il ne s’agit plus tant pour elle de chercher des théories que de développer des modèles aux fonctions épistémiques multiples.} 



%%%%%%
Une erreur dans laquelle est malheureusement tombé \textcite{Yanoff2008}, qui a jugé ce travail sans être au courant des codes, des discussions en cours à l'oeuvre dans cette communauté. Or d'une part il existe une méthodologie implicte à la construction de tel modèles, et d'autres part si on veut parler de la validation en SHS, on est obliger de traiter la question du collectif de part le rôle important qu'il joue cette fois-ci dans la Validation, non pas dans l'absolu et détaché de tout contexte, mais par les pairs en SHS \autocite{Rouchier2013, Ahrweiler2005}. \hl{on en parle aussi plus loin ! } \autocite{}




Critères sont comme les hypothèses, ils n'arrivent pas de nul part. Il faut intégrer le modèle de simulation dans un contexte plus large.

