
L’existence de théories alternatives multiples est une constante dans l’histoire des sciences humaines. L'étude de l'objet social est un construit contextuel qui se nourrit d'une multiplicité des point de vues. C'est à ce titre que Jean-Claude Passeron \autocite{Passeron2006} nous met en garde contre une tentative de vérification des modèles qui serait décorrélée de tout contexte. Le terme \enquote{vérification} \foreignquote{english}{[...] stands for absolute thruth } \autocites{David2009, Oreskes1994} et se rapporte avant tout ici à la notion d'équifinalité \autocite{OSullivan2004} 

Pour Passeron le faillibilisme poppérien qui se cache derrière la méthode Hypothético-Déductive Nomologique-Déductif (HD-ND) ne peut pas s'appliquer à la construction de théorie dans le cadre des sciences humaines et sociales. Une relecture du fameux \enquote{exemple du cygne} introduisant \enquote{la problématique de l'induction} est introduit par \cite{Allard2000} pour illustrer la spécificité de l'epistémologie de Passeron.

\foreignblockquote{english}[\cite{Allard2000}]{Dans une science nomologique, des énoncés illustratifs découlant d’une loi générale (du type: tous les cygnes sont blancs) n’ont aucune valeur, car ils imposeraient un nombre écrasant (infini) de vérifications : il faudrait aller voir si, à tous les endroits $k$, ou bien il n’y a pas de cygne, ou bien il y a un cygne blanc. Mais dans les sciences sociales, les lieux $k_1 ... k_n$ intéressants pour une étude donnée ne sont pas distribués au hasard. Si je recense les endroits où il est le plus probable que je rencontre des cygnes (zoos, niches écologiques...), et si je rencontre toujours des cygnes blancs; alors j’organise un protocole de vérification empirique qui donne plus ou moins de valeur à la présomption associée à la proposition générale : tous les cygnes sont blancs. Dans les sciences historiques, deux moyens permettent d’accroître cette présomption: il faut à la fois \enquote{multiplier et conjoindre sémantiquement des actes d’exemplification}. Les théories qui ne remplissent pas la première condition restent métaphysiques ; les recherches qui oublient la seconde ne peuvent être que des \enquote{entreprises sociographiques}, puisque associer des énoncés qui ne s’inscrivent pas dans un même langage théorique a peu de sens. Et la valeur d’une théorie sociologique se mesure à sa capacité à faire surgir et à rendre intelligibles des faits ou des relations dont la pertinence ne préexistait pas à cette théorie : c’est alors la \enquote{grille conceptuelle de description du monde} qui permet de multiplier les constats. Par conséquent, on est conduit à distinguer \textit{vérité} (qui s’oppose à la fausseté possible comme falsifiabilité) et \textit{véridicité}, qui correspond au type de connaissance auquel les sciences sociales peuvent prétendre. C’est en raison de cette exemplification rigoureuse que les sciences historiques, sciences interprétatives, n’ont rien à voir avec ce que Passeron appelle \enquote{herméneutique inspirée} ou \enquote{interprétation libre}: cette dernière intervient après une observation historique, ne faisant que paraphraser son sens intrinsèque, ou bien lui adjoignant des significations extrinsèques, c’est-à-dire extra-empiriques. En revanche, la théorie interprétative dans les sciences sociales fait apparaître de nouvelles relations par la comparaison avec d’autres descriptions empiriques.}

Toutefois il semble important de relativiser les propos de Passeron en admettant l'existence d'une cadre commun permettant de penser les échanges et la \enquote{cumulativité} \Anote{pumain_cumulativité} intra et inter disciplines des sciences humaines et sociales. Comme l'indique \textcite{Pumain2005} dans un article dédié à ce sujet, \enquote{La condition indiquée par J.C. Passeron (\enquote{ la sociologie n’a pas et ne peut prendre la forme d’un savoir cumulatif, c’est-à-dire d’un savoir dont un paradigme théorique organiserait les connaissances cumulées }, 1991, p. 364) n’est-elle pas excessivement exigeante ? Les connaissances des sciences dites \enquote{ dures }, expérimentales, sont-elles vraiment organisées dans un même paradigme théorique ? [...] La multiplicité des contextes différents, dans l’espace et dans le temps, est aussi invoquée par J.C. Passeron comme un obstacle rédhibitoire à la comparaison des cas et donc à la cumulativité des connaissances.} 

Pour Denise Pumain, qui a déjà experimenté avec d'autres géographes la possibilité de ces transferts bénéfiques (et prudent) entres disciplines parfois éloignés (physique, chimie, biologie) au travers du cadre systémique \autocites{Pumain1989,Sanders1992, Dastes1998}, il ne faudrait donc pas tomber dans un excès de relativisme tel que l'on trouve dans certaines postures postmoderne. Il est possible de travailler à la mise en place de méthodes \Anote{pumain_methode} propre à faire converger ces disciplines vers l'articulation et l'enrichissement de concepts, d'objets au travers de nouvelle grilles de lecture venant supporter la constitution d'un savoir, qui ne sacrifie si possible ni l'originalité, ni la diversité des points de vues engagés. Alors nous dit Denise Pumain, \enquote{Nous pourrions ainsi, tout en produisant des formalismes nouveaux, illustrer la question de la complexité d’une façon bien plus éclairante [...] La complexité d’une notion serait mesurée par la diversité des regards disciplinaires nécessaires à son élaboration, à l’intelligibilité des objets ou des processus étudiés, selon un objectif donné de précision des énoncés et des contextes}

Le modèle de simulation parait être un excellent support pour l'application et la discussion concrete autour de ces hypothèses, nouvelles, pouvant émerger de la mise en place d'un cadre commun. Les projets fortement interdisciplinaire que sont par exemple Archeomedes, TransMonDyn, Alpage, ou GeoDivercity \autocite{Chapron2014} semblent tous démontrer quelle fertilité en terme de formalismes, de modèles de simulation, et de connaissances produites peut avoir une reconstruction commune à partir d'une telle remise à plat initiale.

L'équifinalité donc, un argument souvent utilisé par les detracteurs de la simulation qui voit dans l'introduction de cette variabilité l'échec prévisible de toute explication, est ici vue d'une façon positive car elle permet au contraire d'assumer tout autant la réalité d'une multiplicité de facteurs explicatifs propres à la diversité des points de vue en SHS et en géographie, que sa capture dans un cadre commun support de discussion interdisciplinaire enrichissante. 

Rejoignant par là l'analyse de Passeron et de Pumain, \textcite{Besse2000} expose une version de l'\enquote{explication} qui prend une certaine distance avec l'hypothético-deductivisme et l'explication nomologique dans sa forme logique et rigide, clairement inadapté aux formes sociales (HD-ND). Ce dernier cadre explicatif a déjà reçu un certain nombre d'objection au cours de cette thèse, bien souvent appuyé par l'analyse très juste réalisé par \textcite{Besse2000}. Ces arguments ne sont pas tous remobilisé ici, et on peut se reporter à la section \hl{Section!}. Mais cette volonté nomologique restant au coeur de l'entreprise scientifique de la géographie, cette critique ne serait pas juste si par ailleurs elle ne mobilisait pas une description assouplie, sinon alternative à celle d'Hempel. Celle de Passeron en est une pour la sociologie, Besse propose également une vision assouplie pour la géographie.

\textcite{Besse2000} d'aborder \enquote{l'explication nomologique comme une forme particulière de la \enquote{mise en discours} de la science, et plus précisément encore comme un \textit{moment} au sein de la diversité des actes du savoir scientifique. La question pourrait être posée de manière suivante: faut-il réduire la discursivité scientifique au seul mouvement de la déduction, la science se réduit-elle au seul raisonnement hypothético-déductif, qui en constituerait pour ainsi dire la norme logique ? }
 
L'abduction, comprise non pas comme cadre logique mais comme un argument naturaliste basé sur la prise en compte des aspects cognitif de raisonnement, permet d'aller au delà du cadre logiciste pour penser l'explication en sciences humaines et sociales. Pour \textcite{Besse2000} \enquote{la démarche abductive permet un authentique gain de sens, une progression dans l'élucidation. Elle indique l'émergence d'un niveau \textit{sémantique} par rapport à un niveau formel dans l'activité scientifique, qui nous conduit à envisager celle-ci dans la perspective d'une dynamique de problématisations ouvertes, plutôt que celle d'une normativité déductive.}. Elle apporte à coté d'une \enquote{logique de preuve}, une \enquote{logique de la recherche} comprise comme \enquote{une \textit{logique du sens}, une logique de la progression du sens. A ce titre, une bonne partie du travail de la science consisterait non pas à chercher d'\textit{abord} des causes et des enchaînements déductifs, mais à organiser des points de vue et des tableaux représentant les situations dont elle cherche à rendre compte. En d'autres termes, avant de chercher à expliquer et à \textit{démontrer} des successions de causes et d'effets, la science se préoccupe d'\textit{éclairer} des situations en procédant abductivement à des rapprochements signifiants. La science ne cherche pas seulement à démontrer, mais aussi à éclairer.} 

On peut évoquer l'exemple des loi empirique Processus/Loi (P/L) sous-déterminé \Anote{sous_determination} comme par exemple la loi Rank-Taille \autocite{Varenne2014}. On peut considérer que cette forme de sous-détermination est similaire à l'équifinalité telle que l'on a déjà définit. Différentes disciplines se sont intéressé au mode de production de ce phénomène, qui est là, et qu'il faut non pas expliquer, puisque c'est impossible, mais elles ont tenté au moins de l'éclairer au regard de leur grille d'analyse. L'application de méthode pour une cumulativité des connaissance centré sur cette loi empirique a permis de croiser et de faire émerger de nouvelles hypothèses explicatives entre différentes disciplines, comme le montre par exemple cette analyse croisé entre l'archéologie, économie et  géographie par \textcite{Sanders2012} : \enquote{L’objectif de cette contribution est de mettre en vis-à-vis les travaux de différents champs disciplinaires pour comparer, d’une part, les interprétations avancées des régularités et des écarts relativement à la loi de Zipf, et, d’autre part, les cadres théoriques adoptés pour identifier les mécanismes à l’origine de l’organisation hiérarchique des systèmes de peuplement.}

Sur les aspects nomologiques, des théories de portées peut être moins ambitieuse qu'en physique ont également vu le jour par l'utilisation de cette méthodes d'abduction/ rétroduction en géographie. Ainsi la théorie évolutives des villes de \textcite{Pumain1997} s'appuie sur la formulation de modèles de simulation dont les conclusions lorsqu'elles se vérifient avec succès dans l'empirie viennent renforcer en retour cette théorie. Il est donc possible en géographie de se passer d'un cadre déductif initial pour voire émerger par accumulation et renforcement des nouveaux cadres d'analyse à même d'être ensuite dérivés. 

Enfin, comme l'indique de façon plus générale \textcite{Varenne2014} dans son analyse des rapports historiques des géographes modélisateurs (plus particulièrement francais d'ailleurs) avec différentes sous-déterminations P/L, \enquote{[...] la fécondité propre à la géographie de modélisation contemporaine et à ses différentes formes de manifestation tient en grande partie à sa capacité à affronter cette question de la sous-détermination, à comprendre qu’il ne s’agit plus tant pour elle de chercher des théories que de développer des modèles aux fonctions épistémiques multiples.}

%L'abduction déjà présent dans le paragraphe \ref{p:abduction} se voit doté au travers de ces différentes analyses d'une légitimité renouvellé dans l'établissement de connaissance, sans avoir à justifier d'un cadre logiciste pour les avancer. %L'induction, la déduction peuvent arriver dans un second temps, pour raffiner l'hypothèse. 

Comme l'indique Besse, l'abduction s'attache à un fil de raisonnement, à une progression de sens, ce qui impose pour être honnete la prise en charge explicite du caractère contextuel et temporel qui accompagne le déroulement de celui-ci. Avant de discuter plus longuemment des autres implications que cela suppose, on peut donc d'ores et déjà noter la dissonance qu'introduit cette remarque entre nos pratiques actuelle, et la critique, même mal informé, de Grune-Yannof. 

Entre le présentation d'un modèle de simulation qui s'est lentement développé sur la base de multiples aller-retour entre les réponses du modèle à nos raisonnement, et cette image statique d'un modèle auto-suffisant que l'on trouve dans de nombreuses publications, la critique même infondé, parait justifié : construction des hypothèses et des critères ? construction des données ? raisonnement intermédiaire pour présenter les hypothèses retenues ? etc. Comment justifier de l'honneteté d'une connaissance établie par la progression d'un raisonnement si la solution présentée apparait comme tronqué et dénudé aux yeux du lecteur extérieur ? \autocite{OSullivan2004}

%L'équifinalité est une propriété des systèmes complexes, elle se manifeste donc aussi dans la diversification des hypothèses et des critères accueillant la construction des modèles. Elle n'est pas le seul apanage des critiques, et les modélisateurs y sont eux-même  confrontés puisqu'elle fait partie du jeu abductif.

L'importance de la reproductibilité dans la valorisation des modèles de simulation  apparait ici à peine voilée \autocites{Amblard2006, Wilensky2007a, Rouchier2013}. Or non seulement celle-ci est déjà rarement mise en oeuvre, mais en plus elle ne suffit pas, il faut aller plus loin, vers une reproductibilité du raisonnement, des discours développés avec et autour des modèles, comme le propose \textcite{OSullivan2004} : \foreignquote{english}{The process of model development, the possible outcomes it reveals and interpretations of those outcomes, taken together, constitute a geographical narrative, so that modellers become ‘makers’ of stories.}

Dès lors ce débat sur l'équifinalité apparait comme un malentendu pourvoyé par une mauvaise communication sur les modèles, dont toute la partie contextuelle et temporelle a souvent été évincé. On aurait tort de penser que cette question de la reproductibilité n'est pas importante, car elle n'est pas sans impact sur la Validation des modèles. 

La problématique de la Validation, si elle n'a pas effet dans le cadre des pratiques interne, s'étend toutefois aux jugements par les pairs. C'est le point de vue de \textcite{OSullivan2004} mais également de \textcite{Rouchier2013} que de penser la Validation au delà du premier cercle. La qualité des hypothèses, et des critères mobilisés à tout à gagner à être discuté, voire modifiés par . Ne disait-on pas que le modèle une fois délivré devenait un objet autonome ?

Pour \autocite{OSullivan2004} mobiliser l'argument technique est un leurre. 

\foreignblockquote{english}[\cite{OSullivan2004}]{Connectingthe model back to the world it represents is difficult for a number of reasons, principally the equifinality problem, which makes it impossible to judge the relative merits of alternative models on purely technical grounds. [...] It is clear that assessment of the accuracy of a model as a representation must rest on argument about how competing theories are represented in its workings, with calibration and fitting procedures acting as a check on reasoning. So, while we must surely question the adequacy of a model that is incapable of generating results resembling observational data, we can only make broad comparisons between competing models that each provide ‘reasonable’ fits to observations. Furthermore, critical argument and engagement with underlying theories about the processes represented in models is essential: no purely technical procedure can do better than this.}

Systématiser l'évaluation de cette mise en tension entre hypothèses et critères pouvant en rendre compte est evidemment un développement souhaitable si on veut à la fois maximiser les possibilités d'apprendre du modèle et amener une certaine fiabilité dans les raisonnements avancés. Pris au contraire comme une tentative de garantir une meilleure inférence au réel, cette activité nous conforte dans l'établissement de fausses certitudes. Il y'aura toujours d'autres proposition de modèles pour rendre compte de ces résultats, même avec des critères qualitatif (\enquote{fait stylisés})
Mais il y a pire, aucune exploration de la dynamique interne aux modèles, même idéalement exhaustive, ne nous permettrait jamais de remplacer une discussion autour des hypothèses et des critères mobilisés. Non seulement l'équifinalité est souhaité, mais sa prise en compte de façon explicite dans la communication des modèles est devenu souhaitable, si on veut poursuivre cet objectif proposé par Sullivan. 



Ainsi plus que les solutions techniques, c'est dans le processus de discussion et d'échange autour des hypothèses admises dans les modèles que notre connaissance sur les phénomènes réels est amenée à progresser. Par la mobilisation, l'hybridation, la confrontation de modèles ou briques de modèle issues d'angles de vues inter-disciplinaires,  on met en œuvre une grande discussion à même d'éclairer cette dynamique globale qui serait de toute façon insaisissable dans sa globalité. {cf transcidisciplinarité de morin ?}


L'impact sur la validation aussi est important.

\autocite{Rouchier2013, Ahrweiler2005}

\autocite{Rouchier2013} s'appuyant sur une définition de \todo{Gilbert et Artweiler} décrit cette forme de validation basée sur la réutilisation et l'enrichissement collectif des modèles comme étant post-moderne, \enquote{ dans la mesure ou elle base la valeur d'un modèle au regard de son usage par une communauté d'usagers }. Il y a donc dans le processus d'évaluation des modèles de simulation une dimension collective qui ne peut plus être niés dans l'établissement d'outil et de méthodologie . De façon plus générale, \autocite{Rouchier2013} évoque et décrit bien dans un article récent \enquote{  Construire la discipline \enquote{ Simulation Agent }} la nature de ce mouvement structurant qui œuvre dans la construction de communauté scientifique. Celui ci prend forme autour de revues revendiquant une large ouverture inter-disciplinaire, tel que JASSS, qui font alors office de catalyseur en supportant, relayant ces discussions de fond, à la fois sur le plan méthodologique et technique.

%Pour pousser l'analogie du \enquote{laboratoire virtuel} encore plus loin, il s'agirait alors d'ouvrir ce laboratoire aux autres scientifiques, d'en faire \enquote{place publique} afin de montrer l'histoire de nos protocoles, de nos modèles, de nos résultats \foreignquote{latin}{in vivo}, en assumant au passage toutes les contraintes que cela suppose. 

Dans sa conclusion \autocite{Rouchier2013} mise sur le développement de la crédibilité de cette discipline dans les années à venir, grâce aux revues, aux règles de conduites édictées, et aux modèles repris et discutés au cœur de cette communauté \autocite{Hales2003}.





---




Comme on pouvait toutefois s'y attendre, cette impossibilité d'admettre l'unicité des critères pour juger la structure causale mobilisé s'insère dans un questionnement plus large. Faut-il effectivement juger la valeur des hypothèses constituantes de cette structure uniquement vis à vis de la réponse à ces critères ?

Un des arguments qui ressort toutefois pour expliquer la mobilisation des hypothèses ou des critères dans les modèles de simulation apparait dans cette prise en compte nécessaire du contexte global, dans le positionnement du modèle dans une démarche plus globale de construction des connaissances en géographie qui limite aussi la portée des preuves amenés par le modèle.


Si on en croit \textcite[17]{Besse2000}, pas vraiment, car cela serait oublier qu'\enquote{Une hypothèse possède une signification propre, avant même d’avoir été engagée dans l’aventure hautement improbable des programmes de validation. Cela nous conduit à reconnaître dans l’activité scientifique un moment de la production du sens, a coté du mouvement vers l’établissement des vérités.}





















----------------

seule n'intervient qu'en partie dans cette question, car il s'intègre, au titre de nombreux autres outils, dans une démarche méthodologique plus générale et englobante \autocite{}. Aborder la validation du seul point de vue de l'outil est à mon sens une erreur, car comme l'ont déjà noté Hélène Mathian et Léna Sanders , les géographes concoivent la démarche, ou plutot les démarches de construction de connaissance sur les objets spatiaux en mobilisant non pas un, mais des modèles parmis un ensembles à leur dispositions. 

Tous sont construction, et donc tous sont faux à des degrés divers. La question de la validation prise hors de son contexte thématique n'a pas de sens dans un cadre pratique de construction de connaissances, mais tout cela à déjà été dit, et c'est pour cela que nous préférons dans la pratique nous concentrer en tant que modélisateur, sur l'évaluation des modèles de simulations, plutot que sur leur validation. \autocite{Amblard2006}


---




%%%%%%%%%%%%%%%%%%%%%%%%%%%%%%%%%%%%%%%%%%%%%%%%%%%%
%%%%%%%%%%%%%%%%%%%%%%%%%%%%%%%%%%%%%%%%%%%%%%%%%%%%
%%%%%%%%%%%%%%%%%%%%%%%%%%%%%%%%%%%%%%%%%%%%%%%%%%%%

Il faut donc ajouter à cela, La mise au jour exhaustive et transparente des dynamiques animant la structure causale de nos systèmes complexes (qui ne serait donc plus complexe) par la calibration ou l'exploration complète des modèles, si elle était possible, ne garantirai en rien sa validité, car un tout autre jeu d'hypothèses, d'implémentation d'hypothèses, de paramètres, de valeur de paramètre pourrait très bien conduire au même résultat.

Cette équifinalité, quant on la regarde sous cette forme, peut être à la fois considéré comme une limite dans l'établissement de vérité, voire une faille exploitable par les critiques de la simulation. Seulement, comme on a déjà pu le préssentir, la question de la preuve ou de la vérité n'est pas au coeur des préoccupations des chercheurs en sciences humaines et sociales, qui font appel à la simulation pour une tout autre raison. 




%%%%%%
Une erreur dans laquelle est malheureusement tombé \textcite{Yanoff2008}, qui a jugé ce travail sans être au courant des codes, des discussions en cours à l'oeuvre dans cette communauté. Or d'une part il existe une méthodologie implicte à la construction de tel modèles, et d'autres part si on veut parler de la validation en SHS, on est obliger de traiter la question du collectif de part le rôle important qu'il joue cette fois-ci dans la Validation, non pas dans l'absolu et détaché de tout contexte, mais par les pairs en SHS . \hl{on en parle aussi plus loin ! } \autocite{}




Critères sont comme les hypothèses, ils n'arrivent pas de nul part. Il faut intégrer le modèle de simulation dans un contexte plus large.

