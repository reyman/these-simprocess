% -*- root: These.tex -*-

\section{Introduction}
\label{sec:Introduction}


Plan introduction \enquote{Une plateforme intégrée pour la construction et l'évaluation de modèle de simulation en géographie}

% ARGUMENT INTERDISCIPLINARITE
Ce travail de thèse s'inscrit dans un cadre de forte inter-disciplinarité dans lequel j'ai principalement joué le rôle de médiateur entre géographes, et informaticien. 

Comme le précise \autocites{Pumain2005, Chapron2014}, quelque soit la méthode utilisé pour organiser et s'assurer de la réussite d'une activité de recherche inter-disciplinaire, il est nécessaire de déconstruire les concepts et objets en jeu pour que puisse émerger par une co-construction entre les différentes disciplines \autocite{Banos2013} de nouveaux formalismes, de nouveaux objets à même de supporter, voire de catalyser, des enjeux scientifiques qui dépasse à présent le seul cadre de la seul discipline.

% plateforme pour la construction et l'évaluation des modèles a supporter l'émergence de tout nouveaux enjeux, pour la géographie, mais aussi pour l'informatique.  

Cette déconstruction puis cette reconstruction n'impacte pas que la vie des idées, mais également celle des outils, surtout lorsqu'une équipe intègre des informaticiens. Or comme on le racontera plus en détail dans la deuxième partie de cette thèse, une fois la possibilité d'une convergence potentielle constaté en 2010, il a fallu pour que la coopération puisse être plus bénéfique encore, savoir se dessaisir d'un projet scientifique et technique conçu à l'échelle d'une personne et d'une thèse \autocites{Rey2009, Louail2010}, pour le reinventer au travers d'un projet déjà opérationnel  depuis deux ans, complexe sur le plan conceptuel et technique, conçu et maintenu par deux ingénieurs à temps plein suivant une feuille de route indépendante, et vierge au moins au départ, des désirs provenant des géographes.

Cette thèse est donc traversé par le désir d'une transformation des usages pour la simulation en géographie, ancré dans une histoire, et tourner vers un avenir. 

L'entrée pour moi dans ce processus inter-disciplinaire vient dans le remplacement d'un objet par un autre, plus adapté, plus visionnaire, mais aussi beaucoup moins maitrisé, maitrisable. 

 J'ai donc fait l'expérience de cette transformation, et de cette réorientation assumé vis à vis d'un sujet thèse devant composer avec cette mise en danger et cette perte en autonomie sur cet objet \enquote{plateforme} qui marque . 


%Deux grandes parties dans cette thèse

Cette thèse est découpée en deux grandes parties traitant de la \enquote{Construction et Evaluation de modèle en géographie}, puis de la \enquote{Réalisation d'une plateforme intégré}



%Pourquoi une partie historique ?  




%Pourquoi une partie pratique ?  

La deuxième partie est tourné vers la mise en oeuvre de nouvelles pratiques \enquote{High Perfomance Computing} en géographie.






