
Deux questions différentes vis à vis de la validation : 
Question du scientifique vis à vis des hypothèses de son modèles, de leur validité, et de leur cohérences dans le monde scientifique  foundamentalism vs coherentism ... 
Question du scientifique vis à vis de l'outil simulation, de sa fonction épistémique, et des résultats qu'il compte en tirer (cf varenne)

Evidemment il semble que les deux ne soit pas indépendant l'un de l'autre.

Historique V et V
Naylor fait partie des pionniers
Vision de Naylor domine encore et semble poser probleme (voir Kleindorfer)

"We would like to revisit this subject.In our view, the limitations of the positions outlined by NaylorandFinger are the source of some of the current prob-lems regarding validation in simulation. " Kleindorfer 1998

Probleme de cette vision de Sargent et Balci, c'est qu'elle ne nous dit absolument rien rien rien du tout sur la construction du savoir au niveau des hypothèses. Ok il faut faire de l'incrémental et apres ? C'est le probleme de la majorité des guides, c'est que cela n'est pas parcque le probleme de la validation est contextuel qu'il ne faut pas proposer des outils et des méthodes pour experimenter correctement (VOir aussi ce que dit Numo a ce ce sujet )
Autrement dit, suivant la meme voie generique, il n'est pas question d'imposer des outils statistiques ou autres, ni meme une utiliser forcée, mais de fixer une démarche d'ajout/ retrait de mécanismes qui parle à la plupart des gens.

Avant d'arriver à ce point là, evidemment il faut etre capable de mobiliser une boucle dans son ensemble.


Historique des termes => Les termes « validation et vérification » proviennent à l'origine de l'ingénierie des systèmes, et peuvent être rattaché au concept de « qualité » tel quel est définit par la famille de règle ISO établit par l'organisation mondiale de normalisation. Décomposable en plusieurs branches cette disciplines à part possède une branche dédié à l'expertise logicielle. De ce fait, il n'existe pas réellement de définition ni de théories ou méthodologies officiellement acceptable, l'acceptation des termes pouvant varier fortement selon les branches d'application. Toutefois on trouve dans des livres dédié à la définition d'une terminologie standard dans les disciplines, telle que le {PMBOK guide}, résultats d'un travail certifié par des associations ou des organismes étatiques telle que IEEE et ANSI. Ce dernier propose une définition générale de la vérification et validation en ces termes : 

Verification and validation (V\&V) processes are used to determine whether the development products of a given activity conform to the requirements of that activity and whether the product satisfies its intended use and user needs. 

et revient ensuite plus spécifiquement sur les termes : 

Validation : The assurance that a product, service, or system meets the needs of the customer and other identified stakeholders. It often involves acceptance and suitability with external customers. Contrast with verification.

Verification : The evaluation of whether or not a product, service, or system complies with a regulation, requirement, specification, or imposed condition. It is often an internal process. Contrast with validation.

Hétérogeineité des acceptation => Toutefois, comme le fait remarquer {Kleijnen1995} en citant astucieusement une note de bas de page de {Barlas1990}, il n'est pas impossible de voir les termes inversé dans certaines disciplines, comme par exemple en philosophie. Les termes officieux sont finalement bien trop généraux pour envisager de les appliquer tel quel dans notre domaine de compétence. Ainsi, les définitions de la V\&V dans le cadre de l'ingénierie logicielle dédié à la simulation seront plus éclairantes dans le cadre de notre étude. Le nombre de définition s'en trouve alors réduit, les plus connues et appliqués étant sûrement celles donnés par Kleijnen, Sargent ou encore la plus historique, celle de Naylor {Naylor1967}. Il est possible d'en retrouver par ailleurs en feuilletant les « Proceedings of the Winter Simulation Conference » ou la problématique de la V&V est régulièrement réévalué au regard des nouvelles connaissances. 



Dépasser les définitions classiques => Ces différentes définitions apparaissent régulièrement dans différents papier scientifique, de toute discipline, y compris la géographie, entre autre parcqu'ils offrent un cadre structurant et relativement neutre pour penser le processus de construction des modèles « en général », et propose une terminologie suffisament claire pour la mise en œuvre de pratiques standardisés. 

Si ces définitions sont intéressantes car génériques, nous verrons qu'à l'image de celle de Naylor, elle ne prennent pas suffisament en compte la richesse de l'objet validation quand il s'exprime dans une perspective disciplinaire. La spécificité de la modélisation en géographie , ou le caractère social de la validation sont par exemple totalement absents de ces définitions. 

Ils existent toutefois des définitions beaucoup plus complète et directement applicable fournis par la communauté, tels que celles développés par Frederic Amblard. Celle ci fournissant la principale base de travail pour définir notre approche, nous serons amené à y revenir plus tard lors de l'étude du paradigme agent.

La caractéristique principale de cette notion de validation est sa transversalité, mais également  

Afin d'introduire la notion de connaissance, il est proposé de repartir de la définition de Naylor en 1967. Cette définition, dite  « historique » par Sargent est également celle que l'on retrouve assez régulièrement dans les premiers livre méthodologiques autour de la simulation, et elle témoigne assez bien des l'esprit de l'époque quand à la question de la validation. 
Ce rapprochement épistémologique, le premier si on en crois {Barlas1990} (officiellement, car officieusement la question se pose probablement dans chacune des disciplines ayant accès à la simulation ) fait de la validation « le processus d'extraction de connaissance » permettant de justifier la construction du modèle. 

Jerry Banks dans son livre régulièrement réédité « Discrete-Event System Simulation », prouve que l'approche proposé par Naylor est encore bien représenté en ingénierie en proposant aux lecteurs de s'appuyer sur une version synthétique et modernisé de l'approche proposé par Naylorpour faire la validation de leur modèle : 

1. Build a model that has high face validity
2. Validate model assumptions
3. Compare the model input-output transformations to corresponding input-output transformations for the real system.

{Description des trois points}

Malgré une approche résolument éclectique, l'argumentation de Naylor et Finger's est  empiriste {Barlas1990} : « A simulation model, the validity of which has not been ascertained by empirical observation, may prove to be of interest for expository or pedagogical purposes, [but] such a model contributes  nothing to the understanding of the system being simulated » {Naylor1967} Les deux auteurs excluant tout apports d'information autre que pédagogique si la comparaison avec des données empiriques est absente de la boucle de validation.

Le modèle étant construit pour répondre à une ou plusieurs questions, ce processus de validation ne peux plus être envisagé sur un plan universel, surtout lorsqu'il utilisé dans un cadre de sciences humaines et sociales. 

Les auteurs se borne donc a accepter ou rejetter chaque modèle selon une logique « vrai » « faux » très réductrice qui ne va pas sans poser de problème aux modélisateurs en sciences humaines et sociales {ref}.

« To verify or validate any kind of model (e.g management science models) means to prove the model to be true. But to prove that a model is « true » implies (1) that we have established a set of criteria for differentiating between those models which are « true » and those which are not « true », and (2) that we have the possibility to apply these criteria to any given models »{Naylor1967}

 Au sens de Barlas, la notion de « usulfuness » d'un modèle , qui reviendrait à une évaluation plus souple et contextuelle du modèle, est écarté. 

Pourtant le danger d'une telle approche en science sociales semble entendue si on s'en tient aux remarques qui peuvent être fait dès 1972 par certains auteurs publiant dans des revues méthodologiques reconnues. 
 
 « In fact, utility of simulation is sometimes confused with validity. The one refers to its usefulness for some purposes, whereas the other refers to its degree of correspondence with the real world. Since utility requires some degree of validity, some authors speak of a model as habing been « validated » by some use to which it has been put. Validity of a model, however, is not and end in itself but merely a means of enhancing the utility of the model – and usually only up to a point. Both validity and utility are commonly matters of degree. […] While validity is the ultimate test of a theory, the ultimate test of a model is its utility. » Developments in Simulation in Social and Administrative Science (Randall Schultz et al.1972)

En un sens, celle ci font étrangement écho à notion de usulfuness décrites par Barlas cité juste avant.

Parmis les autres auteurs {Kleindorfer1993} {Kleindorfer1998} Kleindorfer critiquent lui aussi cette approche et propose en 1993 de revoir cette approche, qui selon lui est limitante, sous l’œil des développements épistémologiques plus récent. (popper / faillibilisme)

Ainsi, et toujours en dehors de toute considération technique, la validation en plus d'être processus continu, serait un objet semi-formel et porteur de composantes subjectives.

« […] it is impossible to define an absolute notion of model validity divorced from its purpose. Once validity is seen as “usefulness with respect to some purpose”, then this naturally becomes part of a larger question, which involves the “usefulness of the pur- pose” itself. Thus, in reality, judging the validity of a model ultimately involves judging the validity of its purpose too, which is essentially non technical, informal, qualitative process. »{Barlas1996}

« Validation is a prolonged and complicated process, involving both formal/quantitative tools and informal/ qualitative ones. » {Barlas1996}

Cette thèse défendu par Barlas {Barlas1990} {Barlas1994} propose de positionner  l'objet « validation » dans une épistémologie « relativiste ».

Un témoignage important est quand même celui de Jay W. Forester, figure historique qui a donné naissance aux systèmes dynamiques, qui va affronter dès la publication d' « industrial dynamics » des critiques extrémement vive qui remettent en cause la scientificité de son modèle, en s'attaquant à sa méthode de validation.
