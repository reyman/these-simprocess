
PLAN DE GUERRE : 

- Reecrire  la partie épistémo en lien mais en faisant attention à ne pas empiéter sur le reste 


- Faire une synthèse sur deux points, entre philo et V et V , s'arreter sur la spécifité de la géographie et des shs pour introduire l'inflexion dans les années 80.


Arguments : 

%%%%%%%%%%%%%%%%%%%%%%%%%%%%%%%%%%%%%%%%%%%%%%%%%%%%%%%%%%%%%%%%%%%%%%%%%%%%%%%%%%%%
%%%%%%%%%%%%%%%%%%%%%%%%%%%%%%%%%%%%%%%%%%%%%%%%%%%%%%%%%%%%%%%%%%%%%%%%%%%%%%%%%%%%
%%%%%%%%%%%%%%%%%%%%%%%%%%%%%%%%%%%%%%%%%%%%%%%%%%%%%%%%%%%%%%%%%%%%%%%%%%%%%%%%%%%%
%%%%%%%%%%%%%%%%%%%%%%%%%%%%%%%%%%%%%%%%%%%%%%%%%%%%%%%%%%%%%%%%%%%%%%%%%%%%%%%%%%%%

- Débats entre Mécanisme Générateurs et Generative Science
	- presentation des idées et de la controverses
	- controverse grune yannof / elsenbroich
	- equifinalité n'est pas suffisante

Ouverture sur les mécanismes générateurs ...

%%%%%%%%%%%%%%%%%%%%%%%%%%%%%%%%%%%%%%%%%%%%%%%%%%%%%%%%%%%%%%%%%%%%%%%%%%%%%%%%%%%%

Avec parfois des désacords certains, comme par exemple entre Conte2007 et Epstein2007 qui revient sur la notion de science générative supporté par ce dernier, et largement repris par la communauté agent. Il semblerait que cela soit deux visions qui s'affronte \autocite[698]{Livet2014}, entre d'une part les tenants des \enquote{Mécanisme générateur ?} \autocites{Hedstrom2010, Conte2007, Manzo2007} } et celui des \enquote{générative science ?} \autocite{Epstein1999}, mais quelle différence exactement ?

La phrase d'\textcite{Epstein1999} \textit{If you didn’t grow it, you didn’t explain it.} est moins ambigue si on regarde la clarification faite par l'auteur en 2006 \autocite{Epstein2006} : 

\foreignquote{english}{The scientific enterprise is, first and foremost, \textbf{explanatory} [...] If you didn’t grow it, you didn’t explain it. It is important to note that we reject the converse claim. Merely to generate is not necessarily to explain (at least not well). A microspecification might generate a macroscopic regularity of interest in a patently absurd—and hence non-explanatory—way. For instance, it might be that Artificial Anasazi [Axtell, et al. (2002)] arrive in the observed (true Anasazi) settlement pattern stumbling around backward and blindfolded. But one would not adopt that picture of individual behavior as explanatory. In summary, \textbf{generative sufficiency is a necessary, but not sufficient condition for explanation.}} 

Il n'empeche, soulever le problème de l'équifinalité ne suffit apparemment pas, d'ailleur cela n'avait pas convaincu \textcite{Yanoff2008} qui voyait dans cet aveux une faiblesse dans la capacité causale de ces modèles. 

\enquote{Si une telle simulation « générative » peut être vue comme une condition nécessaire pour une science sociale computationnelle, elle ne suffit pas à fournir une explication ultime du phénomène. Tout d’abord, aux fonctions de la simulation doit correspondre un processus causal (Conte, 2007). De plus, ce type de modèle permet d’identifier un candidat explicatif pour ce phénomène, sans que ce soit nécessairement la seule explication possible, ni même forcément l’explication pertinente dans tous les cas de figure. La position extrême de Joshua M. Epstein a été critiquée pour la modélisation à base d’agents par Michael W. Macy et Andreas Flache dans leur ouvrage de synthèse sur la sociologie analytique (2009), où l’on préfère la notion plus large de \enquote{mécanismes générateurs}} \autocite{Livet2014}

La chaine causale mis en oeuvre dans le modèle est contrainte par les indicateurs que l'on voudrait mesurer.

“Un réaliste récuserait, d’une certaine manière, l’opération d’euphémisation de la notion de cause, qui réduit celle-ci à la simple notion d’une ‘condition initiale’, et, surtout, conduit à identifier l’explication à la simple constatation de relations régulière de succession entre des phénomènes antécédents et des phénomènes conséquents (Vrin, 1988, 130). 
Je souligne : le souci qui s’exprime ici est celui d’une compréhension ou d’une intelligibilité plus profonde du réel, au sens où ce qui est visé, c’est le mode de manifestation des phénomènes. Il ne suffit pas de décrire et mesurer des régularités, il faut en rendre raison, il faut donner une intelligibilité aux lois elle-memes.

%%%%%%%%%%%%%%%%%%%%%%%%%%%%%%%%%%%%%%%%%%%%%%%%%%%%%%%%%%%%%%%%%%%%%%%%%%%%%%%%%%%%
%%%%%%%%%%%%%%%%%%%%%%%%%%%%%%%%%%%%%%%%%%%%%%%%%%%%%%%%%%%%%%%%%%%%%%%%%%%%%%%%%%%%

Evaluer plutot que valider


- Une définition des validations plus adaptée aux construction dans les sciences sociales, Amblard2006. 


%%%%%%%%%%%%%%%%%%%%%%%%%%%%%%%%%%%%%%%%%%%%%%%%%%%%%%%%%%%%%%%%%%%%%%%%%%%%%%%%%%%%
%%%%%%%%%%%%%%%%%%%%%%%%%%%%%%%%%%%%%%%%%%%%%%%%%%%%%%%%%%%%%%%%%%%%%%%%%%%%%%%%%%%%

T : Ok pour les débats et les définitions, mais quid des outils pour guider la construction de tels modèles ?
	- vu la permanence de la problématique, stop aux guides uniquement méthodologiques
	- qui des plateformes techniques pour évaluer les modèles ?

%%%%%%%%%%%%%%%%%%%%%%%%%%%%%%%%%%%%%%%%%%%%%%%%%%%%%%%%%%%%%%%%%%%%%%%%%%%%%%%%%%%%
%%%%%%%%%%%%%%%%%%%%%%%%%%%%%%%%%%%%%%%%%%%%%%%%%%%%%%%%%%%%%%%%%%%%%%%%%%%%%%%%%%%%

Dans l'étude mené par \textcite{Heath2009} entre 1998 et 2008 sur 279 publications, l'auteur considère que seul 35 \% des modèles sont validés conceptuellement et informatiquement, même si une amélioration est à noter entre 2005 et 2008, ou ce chiffre monte à 43\% environ. Toutefois, si dans l'ensemble des modèles agents récupérés, on ne garde que les modèles agents classés en science sociale, ce chiffre chute fortement, avec 28\% de modèle validés, et quasiment 37\% de modèle qui n'aborde même pas ce problème \Anote{survey_heath}.

Si on regarde plus du coté des techniques, comme par exemple les analyse de sensibilités, cité régulièrement pour leur utilité dans la \enquote{validation interne} des modèles \autocite{Amblard2006}, le résultat n'est guère plus encourageant. Du moins si on en croit l'étude de \textcite{Thiele2014} entre 2009-2010 \Anote{survey_thiele}, mais également celle plus restreinte de \textcite{Cottineau2015} sur le volume JASSS de Mars 2014 \Anote{survey_cottineau}.

Finalement, sans même faire intervenir les critiques issue de débats plus épistémologiques (comme ceux que l'on a vu précédemment), cette seule insuffisance dans l'utilisation des moyens existants pour évaluer nos modèles suffit largement à prolonger un cercle vicieux où l'absence d'évaluation nourrit une perpétuelle remise en question de cet outil et de sa scientificité \Anote{serpent_mer}. 
