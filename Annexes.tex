
\graphicspath{{Figure4/}}

\part{Annexes}

\appendix

\chapter{Historique du paradigme systémique}

\subsection{Retour sur la fondation et les apports du \enquote{paradigme systémique} au début du XXème siècle}
\label{ssec:systemique}

De la même façon que les épistémologues des sciences comme ici Olivier Orain \autocite{Orain2001}, l'auteur ne détaillera pas ici une approche inter-disciplinaire de la notion \footnote{Au sens donné par Piaget, voir note de bas de page \autocite {Orain2001}} de \enquote{système}, difficile à envisager dans un cadre global car sa diversité d'acceptation est fonction, d'une part de la rapide évolution de cette notion depuis les années 1940, et d'autre part la règle définissant l'acceptation de cette \textit{notion} dépend non seulement de la variabilité inter-disciplinaire, mais aussi intra-disciplinaire. Le terme \enquote{approche systémique} est alors proposé par \autocite{Orain2001} pour incarner cette diversité d'intégration par les disciplines des sciences sociales de la \enquote{théorie systémique} ou \enquote{systémique}.

La complexité d'approche caractéristique de cette notion est pour Jean Louis Lemoigne grandement lié à la reconstruction épistémologique \textit{a posteriori} de ce qu'il appelle \enquote{paradigme systémique}. Une acceptation qui parait d'autant plus justifié tant l'étude exhaustive de la ramification qui découle du concept est impossible, et sans rentrer dans les détails de querelles entre les différentes chapelles, il est acceptable de voir cette construction comme un processus de raffinement cumulatif. \hl{a dire mieux}

\subsubsection{La Cybernétique}
\label{ssubsec:cybernetic}

\paragraph{Des outils pour penser une nouvelle causalité}

Une des branches communément admises comme fondatrice du mouvement tient dans l'organisation des conférences de Macy entre 1942 à 1953. Celle ci sont considérés comme un des tout premier regroupement interdisciplinaire et marque une période de changement profond dans l'histoire des sciences en général, et particulièrement en science sociale. Celles ci vont réunir pendant plusieurs années autour d'une même table des acteurs majeurs des sciences physiques et sociales pour discuter autour de régularités communément observés, avec pour idée la construction d'un savoir commun que l'on pourra alors qualifier de trans-disciplinaire.

Les conférences naissent suite à la rencontre entre un mathématicien réputé au MIT N. Wiener, un neurobiologiste A. Rosenbluch, et un ingénieur électronicien J.Bigelow qui vont opérer un rapprochement entre l'homme et la machine entre 1942 et 1946 (pour rappel le premier ordinateur ENIAC est opérationel en 1946) par le biais de groupes inter-disciplinaires chargés d'explorer ce \textit{no man's land} à l'interface des deux disciplines.

Plusieurs \enquote{outils} dérivent de ces premiers séminaires organisés dès 1942 à la Josiah Macy, Jr. Foundation : la notion de \enquote{boite noire} ou système téléologique fonctionel, et la notion de \textit{feedback} ou causalité circulaire, avec pour objectif principal l'étude de l'homéostasie introduite auparavant par les travaux pionniers du physiologiste Walter Cannon en 1926.

Si la notion d'homéostasie pour des organismes vivants apparaît pour la première fois cité par Claude Bernard 1865, celle ci est reprise et étendue par Walter Cannon en 1932 dans le livre \textit{The Wisdom of the Body} \autocite{Cannon1932} comme « l’ensemble des processus organiques qui agissent pour maintenir l’état stationnaire de l’organisme, dans sa morphologie et dans ses conditions intérieures, en dépit de perturbations extérieures ». Ainsi dans le cadre de son application biologique cette rétro-action permet de décrire un certain nombre de mécanisme à l'oeuvre dans une cellule en interaction avec son environnement qui tente de maintenir de façon stable dans son milieu la concentration d'éléments comme les ions, la glycémie, etc.

L'attention des discutants dans ces premier séminaire porte donc avant tout sur l'ubiquité du concept et la pertinence de son transfert hors des systèmes biologiques. Wiener fait alors un rapprochement décisif entre les problématiques de calcul de trajectoire en balistique et des maladies nerveuses ayant pour symptôme l'ataxie. De ces discussions émergent alors un même schéma explicatif qui semble à la fois convenir à ces problématiques, la \enquote{causalité circulaire}. \autocite[774]{Pouvreau2013, Rosnay1975}

L'approche néo-béhavioriste retenue par les discutants \enquote{consiste à étudier un objet comme une \enquote{boite noire}, par l'examen de l'extrant de l'objet [i.e tout changement produit dans son environnement] et des relations entre cet extrant et l'intrant [i.e tout événement externe qui modifie l'objet]} \autocite{Pouvreau2013} En adoptant cette approche, le \enquote{comportement} d'une entité est perçu \enquote{comme tout changement extérieur détectable de cette entité par rapport à son environnement} , et par téléologique il faut entendre un comportement \enquote{finalisé} c'est à dire déterminé par un mécanisme de \enquote{rétroaction} négative. De la connaissance de ces entrants et de ces sortants, on peut en déduire qu'il existe une retro-action négative ou positive, ou \textit{feedback} permettant de décrire progressivement le système de commande de la boite noire.

L'introduction de cette \enquote{causalité circulaire} est pour l'époque loin d'être anodine car elle remet en cause le schéma classique linéaire cause \textrightarrow conséquence, qui se traduit dans le temps par la relation avant \textrightarrow après, la cause étant irrémédiablement suivi d'une conséquence. La possibilité de causalité circulaire, positive ou négative, brise ce schéma, et ne permet plus d'isoler un ordre entre cause et conséquence, c'est le problème de \enquote{la poule et de l'oeuf}. En réintroduisant la poursuite d'un but, on injecte une autonomie, une spontanéité, une dynamique entre objets qui était jusque là absente de la causalité linéaire déterministe.

Appliqué à un système servo-mécanique, la stabilité de celui-ci suppose la capacité à anticiper et à annuler les agressions extérieures par une capacité de régulation (flexibilité) qui repose plus alors sur la dynamique des interactions que sur la structure physique en place (rigidité), un mode de fonctionnement impossible si on se place dans le cadre de la \enquote{pensée classique} de l'époque.

%Dans "Behavior, Purpose and Teleology", le terme téléologie est à ce titre utilisé comme un synonyme de "l'objectif controllé par la rétroaction".\footnote{wikipedia}

\paragraph{La réintroduction du concept de \enquote{téléologie}}

Avec la mise en place d'une classification de ces comportements, et en prenant distance du concept de \enquote{causalité finale} qui lui était rattaché, les auteurs espèrent ainsi redorer le concept de téléologie, renouant avec la reconnaissance de l'\enquote{importance du but} qui avait disparu avec la mise au ban de ce concept. Reprenant les explications de \autocite[776]{Pouvreau2013}, celui-ci cite \autocite[23-24]{Rosenblueth1943} \enquote{[...] Puisque nous considérons la finalisation comme un concept nécessaire afin de comprendre certains modes de comportement, nous suggérons qu'une étude téléologique est utile si elle évite les problèmes de causalité et se limite à s'attacher à l'étude du but [...] Le comportement téléologique devient synonyme de comportement contrôlé par une rétroaction négative et gagne donc en précision par une connotation suffisamment restreinte.} La finalité est reintroduite via le concept de \enquote{téléologie}, mais elle est libéré de la notion de \enquote{causalité} qui lui était autrefois associé. Elle redevient l'étude des comportement associé à un but, dont l'importance ne peut plus être nié, et redevient compatible avec le concept autrefois opposé de déterminisme.\footnote{Pour donner un exemple peut-être plus parlant, l'étude en biologie des comportement oeuvrant dans la formation d'un organisme par une méthode téléologique n'empêche pas l'usage d'un cadre de pensée déterministe  correspondant à la formation d'un même organisme à partir d'un même code initial (un déterminisme largement remis en cause depuis, voir par exemple \href{http://www.nytimes.com/2014/01/21/science/seeing-x-chromosomes-in-a-new-light.html?ref=science&_r=0}{New York Times} )}

De ces discussions deux articles fondateurs à la fois des sciences cognitives \autocite[23]{Dupuy2000} et de la cybernétique vont être publiés : \textit{Behavior, Purpose and Teleology}ou Rosenblueth, Wiener, et Bigelow \enquote{ propose de déconstruire la distinction entre action volontaire et acte réflexe, en assimilant la volonté à un mécanisme de rétro-action (\textit{feedback})}; et \textit{A logical calculus of the ideas immanent in nervous activity} où McMulloch et Pitts donne \enquote{une base purement neuroanatomique et neurophysiologique au jugement synthétique \textit{à priori}, et de donner ainsi une neurologie de l'esprit}

\paragraph{ Les limites du transfert des concepts aux sciences sociales}

\subparagraph{Introduction aux sciences sociales}
Parmis les auteurs de ces premiers séminaires organisés entre 1942 et 1944 figurent deux représentant des sciences sociales, Gregory Bateson et Margaret Mead. Enthousiastes, il vont rapidement trouver dans l'étude des concepts développés dans ce premier séminaire (1942) un écho à leur propre travaux sur la dynamique sociale, la notion d'homéostasie n'étant qu'un nouveau mot permettant de rassembler des travaux existants déjà au fait de ces phénomènes. Cette mise au jour de problématiques commune entre le biologique et le mécanique permet d'envisager la construction d'un référentiel lui aussi commun; une prise de conscience qui va amener les auteurs du cercle de réflexion initial à envisager rapidement l'élargissement de celui-çi à l'ensemble des acteurs des sciences sociales.

La suite des conférences de Macy (1946-1952) sera organisés par Arturo Rosenbluch et son ami Warren McCulloch, un autre neurobiologiste. Cette ouverture vers les sciences sociales est timide dans un premier temps, et ce n'est qu'à la 2ème conférence en octobre 1946 sur une suggestion de Lazarsfeld que les conférences concrétise cette ouverture dans le cadre d'un sous séminaire intitulé \textit{Téléogical Mechanisms in Society}. La 4ème conférence acte cette ouverture et introduit pour la troisième fois de suite une modification de l'intitulé, avec cette fois ci l'adjonction d'une dimension sociale à un objet d'étude, qui apparaît encore à cette date difficile à définir : \enquote{la causalité circulaire et des mécanismes de \textit{feedback} dans les systèmes biologiques et sociaux}. Le terme \textit{Cybernetics} est pour la première fois introduit dans les séminaires par Wiener en 1946. Il faut toutefois attendre 1949 et la septième conférence pour que sous l'influence d'un nouveau participant nommé H. Von Foerster, ce terme chapeaute de façon définitive les prochains intitulés de séminaires. Au final, ces dix séminaires vont participer de l'émergence de la \enquote{science cybernétique} en \enquote{permettant l'échange effectif de savoir et d'experiences, tant entre les disciplines qu'entre les sciences et la société}, réalisant par là un des objectifs annoncé par Wiener et Rosenbluch dans leur classification, faisant de la cybernétique une \enquote{[...] science générale des systèmes à comportement finalisé ayant principalement pour objet ceux dont le comportements est \enquote{téléologique} } \autocite{Pouvreau2013}

\subparagraph{Des biais mécanisistes mettent en échec ce premier transfert}

Wiener mais aussi d'autre acteurs de la cybernétique ont vus assez tôt tout l'intérêt que pourrait apporter l'utilisation et le transfert d'outils comme \enquote{la boite noire}, ou le principe de régulation par \enquote{rétro-action} une fois appliqué à l'étude des interactions dans les systèmes sociaux. Mais les difficultés d'applications et les critiques ont rapidement mis à mal cet objectif trans-disciplinaire, pour plusieurs raisons qui tiennent : d'une part à l'existence de restriction mathématiques remettant en cause la scientificité des résultats obtenus : (a) les statistiques sur le long terme étant difficile à obtenir (b) la difficulté à minimiser la distance entre observateur et phénomène observés, et donc le biais qui s'applique aux données dans un tel cadre; et d'autres part au réductionnisme et le biais mécanicistes touchant la vision de certains acteurs des conférences de Macy  : \enquote{[...] la vie était pensée comme un dispositif de réduction d'entropie ; les organismes et leur associations, en particulier les hommes et leurs sociétés, l'étaient comme des servomécanismes ; et le cerveau comme un ordinateur} \autocite[784]{Pouvreau2013}

\autocite[782]{Pouvreau2013} explique très bien les limitations qui font  de l'extension de la cybernétique au sciences humaines une simple \enquote{[...] ressemblance superficielle au niveau du formalisme. Ne serait-ce que parce que dans un système tel que conçu par la \enquote{première} cybernétique, par définition fermé à l'information, la téléologie ne peut qu'être confinée au cercle d'un but déterminé; et que pour cette raison, ce modèle ne permet pas de comprendre de quelle manière un système peut être amené à redéfinir ses buts à partir de ses interactions avec son environnement, la pertinence d'une téléologie relative à des buts \textit{intentionels} restant donc intacte en sciences humaines}.

\subsubsection{La GST ou la théorie des \enquote{systèmes ouverts}}
\label{ssubsec:gst}

Cette incapacité de la première cybernétique à coller aux problématique des systèmes sociaux va trouver un écho plus positif dans un courant qui se développe en parallèle du mouvement cybernétique. Ce mouvement fondé par le biologiste Ludwig Von Bertalanffy en 1937 peut être considéré comme la deuxième branche venant enrichir le paradigme systémique. Tout en apportant de nouveaux concepts, celui ci va se positionner de façon critique par rapport à la \enquote{première cybernétique} tout en englobant par la suite les autres innovations qui proviendront de ce courant, Asbhy jouant le rôle important de médiateur entre ces deux courants.\autocite[]{Pouvreau2013} De cette prise de position va peu à peu découler la construction d'une théorie établissant une méthodologie logico-mathématique à vocation unifiante, accessible à n'importe quel champs disciplinaire pour décrire les lois de structure similaires (isomorphe). \autocite{LeMoigne2006a}.

Ainsi rapporté par LeMoigne en 1977, cette \enquote{vision stupéfiante est celle d'une une théorie générale de l'univers, du système universel} \autocite[59]{Lemoigne1977}. Le mot \enquote{Vision} est ici quasi synonyme de \enquote{Révélation}, car elle amène à voir une tout autre approche du réel pour qui s'en rapporte. Ainsi selon les mots même de Bertalanffy, \enquote{De tout ce qui précède se dégage une vision stupéfiante, la perspective d'une conception unitaire du monde jusque-là insoupçonnée. Que l'on ait affaire aux objets inanimés, aux organismes, aux processus mentaux ou aux groupes sociaux, partout des principes généraux semblables émergent} \autocite[59]{Lemoigne1977} \autocite[220]{Bertalanffy1949}. Une idée déjà existante dans la maxime célèbre de Claude Bernard en 1885, remise au gout du jour par \autocite{Lemoigne1977}, celle-ci résume toute la souplesse offerte par cette notion d'un point de vue de la modélisation :  \enquote{Les systèmes ne sont pas dans la nature mais dans l'esprit des hommes}

Cette théorie nommé \textit{General System Theory} (GST) est évoqué pour la première fois en public en 1937-38 par Bertalanffy, s'ensuit alors la rédaction d'une première ébauche en 1950, et il faudra attendre 1968 pour qu'un ouvrage titré \textit{General System theory: Foundations, Development, Applications} proposent une synthèse de toutes les avancées. La durée de développement de cette théorie n'est pas anodine, et si on en croit Pouvreau \autocite{Pouvreau2013} qui a analysé en détail la très vaste littérature associé à cette thématique, cette théorie n'en est pas vraiment une en réalité. En effet l'état inachevé du projet de Bertanlanfy laisse plus à penser qu'il s'agit là d'un \enquote{projet}, et c'est à ce titre que Pouvreau préfère employer le terme de \enquote{systémologie générale} pour désigner ce qu'il définit alors comme \enquote{le \textit{projet} d'une \textit{science de l'interprétation systémique} du \enquote{réel} } \autocite[9]{Pouvreau2013}. L'hypothèse défendu par Pouvreau étant que cette \enquote{[...]science de l'interprétation systémique du \enquote{réel} se caractérise en fin de compte comme une herméneutique, au sens où elle a pour vocation d'élaborer à la fois les moyens de construire des interprétations systémiques d'aspects particulier du \enquote{réel} sous la forme de modèles théoriques spécifiques et les moyens d'interpréter à leur tour de tels modèles comme des déclinaisons de modèles systémiques théoriques d'un degré de généralité supérieur.}\autocite[9-10]{Pouvreau2013}

Mais avant de même de fonder ce projet unifiant qui par la suite va rayonner et être absorbé (non pas sans déformation ..) dans un grand nombre de disciplines, dont la géographie, il est intéressant de rappeler comment la théorie biologique de Bertalanffy a participé de la formation de grandes notions comme l'\enquote{équifinalité} ou l'\enquote{auto-organisation}, des notions aujourd'hui communément admises comme fondatrice du paradigme actuel de la \enquote{complexité}.

Bertalanffy poursuivant depuis 1937 avant tout cet objectif de dépasser la compréhension des systèmes biologiques  englué jusque alors dans une dualité opposant les \enquote{vitalistes} et \enquote{mécanistes}. La synthèse de ces travaux est organisé dans une \enquote{biologie organismique} qui fonde une troisième voie visant d'une certaine manière la réconciliation entre les deux approches \autocite[55-56]{Lemoigne1977} \autocite[258]{Bertalanffy1949}. Avec cette nouvelle biologie théorique il s'agissait donc d'incarner \enquote{l'avenir de la biologie" en établissant via la mobilisation de moyen scientifique (analyse et analogies physico-chimique et mathématique du vivant) écartant la métaphysique/psychiques, un programme de recherche des \enquote{loi systémiques ou d'organisation à tous les niveaux de la nature vivante} entendues comme \enquote{l'explication de l'harmonie et de la coordination des processus à partir de la dynamiques des forces qui leur sont immanentes}}\autocite[456]{Pouvreau2013}. Principalement \enquote{ordonnées en direction de la conservation de la totalité}\autocite[440-458]{Pouvreau2013} dans une \enquote{tendance à une complication croissante}, cette \enquote{Gestalt organique} de la théorie \enquote{organismique} de Bertalanffy place \enquote{l'Organisation} des processus comme une véritable problématique de recherche, et met de coté la question de la \enquote{finalité} du vivant.\autocite[455-457]{Pouvreau2013}

Déjà tout à fait conscient que \enquote{le tout est plus que la somme des parties} Bertalanffy admet que l'étude des mécanismes physico-chimiques des processus vitaux tient plus d'une heuristique de recherche, une \enquote{méthode téléologique qui permet \enquote{d'examiner jusqu’à quel point le caractère de conservation de la totalité se manifeste dans les processus qui se déroulent en eux}} sans jamais arriver à en donner une complète description.\autocite[464]{Pouvreau2013}

Cette \enquote{biologie théorique organismique} (également appelé de façon synonyme par Bertalanffy \enquote{théorie systémique du vivant}) montre en bien des points toutes les prémisses d'une pensée systémiste et non réductionniste qui dépasse déjà largement le cadre seul de la biologie, et cela même avant 1937 et l'introduction de \enquote{systèmes ouvert} \autocite[499]{Pouvreau2013} qui ont fait la renommée de l'auteur.  Cette \enquote{biologie organismique} de Bertalanffy, bien évidemment construite sur les acquis et l'aide de bien d'autres de ces contemporains (voir \autocite{Pouvreau2013}, arrive à maturité en 1937 \autocite[14]{Pouvreau2013}, et présente déjà à ce stade tout les traits d'une première \enquote{systémologie restreinte}, qui va servir d'\enquote{antichambre} à la formation de la future \enquote{systémologie générale} (la première évocation publique date de 1945, mais des traces indirectes de ses premiers discours semblent remonter à 1937).\autocite[670]{Pouvreau2013} de Bertalanffy.

% D'abord on fait le point sur les principes (ce qui suppose de faire une grosse parenthèse avec tout ce que l'on a décrit sur la thermodynamique) et ensuite on peut passer à la critique, évoquant l'équifinalité et la hierarchisation de processus qui permet de recentrer aussi l'étude des boites noires.

L'articulation entre les deux \enquote{principes organismiques} qui fondent sa théorie apparaît de façon très claire dans une première définition du vivant en 1932, ici cité dans sa version telle que raffinée par Bertalanffy en 1937, date à laquelle selon

%Définition des deux principes organismiques !?

Le premier principe théorique \enquote{organismique} de Bertallanfy s'appuie sur le principe biologique fondamental qu'il a énoncé dès 1929 avec la \enquote{conservation du système organique en équilibre dynamique}. Un équilibre qui parait statique d'un point de vue extérieur, mais qui est en réalité dynamique car son existence même est basé sur la remise en jeu permanente d'une partie du travail effectué par la cellule pour maintenir le système organique loin de l'équilibre \enquote{vrai} (physique, c'est à dire celui qui correspond à une mort thermique, ou chimique qui ne peut pas produire non plus de travail à l'équilibre). Un \enquote{équilibre de flux} qui ne peut être réalisé que parce que l'organisme n'est ni un système fermé, ni un système statique, mais un système dont l'ordre et l'organisation (def à valider ici) est fondé sur un travail issue d'un \enquote{flux} de matière et d'énergie résultat d'une transaction à double sens avec son environnement. \autocite[472]{Pouvreau2013} Je me permettrai de citer ici Morin, qui reprenant Héraclite, évoque très bien cet antagonisme à l'oeuvre dans les systèmes organiques, mais aussi par extension sociaux \enquote{Vivre de mort, mourir de vie} : \enquote{ ne vivons-nous pas de la mort de nos cellules qui vieillissent et se décomposent pour laisser la place à des cellules jeunes ? [...] La vie et la mort sont certes deux ennemies fondamentales, mais la vie lutte contre la mort en utilisant la mort. Néanmoins, il est tuant de se régénérer en permanence. C’est épuisant. Finalement, on mveurt à force de rajeunir. On meurt de vie. } \autocite{MorinXX}

% Critique cybernétique
Le principe d'\enquote{équilibre des flux}, même si il peut être rapproché du concept d'\enquote{homéostasie} définit par les tenants de la \enquote{première Cybernétique} (en analogie avec les systèmes mécaniques) comme la \enquote{conjonction des processus par lesquels, nous autres, être vivants, résistons au courant général de corruption et de dégénérescence} est trop généraliste pour application en tant que tel à toute les notions de régulations organiques. \autocite[194]{Morin1977} \autocite{Wiener1950}. L'\enquote{homéostasie} tel que définit par Wiener dans le cadre de la Cybernétique s'avère en réalité être un mécanisme de régulation organique parmi tant d'autres, tous n'étant pas basé sur le schème de rétro-action. A ce titre, la notion d'\enquote{homéostasie} pourtant quasi semblable dans sa définition à l'équilibre de flux dans un système ouvert, mobilise en réalité un tout autre fonctionnement que le schème de rétro-action Cybernétique, et tient plus de l'extension aux systèmes ouverts du principe dit de \enquote{Le Chatelier}. De la même façon la régulation intervenant dans le processus de croissance des organismes qui nécessite la régénération, et l'évolution des structures dans le temps n'est pas compatible avec l'ordre structural pré-établi des machines et le scheme de rétro-action promis par la Cybernétique. La vision \enquote{machinaliste} limité/biaisé des premiers cybernéticiens n'est donc pas satisfaisante pour une application aux systèmes organiques, dès lors qu'il faut accepter la constance non pas des structures mais des interactions entre les structures. Bertalanffy développe une classification plus complète de ces régulations qu'il considère selon le type de leur téléologie, et introduit le concept d'\enquote{équifinalité} comme téléologie dynamique moteur dans la construction et le maintien des systèmes organiques. Dans ce contexte, le principe d'équifinalité \autocite[131]{Pouvreau2013}, est ainsi évoqué pour la première fois comme la possibilité d'atteindre le même état finalisé à partir de trajectoires quelconques, un processus impossible dans le cadre de système fermé où les condition initiales définissent par avance l'état final. Ce faisant, Bertalanffy introduit la primauté de l'ordre dynamique sur l'ordre structurel et fait de l'équifinalité un concept qui dérive de l'ouverture des systèmes. \autocite[489]{Pouvreau2013} \autocite[647]{Pouvreau2013} Un exemple illustrant les effets de l'équifinalité dans les organismes vivants peut être montré avec le processus de division embryonnaire. Ainsi un organisme a qui ont impose la fragmentation, la régénération, ou des blessures d'unités biologiques élémentaires comme les gènes ou les chromosomes va de façon constante s'organiser suivant un plan pré-établi menant à la \enquote{constitution d'un tout}, autrement dit un organisme complet.

%Il nécessite un autre mode d'explication de processus téléologique, celui de la cybernétique s'avérant incompétent au regard du principe d'équifinalité observé dans les systèmes organiques.

% Bertalanffy s'appuie dans sa critique à raffiner sa classification des téléologies, ce qui lui permet d'introduire le concept d'équifinalité comme sous-type de téléologie dynamique, un type de processus de régulation qui selon lui ne peut pas être expliqué par les schèmes cybernétique initiaux, seulement capable de mobiliser le concept de finalité en regard d'une explication basé sur un arrangement structural pré-établi (une machine faites de composants) et non pas l'ordre  dynamique propres au système en équilibre de flux.

La combinaison des deux principes \enquote{organismique} menant à la théorie des \enquote{système ouvert en équilibre de flux} deux heuristiques de recherches \autocite[481]{Pouvreau2013}:
\begin{itemize}
\item La subordination du \enquote{principe de hierarchisation} à celui du \enquote{système ouvert en équilibre de flux}, autrement dit la genèse et le maintien de l’ordre hiérarchique d’un \enquote{système organique} est conditionné par l'existence d'un \enquote{système ouvert en équilibre de flux}
\item  La relation précédente est un principe ubiquitaire s’appliquant à tous ses niveaux
\end{itemize}

Cet idée sera particulièrement fructueuses une fois articulé avec le principe d'un enboitement des systèmes, l'accroissement du degré de liberté dans un système résultant de l'équifinalité.
 \autocite[38]{Bertalanffy1973} \autocite[786-788]{Pouvreau2013}

%Developpement rendu possible uniquement par l'apport des théories de la thermodynamique ... l'expression d'une trajectoire indépendamment de l'état final, celui ci n'est qu'un processus de régulation parmis d'autres, car ce même système organique est non seulement capable de maintenir son état mais choses plus importante, il permet surtout de produire de l'organisation, de la complexification.

% Relation avec science sociale ??
% => entéléchie /
Cette notion d'équifinalité reliant un niveau micro à un niveau macro pourra par la suite être transposé dans les système sociaux, le parallèle de l'individu comme acteur réflexif dans la société sera mobilisé par ?

De ce fait la Cybernétique n'est pour Bertalanffy qu'un cas particulier dans une systémologie dont il pense qu'elle peut être beaucoup plus universelle... ++ Homéostasie avec Ashby ? ++

Tel que définie, cette notion d'équilibre dynamique de Bertalanffy est bien différente de celle produites en physique et en chimie, qui se caractérise justement par l'absence de travail disponible, l'énergie disponible étant minimale. Pour que la permanence d'un ordre puisse être effective dans la théorie organismique, il faut qu'il y ai un échange, un flux d'énergie mais aussi de matière possible avec l'environnement; une différenciation qui amène Bertalanffy à développer dès 1937 une théorie des \enquote{systèmes ouverts}, la seule capable de s'appliquer également à des systèmes sociaux par la suite.

% Sur l'ouverture des systèmes
Pour mieux comprendre en quoi cette ouverture est importante pour l'application du paradigme systémique aux sciences sociales, il faut revenir quelques décennies en arrière pour définir les limitations des premier systèmes issue de la thermodynamiques, limitations qui par la suite ont irrigués les réflexions initiales des cybernéticiens tout autant que les motivations de Bertalanffy pour les dépasser dans le cadre de sa théorie \enquote{organismique}

La seconde loi de la thermodynamique esquissé par Carnot et formulé par Clausius en 1850 montre que l'energie calorifique ne peut se reconvertir, elle se \textit{dégrade} et perd son aptitude à effectuer un \textit{travail}. Clausius nomme \enquote{entropie} cette diminution irréversible de l'aptitude à se transformer et à effectuer un travail, propre à la chaleur.\autocite[35]{Morin1977} Prigogine dans la \textit{fin des certitudes} écrit à propos de l'entropie qu'elle \enquote{[...] est l’élément essentiel introduit par la thermodynamique, la science des processus irréversibles, c’est-à-dire orientés dans le temps.}

C'est Boltzmann, Gibbs et Planck qui vont par la suite faire le lien entre le niveau micro des particules et la notion de chaleur. Parce que la chaleur est caractérisé par l'agitation désordonné des molécules dans un systèmes, l'entropie devient plus qu'une simple réduction du travail, c'est aussi l'ordre et le désordre des molécules qui en est la cause. Cette transformation s'effectue avec création d'entropie, une \enquote{quantité de désordre} qui ne peut que croître dans le temps et cela jusqu'à atteindre une valeur maximale équivalente à ce nouvel état d'équilibre. De ce fait et de façon générale celle-ci définit comme évolution irréversible toute transformation réelle dans un système isolé (Système où la frontière est totalement imperméable : l'Univers est par définition un tout englobant) ou fermé (Système ou la frontière est perméable aux flux entrant ou sortant d'énergie mais imperméable aux échanges de matière : La Terre reçoit de l'énergie du soleil) partant d'un état non stable et se dirigeant vers un nouvel état stable.  Ainsi si on considère l'univers comme un méta-système isolé englobant tout les autres, alors ce second principe a pour corollaire que l'entropie de l'univers augmente vers un état de désordre maximal qui se traduit en définitive par une mort thermique.

L'intuition de cette possible analogie entre loi gouvernant systèmes physiques et biologiques est issues des réflexions menés par Boltzman, qui comme ces contemporains du XIX siècle est admiratif pour la récente théorie évolutive de Darwin \autocite[27]{Prigogine1996}. Celui ci tente alors un parallèle avec ses propres travaux sur la seconde loi de thermodynamique, que l'on retrouve dans une des fameuses citations présente dans son livre \enquote{second law of thermodynamic} : \enquote{ The general struggle for existence of living beings is therefore not a struggle for raw materials — the raw materials of all organisms in the air, water and soil are in abundance there — nor about energy, which in the form of heat, unfortunately, is contained abundantly [but unfortunately] [in]convertible in each body, but a struggle for entropy, which is available [disposable] by the transfer of energy from the hot sun to the cold earth.}

% Le sys ouvert/fermé , de la thermodynamique à la biologie ?
Le point de vue de Boltzmann est repris et théorisé par Alfred J. Lotka, un mathématicien, chimiste et statisticien qui va largement influencé par la suite Bertalanffy dans la formation de sa \enquote{systèmologie générale} \autocite[178]{Pouvreau2013} par ces études de la démographie des populations et des flux de matières dans le monde biologiques \autocite[545-546]{Pouvreau2013} , toutes deux usant largement des équations différentielles (un premisse d'isomorphisme mathématique applicable à diverses disciplines pour qui quiconque tente de rentrer dans le formalisme de Lotka, et par la suite Lotka et Volterra \autocite[550]{Pouvreau2013}). De la même façon que Bertalanffy par la suite, celui ci ignore sciemment les débats entre \enquote{vitalistes} et \enquote{mécanicistes}, et adopte un point de vue unificateur qui vise la réconciliation entre système physique et système biologique, et part à la recherche d'isomorphisme en s'appuyant sur le processus d'irréversibilité commun aux deux paradigmes : \enquote{[...] the law of evolution is the law of irreversible transformation; that the \textit{direction} of evolution [...] is the direction of irreversible transformations. And this direction the physicist can define or describe in exact terms. For an isolated system, it is the direction of increasing entropy.  The law of evolution is, in this sense, the second law of thermodynamics} \autocite[26]{Lotka1925}.

Dès 1922 \autocite{Lotka1922a} \autocite{Lotka1922b} Lotka une nouvelle théorie qui acte la capacité de capturer de l'énergie comme un optimum à atteindre guidant la sélection tel quel est décrite par l'évolution Darwinienne. Il est également l'un des premier à percevoir les limites des lois actuelle de la thermodynamiques pour expliquer les processus du vivants, ainsi \enquote{Tenant pour légitime de traiter les êtres vivants et leurs associations comme des systèmes physiques, Lotka insistait toutefois sur le fait qu’il s’agit de « systèmes ouverts » aux flux de matière et d’énergie (ainsi que Raymond Defay (en 1929) et Bertalanffy (en 1932) les qualifièrent plus tard), capables d’échapper à l’équilibre thermodynamique défini par un maximum d’entropie promis aux systèmes fermés par le Second Principe, et d’évoluer vers une structuration croissante.} \autocite[179]{Pouvreau2013}

En effet pour un système vivant, l'état d'équilibre tel que décrit pour des systèmes clos ou isolé, correspond à un état de mort cellulaire. Hors, il est prouvé empiriquement à cette période que les systèmes vivants évolue dans un environnement chimique en perpétuel évolution loin de l'équilibre, et sont de fait capable de maintenir un haut niveau d'organisation par l'échange d'énergie et de matière avec l'environnement. Autrement dit, il n'est pas possible de concevoir l'équilibration permanente des systèmes vivants comme le résultat d'une évolution entropique croissante \autocite[248]{Lemoigne1977}. Des résultats énoncés sous forme de loi en 1929 par Bertallanfy, qui fait de \enquote{la conservation de système organique en équilibre dynamique} un \enquote{principe biologique fondamental}, et qui deviendra plus tard dans sa théorie \enquote{organismique}, le premier principe de  \enquote{système ouvert} en \enquote{équilibre de flux}. \autocite[492]{Pouvreau2013}

Mais en voulant faire l'analogie entre ces deux systèmes, une question va rapidement se poser aux scientifiques. \enquote{Comment la progression irréversible du désordre pouvait elle être compatible avec le développement organisateur de l'univers matériel, puis de la vie, qui conduit à homo sapiens ?}, une question qui va engendrer la problématisation et un changement de point de vue radical. Comme le résume bien \textit{a posteriori} Morin dans son premier tome de \textit{La Méthode}, \enquote{A partir du moment où il est posé que les états d'ordre et d'organisation sont non seulement dégradables, mais improbables, l'évidence ontologique de l'ordre et de l'organisation se trouve renversée. Le problème n'est plus : pourquoi y a-t-il du désordre dans l'univers bien qu'il y règne l'ordre universel ? C'est : pourquoi y a-t-il de l'ordre et de l'organisation dans l'univers ? } \autocite[37]{Morin1977}

Avec de tel propos se pose alors rapidement la question des mécanismes à l'oeuvre dans le vivant qui permettrait en quelque sorte de rétablir l'universalité de la seconde loi thermodynamique. Bien qu'intuité par de nombreux chercheur comme Lotka ou Bertalanffy, il faudra attendre les années 1940 pour que s'amorce plus concrétement ce rapprochement entre paradigme évolutionniste et domaine de la thermodynamique, concrétisé par le partage des théories entre biologistes et physiciens, qui va se réaliser notamment sous le couvert des récents progrès de ce dernier, permettant l'émission de nouvelle hypothèses.

Reprenant l'acceptation d'un système ouvert, c'est le livre \textit{What is Life} de Schrödinger (1944) qui va marquer le plus les esprits, et soulève le mieux ce paradoxe à la croisée des deux théories. Deux choses au moins fascine celui-ci \autocite{Foerster1959}, d'une part l'existence d'un code héréditaire qui définit au niveau micro la formation, l'organisation d'organisme au niveau macro (le principe \enquote{order-from-order}), d'autre part l'étonnante stabilité de ce code héréditaire immergé à 310 Kelvin \autocite[47]{Schrodinger1944}, et qui ne répond donc pas au fameux principe statistique \enquote{order-from-disorder} établit précédemment par Boltzmann.

En inscrivant comme nécessaire l'existence d'un code génétique comme un plan guidant l'évolution (tout comme Bertalanffy qui développe des théories similaires à la même époque), il introduit avec son concept de d'"entropie négative" un principe qui rend de nouveau compatible la seconde loi de thermodynamique avec l'évolution des systèmes biologiques : \enquote{le physicien attribuait le maintien de l’organisme dans un état \enquote{ stationnaire } éloigné de l’équilibre vrai à sa capacité de se \enquote{ nourrir } d’\enquote{ entropie négative } grâce à son ouverture sur son environnement. Une \enquote{ néguentropie } interprétée comme une \enquote{ création d’ordre à partir d’ordre } -- l’organisme créant un ordre spécifique à partir de la matière déjà ordonnée, structurée d’une manière déterminée mais devant être transformée pour ses besoins énergétiques, qu’il trouve dans son environnement} \autocite[502]{Pouvreau2013} Autrement dit, le maintien de l'organisation est un équilibre dynamique, un jeu à somme nulle où la création d'entropie est annulé par la capacité des organismes à transformer l'énergie, l'ordre puisé dans l'environnement pour maintenir ce degré d'organisation, un processus qualifié de néguentropique. Ce concept, déjà difficile à accepter tel quel dans sa généralité \autocite[225]{Lemoigne1977} va par la suite être raccroché à théorie de l'information de Shannon après son introduction en 1948 dans le microcosme Cybernétique. L'introduction de cette théorie étant un autre moment fort (avec la thermodynamique) ayant inspiré de nombreux développement dans la cybernétique. Mais les tentatives d'unification entre les deux théories débouche sur deux rapprochement possible, avec d'une part la qualification d'une \enquote{information pensé comme quantité physique} ou d'autre part l'expression des \enquote{quantité physique pensé comme de l'information}, selon que l'on adopte le point de vue de Wiener ou de Brilloin 1956 (auteur de la néguentropie qui associe qui associe \enquote{information} et principe de négentropie ). Ces points de vues font encore à l'heure actuelle l'objet de nombreux débats, certains voyant la physique de l'information comme un point de départ à creuser pour appeller une théorie de l'"organisation" \autocite[37-38]{Morin2005}, alors que d'autres n'y voient qu'un concept flou seulement basé sur la similitude des deux formules. Autant de ramifications naissent de ces positions, et leur présentation dépassent de loin le seul cadre d'étude de cette thèse, mais le lecteur pourra se référer au travail de \autocite{Triclot2007} pour mieux comprendre le point de départ d'un malentendu qui dure toujours /footnote{Voir par exemple la différence de ton qui existe entre le site http://www.eoht.info/page/Information+theory, mais aussi les notes de bas de pages de \autocite[277]{Lemoigne1977} }.

\autocite[482]{Pouvreau2013} Mais finalement plus que les idées développés par Shrödinger, la plupart étant déjà largement sous entendu dans les travaux des biologistes de l'époque, il semblerait plutôt que cela soit avant tout ce nouvel éclairage physiciste apporté à la biologie {REF}, et l'espoir déguisé (finalement non réalisé) de trouver de nouvelles lois physique à l'oeuvre dans la construction du vivant associé à la grande diffusion du petit livre dans le grand public qui amèna peut être de nombreux physiciens à ne plus ignorer les avancés dans ce domaine, notamment durant les années 1940 / 50, tel que Prigogine \autocite[77]{Prigogine1996}, Von Foerster, etc. \autocite[73]{Lemoigne1977}

Mais conscient des manquements et des reproches faites à son approche, alors incomplète, car focalisé sur la cinétique, celle ci n'est pas relié à une théorie plus explicatives sur les mécanismes energétiques à l'oeuvre justifiant l'existence de ces propriétés des systèmes vivants dans le cadre des systèmes ouverts. C'est les récents développements sur la \enquote{Thermodynamique des processus irréversibles} qui va introduire a posteriori la possibilité d'une thermodynamique des systèmes ouverts compatible avec l'approche de Bertlanffy. Des physiciens ayant participé à ces travaux sur la thermodynamique des systèmes ouverts loin de l'équilibre (Osanger, etc.) c'est les travaux de Prigogine  en 1946 \autocite{Prigogine1946} qui vont le plus attirer l'attention de Bertalanffy. Lorsque celui ci découvre vers 1948 ces récentes avancées qui semble faire parfaitement écho à ces travaux ( Prigogine n'hésitant pas à citer Bertalanffy comme un de ses modèles d'inspiration \autocite{Prigogine1996}), le rapprochement se fait assez rapidement et Bertalanffy n'hésite pas à promouvoir cette nouvelle thermodynamique comme le parfait support physique justifiant des principes qu'il a établi dans sa propre théorie des système ouvert en équilibre des flux ! \autocite[653-658]{Pouvreau2013}

Pas étonnant donc de voir Bertallanffy s'appuie sur les écrits de Schrödinger pour re-formuler et préciser ses premières intuitions,
+

Malheureusement le \enquote{théorème de Prigogine} de \enquote{minimum de production d'entropie} ne s'exprime que dans des conditions semblent il très drastiques \autocite[53]{Lebon2008} et limité à des systèmes très proche d'un état d'équilibre tel que le prouve les travaux de Denbigh : \enquote{ It is possible that certain reactions in biological systems may be sufficiently close to equilibrium for the rate of entropy production due to them to be very small. But in general it seems that the notion of minimum entropy production has no real significance as applied to chemical reaction in open systems [...] it is incorrect to regard the tendency of an open system to approach a stationary state as being determined by thermodynamic factors. The stationary state may or may not coincide with a state of minimum entropy production, according to whether the rates of the individual processes are linear functions of thermodynamic variables. In the above we have assumed this to be the case for diffusion (eqn. (ll)), but it is known not to be true for chemical reaction.} \autocite{Denbigh1952}

Hors l'état des systèmes biologiques est semble t il loin d'être proche d'un état d'équilibre thermodynamique.. Bertalanffy qui jusqu'à présent se contentait de relier les résultats à son programme organismique ne cache alors plus sa déception lorsque en 1953 il écrit \enquote{Un minimun de production d'entropie ne caractérise donc pas l'équilibre des flux dans les systèmes ouverts [...]}; autrement dit \enquote{la thermodynamique [...] ne nous dit jamais ce qui peut se passer dans un système, ce qui est permis [...] Et le problème de l'organisation progressive, la tendance néguentropique de l'évolution des organismes simples aux organismes compliqués, reste à présent non résolu.} Bien qu'ils n'abandonne pas l'idée de voir expliquer un jour sa théorie organismique par une théorie thermodynamique adapté, il abandonne en 1953 l'étude de la biophysique des systèmes ouverts et se consacre par la suite uniquement à la construction de sa théorie du système général.

Le fait est qu'il y a réduction d'entropie dans les systèmes en équilibre de flux, et qu'il y a maintient et augmentation du niveau d'organisation, sans que l'on sache pourquoi pour le moment dans le monde du vivant. Si l'analogie et le pont entre tissé entre physique et biologie semble donc encore soumis à questionnement, les travaux de Prigogine sur la thermodynamique des systèmes ouverts va continuer quand a elle à ouvrir bien d'autres perspectives, notamment dans les systèmes sociaux.

%paragraphe dimension reflexive auto-orga ...
Elle dépasser largement ce cadre, et appuie sur des bases physiques le concept d'"auto-organisation", une notion déjà introduite dans le mouvement cybernétique par Ashby, un homme clef dans la convergence des idées entre Cybernétique et GST.

Ashby, tout comme Von Foerster interviennent dans la création de la seconde cybernétique, et introduise une dimension réflexive aux débats.

Inspiré par Von Foerster, vont alors introduire un autre concept \enquote{d'order from noise}, totalement différent du \enquote{order-from-disorder} de Schrodinger.

TODO : Partie plus axé sur les changements de causalité ? (vient avant ou apres ici ?)

L'équifinalité

Un autre concept important est introduit par Ashby dans le mouvement Cybernétique, le concept d'auto-organisation, l'introduction du mot \enquote{auto} amorcant ainsi un virage réflexif qui annonce la seconde Cybernétique, piloté par Von Foerster.


%Des auteurs comme Prigogine en 1947 >> clairement inspiré par bertalanffy/ Schrodinger...  cf Pouvreau et internet
%Il fait le lien avec processus physique =>
%http://www.informationphilosopher.com/solutions/scientists/prigogine/
%http://www.informationphilosopher.com/solutions/scientists/schrodinger/

%http://en.wikipedia.org/wiki/Entropy_%28information_theory%29#Relationship_to_thermodynamic_entropy

C'est également à cette époque, que relayant les premiers travaux de Prigogine sur les systèmes dissipatifs, Bertalanffy va catalyser ainsi ces idées dans sa GST.

Ce procédé sera transféré au réel par Ashby, un autre cybernéticien qui travaillera dès 1946 à la mise au point d'une machine expérimentale capable de reproduire de façon mécanique cette dynamique de stabilisation face aux variations de son environnements. Nommé \enquote{homéostat} celle çi sera construite en 1948, et présenté aux conférences de Macy en 1952.

WIkipedia => L'implication de la cybernétique dans la systémique est historiquement plus liée au « deuxième mouvement cybernétique ». En effet, si selon Norbert Wiener la cybernétique étudie exclusivement les échanges d'information (car c'est « ce qui dirige » les logiques des éléments communicants d'où le mot cybernétique), dans son évolution qui engendrera la systémique, on réintègre les caractéristiques des composantes du système, et on reconsidère les échanges d'énergie et de matière indépendamment des échanges d'information.

La dégradation de l'énergie nécessaire pour maintenir une organisation implique l'irréversibilité des transformations.


The history of an open system is part of its structure, and Prigogine links open systems to irreversibility. Prigogine calls open systems dissipative. Put more simply, this means that matter does not tend to organise itself in a particular location unless there is some external energy source powering it. Evolution can be seen as matter organising itself.


The term \enquote{self-organizing} was introduced to contemporary science in 1947 by the psychiatrist and engineer W. Ross Ashby.[9] It was taken up by the cyberneticians Heinz von Foerster, Gordon Pask, Stafford Beer and Norbert Wiener himself in the second edition of his \enquote{Cybernetics: or Control and Communication in the Animal and the Machine} (MIT Press 1961).

Self-organization as a word and concept was used by those associated with general systems theory in the 1960s, but did not become commonplace in the scientific literature until its adoption by physicists and researchers in the field of complex systems in the 1970s and 1980s.[10] After Ilya Prigogine's 1977 Nobel Prize, the thermodynamic concept of self-organization received some attention of the public, and scientific researchers started to migrate from the cybernetic view to the thermodynamic view. WIKIPEDIA


Malgré les critiques soulevés de part et d'autres, du faite entre autre d'un objectif peut être un peu sur-évalué voire immodeste, celle ci aura un large écho auprès des sciences humaines, et notamment en géographie; d'abord anglo-saxonne \autocite{Haggett1965, Chorley1962}, puis par diffusion en France \autocite{Raymond}.



L'avénement de la deuxième cybernétique :
La régulation apparaît en effet comme un phénomène majeur chez les organismes vivants, puisqu’elle « retarde la dégradation de l’énergie et donc l’augmentation de l’entropie » (p 129), et associée au retard d’entropie et à la computation, elles forment l’essence même de la cybernétique

\paragraph{La mise en avant de concepts à l'héritage complexe}
\label{p:heritage_complexe}

\hl{T : Complexité , concept clef d'auto organisation } 

Une première influence est d'abord à chercher dans l'émergence de ce que l'on appelle aujourd'hui \enquote{Cybernétique de Second Ordre}; et dont on trouve les premières traces à la charnière des années 40-50, avec l'introduction par l'influent McCulloch du physicien Viennois Von Foerster comme orateur en 1949 puis secrétaire jusqu'en 1953 des importantes conférences inter-disciplinaire de Macy. 

L'homme qui nous intéresse ici, McCulloch, est donc d'autant plus influent par ses travaux qu'il figure également comme participant et organisateur dès les toutes premières et importantes conférences de Macy (1942). Si on peut encore discuter sur la part d'influence qu'il convient d'attribuer à McCulloch ou à Wiener sur la structuration des idées dans le groupe Macy, il n'y a aucun doute sur l'importance des travaux menés par ce dernier avec Pitts et Von Neumann \enquote{ sur la logique mathématique comme instrument d'une théorie unifié liant fonctionnement du cerveau et des ordinateurs }. Malgré le biais mécaniciste réductionniste \textcite[783-784]{Pouvreau2013} induit par le discours de ces derniers autour de leur modèle de réseau de neurone formels, \enquote{ ce fut surtout parce qu’il contribuait à l’extension du domaine de la science \enquote{ exacte } à la neurophysiologie, parce qu’il permettait dans un même mouvement de connecter celle-ci à la théorie des automates, et parce qu’il nourrissait le consensus autour de l’idée que la pensée a pour structure physique sous-jacente des réseaux de neurones biologiques analogues aux réseaux constitutifs des automates de calcul} que \autocite[777]{Pouvreau2013} considère les travaux de McCulloch comme un des quatre moments clef dans la construction de la cybernétique. Si le ton des conférences de Macy porte une vision réductionniste \Anote{reductionisme_pouvreau_macy}, le point de vue de Von Neumann et McCulloch se différencie toutefois des position plus modéré de Wiener ou Rosenblueth. Personnage complexe, on pourra se rapporter aux écrits de \textcite{Dupuy2005}, et \textcite{Levy1985} afin de mieux comprendre et replacer l'énorme héritage laissé par McCulloch, notamment par rapport à l'intelligence artificielle, dont il est un éminent précurseur. Car selon \textcite{Dupuy2005} bien que celui-ci se range plus souvent dans sa carrière du coté des biologistes que des ingénieurs, ce fut paradoxalement par les psychologues et les embryologistes qu'il fut plus particulièrement rejeté. Un point de vue partagé par \textcite[778]{Pouvreau2013}, sa théorie ayant eu une bien plus grande influence dans le domaine des automates que dans le domaine biologique \Anote{influence_turing}. 

Influent McCulloch l'est également par le vaste réseau de relation international qu'il est amené à mobiliser dès lors qu'il découvre des travaux originaux \autocites{Dupuy2005, Husbands2012, Levy1985}. Ainsi tout comme le soutient important qu'il a pu apporté aux travaux du jeune Von Foerster, c'est également McCulloch qui recrute a plusieurs reprises des \enquote{cybernéticiens avant l'heure} membres du \textit{Ratio Club} anglais \Anote{mcculloch_ratioClub}, dont le psychiatre et ingénieur anglais Ashby - une figure clef par la suite dans l'évolution du projet systémique - pour participer aux 9ème conférences de Macy en 1952. Une inflexion scientifique qu'il maintient également dans le projet du BCL de Von Foerster, où il place Günther en 1967 comme scientifique titulaire , et que l'on peut entrevoir lorsque Ashby est lui aussi titularisé par Foerster en 1961, il y restera 9 années. (\hl{ref cite officiel ashby}

Von Foerster est reconnu comme le chef de file d'une transformation de la pensée cybernétique. La fin des conférences de Macy en 1953, et l'absence de véritable lieu physique inter-disciplinaire pour discuter de cette problématique sous un angle véritablement biologique semble être moteur dans le projet initié par Von Foerster. Soutenu et initié à la biologie par McCulloch et le mexicain Rosenblueth pendant cette période d'entre deux, Von Foerster semble plus intéressé pour poursuivre l'investigation de la \enquote{computation au sens biologique} déjà incarné dans la figure de McCulloch que par les problématiques purement cybernétique \Anote{foerster_interview}, la causalité circulaire dans sa spécificité biologique n'ayant semble t elle été que très peu traité par la cybernétique \Anote{dupuy_causalite}. 

Ce sont probablement ces éléments qui vont pousser Von Foerster a fonder en 1958 le \textit{Biological Computer Laboratory} (BCL) au cœur de l'université de l'Illinois. Un foyer inter-disciplinaire initié et dirigé par ce dernier jusqu'à son départ et la fermeture qui s'ensuit au milieu des années 1970. \autocite{Proulx2003}.

C'est donc dans ce creuset accueillant du BCL où sont invité à défiler un certain nombre de chercheurs, de façon permanente ou temporaire, que vont être amenés à discuter de nombreuses et très différentes problématiques dont la notion aujourd'hui bien connu d'\enquote{auto-organisation}. Nous ne rentrerons pas ici dans les détails d'une généalogie du concept dont \textcite{Stengers1985} \Anote{livret_CREA} a pu montrer qu'elle était en réalité d'un point de vue épistémologique un puzzle de lecture extrêmement complexe, mais nous pouvons d'ores et déjà rappeler quelques éléments saillants, évoquant par le biais des influences de certains acteurs majeurs de cette réflexion le différentiel de points de vue pouvant animer les débats sur cette question.

Les principales discussions du BCL sur la notion sont données à voir par le biais de \textit{proceedings}, résultat de trois conférences voulues par Von Foerster : \autocite{Yovits1960}, \autocite{Yovits1962} et \autocite{Foerster1962} Dans ce cadre, l'intérêt biologique est également amené à croiser l'intérêt informatique. Le BCL côtoie ainsi dans ces conférences les contributions de ce qui est en train de devenir depuis 1956 à Darmouth la toute jeune discipline de l'Intelligence Artificielle. Il n'est pas anodin alors de citer l'influence de McCulloch qui opère depuis 1952 justement dans la division électronique du MIT, et travaille avec les pionniers Minsky (projet SNARC), Papert, etc. Il est ainsi intéressant de voir réuni dans ces conférences sur l'auto-organisation de 1960 tout les précurseurs de ce domaine, réuni autour d'une cause commune, alors même que les tensions entre partisans du \enquote{symbolisme} et \enquote {connexionisme} ne semble pas encore avoir éclaté \Anote{connexionisme_symbolisme}. Sont ainsi présent lors des conférences, Herbert Simon, Allen Newell, John Shaw, Marvin Minsky, John McCarthy ainsi que le pionnier des réseaux neuronaux Frank Rosenblatt, et les cyberneticiens Warren McCulloch, Gordon Pask, et évidemment Von Foerster. \autocites[256]{Asaro2007}{Yovits1960}.

Toutefois, malgré le fait que ces conférences attirent des cybernéticiens brillants, \textcite[87]{Stengers1985} fait état d'un bilan en demi-teinte, ces \textit{proceedings} faisant plus penser à un catalogue de points de vue hétérogènes qu'à une réelle volonté de synthèse. Ainsi à l'instar de Stengers, l'histoire retiendra principalement de ces publications les auteurs des points de vues alors déjà célèbres (homeostat en 1952, loi de la variété requise en 1956) du psychiatre et ingénieur \autocites{Ashby1947, Ashby1962}, et ceux plus contemporains de cette époque du physicien \textcite{Foerster1959} \autocites{Muller2007a}[55-56]{Stengers1985} Si le sens du concept d'auto-organisation semble nous filer entre les doigts tant il est polymorphe, il n'en représente pas moins pour \textcite[106-110]{Livet1985} un mot d'ordre que l'on aurait tort de négliger dans l'analyse des travaux au BCL, car il constitue un drapeau de ralliement qui marque par un horizon de pensée, la spécificité de ces questionnement par rapport à la première cybernétique. 

Selon Umpleby \hl{ref}, pour Von Foerster la première cybernétique est définitivement effacé par la seconde, la seule qui devient acceptable d'un point de vue scientifique. Comment alors le réductionisme fervent de McCulloch se transforme-t-il dans la filiation de questions opérés au travers des positions de Foerster et des projets menés au BCL ? Selon \textcite[120-122]{Livet1985} \enquote{la cybernétique de \enquote{second ordre} du BCL à conservé l'hypothèse de Mac Culloch d'une computation universelle, mais elle a aussi accentué les aspects non-réductionnistes, et tout d'abord le refus du behaviorisme [...]} Ni totalement réductioniste, ni holiste au sens le plus simple, Levy y voit une certaine parenté avec l'\enquote{organicisme} sans toutefois pouvoir l'y rattacher, car si les cybernéticiens semble bien admettre des différences entre l'organique et l'inorganique, l'organique est quand même étudié ici comme \enquote{machine biologique} capable de \enquote{computation}, à la différence des embryologistes organicistes.

La rencontre de Foerster avec le biologiste Maturana (Leiden 1962) et de son disciple Varela (1965) donnera naissance à la notion d'auto-poeise. Pour ces deux scientifiques cette notion n'a rien à voir avec le concept d'auto-organisation tel qu'il est abordé lors de leur passage au BCL, et cela même rétrospectivement lorsque ceux-ci découvre en 76-77 l'autre sens thermodynamique prise par la notion. Si l'article de référence sur l'auto-poeise date de 1974 \autocite{Varela1974}, la notion se cristallise certes dans l'historique des pratiques expérimentales des deux biologistes mais également surtout par la pratique de cet environnement fécond qu'est le BCL. Ainsi c'est au détour d'une publication interne du BCL (1970) qu’apparaît pour la première fois ce terme; une preuve de cette synergie féconde orchestrée par et autour de Von Foerster, le seul en réalité capable de discuter ces idées et d'opérer une synthèse au travers des différentes approches - parfois opposé-  qui traverse son laboratoire. \autocites[283-287]{CREA1985}{Muller2007b, Varela1995} Ainsi par exemple, à la lecture des interviews de \textcite{Varela1995}, Maturana \autocite{Muller2007b}, ou Von Foerster \autocite{Franchi1995} on comprend que les relations déjà complexe de certains membres avec le \textit{MIT AI group} fondé en 1958 par Minsky et McCarthy vont se renforcer avec la disparition des financements supportant le BCL. En désaccord avec la vision du cerveau comme machine de traitement symbolique \hl{ref maturana}, ces derniers expriment également toute leur méfiance envers un certain nombre de mot clef de la cybernétique, et s'associe même pour Maturana à une difficulté de formalisation assumé \autocite[258-263]{CREA1985} dont on trouve trace encore aujourd'hui dans les contours difficile à cerner qu'est la notion d'auto-poeise.

Toutefois, et comme discuté par la suite dans cette section, il existe probablement une piste à explorer entre la direction prise par Von Foerster dans le courant des années 1960 sous l'influence réciproque de Maturana et Varela, et le concept d'auto-organisation tel qu'évoqué pour la constitution du vivant en biologie théorique. Une autre filiation pour la notion d'auto-organisation est exploré par Stengers, dans son sens physico-chimique le terme n’apparaît en tant que tel chez Prigogine que tardivement \hl{en 1969}. Or pour \textcite[64]{Stengers1985}, une explication pour justifier l'apparition spontané de ce terme dans les textes de Prigogine tient de sa familiarité originelle avec la biologie, où le terme est utilisé depuis longtemps, notamment en embryologie.

Or on sait que sur les réflexions théoriques sur les systèmes ouverts éloignés de l'équilibre extrait des travaux de Von Bertalanffy sont entrés très tôt en résonance étroite \autocite[653-661]{Pouvreau2013} avec les réflexions de Prigogine \autocite{Prigogine1996}, ce dernier ne cachant pas son inspiration pour la biologie comme tendent à le montrer plusieurs de ses collaborations et publications \autocites[59-67]{Stengers1985}{Prigogine1946}. 

Mais on ne peut aller plus loin sans évoquer la part d'héritage que doivent ces réflexions aux travaux antérieurs de Von Bertalanffy. La construction de sa théorie organiciste  entamé dans les années 1930 fait de lui un des acteurs incontournables dans l'établissement d'une biologie théorique.

D'un tout autre coté, dans l'interview donné pour le \textcite[255]{CREA1985}, Von Foerster indique bien ne pas avoir pensé lorsqu'il étant au BCL à appliquer les mathématiques des systèmes non linéaires à la problématique de l'auto-organisation, mathématiques dont il connaît pourtant l'existence de par sa formation. 

% Voir page Dupuy : https://books.google.fr/books?id=bwlm7kVy5WoC&pg=PP53&lpg=PP53&dq=mcculloch+foerster&source=bl&ots=lD2chp1gL5&sig=QRk4AgrqRe7jmCgI7_ERrqVdyPo&hl=fr&sa=X&ei=jyqPVPr5Bo3SaKbAgagD&ved=0CGwQ6AEwCg#v=onepage&q=mcculloch%20foerster&f=false

Des observations qui tendent à avaliser cette hypothèse forte donnée par Stengers, pour qui cette branche de réflexion double abordant la notion sous l'angle biologique et thermodynamique évolue dans une relative indépendance par rapport à la réflexion menée au BCL. \textit{Pourquoi relative ?} Car il faut prendre en compte l'existence tout à fait plausible d'une forme de recoupement entre ces deux voies de réflexions. Mais avant d'aborder la possibilité d'une telle piste, il faut donner un aperçu de la spécificité de cette seconde réflexion sur l'auto-organisation.

La question de l'auto-organisation s'inscrit en biologie dans une tradition beaucoup plus ancienne que celle évoqué dans la tradition cybernétique ou physico-chimique. On pourra ainsi retrouver dans les débats des biologistes de multiples références à la philosophie, comme par exemple celle de Kant, qui critiquait déjà en 1789 l'hypothèse mécaniste pour justifier de la vie et \enquote{considérait déjà l'auto-organisation comme principe distinctif du vivant} \autocites[76]{Pouvreau2013}[275]{Mossio2010}[6]{Mossio2014}. Ce concept d'auto-organisation \autocite[68]{Stengers1985} est rediscuté à la lumière des débats opérant à la charnière des années 1920-1930, dans l'émergence d'un courant de biologie théorique dont la volonté nomothétique se fait l'écho conjoncturel d'une discipline biologique en crise \autocites[421-434]{Pouvreau2013}. C'est appuyé par la pensée pionnière de quelques scientifiques opérant principalement dans le monde germanique (Allemagne, Autriche) \autocite{Drack2007b}, en Grande-Bretagne et aux Etats-Unis que va se constituer un mouvement de chercheurs porteurs de perspectives holistiques capable de caractériser et de prendre le contre-pied des dérives et des débats jusque là stériles (vitaliste/mécaniciste, darwinisme/lamarckisme, etc.) qui décrédibilisent la biologie à cette période \autocite[153-154]{Pouvreau2013}. Un héritage qui va influencer les travaux de Von Bertalanffy tant sur les aspects philosophiques, que mathématiques, une discipline dont il va questionner sa relation avec la biologie \autocite{Pouvreau2005} jusqu'en 1932, date à laquelle il finit par accepter sa nécessité dans l'établissement de son projet de systèmologie générale. Les travaux biomathématiques de cette période sont alors assimilés de façon tout à fait sélective et congruente à son programme organismique, comme tâche de le montrer \textcite[515]{Pouvreau2013} dans sa thèse. Il est intéressant de garder en mémoire pour la suite de notre exploration qu'une partie de cette prise de contact avec les biomathématiques se soit faite par les préoccupations communes, l'amitié et le travail de Woodger avec Bertalanffy \autocite[347,433]{Pouvreau2013}, un embryologiste et philosophe anglais qu'il a rencontré en 1926, et avec qui il correspond de façon intense entre 1930 et 1932 \autocite[165]{Pouvreau2013}. 

Parmi les différents foyers intégrant ce courant holistique, on s'intéressera donc plus à celui représenté par les membres du \textit{Theoretical Biological Club} (TCL) (connu aussi sous le nom de \textit{Biotheoretical gathering}) opérant de 1932 à 1938, et continuant ensuite après guerre jusqu'en 1952. Co-fondateur de ce club inter-disciplinaire, le biologiste philosophe Joseph Henri Woodger a constitué et défendu pour l'époque des anti-thèses importantes dans la constitution d'une biologie théorique, une importance qui après de nombreuses critiques lui est aujourd'hui justement restituée \autocite{Nicholson2013}. Le club regroupe initialement les biochimistes Dorothy et Joseph Needham, la mathématicienne et philosophe Dorothy Wrinch, le physicien cristallographe John Desmond Bernal, l'embryologiste fondateur de l'epigénétique Conrad Hal Waddington; des scientifiques dont l'originalité et la portée des réflexions va rapidement attirer d'autres personnalités, comme Karl Popper, Alfred Tarski et bien d'autres \autocite[14-43]{Niemann2014}. Si le club est effectivement amené à couvrir un large panel de sujets, celui-ci vise collectivement le \foreignquote{english}{[...] development of a ‘mathematico-physico-chemical morphology’ that would enable an interdisciplinary engagement with the problem of biological organization at the supracellular, cellular, and subcellular levels.} \autocite [277]{Nicholson2013}. Toutefois, si l'influence de ce courant anglo-saxon dans le projet de Bertalanffy est notable, celle-ci ne constitue pas la voie unique de ses influences, et sa vision des choses puise dans l'ensemble du champs des biomathématiques de l'époque (Rachevsky, Lotka, etc.) \autocite[574-585]{Pouvreau2013}.

Autre moment important de la biologie théorique, représentative des points de vue hétérodoxe de la biologie moléculaire, est celle de l'embryologiste et généticien Waddington. Dans une forme de continuité de réflexion par rapport au TCL celui-ci organise au début des années 1970 un ensemble de symposium intitulé \enquote{Towards a Theoretical Biology} tous les ans de 1966 à 1969 en Italie à Bellagio. Pour \textcite[512-513]{Nanjundiah2010}, les problématiques soulevés dans les deux premières conférences tel que résumé par \textcite{Waddington1968} sont triples : \foreignquote{english}{There was the high level of complexity of biological systems in terms of both the number of variables that had to be taken into account for describing them and the number of interactions among those variables. Next, the prevailing gene-centred view failed to take into account the fact that genes were as much responders as actors. Third, evolution had to be integrated into any theory of development. One needed to understand organisms and their development by including the workings of genes and the environment in one conceptual whole.}

%Ainsi la notion d'attracteur réaparait sous des formes différentes, tout autant dans la notion d'équifinalité repris et développé par bertalanffy et dont on trouve écho dans les premier travaux de Prigogine (voir Annexe), que dans la métaphore de paysage génétique de Waddington dont la traduction en système dynamique démarre avec Réné Thom (participant des premieres conférences), et se poursuit encore aujourd'hui au travers de nombreux projets.

% Notion d'auto organisation, on la retrouve par exemple dans le cadre de l'embryogenese.

Les conceptions épigénétiques de la morphogenèse de Bertalanffy vues au travers de son second principe organismique \Anote{Pouvreau_secondprincipe}, couplées à celles développées par d'autres embryologistes comme Weiss, Woodger, Waddington - dont on doit entre autre l'origine du mot épigénétique -  forme un cadre de réflexion historique où la trajectoire de la notion d'auto-organisation, bien que partageant dès le départ certaines similarités avec l'angle de vue physico-chimique \autocite{Prigogine1946}, sont d'emblée amenées à être dépassé.

Il ne s'agit pas de nier ici l'importance de ces dernières, car elles fourniront le matériel conceptuel et mathématique nécessaire à l'engagement d'une toute nouvelle réflexion dans d'innombrables disciplines, notamment en géographie (voir \ref{sssec:progressive_systemique}). Il s'agit plutôt ici de traduire leur insuffisance à fournir à elles seules une explication universelle en biologie. Chez Von Bertalanffy, c'est finalement dans l’acceptation (voir Annexe 1 et \autocite[657-661]{Pouvreau2013}) de cette faiblesse dans la partie thermodynamique de sa théorie organismique que se révèle toute la richesse d'une théorie dont les problématiques sous-jacentes à l'articulation des concepts dépassent le seul questionnement de son opérationalisation physico-chimique. 

%Ainsi il parait impossible de négliger l'émergence dans les débats sur l'embryogénèse des années 1930 d'un point de vue intégrant tout autant l'importance d'un présuposé matériel génétique, que son interaction avec l'environnement (phenotype). \hl{Ref}

Une acceptation largement partagé par la communauté des biologistes, d'autant plus lorsqu'elle est appuyé par les dires d'un des collaborateurs les plus proche de Prigogine. Le physicien et biologiste Jean-Louis Deneubourg affirme ainsi avec ses collègues dans le livre \textit{Self-Organization in Biological Systems} \foreignquote{english}{The mechanisms of self-organization in biological systems differ from those in physical systems in two basic ways. The first is the greater complexity of the subunits in biological systems. [...] The second difference concerns the nature of the rules governing interactions among system components. In chemical and physical systems, pattern is created through interactions based solely on physical laws. [...] Of course, biological systems obey the laws of physics, but in addition to these laws the physiological and behavioral interactions among the living components are influenced by the genetically controlled properties of the components. In particular, the subunits in biological systems acquire information about the local properties of the system, and behave according to particular genetic programs that have been subjected to natural selection.} \autocite[12-13]{Camazine2003}

C'est également le point de vue capturé par \textcite{Mossio2014}. Celui-ci  appelle la notion supplémentaire de \enquote{clôture organisationnelle} développé par Piaget pour faire une lecture originale des spécificités du vivant. Biologiste de formation, celui-ci s'appuie de façon précoce sur les idées de Waddington, comme on peut le constater dans le chapitre d'ouverture de \textit{Biologie de la connaissance} (1967). Piaget ayant également eu des interactions fortes avec Von Bertalanffy dès 1953 \autocite[310-311]{Pouvreau2013}, notamment dans le cadre plus général du transfert fructueux de sa théorie organismique à la psychologie \autocite[945-951]{Pouvreau2013}, il n'est pas étonnnant de voir que Mossio inscrit ce concept comme complémentaire de l'ouverture thermodynamique de Von Bertalanffy.

Tel qu'utilisé par \textcite{Mossio2014} ce concept \Anote{piaget_mossio} fonde un support conceptuel spécifique au vivant sur lequel peuvent se greffer les contributions de Maturana et Varela, ou encore celle de Pattee et Rosen dont les réflexions s'inspirent en partie des travaux de Waddington.

Si on ne peut qu'être d'accord avec la lecture de Mossio établissant l'insuffisance du concept d'auto-organisation au sens thermodynamique des structures dissipatives, la frontière entre les deux notions est beaucoup plus flou dès lors qu'on envisage les discussions des biologistes organicistes autour du sens Kantien initial. Si les contributions de Maturana et Varela sont effectivement lisible par le biais du concept théorique de cloture hérité dont la construction doit beaucoup à Waddington et au courant embryologiste, on peut effectivement se poser la question de l'existence d'une boucle reliant la notion d'auto-organisation telle que décrite par les biologistes organicistes et l'émergence courant 1960 d'une réflexion similaire en \enquote{apparente} contradiction avec l'auto-organisation au sens du BCL, puis au sens thermodynamique.

En ce qui concerne l'influence de Bertalanffy sur les discussions de la notion au BCL, on peut considérer que malgré l'absence de communication lors de la conférence sur l'auto-organisation en 1960, sa présence suffit en quelque sorte à établir l'importance de son point de vue sur cette notion. 

Une autre influence de ce dernier, plus indirecte, passe par la présence d'Ashby au BCL. En effet pour \autocite[791]{Pouvreau2013} le cybernéticien Ashby est un homme singulier non seulement par la nature précoce de ses questionnements (1940) et des réalisations mise en œuvre (1948) pour étudier les comportements adaptatifs, mais également par les échanges et la médiation que ces travaux ont permis d'enclencher entre le point de vue cybernétique et l'évolution du projet systémique tel qu'entamé par Bertalanffy depuis sa théorie organismique. Alors même que les premiers contacts de celui-ci avec les écrits de Bertalanffy date au moins de 1952, \textcite[793]{Pouvreau2013} tend à montrer que malgré des désaccords de façade, il existe dans la comparaison de leur travaux d'étonnantes accointances. Sachant cela, l'\enquote{impossibilité d'une auto-organisation} telle qu'évoquée par \textcite{Ashby1962} \Anote{ordre_desordre} dans le cadre des conférences du BCL est alors d'autant plus évocatrice de l'influence implicite des travaux de Von Bertalanffy que cette réflexion d'Ashby va être considéré par Foerster au sein du BCL. A ce constat, il ne faut pas oublier d'ajouter que pendant une large partie de sa présence au BCL Ashby est également président (1962-1965) de la \textit{Society for General Systems Research} (SGSR) entre autres fondée par Von Bertalanffy ! \autocite[826]{Pouvreau2013}.

Sachant l'accrochage dès 1948 de Weiss et Culloch à Hixon, la proximité de Maturana avec McCulloch, puis Foerster, la présence de von Bertalanffy à la conférence de 1960, et la présence que l'on suppose marquante d'Ashby au BCL entre 1961 et 1970, il semble légitime de questionner quels transferts peuvent être établis entre les principes au cœur de la théorie \enquote{organismique} représenté ici par Von Bertalanffy et la formalisation à posteriori du concept d'auto-poeïse. Or en dehors des influences fortes et réciproques constatés entre Von Foerster et Maturana \autocites{Muller2007b}[255-273]{CREA1985}, ce dernier reste relativement discret sur les références qui ont pu guider sa réflexion en tant que biologiste, un fait largement reconnu par ailleurs \autocite[161]{Pangaro2007}. \Anote{etude_pouvreau_mossio} \Anote{piquant_weiss}

\paragraph{L'inscription de la Vie Artificielle dans les problématiques biologiques}
\label{p:va_bio}

\hl{debut zone en travaux sur PATEE}

Parcours Pattee : 

parmi les participant des quatres conférences, le biophysicien Howard H. Pattee va développer durant ces quatre années des réflexions qui sont encore aujourd'hui entretenu et discuté tout autant par les philosophes biologistes que les informaticiens développant des programmes de Vie Artificielle.

C'est dans ce cadre qu'apparait un lien de filiation faisant écho avec les questionnements ultérieurs posés par la Vie Artificielle, la question de la détermination d'un organisme vivant par la seule reproduction, replication de code génétique en dehors de tout modèle physique ou chimique comme cela a été le cas dans un certain nombre de modèle de simulation n'abordant en réalité que la moitié du problème, laissant en partie de coté les problématiques d'autodétermination, et de l'évolution caractéristique du vivant \autocite{Mossio}.

On retrouve problématique de l'émergence en toile de fond, l'auto organisation dans le cadre biologique supposant l'émergence de structure de plus haut degré de complexité, ce constitue pour le moment un horizon indépassable dans le cadre de nos simulation.

%du francais Réné Thom suffit à qualifier l'ouverture des discussions qui y sont engagés.

Le parcours fortement inter-disciplinaire de Pattee l'amène tout au long de sa carrière à développer des idées pertinents dans le champs de plusieurs disciplines scientifiques \textcite{Umerez2001}. Son principal biographe  \textcite{Umerez2009} révèle ainsi comment la toute nouvelle \enquote{biosémiotique} trouve écho à ces problématiques dans les travaux de Pattee alors que lui-même avoue dans une réponse à Umerez \autocite{Pattee2009} qu'il n'avait jusqu'alors que peu de connaissance de ce domaine, de cette trajectoire historique qui l'a sous-tend, et duquel maintenant il fait partie. 

Pour la VA, mais aussi pour les écologistes, il est plus connu comme étant l'initiateur avec son disciple informaticien et biophysicien Conrad d'une première expérience de simulation d'un d'écosystème \Anote{conrad_explanation}. Nommé EVOLVE, ce programme \Anote{conrad_model} voit sa première version daté de 1970 \autocites{Conrad1970, Pattee2002}. Il apparait également comme un penseur critique indispensable dans ce puzzle disciplinaire réunis en 1987 par Langton, en posant déjà un certains nombres de questions essentielles qu'il tire d'une réflexion qu'il a lui même démarré comme plusieurs de ces collègues physiciens dans le courant des années 1960. \hl{Impact Schrodinger Pattee, Prigogine} \autocite{Pattee1988}

\Anote{patte_deception}

Les résultats des premières expériences l'amènent à évoquer sous un jour à peine déguisé, des problématiques classiques se rapportant à la validation, évoquant au travers du substrat support de la simulation la question délicate du rapport entre univers simulé et réalité. 

Une question d'autant plus délicate qu'un courant baptisé de \textit{strong life} s'attend à voir émerger la vie au travers de créatures virtuelle. A la différence peut être d'autre discipline, la question du substrat support des simulations est ici d'autant plus problématique que le vivant semble en partie s'auto-définir dans et par sa nature de substrat spécifique. Si Pattee est un scientifique tout à fait partisant de l'usage de la simulation, il n'est pas dupe du biais qui sous tend les capacités de représentation universelles de l'ordinateur. La manipulation spécifique d'opérateur \enquote{symbolique} devant se conformer à des critères de plausibilité qui repose sur la théorie. 


Les formes décevantes de comportements chaotique observés dans ses premières simulations avec Conrad l'ont amenés à penser qu'il était nécessaire de pousser non pas tant le réalisme que la cohérence de l'univers physique simulé, condition siné qua none pour dépasser cette limitation

l'expérimentation d'une coupure épistémique (\textit{epistemic cut}) L'évolution se construisant dans la relation entre l'organisme et l'environnement, la fidélité d'implémentation des processus à l'oeuvre dans la construction du génotype, du phénotype 

Il semble partisant d'une forme de réalisme permettant d'opérer ce qu'il nomme par la suite \enquote{epistemic cut}

et ne permet en l'état de remplir ce qui ne représente pour Pattee qu'une toute petite partie du contrat dans l'exploration du passage de l'inanimé à l'animé, de la non-vie à la vie. Il n'est alors pas difficile d'imaginer à quel point les conclusions de Pattee ont du passé pour décéptive quand on considère le contexte fédérateur dans lesquelles elles ont été énnoncés.

Une expérience qui en fait avec Von Neumann un des pionniers de l'open-ended evolution tel que définit ainsi par \autocites{Taylor1999,Taylor2012}  

\foreignquote{english}{Taylor One of the major achievements of von Neumann’s work was to clarify the logical relation-ship between description (the instruction tape, or genotype), and construction (the execution of the instructions to eventually build a new individual, or phenotype) in self-replicating systems. However, as already mentioned and as emphasised recently by McMullin (1992), his work was always within the context of self-replicating systems which would also possess great evolutionary potential.}


On retrouve par exemple dans les travaux de Tim Taylor la volonté d'analyser et de construire des simulateurs capable de répondre aux  problématiques posés en amont par Pattee et Waddington \autocite{Taylor1999}. 

Sont ainsi fait référence à la notion de processus d'évolution créatif rendu possible par la mise en oeuvre d'un environnement ouvert, et pour lequel la notion de cloture sémantique est importante ... (nul)

McMullin 2000

 font Bien que découvrant l'émergence de cette nouvelle discipline qu'est la biosémiotique, Pattee par les fondateurs de cette nouvelle discipline, son intérét comme physicien sur le problème de la vie rejoint celui d'illustre physicien comme Bohr's ou Schrodinger. 

T: Pattee Robot, Brooks cognitif versus autre théorie...


%Le point commun de ces reflexion est leur opposition à la vision de Schrodinger, l'ordre par l'ordre, le système s'organise en dévorant l'ordre de son environnement. En effet, la cybernétique de second ordre c'est l'inclusion de l'observateur dans le système, traduit ici par la capacité de l'organisme à savoir ce qu'est pour lui l'organisation. 

%Quant à l'auto-organisation telle qu'elle est investi par la suite dans l'auto-poeise, les récentes relectures sur les travaux de Bertalanffy soulève à mon sens aux moins deux questions.

Ces hypothèses qui font plus état de lecture de seconde main que d'un véritable travail historiographique nécessitant une immersion poussé pour la compréhension des concepts, le lecteur pourra sur ce point se référer aux publications passionantes de Pouvreau, Drack, et Mossio \autocites{Pouvreau2006, Pouvreau2013, Drack2015} mais également au livret édité en 1985 par le CREA, déjà plusieurs fois cité. (voir également l'annexe \ref{ssubsec:cybernetic})


\hl{fin zone en travaux}

% A retravailler avec les remarques de Pouvreau...
% Retour de la biologie systémiste Braillard2008
%On y retrouve également l'influence de concept propre au paradigme systémique partant de la seconde cybernétique, dont on peut ancré tout ou partie des concepts initiaux dans l'étude du vivant tel que ceux mené dans les années 1950 par Von Bertalanffy (théorie organismique \autocite{Pouvreau2013}) ayant inspiré par la suite les travaux de Varela (auto-poeise \autocite{Varela1979?}) \Anote{varela_modele_ca}, que l'influence des multiples travaux informatiques mimant les processus évolutif décrit par la théorie darwiniste.


!! Comme semble nous le dire Pattee, l'automate cellulaire prend racine aussi dans les questionnement sur la vie opéré par Von Neummann. !!


\printbibliography[heading=subbibliography]

\chapter{Le double foyer d'apparition des SMA en SHS}


\paragraph{Les principaux initiateurs de la simulation Agent pour les SHS en Europe}
\label{p:communautes_europe}

%http://books.google.fr/books?id=2YJTAQAAQBAJ&pg=PT326&lpg=PT326&dq=james+doran+1982+archaeology&source=bl&ots=04tyzJ0HoM&sig=T_OpaK1gtQVjlJv-R4qPG0GHUmk&hl=fr&sa=X&ei=aNARVOaVOMSWauXwgeAO&ved=0CCwQ6AEwAQ#v=onepage&q&f=false


% SMALLTALK premier SIMPOP, deuxième grand moments pour les sciences urbains (Sanders2013); trouve une réponse encore plus adapté au concept

En Europe, l'ingénieur et sociologue Nigel Gilbert fait partie de ces personnalités qui ont oeuvré très largement pour la diffusion et la vulgarisation de la modélisation multi-agent (\textit{Agent Based Model}) en sociologie, mais également en sciences sociales dans la communauté internationale \Anote{gilbert_date_clef}.

En 1985, il participe et édite le recueil de papier tiré de la conférence \foreignquote{english}{Social Actions and Artificial Intelligence} qui s'est tenu à Surrey en 1894 \autocite{Gilbert1985}. De cette confrontation de points de vue entre chercheurs en intelligence artificielle et sociologue, on retiendra particulièrement l'article \foreignquote{english}{The computational approach to knowledge, communication and structure in multi-actor systems} de James Doran \autocite{Doran1985}, un informaticien de l'université ESSEX formé par Donald Mitchie, déjà très actif dans la communauté des archéologues durant les années 1970 (voir la section \ref{ssec:engouement_sciencesociale}). Suite à cette rencontre (\hl{Lien vers correspondance privé}) s'établira une collaboration sur le long terme entre Doran et Gilbert; une façon ici de rapeller que ce dernier s'est par la suite largement appuyé pour ses développement théoriques sur l'émergence du projet EOS (\foreignquote{english}{Emergence of Organised Society}) dirigé James Doran et Mike Palmer, un autre informaticien spécialisé en archéologie \autocite{Doran1994a, Gilbert1995a}. Car si Nigel Gilbert se dit impliqué dans ce projet depuis sa création, il avoue lui même ne pas être le principal réalisateur du projet \Anote{gilbert_EOS}. \autocite[122-131]{Gilbert1995a}

%Dans son article \foreignquote{english}{Emergence in Social Simulation} \textcite{Gilbert1995} s'appuie sur le peu de questionnements réels dans la littérature reliant DAI et Sociologie \Anote{note_bond_liens}.

La première publication évoquant de façon implicite le futur projet \foreignquote{english}{EOS} date de 1982 \autocite{Doran1982}, et paraît dans l'ouvrage collectif publié par \textcite{Renfrew1982}.

Comme on va pouvoir également le constater dans la partie suivante pour les géographes (section \ref{sssec:progressive_systemique}), les archéologues sont déjà depuis les années 1970 sensibilisé aux possibilités de formalisation offertes par la systémique (section \ref{ssec:engouement_sciencesociale}). Les années 1980 concède l'accès à de nouveaux concepts pour penser et explorer la complexité, au travers d'une mise en application de la dynamique des systèmes commencé avec Forrester, et étendue depuis aux regards de nouvelles découvertes et redécouvertes sur les mathématiques relative au concept de bifurcations, d'auto-organisation. La publication côte à côte de \textcite{Doran1982} et \textcite{Allen1982} dans l'ouvrage déjà cité de \textcite{Renfrew1982} introduisant ces concepts aux archéologues montre que cette petite communauté d'archéologue modélisateur ne se contente pas d'explorer la seule voie mathématique de la dynamique des systèmes pour construire des modèles dynamiques, mais abordent également les prémisses prometteuses \Anote{renfrew_futur_archeology} offertes par un futur usage des DAI, comme en témoigne certains passages de \textcite{Doran1982} \Anote{doran_82_DAI} et \textcite{Doran1986b} \Anote{doran_86_DAI}.

Ainsi, presque douze ans après sa publication de 1970 \autocite{Doran1970}, déjà visionnaire par les descriptions de simulations qui y sont imaginés \Anote{description_imagine_simulation}, Doran se retrouve une deuxième fois avec ses collègues en position de pionnier avec la mise en œuvre des toutes dernières techniques de l'intelligence artificielle distribué pour l'archéologie \Anote{doran1982_reclamation}, mais également en sociologie \autocite{Doran1985}. Une greffe dont le succès repose là aussi probablement sur un existant riche d'une histoire en simulation dont on a déjà donné quelques éléments de réussite dans la section (\ref{ssec:engouement_sciencesociale}).

%\hl{Reintroduire rapidement la référence à la simulation en sociologie, et le rapprochement initial qui peut être fait avec l'héritage systémique opéré en sociologie, voir \ref{sssec:progressive_systemique}}

% Comme le dit Sanders2013 il est fort probable que comme en géographie, les outils ne fassent que rejoindre des concepts déjà bien intégrés.

% L'objectif affiché ici par Doran et son équipe est très clair, il s'agit de tester si les théories développés en inteligence artificielle distribué peuvent être transferable à un modèle archéologique au préalable déjà formalisé par Paul Mellars en 1985 {Mellars1985}.

% Si l'on se tient aux définitions donnés par Jacques Ferber quand à la nature des agents, soit «cognitifs», soit «réactifs» il semblerait que se découpe déjé une délimitation nette dans les modèles apparaissant dans ce premier et ce deuxième ouvrage. Nigel Gilbert et James Doran utilise par exemple des agents cognitifs pour leur plateforme EOS, alors que MANTA est un modèle qui tente de reproduires le fonctionnement d'une fourmillière en utilisant des agents réactifs.


\hl{T : ?}

%%%%%%%%%%%%%%%%%%%%%%%%%%%%%%%%%%%%%%%%%%%%%%%%%%%%%%%%%%%%
%% INSERTION DAI
%%%%%%%%%%%%%%%%%%%%%%%%%%%%%%%%%%%%%%%%%%%%%%%%%%%%%%%%%%%%

\paragraph{Une inspiration provenant de la branche des DAI}
\label{p:communautes_usa}

Carl Hewitt, figure assez importante dans le paysage de l'informatique et de l'IA distribué, développe avec d'autres et cela dès le début des années 1970, des travaux innovants qui vont inspirer par la suite les futures recherches en DAI (\textit{Distributed Artificial Intelligence}) et sur les systèmes multi-agents \autocite{Ferber1995}.

Dès le départ les initiateurs de l'intelligence artificielle distribué se sont tournés vers l'analyse des phénomènes sociaux existants pour formuler une forme d'intelligence distribué à même de résoudre des problèmes complexes \Anote{hewitt_metaphore_sociale}. 

Les \textit{blackboard system} souffre très vite d'un problème qui ralentit la progression pour le développement des aspect concurrentiels d'une telle approche. La présence d'une ressource partagé, le tableau, qui représente un goulot d'étranglement pour la communication avec les experts KS (\textit{Knownledge Source}) poussent rapidement les chercheurs à envisager une autre forme de parallélisme \autocite{Wooldridge2009}

Pour les experts du domaine comme Wooldridge \Anote{inspiration_wooldridge} et Ferber \Anote{inspiration_ferber} les travaux de Carl Hewitt semble jouer un grand rôle dans l'histoire dans la formation du paradigme multi-agent.

En 1971 Carl Hewitt obtient son doctorat pour son implication dans la construction du système de démonstration de théorèmes \textit{PLANNER}. Ce langage est largement inspiré des méthodes dites de \textit{blackboard system} \Anote{blackboard}, qui s'appuie sur une analogie avec une société d'expert pour l'analyse et la résolution de problème complexe. Mais c'est à la suite de son travail au MIT sur SMALLTALK \Anote{inspiration_double_small} que naît le formalisme \enquote{Acteur} \Anote{acteur_definition_ferber} qui va être repris et opérationnalisé par la suite dans de nombreux autres travaux\Anote{futur_histoire_acteur}. Les chercheurs œuvrant dans le cadre des systèmes multi-agents, une des branches composante des DAI, s’appuieront ensuite largement sur cette frontière très mince entre les notions d'acteurs et d'agents pour appliquer des versions plus ou moins dérivés de ces protocoles d'échanges de messages dans le cadre de plateforme ou \textit{Testbeds}. Pour ces dernières on retiendra les très connus et influent MACE (\textit{Multi-Agent Computing Environment}) développé à l'\textit{university of Southern California} \Anote{mace_systeme}, ou DVMT (\textit{Distributed Vehicle Monitoring Testbed}) développé à l'\textit{University of Massachusett}, intégrant un système formalisé pour l'échange d'information structurés entre entité expertes autonomes.

\hl{Suite de l'histoire ? }

On trouve un historique et une descriptions des influences sur l'IAD beacoup plus complète dans l'article de \textcite{Bond1988} couvrant la période de recherche jusqu'au année 1990, et de façon plus générale dans les ouvrages de \textcite{Wooldridge2009} et \textcite{Ferber1995}.

%http://link.springer.com/chapter/10.1007/978-1-4471-1831-2_13
%MCS multiple agent software testbed which has been developed as a research tool in the University of Essex, Department of Computer Science. 

% LAPPROCHE MULTI AGENT ACTUELLE SE NOURRIT A LA FOIS DE L'IAD ET DE LA VA (voir page 28 de Ferber) Il est intéressant de voir au travers des deux foyers initiaux américains et européen l'influence plus ou moins prononcé de l'une ou de l'autre approche, tout en suivant le meme objectif, l'émergence. Alors que le pole gilbert, conte, doran est plus orienté vers la mise en oeuvre d'agent cognitif traditionel en IAD,  Epstein et Axtell qui s'inspirent avant tout de ce qui est fait au Santa Fe Institute en terme de vie artificielle.

% Evidemment dans les fait, les deux approches cognitiviste et réactive, sont représentés dans les ouvrages, et partage finalement ce socle commun. 

%L'approche KISS a tendance à favoriser l'émergence de modèle agent plutot reactif, les approches cognitivistes mobilisant d'emblée beaucoup plus d'expertise. Le débat de façon générale dans les SMA s'est transmis à la modélisation orienté agent.

% PROFITER APRES CETTE INTRODUCTION POUR INTRODUIRE LE FAIT QUE LA MISE EN OEUVRE (PAR QUI? , COMMENT ? ) DU PROTOCOLE DE CONSTRUCTION JOUE DANS LA VALIDATION, NE SERAIT CE QUE PAR LA PERCEPTION QUI EST FAIT DES OBJETS MANIPULÉ. UNE REVELATION FAITE ÉGALEMENT PAR DROGOUL2003 QUI SOULEVE LA PROBLEMATIQUE DE LA MODELISATION AGENT, entre concept et implémentation. MAIS DONT ON TROUVE ÉGALEMENT EN GEOGRAPHIE LE TEMOIGNAGE DE GLISSE.

% NE PAS OUBLIER LE PASSAGE DE LENA SUR LE FAIT QUE LES CONCEPTS SONT RATTRAPÉS PAR LES OUTILS, c'est IMPORTANT POUR APPUYER LE FAIT QUE LA VALIDATION SOIT UN PROBLEME DE PLUS LONGUE DATE, et NE DISPARAISSENT PAS AUSSI FACILEMENT.

% OUTILS PAR LEUR APPARITION, PERMETTENT DE DEVELOPPER DE NOUVELLES QUESTIONS EN RETOUR, ne serait ce que par exemple en contraignant le discours du modélisateur en donnant à voir le comportement du modèle...

% DONC IL Y A UNE FORME DE PARADOXE, entre d'un coté la préexistence de questionnement, et l'apparition de nouveau questionnement.

%%%%%%%%%%%%%%%%%%%%%%%%%%%%%%%
%%%%%%%%%%%%%%%%%%%%%%%%%%%%%%%
%%%%%%%%%%%%%%%%%%%%%%%%%%%%%%%

\paragraph{Le deuxième foyer américain, et l'inspiration majeure d'une nouvelle discipline, la \enquote{vie artificielle}}

Le terme d'\foreignquote{english}{Artificial Societies} \Anote{artificial_societies} qui consacre les usages alors naissant des modèles individu centré dans la discipline aurait été selon \textcite{Gilbert2000a} plus ou moins inventé en même temps en Europe et aux Etats-Unis, cela de façon indépendante à la fois par Epstein en 1996 et Gilbert and Conte en 1995.\Anote{gilbert_confidence}. Mais il est intéressant de voir que derrière un terme et un objectif finalement similaire (produire des expériences \textit{in silico} , mettre en oeuvre le concept d'émergence) les motivations et les sources d'inspirations mise en avant diffère légèrement. Là où l'expertise de Doran et de Gilbert s'appuie sur cette triple compétence mêlant intelligence artificielle, question théorique en sociologie et ancrage archéologique,  \autocite[17-19]{Epstein1996} révèle une approche initiale plus abstraite de ces notions au travers de \textit{Sugarscape}, inspiré principalement par le domaine de l'\textit{Artificial Life} ou Vie Artificielle (VA), un domaine de recherche alors très actif au \textit{Santa Fe Institute} (SFI).

%Doran and Gilbert (1994) argue that computer simulation is an appropriate methodology whenever a social phenomenon is not directly accessible, either because it no longer exists (as in archaeological studies) or because its structure or the effects of its structure, i.e. its behaviour, are so complex that the observer cannot directly attain a clear picture ofwhat is going on (as in some studies of world politics). The simulation is based on a model constructed by the researcher that is more observable than the target phenomenon itself. This raises issues immediately about which aspects of the target ought to be modelled, how the model might be validated and so on. However, these issues are not so much of an epistemological stumbling block as they might appear. Once the process of modelling has been accomplished, the model achieves a substantial degree of autonomy. It is an entity in the world and, as much as any other entity, it is worthy of investigation. Models are not only necessary instruments for research, they are themselves also legitimate objects of enquiry. Such “artificial societies” and their value in theorizing will be the concern of the first part of this chapter.

Joshua Epstein et Robert Axtell se sont rencontrés au \textit{think tank} de \textit{Brookings} en 1992. C'est peu de temps après, lors d'une conférence sur la \enquote{Vie Artificielle} au \foreignquote{english}{Santa-Fe institute} (SFI), qu'il trouve l'inspiration pour la réalisation du modèle de simulation SugarScape \Anote{histoire_sugarscape}. Un travail qui donne lieu à un livre \textit{Growing Artificial Societies: Social Science from the Bottom Up} réunissant différentes expérimentations autour de variations du modèle de simulation original \hl{Préciser que le code source n'a jamais été fourni, et que la plupart des implémentations sont des réécritures}, et une vision de la construction des modèles en science sociale résumé dans un simple motto (\textit{If you didn’t grow it, you didn’t explain its emergence}) sur laquelle nous aurons l’occasion de revenir d'un point de vue plus épistémologique. \hl{Référence à la section}

% Artificial Social Life (ASL) Epstein / Axtell

Le SFI est centre de recherche inter-disciplinaire indépendant ouvert en 1984 au Nouveau Mexique, principalement dédié à l'étude de la complexité au travers des Complex Adaptative System (CAS) sous toutes leurs formes : physiques, biologiques, sociaux, etc. Un des axes de développement important à SFI durant la fin des années 1980 tient dans l'émergence (en réalité la ré-émergence) du concept de Vie Artificielle sous l'impulsion principale de Christopher Langton, l'inventeur du terme. Cette acceptation permet tout à la fois de regrouper et de rendre visible sous une bannière identifiable les travaux de plusieurs décennies de recherches dans différentes disciplines (mathématique, informatique, robotique, biologie, écologie, etc.)  \autocite{Taylor1999}. Il en ressort également une forme de questionnement commun autour du concept de \enquote{Vie} lorsqu'il est appliqué à un environnement \enquote{informatique}.

Attention toutefois à ne pas voir le Santa Fe institute comme le lieu de création \textit{ex-nihilo} des concepts sous-jacents aux \textit{Complex Adaptative System} et à la nouvelle discipline de l'\textit{Artificial Life} de Langton. En effet, les problématiques et les discussions abordés dans ces \enquote{nouvelles disciplines} puisent matière dans les riches échanges inter-disciplinaires datant du début et milieu du XXième siècle, cela avant bien avant que le SFI ne sorte de terre au nouveau mexique en 1984.


%\hl{Années d'or 1977}



En France, les travaux sur la Cybernétique sont déjà observé de près depuis les années 1950 par le polytechnicien Robert Vallée et ses collègues dans le cadre du \enquote{Cercle d’études cybernétiques} \autocite{Bricage1990}. 

Le livre de Wiener \textit{Cybernetics or Control and Communicat
ion in the Animal and the Machine} est publié en Français en 1948, et l'ouvrage \enquote{Les problèmes de la vie} \Anote{pouvreau_livre1949} qui consacre le travail de Bertalanffy démarré dans les années 1930 parait en allemand en 1949, en anglais en 1952, et la première traduction francaise date de 1961. \autocite{Vallee2005}

% Marois1971 et Marois1969
Autre événément important dans l'histoire du rapprochement entre discipline, c'est à l'Institut de la Vie fondé en 1960 à Versaille et voulu par Maurice Marois que se réunissent en 1967 des chercheurs de tous horizons pour une première grande conférence internationale de physique théorique et de biologie. Première d'une longue lignée, celle-ci est ouverte par le zoologiste et président de l'académie des sciences Pierre-Paul Grassé (inventeur entre autre du terme \textit{stigmergie} \autocite{Theraulaz1999}), alors entouré d'un comité scientifique non moins prestigieux : P.Auger, A. Fessard, H.Frolich, A.Lichnérowicz, I.Prigogine, L.Rosenfeld. \autocites{Marois1969,Marois1971}

Des conférences qui vont se poursuivre à Versaille jusqu'en 1973, puis à Edinburgh par la suite, avec cette volonté toujours renouvelée de défricher toutes les passerelles plausibles qui constituent le lien entre physique et biologie autour de cette thématique universelle \enquote{Qu'est-ce-que la vie ?}. parmi les participants réguliers on retrouve Prigogine, mais également Hermann Haken. Ce dernier, déjà présent lors des premières conférence en 1967, sera amené dans un futur proche à porter le concept de \enquote{Synergétique} en tant qu'orateur en 1971 \autocite{Kroger2012, Kroger2015}. De son coté,  Prigogine est amener à introduire le concept des \enquote{structures dissipatives} bien plus tôt, dès les premières conférence \autocite[60]{Stengers1985}

Une inspiration qui se poursuit dans les 1970-80 avec l'introduction de ces nouveaux concepts dans une communauté enthousiaste (Morin, Le Moigne, Dupuy, etc.), 1977 étant souvent qualifié d'\textit{Annus mirabilis} car marqué par la sortie de nombreux ouvrages majeurs. Structuré autours d'associations comme l'AFCET (devenu depuis 1999 AFSCET) qui coordone depuis sa création en 1968 \autocite{Hoffsaes1990} les réflexions de centaines de chercheurs et ingénieurs autour de groupes de travail inter-disciplinaire, de publications, de conférences internationales. Ainsi plusieurs événements majeurs ont lieu autour de l'auto-organisation au début des années 1980, le colloque de Cerisy organisé en 1981 intitulé \enquote{L'auto-organisation: De la physique au politique} \autocite*[postnote]{key}{Dumouchel1983}, et la conférence de 1982 à Bruxelle sponsorisé par l'AFCET-SOGESCI et organisé par Bernard Paule, le point culminant d'une série de conférences démarré en 1975 sur les Systèmes Dynamiques.

\hl{retravail AVEC CITATION DE l'annexe A}
% Présence de Deneubourg à LOS ALAMOS... retrouver la ref
%\hl{Travail de Deneubourg (sur les deux plans), Brooks (retour au subsymbolisme) à intégrer ici ?!}

On comprendra avec ce bref eclairage sur l'historique complexe de la notion d'auto-organisation les quelques grincements de dents des européens \autocite{Varela1995} lorsqu'il s'agit d'évoquer l'origine des CAS et de la notion (trop?) computationalisé de \textit{ALife}, qui bénéficie d'une couverture médiatique et institutionelle importante, dans la pure tradition des financements américain. \Anote{helmreich_IA} 

%Citation du livre Handbook of archeological method 
%Edited by Herbert D.G.Maschner et Christopher Chippindale
%2005
%One of the key insights claimed for CAS structures is their ability to self-organize (Holland 1992 / Kauffman 1993)
%Despite the implication from Americanist litterature that self-organized phenomena are a recent product of CAS research at Santa-Fe (Gumerman and Gell-Man 1994, Kauffman 1995) , it need to be remenbered that the paradigm of self-organization  has a somewhat longer history mainly because of the work of Ilya Prigogine on nonlinear dynamics and dissipative structures (Nicolis And Prygogine 1977, Prygogyne 1978 1980)
%In fact the paradigme was first introduced to an archeological audience a decade ago by Prigogine's colleague Peter Allen(1982a 1982b) and to Anthropology by Adams (1988) ! 


\paragraph{Automate Cellulaire}
\label{p:va_automate_cellulaire}

Conscient maintenant du recul historique nécessaire pour évaluer à leur juste valeur les travaux initiés au SFI dans les années 1980, on peut évoquer les racines historiques de l'outil qui a servit de support principal à ces développements. Ainsi à ce titre, et en parallèle des développements mathématique abordant l'auto-organisation sous l'angle de la thermodynamique \Anote{liaison_prigogine_foerster}, l'automate cellulaire s'est avéré très tôt comme un outil capable d'intégrer ces multiples influences, notamment du fait des très nombreuses propriétés que ce type de formalisme continue d'exposer \autocite{Ganguly2003}. %classification des automates cellulaires de Wolfram, Temps discret etat de Zeigler 1976

parmi les différentes propriétés qu'il est possible d'étudier dans les automates cellulaires, on retiendra pour l'étude de la VA la réplication, ou la reproduction \Anote{taylor_reproduction} d'entité autonome évoluant dans un environnement ouvert, qualifié aussi par Taylor de \textit{Open-Ended Evolution (OEE)} \Anote{taylor_openended}. 

Sans rentrer plus en avant dans les subtilités qu'amène une telle définition, on observe sur ces différentes questions des publications marquantes inspiré le plus souvent des travaux initiaux de Von Neuman et Ulmman (auto-reproduction), mais aussi les travaux très concrets et souvent oubliés \autocites[111-130]{Dyson1997}{Fogel1998, Taylor1999, Hackett2014} du mathématicien et biologiste Italo-Norvégien Nils Aall Barricelli (1957) (la notion de \foreignquote{english}{symbioorganism}) : la proposition d'automate cellulaire évolutionnaire pour l'auto-organisation du cybernéticien du BCL Gordon Pask \autocite{Pask1961}, jeu de la vie de Conway, Conrad \textcite{Conrad1970}, alpha univers de Holland \autocite{Holland1976}, boucle reproductible contenant du matériel génétique de \textcite{Langton1984}, automate cellulaire illustrant l'auto-poeise de \textcite{Varela1974,McMullin1997b, McMullin1997, McMullin2004}.

De façon encore plus générale, la VA va s'appuyer sur cette large classe d'algorithmes inspiré par la biologie (Biological computing). Ainsi et dans la continuité des travaux évoqué au dessus, la VA va utiliser pour la mise en œuvre des aspects évolutionnaires de ces programmes des travaux regroupés sous le terme générique de \textit{Evolutionary Computation} (EC) \autocites{Back1997, Fogel1998, Fogel2006a}. Une sous classe de techniques issues de l'Intelligence Arficielle principalement inspiré des mécanismes d'évolution biologique, eux même subdivisé en différentes familles (Genetic Algorithm (GA), Genetic Programming (GP), Particle Swarm Optimization (PSO), Ant Colony Optimization (ACO), etc.) parfois difficile à distinguer. Ils peuvent être appliqué à différentes classes de problèmes, et ne relèvent pas forcément d'une fonction fitness explicite : évolution de comportement, évolution de forme, évolution artistiques, etc. Tout dépend donc à quelle échelle \Anote{echelle_optimization} ont considère le problème d'optimisation; l'individu amené à être évalué peut tout à la fois dénoter une entité virtuelle autonome évoluant dans un environnement comme un robot dans une simulation, ou les paramètres d'un modèle de simulation, ou encore une sous ensemble de fonction mathématique dans un polynôme.

%Ecologie + GA Hamblin2013

Mais elle s'inspire également de ce que l'on peut considérer comme le chemin inverse (Computational Biology) qui constitue à simuler le vivant en s'appuyant sur l'informatique, ce qui peut inclure l'emploi de technique évolutionnaire, les aller retour entre les deux approches (\textit{Biological computing} et \textit{Computational Biology}) étant bien établis \autocites{Giavitto2002, Hogeweg1992} \Anote{terme_bioinformatique}.

% A placer avant surement, vu que sfi est un peu la synthèse des deux courants aux USA ? celui de burks et celui de foerster...

En effet, Arthur Walter Burks (mathématicien, physicien, philosophe), bien connu pour son travail sur ENIAC ( \textit{Electronic Numerical Integrator and Computer} ), et sa collaboration fructueuse avec Von Neumann sur de nombreux sujet, comme la \textit{Theory of Self-Reproducing Automata}, publié par Burks en 1966, soit presque 10 ans après sa mort. Etabli à l'université du Michigan , il créé en 1949 le \textit{Logic of Computers Group} rattaché au département de philosophie, un fait pas si étonnant quant on sais que Burks a disserté en 1941 sur les fondations logiques du scientifique et philosophe Charles Sanders Peirce (\textit{The Logical Foundations of the Philosophy of Charles Sanders Peirce}). Après rapprochement avec le département de linguistique de Peterson, un comité est formé et devient capable de délivrer des diplome dès 1957. Fait d'enseignement inter-disciplinaire délivré dans chacun des départements respectifs, le \textit{Computer \& Communications Sciences}\Anote{nature_ccs} passe de programme à département en 1965. Dédié à l'étude du \textit{computing} est inter-disciplinaire, ce dernier va former de nombreuses figures connu de l'informatique et de la simulation. Si on en croit la base de données de \textit{Mathematics Genealogy Project} celui ci n'a eu que deux élèves, John Holland en 1959, premier professeur du CCS (qui a encadré par la suite plus de deux cent chercheurs), probablement un des premiers \textit{phd} en informatique, et Christopher Langton en 1991.

%devoted to the interdisciplinary study of complex information processing systems of all kinds, both natural and artificial

%http://www.lsa.umich.edu/cscs/aboutus/bachgroup
Dans les années 1980, Burks fera partie du groupe inter-disciplinaire nommé BACH, réunissant Bob Axelrod, Michael Cohen and John Holland, et qui préfigure le futur CSCS en 1999.

\hl{ ajout note sur Weinberg1971 }

Ainsi on constate par exemple en biologie la similarité \autocite{Hermann1973, Hogeweg1974, Stauffer1998} des automates cellulaires (issue au départ d'une analogie avec le vivant) et des L-System \autocite{Prusinkiewicz1999} de Lindenmayer (1971), également étudié et mis en application dans des simulations utilisant des automate cellulaire \autocites{Hogeweg1978, Frijters1974}.

Ermentrout1993

On trouvera une description plus exhaustive de l'apport de ces chercheurs dans leurs publications respectives et dans les ouvrages de synthèse suivants dont sont tirés la majorité des références cités au préalables \autocites{Dyson1997,Fogel1998, Sipper1998, Fogel2006a}[46-66]{Taylor1999} 

%1957
%Nils Aall Barricelli. 
%Symbiogenetic evolution processes realized by artificial methods. 
%Methodos, 9(35-36), 1957.
%
%Dyson1998
% “symbioorganism” defined as a “self-reproducing structure constructed 
%by symbiotic association of several self-reproducing entities of any kind” 


% 1976 AFCET RECENSEMENT DS : J.F Le Maitre( 2 projets sur ibm 360),P. Uvietta dès juin 1975 (probablement amoral, ibm 360), Ch. alexandre (démarre en 73)

% ANNUS MIRABILIS pour la self organization avec la publication de multiples ouvrages (voir ce que dit jean louis lemoigne + Pelster)
% Prigogine1977 AFCET versailles => Urbain est présenté par prigogine, allen ,etc. A mettre en parallele avec la liste donné pour SD par Karsky

% symbioorganism self-reproducing structure constructed by symbiotic association of several self-reproducing entities of any kind

%VENUS : http://www.cs.cmu.edu/Groups/AI/html/faqs/ai/genetic/part3/faq-doc-4.html
%rasmussen : http://www.scoop.co.nz/stories/HL1212/S00060/steen-rasmussen-the-flag-bearer-of-artificial-life.htm
% rasmussen bio : http://flint.sdu.dk/index.php?page=steen-rasmussen



%TIERRA : https://www.cs.cmu.edu/afs/cs/project/ai-repository/ai/areas/alife/systems/tierra/
% http://life.ou.edu/
%http://books.google.fr/books?id=DGdwAwAAQBAJ&pg=PA284&lpg=PA284&dq=tierra++%2B+VENUS+Rasmussen&source=bl&ots=mZtaVfMpUN&sig=2ZAUtx-yQoET3jPUP7pxudewbzo&hl=fr&sa=X&ei=jiwXVOavCciUavbLgKgI&ved=0CDgQ6AEwAg#v=onepage&q=tierra%20%20%2B%20VENUS%20Rasmussen&f=false

Plusieurs chercheurs interne ou externe au SFI se concentre au début des années 1990 sur le développements de support logiciel innovants, au delà des pratiques courantes d'utilisation d'automates cellulaires, comme le démontre ces trois différents projets : 

\textbf{a)} la famille de logiciels ECHO tient d'une commande faite par Murray Gell-Mann, un des fondateur du SFI, à John Holland pour développer un logiciel d'écologie virtuelle pour les CAS. L'initiateur des \textit{Genetic Algorithms} (GA) reprend avec ses collègues Mitchell et Forrest des travaux usant des GA dans une perspective écologique et individu centré compatible avec ce que l'on peut attendre de la Vie Artificielle. \autocites{Holland1993, Mitchell1993, Smith2000}

\textbf{b)} Le logiciel TIERRA de Thomas Ray, un biologiste tropical converti à la VA au début des années 1990. Inspiré entre autre par le développement des programmes CoreWar (Dewdney1984) et CoreWorld/VENUS (Virtual Evolution in a Nonstochastic Universe Simulator) du chimiste danois Rasmussen \autocite{Rasmussen1990}, Thomas Ray développe un écosystème virtuel sous forme de métaphore de l'ordinateur. Des morceaux de code vivent, luttent, mutent et se reproduisent dans dans un espace mémoire virtuel limité en utilisant de l'énergie tiré d'un CPU lui aussi virtuel. Par les surprenantes formes de vies artificielle qu'il a mis à jour, le travail toujours actif de Ray a inspiré d'autres recherches et d'autres logiciels similaires comme Avida (1998) ou Cosmos (\autocite{Taylor1999})

\textbf{c)} enfin, le logiciel SWARM \autocite{Minar1996} est initié puis supervisé par Langton au début des années 1990. Créateur du terme fédérateur de \enquote{Vie Artificielle}, c'est lui qui organise au SFI en 1987 les premières conférences sur ce vaste sujet. La plateforme SWARM, initié et développé par Langton, est une idée dont l'origine prend racine dans les expérimentations de Langton courant des années 1980. Cette fois ci, le SFI met à disposition de Langton une équipe de développeur dédié à la création d'une plateforme Agent spécifique aux CAS. Mais celle ci repose sur des bases différente des deux autres, dans le sens ou elle ne s'attache pas spécifiquement aux aspects évolutionnaires des CAS pour fournir dans une librairie développé au dessus du langage Objective-C des objets de plus haut niveau d'abstraction permettant aux scientifiques de manipuler plus rapidement des agents, quelqu'il soit.

En science sociale, on connaît bien évidemment les travaux isolés, pionniers et souvent oubliés \autocites{Hegselmann2012, Aydinonat2007} du psychologue James Minoru Sakoda \autocite{Sakoda1949} (implémenté sur ordinateur en \autocite{Sakoda1971}), et ceux plus connu de de Schelling(1971, 1978) qui dans une interview dit ne pas avoir connu ces même travaux de Sakoda \autocite{Aydinonat2005}, et enfin Axelrod(1984). Mais le premier à évoquer spécifiquement la technique d'automate cellulaire pour l'analyse de phénomènes socio économiques serait de l'avis général l'économiste américain Peter S. Albin dans son livre \textit{The Analysis of Complex Socioeconomic Systems} \textcites{Smith1975, Ganguly2003, Benenson2004, Portugali2000}. En géographie, plusieurs géographes ont également perçu très rapidement l'intérét de proto automates cellulaires pour l'évolution de modèle de simulation spatio-temporel (Hagerstrand1953, Tobler1975, Tobler1979, Couclelis1985).

%Kaiser 1979, Lomnicki 1978,1988 mais Neither the work of Kaiser nor that of ?Lomnicki had a strong impact on the early development of the IBM approach.; cf Grimm2004 

%Sarkar2000
%L-Systems et A-Life : hogeweg, smith

L'écologie est également par de nombreux aspects une discipline intéressante à évoquer de notre point de vue. D'une part car elle est également pionnière sur la mise en oeuvre de ces techniques individu-centré (le terme consacré dans la discipline étant \textit{Individual Based Modelling} ou IBM ) usant très tôt des automates cellulaires, et d'autres part car elle tient une place et une influence particulière dans la géographie à la fois passé et probablement à venir. Enfin il faut souligner que des écologues comme Volker Grimm ont produit des ouvrages qui se sont avérés fondateurs pour l'évolution et la formalisation des IBM en écologie \autocites{Grimm2004, DeAngelis2014}, ouvrages dont ils ont tirés des publications méthodologiques et techniques \autocite{Railsback2012} tout à fait remarquables par la pédagogie (Netlogo) et la pertinence des solutions proposés, tout à fait applicables dans d'autres communautés utilisant ce méta-formalisme agent (ODD \autocite{Grimm2010}, POM \autocite{Grimm2005,Grimm2011}, Analyse de sensibilité avec R et Netlogo \autocite{Thiele2011,Thiele2014a}). % Thiele2014b ?

Dans un article de \textcite{DeAngelis2005} les deux auteurs réalisent un état de l'art des usages de l'IBM sur 900 références. Un travail conséquent permettant d'établir une première grille de lecture en cinq axes pour qualifier la nature des variations individuelles : \textit{(a) spatial variability, local interactions, and movement; (b) life cycles and ontogenetic development; (c) phenotypic variability, plasticity, and behavior; (d) differences in experience and learning; and (e) genetic variability and evolution.} Des variations qui peuvent être mobilisé pour mettre en lumière sept types de processus écologiques et évolutionnaires différents : \textit{movement through space, formation of patterns among invidivual, foraging and bioenergetics to population dynamics, exploitative species interaction, local competition and community dynamics, evolutionary process, management-related processes}. A partir de ces deux caractérisations, on ne peut que constater l'existence d'un lien forcément étroit \autocite{Dorin2008} et ancien \autocites{Hogeweg1988, Hogeweg1990, DeAngelis2014} entre certaines pratiques mise en oeuvre dans les étude de Vie Artificielle et celle des écologues pionniers dans l'\enquote{écologie virtuelle}, ne serait ce que sur l'axe des processus évolutionnaire.

Si l'approche individualisé en écologie ne devient vraiment significative qu'à partir des années 1990, c'est grâce notamment à la publication d'articles théorique fondateurs \autocite{Huston1988} et de plusieurs reviews faisant état de formalismes \autocite{Hogeweg1988} et de modèles pionniers \autocites{Hogeweg1990, DeAngelis1992, Judson1994} dans diverses branches de l'écologie. Cette piste nous pousse à évoquer les prémisses opérationels d'expérimentations similaires à la VA, qui opère bien en amont du SFI. Ces travaux pionniers sont réalisés dans plusieurs branches de l'écologie, de la biologie, usant d'écosystème virtuel pour mener leur expérimentations. Pour n'en citer que quelqu'uns, on trouve une longue suite de modèles de simulations innovants dans l'étude des dynamiques forestière \autocite{Bugmann2001} tel que JABOWA (1970) \autocite{Botkin1972}, FORET (1977), FORTNITE (1982); des pionniers que l'on trouve également dans l'éthologie (étude des comportement animaux) avec les travaux sur l'auto-organisation et le multi-niveau de Hogeweg et Hesper appuyé par leur système de simulation MIRROR (Hogeweg1979, Hogeweg1983); enfin il est également noté la famille de simulateur EVOLVE initié en 1970 et amélioré depuis au fil des ans par Michael Conrad et Howard Pattee. \autocites{Conrad1970, Pattee2002}

%smith_bio

Toutefois comme semble le souligner \textcite{Dorin2008}, bien que ces simulations de VA possède une utilité de par les questions génériques qu'elles abordent (un peu à la façon du modèle de Schelling), il n'est pas raisonnable à l'heure actuelle d'en faire un usage comparé avec l'écologie réelle. Ce dernier invite toutefois à un rapprochement mutuel qu'il estime à terme bénéfique pour les deux parties, l'utilisation de méthodologies adaptés (POM) permettant d'assurer l'approche réelle de ces questions d'évolutions, jusqu'alors relativement peu pris en compte avec l'approche agent actuelle, un constat étonnant quand on connaît l'ancienneté des premières solutions exposé ci dessus.

Cette écologie virtuelle qui coexiste de façon relativement proche à la VA fournit une inspiration importante aux informaticiens qui vont par la suite évoluer au contact des scientifiques en sciences sociales, notamment en France. Si Doran fait partie des scientifiques s'appuyant sur la branche des agents \enquote{cognitif}, la VA vient de façon complémentaire nourrir les réflexions d'une branche divergente en DAI, celle des agents dit \enquote{réactif}. Comme le disent deux acteurs importants dans la formalisation et la diffusion de cette branche \autocite[31-32]{Ferber1995} et \textcite[7-10]{Drogoul1993}, cette résurgence de la VA coincide avec l'approche réactive dans le contrepied pris face à l'approche \enquote{cognitiviste}; le concepts d'auto-organisation est évoqué plus simplement au travers des concepts d'autonomie, de viabilité et d'une intelligence plus simple de type stimulus/réponse montrant qu'il est déjà possible d'obtenir des comportements complexes à partir de mécanismes simples.



\printbibliography[heading=subbibliography]