
\graphicspath{{Figure1/}}

\chapter{Construire et Évaluer des modèles en géographie}

\startcontents[chapters]
\Mprintcontents

\section {Validation, Évaluation de modèles agents} 

Tout au long de ce chapitre, il serait fait régulièrement référence aux différents travaux et publications de Frédéric Amblard, car ils constituent à bien des égards des points d'entrées importants dans notre réflexion sur la validation dans la simulation de modèle en géographie.

\subsection{Un transfert de l'ingénierie à la simulation en science humaine et sociale}

Les références à Sargent, Balci,sont de par leur nature générale couramment reprise dans différents ouvrages ou publications.

Sur la question du transfert des épistémologues se sont déjà penchés sur la question ... voir Numo

Partir de Naylor pour évoquer la nécessité d'une prise de recul dès lors que l'on cherche à caractériser le retour sur investissement

Toutefois plusieurs autres auteurs viennent nous rapeller que le jeu joué par Naylor est dangereux, car sans tomber dans un relativisme qui se voudrait naïf et dangereux dans une perspective inter-disciplinaire, meme si la validation est avant tout problème qui doit être abordé à l'aube de la discipline qui la mobilise. Il ne s'agit donc pas de réduire la validation à un question sociologique ou psychologique lié aux individus ou groupes d'individu ici, mais simplement d'exposer la validation à une forme d'expertise plus large, à meme de comprendre les enjeux et les moyens mobilisés pour tenter de mener à bien une évaluation qui se veut un tant soit peu objective.

Pour prendre un exemple plus concret, si on considère la mobilisation dans nos modèle de simulation du  modèle gravitaire, qui est une forme stylisé et éprouvé par l'expérience de processus réel en jeu dans la réalité, alors il est admis que la connaissance produite par ce modèle puisse être jugé de façon objective compte tenu des théories ainsi injectés.


\section{Objectiver les processus à l'oeuvre dans le cadre d'une évaluation par les pairs}

\subsection{au regard de la littérature en géographie}

\subsection{au regard de la notion d'équifinalité}

=> Osullivan nous propose (dit de facon implicite) de se tourner  vers l'évaluation collective, la seule pour lui à même de dégager une connaissance. KISS en lui même {Axelrod1997} porte cet idéal de simplicité (qui ne renie pas en tant que telle la complexité et la richesse descriptive des mécanismes) pour favoriser l'échange, la diffusion des modèles. Hors de ce coté, le constat est maigre. Le mouvement M2M {Amblard, Rouchier},{Rouchier}, {M.Batty/P.Allen}

Sur ces points Hedstrom rejoint OSulivan et les autres, la Discussion de la scientificité des modèles se fait avant tout sur la qualité des mécanismes et des causalités à l'oeuvre (Entities / Activities de Machamer).  

Présentation du plan lié à ce paradoxe, il faut mobiliser des moyens techniques et humains important pour que puisse mettre  en place des outils standardisé à même d'externalisé le débat non plus sur la question de l'évaluation, mais sur l'objet de cette évaluation. Autrement dit pour pouvoir discuter de la connaissance apporté par le modèle, il faut être capable de produire des coupes fiables des multiples dynamiques à l'oeuvre dans nos modèles .: corrélation entre paramètres, 

Ces outils existent, non seulement car ils font l'objet de recherche en eux même, mais aussi parcqu'ils sont appliqués de façon systématique dans certaines disciplines. Pourtant en modélisation en géographie et dans d'autre disciplines des sciences humaines, il semble peu mobilisé, et cela de façon historique.

L'évaluation de par sa nature contextuelle doit être faite au préalable par les pairs, tout en restant ouverte à la critique interdisciplinaire. Toute la difficulté de tels modèles résidant donc dans le placement du curseur entre ces deux pôle que l'on pourrait qualifier d'attracteurs, de par les moyens qu'il faut mobiliser pour s'en rapprocher dans un cas ou dans un autre.

\subsection{Le processus de création intègre la validation}

Présentation concept de validation interne / externe.

De très/trop nombreux guides méthodologiques pointent trop souvent la construction de modèle comme la recherche d'un état fini, donné, qui n'est pas compatible, ni avec la notion d'équifinalié, ni avec l'idée qu'un modèle est souvent révisable continuellement.

\subsubsection{État des lieux, des moyens inadaptés à la création des modèles}

Complexification, Généralisation, et l'infernal aller retour entre les deux, voir concept théorie précédant à l'observation en épistémologie ?

=> Pour être à même de mesurer le retour de connaissance dans un modèle qui suit une courbe de complexification, deux concepts au moins doivent pouvoir être mobilisé : 

\begin{itemize}
\item La ou les briques ou incrément unitaire qui encapsule l'ajout de connaissance qui concrétise un différentiel avec le modèle précédent, que l'on peut résumer ainsi "En quoi ce nouveau modèle est un modèle différent vis à vis de la question posé par le modèle"
\item Le processus de mobilisation de cette brique unitaire de reflexion dans l'exploration de la dynamique, que l'on peut evoquer de la facon suivante "quel est l'impact d'une insertion, de la modification ou d'un retrait d'une brique, et qu'est ce que je peux en dire vis à vis de la question posé par le modèle"
\end{itemize}

\subsubsection{Le choix d'une unité de raisonnement aproprié}

Une prise de recul reflexive nécessaire pour comprendre ou se situe la création de connaissance vis à vis de la question posé.
Ce qui nous amène à nous intéresser autant aux processus de création de cette connaissance qu'à cette connaissance elle même.

Grille de lecture existante : 
> Guide de bonne pratiques
> Fer à cheval d'Arnaud, etc.

=> ok mais ca suffit pas, on a vu dans le cadre des questionnements originaux en géographie, la notion de mécanisme

\subsubsection{La spécificité du questionnement géographique}

Discussion / Remise en cause de la notion de mecanismes avec la géographie ? 
A regarder la littérature actuelle sur la relation micro/macro dans les modèles agents, il n'y a peu ou pas de prise en compte du spatial dans la construction des relations micro macro observé dans les modèles.

Le modélisateur doit toutefois être attentif sur au moins deux points : 
 a) La biologie n'est pas la géographie, et les analogies ont le sais peuvent rapidement être dangereuse (ex ville comme organisme vivant, etc.)
 b) Ne pas s'enfermer dans les modèles edictés par de tels raisonnements, cf exemple de René pumain sur le passage de l'influx nerveux, et le fait que la mitocondrie se met tout à coup à pariticper au passage des ions, alors qu'en fait on pensais ce truc complétement passif dans l'échange d'ions ... 

Equivalent pauvre en géographie du "schemata" evoqué par Machamer serait les chorème de Roger Brunet, toutefois ceux ci sont limités dans le sens ou ils sont statiques. En ce sens, en terme d'outil pour penser, la simulation accède au rang de "schemata", support dynamique à la réflexion.

\section{Une double contrainte sur la réalisation des modèles, explicativité et parcimonie}

Si on considère l'ajout de mécanismes au modèle, celui ci est contrainte au moins de deux façon : 
> La recherche d'une parcimonie minimale explicative du point de vue de la question
> La recherche de la meilleur estimation rapport à une série de données, ou à un ou plusieurs fait stylisés

Quels sont les cas possibles :
=> Il y a correspondance parfaite entre réel et simulé, le modèle est surdéterminé
=> il y a non correspondance entre réel et simulé, mais l'ajout de mécanisme apporte un gain
=> il y a non correspondance entre réel et simulé, mais l'ajout de mécanisme n'apporte pas de gain, du moins pas dans la dynamique actuelle

Schéma évoquant la double contrainte, permet d'évoquer les problématiques associés et de dérouler par la suite sur l'absence technique de support pour une telle exploration : algorithme génétique, description de la contrainte, objectif à réaliser, ajout mécanisme, etc.

\subsection{Le processus de diffusion}
\subsubsection{Des moyens inadaptés à la diffusion des modèles}
Absence de plateforme de publication
Pas de protocole standard ou de pratiques etablies dans la discipline

\section {Isoler les pratiques et des outils existant}

Ce travail a été largement initié suivant des problématiques déjà très bien décrite par Thomas Louail dans sa thèse...
Ajouter tout le travail réalisé pour la présentation ecqtg2013


\subsection {La modélisation agent en géographie} 

Cette section est volontairement détaché du reste, car si la modélisation agent amène certe de nouvelles problématique à la notion d'évaluation, il n'empêche que cette problématique peut être traité séparément de par son ancienneté, et sa relative indépendance vis à vis des techniques employés.

Les technique de construction que nous présentons sont évidemment à rattacher aux pratiques actuelle de modélisation utilisant les agents, il n'en reste pas moins.

\stopcontents[chapters]

%\begin{table}
%\centering
%\subfloat[Source: élaboré à partir de Courel et al. (2005)]{
%    \includegraphics[width=160mm]{ClassificationMotifs.pdf}
%    }
%    \captionstyle{\centerlastline}
%    \caption{Typologie classique des motifs de déplacements}
%    \label{tab:classificationmotifs}
%\end{table}



