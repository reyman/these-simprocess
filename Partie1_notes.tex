% -*- root: These.tex -*-

\Anotecontent{holland_multi_utilisation}{ Holland a développé les GA avant tout pour leur capacité de \foreignquote{english}{robust adaptive systems} et pas seulement pour leur capacité d'optimisation comme le rapelle \textcite{DeJong1993a} : \foreignquote{english}{However, with all this activity, there is a tendency to equate GAs with function optimization. There is a subtle but important difference between \enquote{GAs \textbf{as} function optimizers} and \enquote{GAs \textbf{are} function optimizers}} ; L'investissement d'Holland dans l'étude des \foreignquote{english}{Complex Adaptive Systems} s'inscrit dans une trajectoire de recherche resté proche des thématiques de ce qui deviendra plus tard la méta-discipline \textit{Artificial Life}. Son investissement continue dans cette branche de développement est d'ailleur lisible au travers de deux plateformes successive sur ce thème : \textit{$\alpha$-universe} et \textit{Echo} dont on trouve une analyse dans les travaux de \autocites{Taylor1999, Taylor2001} }


\Anotecontent{difference_objective_heuristique}{Il n'est pas forcément évident de faire la différence entre ces termes très proche, dont le sens se recoupe parfois, voici donc une aide à la désambiguisation inspirée de celle de \textcite[36]{Weise2011} :

\begin{enumerate}[labelindent=\parindent,leftmargin=*]
\item La fonction objectif (\textit{objective function}) peut etre considérée comme une forme d'heuristique, à la différence que celle ci est une mesure forcément directe du potentiel d'un aspect de la solution, alors que l'heuristique peut être de mesure directe ou indirecte, en ne fournissant par exemple qu'une approximation de la distance séparant une mesure de l'optimum. En ce sens, la fonction objectif mobilise souvent plus d'expertise sur le système que l'heuristique.
\item Une fonction \textit{fitness} est une fonction d'utilité secondaire, conçu comme une combinaison possible de fonction objectifs, et/ou d'heuristiques. Celle-ci peut également être une mesure relative, pour quantifier par exemple la différence existante entre deux solutions.
\end{enumerate}}

\Anotecontent{barricelli_multi_utilisation}{ Tout comme les travaux de McMillan ont permis de voir plus clair dans les intentions de Von Neumman derrière la notion de \textit{self-reproduction automata} ..., les travaux de Dyson \Autocite{Dyson1997}, de Fogel \autocite{Fogel2006a} sur l'histoire de cette discipline a permis également de redécouvrir les recherches de Barricelli comme celle d'un véritable pionnier en ALife, mais également comme celui d'un pionnier dans l'idée d'utiliser l'évolution comme support à la résolution de problème.}

\Anotecontent{fraser_comment}{\foreignquote{english}{Fraser was one of the first to conceive and execute computersimulations of genetic systems, and his efforts in the 1950s and1960s had a profound impact on computational models of evo-lutionary systems. The simulation algorithms he used were im-portant not only in the simulation of genetical problems, but pro-vided a menu of techniques that enriched the entire simulationeffort in any problem that involved probability sampling amonga population of alternatives, the heart of Monte Carlo methods. }\autocite[429]{Fogel2002}}

\Anotecontent{note_pattee_semantic_closure}{ \foreignquote{english}{Additionnary, from an epistemological point of view, Pattee(1995b) points out taht symbolic information (such as that contained in an organisms's genes) has \enquote{no instrinsic meaning outside the context of an entire symbol system as well as the material organization that constructs(writes) and interprets(reads) the symbol for a specific function, such a classification, control, construction, communication ...}. He argues that a necessary condition for an organism to be capable of creative open-ended evolution is that it encapsulates this entire self-referent organisation (Pattee refers to this condition as semantic closure). From this it follows that organisms should be constructed \enquote{with the parts and the laws of an artifical physical world} Pattee (1995a)(p.36). In other words, the interpretation (phenotype) of the symbolic information (genotype) of an artificial organism should be constructed and act within the artificial physical environment of the system. Additionally, if the system is to model the \enquote{origin} of genetic information, then the genotype itself must also be embedded within the environment; that is, the complete semantically-closed organisation -- the \enquote{entire organism} -- must be completely embedded within the physical environment.} \autocite{Taylor2001}} 

\Anotecontent{billet_weise}{Voir le \href{http://blog.it-weise.de/p/309}{@billet} daté de juin 2014.}

\Anotecontent{note_pengouin}{Que faut il penser par exemple d'un algorithme bio lorsqu'il est nommé \foreignquote{english}{Pengouin Search Optimization Algorithm} (PSOea) \autocite{Gheraibia2013} ? }

\Anotecontent{stochastic_note}{Si l'optimisation stochastique (\textit{stochastic optimization}) ou approche probabiliste de l'optimisation (\textit{probabilistic approaches} apparait comme un autre chapeau susceptible de pouvoir englober l'ensemble de ces techniques, le schéma \ref{fig:S_OverviewOptimisation} de Weise contredit ce constat. Il existe en effet dans cette vaste catégorie tout un ensemble de techniques (\textit{Hill Climbing}, \textit{Simulated Annealing}, etc.) qui diffèrent très fortement dans leur structure, leur définition, ou leur inspiration, de la branche de techniques qui nous préoccupe ici, à savoir l'EC. }

\Anotecontent{equipe_mixite}{Suivant les travaux menés dans notre équipe par \textcite{Reuillon2014}, la validité de ce dernier paragraphe est clairement remis en question. L'originalité de ces derniers résident dans la mixité de ces deux objectifs. En intégrant \enquote{la capacité d'extension spatiale} dans l'exploration de ces espaces comme un critère d'optimisation supplémentaire aux objectifs plus classique de recherche de minima, une cartographie dirigé et plus exhaustive est devenu possible.}
%En intégrant l'\enquote{exploration de cette espace} des solutions (si on veut découvrir une carte  exhaustive des solutions optimisés), ou de l'espace de recherche (si on veut cartographier l'espace de recherche menant à cet espace de solution optimisé) comme un objectif d'optimisation supplémentaire aux objectifs plus classique de recherche de minima, une cartographie dirigé et plus exhaustive de certaines zones de cet espace est devenu possible.}


\Anotecontent{def_meta_sorensen}{\foreignquote{english}{A metaheuristic is a high-level problem-independent algorithmic framework that provides a set of guidelines or strategies to develop heuristic optimization algorithms (Sörensen and Glover, 2013). [...] A problem-specific implementation of a heuristic optimization algorithm according to the guidelines expressed in a metaheuristic framework is also referred to as a metaheuristic. The term was coined by Glover (1986) and combines the Greek prefix meta- (metá, beyond in the sense of high-level) with heuristic (from the Greek heuriskein or euriskein, to search)} \autocite{Sorensen2013} } 

\Anotecontent{def_meta_weise}{C'est également ainsi que \textcite[36, 225]{Weise2011} comprend ce terme \foreignquote{english}{A metaheuristic is a method for solving general classes of problems. It combines utility measures such as objective functions or heuristics in an abstract and hopefully efficient way, usually without utilizing deeper insight into their structure, i. e., by treating them as black box-procedures}}

\Anotecontent{greedy_description}{Un \enquote{choix optimal local} est réalisé à chaque itération durant l'optimisation, ce qui produit en général des solutions viables mais très rarement optimales.}

\Anotecontent{q_ppr}{Questions tirés du wiki \textit{Portland Pattern Repository} (\href{http://c2.com/cgi/wiki?MetaHeuristic}{@PPR}), qui est au passage un des premier wiki sur le web (1995)}