% -*- root: These.tex -*-


\Anotecontent{note_metaheuristique}{\hl{voir wikipedia pour ref definition}}

\Anotecontent{holland_multi_utilisation}{ Holland a développé les GA avant tout pour leur capacité de \foreignquote{english}{robust adaptive systems} et pas seulement pour leur capacité d'optimisation comme le rapelle \textcite{DeJong1993a} : \foreignquote{english}{However, with all this activity, there is a tendency to equate GAs with function optimization. There is a subtle but important difference between \enquote{GAs \textbf{as} function optimizers} and \enquote{GAs \textbf{are} function optimizers}} ; L'investissement d'Holland dans l'étude des \foreignquote{english}{Complex Adaptive Systems} s'inscrit dans une trajectoire de recherche resté proche des thématiques de ce qui deviendra plus tard la méta-discipline \textit{Artificial Life}. Son investissement continue dans cette branche de développement est d'ailleur lisible au travers de deux plateformes successive sur ce thème : \textit{$\alpha$-universe} et \textit{Echo} dont on trouve une analyse dans les travaux de \autocites{Taylor1999, Taylor2001} }

\Anotecontent{barricelli_multi_utilisation}{ Tout comme les travaux de McMillan ont permis de voir plus clair dans les intentions de Von Neumman derrière la notion de \textit{self-reproduction automata} ..., les travaux de Dyson \Autocite{Dyson1997}, de Fogel \autocite{Fogel2006a} sur l'histoire de cette discipline a permis également de redécouvrir les recherches de Barricelli comme celle d'un véritable pionnier en ALife, mais également comme celui d'un pionnier dans l'idée d'utiliser l'évolution comme support à la résolution de problème.}


\Anotecontent{note_pattee_semantic_closure}{ \foreignquote{english}{Additionnary, from an epistemological point of view, Pattee(1995b) points out taht symbolic information (such as that contained in an organisms's genes) has \enquote{no instrinsic meaning outside the context of an entire symbol system as well as the material organization that constructs(writes) and interprets(reads) the symbol for a specific function, such a classification, control, construction, communication ...}. He argues that a necessary condition for an organism to be capable of creative open-ended evolution is that it encapsulates this entire self-referent organisation (Pattee refers to this condition as semantic closure). From this it follows that organisms should be constructed \enquote{with the parts and the laws of an artifical physical world} Pattee (1995a)(p.36). In other words, the interpretation (phenotype) of the symbolic information (genotype) of an artificial organism should be constructed and act within the artificial physical environment of the system. Additionally, if the system is to model the \enquote{origin} of genetic information, then the genotype itself must also be embedded within the environment; that is, the complete semantically-closed organisation -- the \enquote{entire organism} -- must be completely embedded within the physical environment.} \autocite{Taylor2001}} 