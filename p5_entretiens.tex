
% -*- root: These.tex -*-

%%ENTRETIENT
\chapter{Entretiens}
\label{chap:entretiens}

\section{Entretien avec le professeur historien et médiéviste Jean-Philippe Genet}
\label{sec:entretient_genet}

\noindent\textbf{- Entretien réalisé le : } 13 mai 2015 \\
\noindent\textbf{- Transcription faite le :} 14 mai 2015 \\
\noindent\textbf{- Validation faite le :} en attente de confirmation, la version définitive apparaitra dans quelques semaines dans la version de la thèse déposée sur Internet.

\paragraph*{Sébastien Rey : Si vous pouviez m'en dire plus sur le type de matériel, l'ouverture du centre ?}

\noindent\emph{Jean-Philippe Genet}: Le centre je ne l'ai connu que quand je suis revenu ici. J'ai été assistant à la Sorbonne en 1967, et à l'époque il n'y avait pas encore Paris 1, c'était la Sorbonne, tout était donc ensemble. Je n'ai pas entendu de suite parler d'informatique, en tout cas chez les historiens. Il y avait bien des historiens qui en faisaient à cette époque, mais ils n'en faisaient pas ici, ou alors dans des endroits relativement, disons, je dirais occultes. A ma connaissance, les deux personnes qui s'intéressaient à l'informatique à l'époque, donc en 1967, c'était surtout \textbf{Antoine Prost}, qui travaillait dans le centre d'histoire contemporaine, qui à l'époque réunissait le dix-neuvième et le vingtième siècle. Il y avait aussi, quelqu'un, je pense qu'il était déjà là, je n'en suis pas absolument sûr, qui s'appelait \textbf{William Serman}. Et donc il faisait de l'analyse de discours, avec un informaticien qui s'appelait \textbf{Édouard Cadet}, un Haïtien, un type très bien, je l'ai un peu fréquenté. Je savais que cela existait, mais on ne pouvait pas être au courant si on n’appartenait pas spécifiquement à ce centre-là, il n'y avait pas d'activité en cours.
Donc ensuite j'ai fait mon service militaire complet à Paris 1, c'était en douce, et après je suis parti à Oxford pendant un an. Quand je suis rentré, je me suis mis sérieusement à l'informatique, j'ai donc pris des cours et çà je l'ai fait à Paris 5. J'ai suivi les cours de \textbf{Marc Barbut}, et puis des gens qui étaient dans son équipe. J'ai ensuite commencé à me préoccuper pour en faire  dans le cadre de mes recherches. A ce moment-là, je n'avais pas vraiment entendu parler du centre de calcul de Paris 1, et c'est seulement à l'automne 70 que je me suis vraiment préoccupé de faire quelque chose. Alors, ce que j'ai trouvé à ce moment-là durant l'année 70-71, c'est un ordinateur qui a été remplacé très rapidement, un IBM classique assez poussif. Il a été remplacé rapidement par une machine qui était un  Philips P880.

\paragraph*{Sébastien Rey : Ce numéro vous a marqué en tout cas.}

\noindent\emph{Jean-Philippe Genet}: Oui, car celui-là on a beaucoup travaillé dessus. On avait deux types d'activités : de la prosopographie (analyse de biographie avec des méthodes statistiques y compris l'analyse factorielle) faite sur le P880, et puis je faisais de l'analyse lexicale sur les grosses machines d'Orsay sur des crédits des CNRS, sur lesquelles on pouvait passer la nuit, ou le samedi et le dimanche. J'avais une Diane à l'époque, que je chargeais de bac de cartes perforées, je les trainais sur le sol, et puis on allait passer le samedi jusqu'à ce que mort s'ensuive à Orsay. On travaillait en mode batch, vous lancez l'exécution et vous attendez le retour, qui est une erreur en général. Une virgule qui manque quelque part, vous cherchez la faute, vous la corrigez, et puis vous relancez jusqu'à ce que ça finisse par passer. La grande différence par rapport à ce qu'on faisait en local, c'est que pour travailler à Orsay, il fallait des crédits CNRS.

\paragraph*{Sébastien Rey : Cela coûtait cher pour l'époque ?}

\noindent\emph{Jean-Philippe Genet}: Ce n’est pas que c'était très cher, mais il fallait surtout demander au comité national du CNRS l'autorisation, et obtenir un crédit, faire un dossier, etc. bref c'était compliqué. Donc je l'ai fait, j'ai obtenu un crédit, je m'en souviens très bien, mon premier crédit était de mille francs, et je recevais un carnet de chèques. C'était très bien d'ailleurs, la gestion n'était pas très compliquée, mais ça marchait très bien. Ceci dit cela ne servait à rien dans le cas d'Orsay, puisque là c'était des virements internes, donc je n’ai jamais touché au carnet de chèques. Avec ce travail à Orsay on est monté en puissance, on est devenu une vraie équipe CNRS,  puis un laboratoire du CNRS (date?). Tout cela a pris beaucoup d'importance et bientôt on n'était plus obligé d'aller à Orsay, et on a pu travailler au Laboratoire Informatique des Sciences de l'Homme.

\paragraph*{Sébastien Rey : Donc à la MSH, au LISH}

\noindent\emph{Jean-Philippe Genet}:Voilà, à la MSH, boulevard Raspail, c'est là que j'ai connu Philippe Cibois.

\paragraph*{Sébastien Rey : À quelle date environ ? }

\noindent\emph{Jean-Philippe Genet}: Cela doit être dans les années 75-80, franchement je suis très mauvais sur les dates, je suis médiéviste alors tout ça ... je vois plus loin :)

Par contre pour tout le travail, disons sur les bases de données, on travaillait sur le Philips P880, et c'est là où il y avait un club d'utilisateur assez restreint. En économie il y avait Pierre-Yves Hénin, et une personne qui s'appelait \textbf{Pouchain} qui était un économiste qui travaillait avec \textbf{Hénin}. Il y avait les trois géographes \textbf{Thérése Saint Julien}, \textbf{Denise Pumain}, et puis un spécialiste de la France qui s'appelait \textbf{Yvan Chauviré} je crois. Il est resté à Paris 1 assez longtemps, mais il n'y est plus maintenant.


\paragraph*{Sébastien Rey : Ça donc, c'était l'équipe de base ?}

\noindent\emph{Jean-Philippe Genet}: Oui l'équipe de base, et puis il y avait d'autres utilisateurs, bien sûr, mais le fonctionnement était assez bon grâce à cette équipe. Il y avait le dieu tutélaire qu'on ne voyait jamais qui passait de temps en temps, un économiste, qui était vital, car c'était de lui que dépendait le centre, et puis comme il était très respecté chez les économistes ... il s'appelait \textbf{Claude Fourgeaud}. Celui-ci était un professeur d'économie déjà âgé, qui avait déjà eu une attaque, un peu bizarre comme ça, mais c'était un universitaire, semble-t-il très fort et en tout cas très respecté. Ce qui fait que le centre de calcul a toujours eu les moyens qu'il lui fallait. La tête pensante pour les machines, les ingénieurs, etc. c'était \textbf{Édouard Valensy}. Celui-ci enseignait à Paris 1, mais c'était un général polytechnicien, qui s'occupait du centre de recherche de l'armée à Saclay, et qui avait donc cette grande qualité de nous fournir en informaticiens quand on voulait travailler sur des projets scientifiques, des gens d'une très très grande qualité intellectuelle. Pour éviter le problème d'aller à Orsay, j'ai pu développer justement mon logiciel d'analyse lexicale avec deux garçons : \textbf{Jacques Mondelli} qui est devenu un cadre très important chez Bull, et qui est mort malheureusement de la maladie de l'amiante, car les centres de calcul à l'époque n'étaient pas non plus des endroits très sains. Et François Hucher qui a travaillé par la suite à la C2I. L'un était de l'école des mines Berkeley, et l'autre était Centralien. Donc évidemment c'était des gens qui étaient à la pointe du combat en informatique, et on a pu faire vraiment à ce moment-là du bon travail. Bien qu'ayant des moyens dérisoires. Le centre de calcul lui-même c'était toujours le P880, ici à Paris 1. Celui qui le faisait tourner pour nous surtout, c'était un gars qui s'appelait \textbf{Xavier Debanne}. Je crois qu'il est devenu catholique convaincu, il est parti à Rome en tout cas et il a fondé plus ou moins une boîte, et il a travaillé plus ou moins avec le Vatican. Il travaillait à cette époque avec \textbf{Jean-Paul Trystram}, quelqu'un qui était un prof un peu inclassable, ayant également fait un manuel d'informatique pour les sciences sociales. \textbf{Xavier Debanne} avait développé pour un logiciel qui s'appelait BDP2 (banque de données paris 1 version 2), devenu ensuite BDP3, jusqu'à BDP4, et qui marchait très bien. En gros c'était un SPSS miniature, mais qui vous permettez de faire tous les tests, tout çà sur cartes perforées. On faisait ça avec le P880, car il tolérait d'assez gros fichiers, donc on pouvait traiter des bases de données importantes, faire des tris croisés, des analyses factorielles, et tout ça c'était quand même pas mal. 

Le centre de calcul était organisé ainsi, il y avait la salle avec l'ordinateur, et à côté il y avait une petite salle où les dames  faisaient les perforations. Il y avait deux perforeuses, et puis nous, enfin tous les chercheurs, on se faisait les perforations nous-mêmes la plupart du temps. 

\paragraph*{Sébastien Rey : Denise m'a également parlé de ces deux dames.}

\noindent\emph{Jean-Philippe Genet}: C'était les piliers de la maison, elles avaient leur franc-parler, et c'était assez pittoresque, surtout une. Et elles ne s'envoyaient pas dire quand quelqu'un faisait une erreur, cela fusait. 
 
\paragraph*{Sébastien Rey : Donc celles-ci étaient employées à plein temps.}

\noindent\emph{Jean-Philippe Genet}: Oui elles faisaient des petits trous toute la journée, avec une vitesse évidemment qu'enviaient beaucoup les malheureux qui faisaient leurs cartes perforées eux-mêmes.

\paragraph*{Sébastien Rey : Vous savez jusqu'à quand ça a duré ?}

\noindent\emph{Jean-Philippe Genet}: Tout ça a duré assez longtemps, car le CNRS a beaucoup renâclé à pousser la micro-informatique. C'est çà un des problèmes, d'ailleurs Cibois vous en a peut-être parlé.

\paragraph*{Sébastien Rey : Oui c'est arrivé vers les années 1980 - 1984, plutôt sur initiative de quelques personnes.}

\noindent\emph{Jean-Philippe Genet}: C'était beaucoup dans le cadre du CRH (centre de recherche historique). Enfin, il y avait certaines personnes qui connaissaient les universités américaines, et qui dès lors qu'elles sont revenues en France ont dit à ce moment-là \enquote{ça ne va pas} . Moi de mon point de vue, je n'étais même pas au courant. J'ai découvert ça presque un peu tard. Ici à Paris 1 on n’avait pas de micro-ordinateur, on travaillait toujours sur le P880, et surtout on n’avait pas les logiciels, donc on continuait à travailler avec le logiciel BDP4 qui fonctionnait bien. J'avais mon système qui s'appelait ALINE pour l'analyse linguistique qui fonctionnait également bien, donc je n'allais pas changer pour des micro-ordinateurs pour lequel il n'y avait aucun logiciel. D'ailleurs le CNRS ne nous y encourageait pas, il ne voulait pas nous donner d'argent pour qu'on achète des micro-ordinateurs, c'était explicite. Et parce que'il voulait surtout continuer à nous faire financer. En fait il donnait aux équipes des sciences humaines des crédits, pour faire de l'informatique, mais ces crédits étaient ensuite convertis automatiquement en crédits reversés au centre de calcul du CIRCE à Orsay. En fait l'argent ne faisait qu'un détour par nous, il tournait en rond et revenait dans leur système. Donc il ne nous poussait pas du tout à faire de la micro-informatique. Sans compter qu'évidemment on est toujours méfiant, dès que le fonctionnaire va se trouver autonome, il va regarder YouTube, enfin à l'époque ça n’existait pas. À l'époque le truc le plus « in », c'était vous savez les balles de tennis qui rebondissent, pong ... Donc il fallait que l'on continue à travailler avec les grosses machines.

Tout çà a quand même eu une fin. De toute façon j'avais déjà commencé à faire de l'enseignement de l'informatique pour les historiens à Paris 1. J'ai dû commencer dans les années 1977-78 quand les deux logiciels commençaient à bien tourner. La difficulté c'est que comme on n’avait pas de micro-ordinateur, on allait en procession rue cujas jusqu'au bâtiment de la fac de droit pour faire du batch sur le Philips P880.

\paragraph*{Sébastien Rey : Le Philipps était situé où dans le bâtiment ?}

\noindent\emph{Jean-Philippe Genet}: Quand vous rentrez au panthéon, vous voyez la cour, vous allez au coin droit, vous rentrez dans le bâtiment, et c'était là juste à droite. Au pied du grand escalier qui monte, c'était là, les trois-quatre pièces qui étaient là. Ce n’était pas grand.
Il y avait d'autres gens qui faisaient de l'informatique à Paris 1, par exemple \textbf{Colette Roland}, mais elle ne s'est jamais intéressée à ça, je ne l'ai jamais vu travailler sur une machine du centre. Elle faisait plus de la théorie informatique. La seule personne qui vraiment a été l'âme du centre, et qui l'a fait fonctionner, c'était \textbf{Yvonne Girard}, une mathématicienne. C'est elle qui a assuré le fonctionnement du centre pendant de longues années.

Je ne me souviens plus de l'année exacte 1984-85, mais  il y a \textbf{Jean-Pierre Bardet}, un historien démographe de Paris 4, qui a été ensuite au ministère de la Recherche. Un jour il en a eu marre de tout ça, car à Paris 4 c'était pire, nous on avait un centre de calcul, mais eux il en avait à peine. \textbf{Jean-Pierre Bardet} travaillait beaucoup au LISH, et je travaillais avec lui. On lui avait même prêté nos machines pour imprimer sa thèse, car la sienne était tombée en panne la veille de la soutenance, le type d'événement qui crée des liens. Et quand il s'est trouvé au ministère, \textbf{Jean-Pierre Bardet} a changé tout ça. Il a envoyé aux départements d'histoire, un petit peu partout en France, des micro-ordinateurs. C'était des Bull 45, à l'époque c'était bien, c'était des PC de base si vous voulez. Et il en a envoyé 12 à Paris 1 à l'UFR d'histoire. Le directeur de l'époque c'était \textbf{Robert Fossier}, cela tombait bien, car j'étais très proche de lui, et celui-ci ne savait pas trop quoi en faire. On a d'abord pensé distribuer les ordinateurs aux professeurs, et puis en fait ils n'étaient pas vraiment intéressés parce que la plupart ne savaient pas à quoi ça servait. Donc moi j'ai dit qu'il fallait trouver un endroit où les mettre tous ensemble, pour lancer ensuite un enseignement d'histoire et informatique. De façon plus solide, car cette fois-ci au lieu d'aller au panthéon, on fera l'enseignement sur place. \textbf{Robert Fossier} a dit d'accord, et on a donc cherché un local, que l'on a trouvé au sous-sol, là où il y a les salles d'informatique encore aujourd'hui. On a équipé la première avec ces Bull 45 et on a alors pu commencer une formation disons plus systématique pour les historiens. Il y avait à l'époque un bon prétexte, qui a disparu malheureusement par la suite, qui était l'existence de ce que l'on appelait un DEUG et une Licence MAS (mathématiques appliquées aux sciences sociales). 

\paragraph*{Sébastien Rey : Une formation qui a disparu il n'y a pas si longtemps non ?}

\noindent\emph{Jean-Philippe Genet}: Enfin si, en fait cela fait un bon moment. En tout cas les historiens en ont été exclus depuis un bon moment. Puisqu'en fait le gros du public c'était des économistes. Mais avec les collègues géographes justement on avait réussi à obtenir un quota, on avait droit à deux ou trois historiens, ou deux ou trois géographes, pour les intégrer à la formation. Et donc on était bien obligé de leur fournir une UV d'informatique appliquée à l'histoire, et il y en avait également une appliquée à la géographie. Et du coup l'idée a été simplement d'ouvrir cette UV d'informatique appliquée à l'histoire à d'autres étudiants qu'à ceux qui étaient seulement pour le DEUG MAS. Ce qui a permis d'avoir un effectif  assez vite important, car cela a plu aux étudiants. À la fin des années 1990, on a rendu ensuite rendu ça quasiment obligatoire dans l'UFR d'histoire. Cela fait partie de la licence, où il y a une série d'UV spécifiquement d'informatique. De toute façon c'est obligatoire, car il faut que les étudiants aient le C2I et des choses comme ça.

Et donc l'existence du centre de calcul, je suis sorti du cadre du centre de calcul stricto censu, mais voilà, cela a été le point de départ de tout çà. Je pense que cela a été vraiment utile jusqu'aux années 1985, et même peut-être après, mais évidemment quand on est passé à la micro-informatique, c'est devenu beaucoup moins intéressant et à Orsay il a pratiquement disparu. Tant que les micro-ordinateurs étaient très chers, évidemment cela avait son utilité, moi le premier Apple que j'ai acheté, je m'en souviens, car cela m'avait coûté très cher. À l'époque on achetait son ordinateur sur fonds propres, car il n'y avait pas de crédit recherche; les crédits recherches c'est une belle chose, mais ça date d'Allègre. Avant les années 1999-2000, on n’avait pas de grands budgets, et donc on s'achetait nous-mêmes nos ordinateurs. Le premier que j'ai acheté il m'en a coûté plus de 10000 francs, l'équivalent de 1500 euros. C'était un Apple 48K de mémoire, ça ne faisait pas grand-chose par rapport au Philips en comparaison. Vous rentriez une disquette avec un alphabet et puis les 4 accents, car vous aviez droit à seulement 4 caractères accentués. Ensuite le contenu était chargé en mémoire, et vous pouviez rentrer une disquette pour écrire et copier ce que vous faisiez. C'était des disquettes souples, là aussi on ne rentrait pas grand-chose. Ceci dit cela permettait de faire un certain travail. En histoire il y a une thèse qui a été faite expressément avec ce type d'appareil, qui le revendique, et qui a accepté les limitations pour montrer qu'on pouvait faire du bon travail. C'est la thèse de \textbf{André Sitzberg} sur les galériens. C'est un ouvrage important sur le plan historique. Celui-ci a quand même dépouillé tous les registres du bagne de Toulon, il a eu les données sur plus de 60000 galériens, qui ont permis une démonstration fondamentale c'est-à-dire de démontrer que les galériens ne servaient à rien. Ils étaient essentiellement là pour faire peur, et imposer la majesté du roi, et à travers la majesté du roi, assurer la rentabilité des fermes, des tabacs, et la gabelle puisque les galériens avant tout, ce ne sont pas des brigands, ce sont uniquement des gens qui essayaient de frauder sur le sel, le tabac, et puis il y avait aussi les déserteurs. Et ils ne ramaient pas du tout pour faire la guerre,  ils ramaient pour montrer que les galères étaient très belles. Si on n'avait pas eu les 60000 fiches, cette démonstration n'aurait pas pu exister. C'est une thèse qui a eu un vrai résultat historique, et avec un Apple 48K. Alors évidemment les gens ont protesté, et ils ont dit \enquote{ah ouais, vous avez codé les métiers alors les charpentiers il y a 50 dénominations possibles, il y a des tas de différences et on ne peut plus les retrouver}. Avec un 48K on faisait ce que l'on pouvait.

Ensuite j'ai acheté un ordinateur de cette série-là, pour l'équivalent de 2000 à 3000 euros. Alors tant que les ordinateurs étaient comme ça - il n'y avait pas de portable - le centre de calcul a gardé son intérêt, et les gens allaient travailler sur les micro-ordinateurs du centre de calcul. Ils ont remplacé assez vite le P880  par des micro-ordinateurs et les étudiants allaient travailler dessus. On pouvait utiliser ces micro-ordinateurs comme terminaux pour envoyer à Orsay, et travailler sur les logiciels installés là-bas. 
 Je ne sais pas jusqu'à quand les ordinateurs du centre de calcul ont été utilisés, car j'ai cessé de les utiliser, car à partir du moment où on était CNRS j'allais au LISH en cas de besoins plus importants. Le LISH a assez vite coulé, et Philippe Cibois est même venu à la rescousse à un moment pour le diriger.

\paragraph*{Sébastien Rey : En effet il semblerait qu'il y ait eu plusieurs directeurs successifs au LISH dans les années 80}

Celui qui a vraiment engagé le bras de fer, c'est \textbf{Michaël Hainsworth}. C'était un égyptologue, et donc en tant qu'égyptologue, il avait une grosse protection, car il faisait la partie informatique du travail de \textbf{Jean Leclant}, qui est devenu très vite le secrétaire perpétuel de l'institut. Autant dire qu'il avait le bras plus que long. Fort de cette protection, \textbf{Michaël Hainsworth} a essayé d'imposer un développement de la micro-informatique en accès libre très rapide, et ça a posé problème rapidement, car la direction du CNRS ne voulait pas lâcher la bride. Je le sais d'autant plus, car à l'époque j'étais à la direction scientifique du CNRS, et j'ai donc assisté de près au clash entre Hainsworth avec qui je travaillais, et les gens de la direction scientifique. Une direction censée être de gauche, ouverte, mais les réflexes administratifs français ... La priorité du CNRS, les physiciens y veillaient, c'était les gros équipements, c'était le CIRCE, et il ne fallait surtout pas aider le LISH à pousser et à éparpiller la micro-informatique chez les utilisateurs.  

\paragraph*{Sébastien Rey : À partir de là, ça a décliné ?}

\noindent\emph{Jean-Philippe Genet}: Les laboratoires se sont équipés, et les centres se sont déplacés. Les centres de calcul comme le LISH, comme le CIRCE, comme celui de Paris 1, c'était des endroits formidables, car on travaillait tous ensemble. Et on avait du temps, car on attendait les résultats. Donc quand on se trompait, c'était la consultation générale, qu'est-ce-que ça peut être, etc.. Il y avait un échange réel ! J'ai revu plusieurs fois \textbf{Denise Pumain}, car j'ai été président du comité de la section histoire, et par la suite je l'ai aussi croisée dans toutes les instances, mais je n'ai plus jamais parlé avec elle comme je pouvais parler quand on se disait \enquote{mince, qu'est-ce qui n'a pas marché cette fois-ci ?}. C'est là que se faisait le vrai travail, c'est là où on pouvait vraiment discuter. \textbf{Philippe Cibois} a été une aide précieuse pour tous les gens qui ont importé, car c'était un des esprits les plus clairs que je connaisse dans le domaine de l'informatique. \textbf{Michaël Hainsworth} aussi, c'était des gens avec qui on pouvait vraiment travailler. Et puis quand il y avait vraiment de gros problèmes, il y avait des mathématiciens ou des physiciens, qui avaient un niveau informatique d'un niveau bien plus élevé encore.

\paragraph*{Sébastien Rey : Le CIRCE mettait également une littérature grise, des écoles d'étés, des formations également non ? }

\noindent\emph{Jean-Philippe Genet}: Il y avait tout çà oui, mais après il fallait avoir du temps. Je n'étais pas personnel CNRS, mais oui j'en ai fait quelqu'unes, comme celle sur les Analyses Factorielles au CIRCE. Quand j'ai commencé, j'ai fait du Fortran, mais je n'ai jamais été capable de programmer des choses trop ardues. J'ai tout de suite vu que si l'on voulait programmer un logiciel d'analyse lexicale, ce n'était pas la peine d'essayer sans aide extérieure. Ce que \textbf{Jacques Mondelli} et \textbf{François Hucher} ont fait, c'était de l'informatique de très haut niveau. Il était passé par l'école des mines et l'école centrale, pas moi. Ils connaissaient par exemple toute une série de tests statistiques qui permettaient de ranger les lemmes par ordre alphabétique en gagnant du temps, et surtout de l'espace mémoire, car à l'époque les contraintes étaient très grandes, donc ils avaient des tas d'astuces statistiques pour faire avancer plus rapidement la conception des logiciels. 
 

\paragraph*{Sébastien Rey : C'était donc du travail main dans la main avec les informaticiens ?}

\noindent\emph{Jean-Philippe Genet}: J'ai des amis qui ont programmé des analyses factorielles, des choses comme çà. On avait des modèles que l'on pouvait réadapter. Mais quand on prend un système comme BDP4, c'est vraiment un logiciel de base de données, qui est entièrement paramétré, vous rentrez les données, et après vous pouvez faire beaucoup de choses. C'est un peu ce que l'on peut faire aujourd'hui avec des logiciels comme R, que l'on peut adapter à son propre usage, mais à l'époque il n'y avait rien de tel. 

\paragraph*{Sébastien Rey : SPSS arrive dans les années 1976 ? }

\noindent\emph{Jean-Philippe Genet}: Il y a une version pour les gros systèmes, c'est celle que tout le monde a utilisée. Il y avait trois logiciels de ce type que l'on utilisait : SPSS, SAS, BMD (biometrical data). Maintenant il y a des versions pour micro-ordinateurs. On pouvait déjà rentrer des fichiers concernant plus de 80000 individus. Le problème c'est que cela coûtait cher, car il fallait les acheter ces logiciels, et Paris 1 ne les avait pas. Il n'y avait pas de crédit recherche. À partir de 1998-99, il y a eu des crédits recherche, et il y a également eu tous ces masters professionnels qui ont rapporté de la taxe d'apprentissage. Les économistes ont par exemple eu très vite de gros moyens. En base de données, ils travaillaient sur Oracle, alors que nous en histoire, en géographie, on n’a jamais pu atteindre le niveau suffisant pour travailler sur Oracle, le standard aujourd'hui dans les entreprises. La question des moyens certes demeurait, mais à partir de 1998-99, on était passé dans un monde différent, cela marchait beaucoup mieux.

\paragraph*{Sébastien Rey : Et pour la formation en Fortran chez les historiens ? } 

\noindent\emph{Jean-Philippe Genet}: Ah non, chez les historiens on y a renoncé très vite. J'ai pu relancer les choses quand on a été vraiment équipés en micro-ordinateur, et à ce moment-là j'ai essayé d'obtenir des postes. Ainsi il y a eu la possibilité, dans les années 1995 - 1996, d'obtenir des postes de PRAG. On a réussi à créer quatre postes, et un de ces postes a été transformé par la suite en poste de maître de conférences. À l'heure actuelle, il y a 4 PRAG, et un maître de conférences, et il devrait y avoir un cinquième, car on a été obligé de le lâcher pour faire de l'enseignement de statistiques. On devrait le récupérer plus tard sous une forme statistiques/informatiques. Cela fait quand même 6 postes qui ont été créés, et donc qui n'ont pas été pris sur les contingents existants. Si j'ai quand même réussi quelque chose dans la vie, c'est ça, vous êtes trop jeune pour savoir ça, mais faire créer un poste dans l'université, c'est difficile. Ce ne sont pas forcément de bons postes, car ils ne sont que PRAG, mais c'est mieux que rien. Et cela a permis de développer cet enseignement à l'informatique. Je crois que les autres disciplines ont fait de même, il y a des PRAG informatiques ailleurs.

\paragraph*{Sébastien Rey : L'enseignement informatique s'est-il transformé pour devenir seulement l'activité de « bonne  utilisation des logiciels » ?  }

\noindent\emph{Jean-Philippe Genet}: C'est plus que çà, c'est vraiment un enseignement d'histoire. On essaye de développer depuis le niveau de la problématique, de voir ce qu'il est possible de faire face aux sources, ce que l'informatique va vous permettre de faire par la suite. Ce qui suppose d'avoir déjà une idée sur la façon de faire, notamment pour les traitements statistiques. De même, si vous voulez faire des traitements lexicologiques, il faut que vous ayez déjà une petite idée sur la linguistique, sur la lexicologie, la sémantique quantitative, et qu'est-ce qu'on peut tirer d'un contexte. De même pour la prosopographie. J'avais conçu cet enseignement de cette façon, et je crois qu'ils l'ont gardé ainsi, comme une formation pour l'informatique par la recherche. C'est-à-dire, après 3 ou 4 mois de cours magistraux et de rodage sur les logiciels pour qu'ils sachent les utiliser, on demande aux étudiants de chercher un sujet de recherche. Ils fabriquent de petites bases de données, sur des bases textes, sur des choses variées : les menus de Louis XV, le discours des présidents de conseil général de l'Orne, la date d'arrivée des clubs de football dans la première division en France, la programmation des cinémas parisiens entre telle date et telle date, etc. Les étudiants cherchaient quelque chose et il fallait que cela soit eux qui le trouvent, et à partir de là, on leur disait : ou vous construisez une base de données de type BDP4, SAS, SPSS , ou vous faites un traitement textuel avec une base texte sur laquelle vous allez faire des traitements linguistiques. Vraiment un enseignement par la recherche. C'était le travail de base, et ça a un peu évolué par la suite, car c'était en licence. On a remi quelque chose de plus banal pour faire le C2I basique.

\paragraph*{Sébastien Rey : Effectivement le C2I c'est un enseignement de base, une utilisation finalement assez passive de l'informatique.}

\noindent\emph{Jean-Philippe Genet}: On a surtout beaucoup rajouté derrière. C'est ce qu'on a probablement fait de plus intéressant. On a fait des séminaires au niveau de la maîtrise, qui sont des vrais cours d'informatique, par exemple on apprend à programmer du XML, à travailler sur BDP5, à faire du SIG, etc

\paragraph*{Sébastien Rey : et le logiciel R ?}

\noindent\emph{Jean-Philippe Genet}: R, non car c'était la partie statistique, mais on essaie de s'y remettre avec Stéphane Lamassé. Celui-ci  qui permet de travailler justement avec R, qui est adapté aux données historiques. On essaye de fabriquer des choses qui facilitent le travail pour les historiens. Et puis on a créé des ateliers, des séminaires, de suivis des thèses, et des maîtrises. En maîtrise c'est obligatoire, tous les étudiants doivent faire un semestre d'informatique, et s'ils prennent vraiment un traitement informatique dans le cadre de leur maîtrise, ils ont droit à un suivi individuel, même chose pour les thèses. C'est très bien, mais également très prenant.

\paragraph*{Sébastien Rey : Si on revient sur le centre de calcul de Paris 1 , avant l'installation des micro-ordinateurs, y avait-il eu des terminaux ? }

\noindent\emph{Jean-Philippe Genet}: Non non, il fallait aller au Philips, ou alors on allait au LISH, mais ça ne marchait pas pour l'enseignement, c'était uniquement pour la recherche. 

Je ne les ai pas utilisés, je ne me rappelle pas. Comme nous avons eu nos propres micro-ordinateurs, on a cessé d'utiliser ces salles. Et surtout, \textbf{Jean-Paul Trystram} a pris sa retraite, et \textbf{Xavier Debanne} est parti. Xavier c'était un type curieux, il a été en cinquième ou sixième année de médecine. Puis, il a plaqué la médecine, il a fondé sa boîte d'informatique, et il est parti en Italie, pour je ne sais quelle raison.

\paragraph*{Sébastien Rey : Ces postes n'ont pas été remplacés ?}

\noindent\emph{Jean-Philippe Genet}: Ce n'était pas vraiment des postes, je ne sais pas vraiment comment ils étaient payés. Ils devaient être vacataires ou quelque chose comme ça. Les gens qui travaillaient ici, \textbf{Hucher}, \textbf{Mondelli}, c'étaient des gens que je payais avec des vacations CNRS. Ils faisaient leur service militaire en réalité. Cela nous venait par \textbf{Édouard Valensy} qui les avait recrutés à Saclay et travaillaient pour lui. Comme ils n'avaient pas un énorme travail à réaliser pour l'armée, ils se faisaient un peu d'argent en travaillant ici. On travaillait surtout le soir ou la nuit. 

Il se trouve que je suis devenu ami avec \textbf{Hucher} et \textbf{Mondelli}, et donc après on a continué à travailler ensemble, mais c'est devenu difficile... Je me suis battu en particulier pour \textbf{Jacques Mondelli}, Dieu sait que si j'avais pu le faire ça aurait pu lui sauver la vie puisqu'il est mort à cause de l'amiante. J'ai essayé de le faire nommer ici. Il sortait des mines, il avait un doctorat de Berkeley. Ils n'ont pas voulu de son doctorat, et il lui ont dit \enquote{ok, on vous prend, mais il faut que vous repassiez un doctorat ici}.

Après des années de lutte, j'ai fini par obtenir un poste, converti depuis un poste de sergent de pompier. C'est quand même un type qui travaillait avec nous, qui s'appelait \textbf{Marc Turket}, probablement encore en poste au centre de calcul. C'était un archéologue qui avait fait de l'informatique, et qui a travaillé avec \textbf{Olivier Buchsenschutz}. Ce dernier travaillait sur la carte archéologique de la Gaule, une grosse base de données des archéologues au CNRS. Il était rattaché ici, à l'UFR d'art et d'archéologie de Paris 1.

\textbf{Marc Turket} était un spécialiste des ossements. Donc évidemment il faisait des classifications automatiques à n'en plus finir. Il était assez compétent. Mais bon, il avait peu d'espoir de trouver un poste dans le monde de l'archéologie, donc il s'est contenté d'un poste pas très bien payé, mais qui lui permettait de continuer à faire de l'informatique, au centre de calcul.

\paragraph*{Sébastien Rey : Le centre de calcul, qu'est-il devenu par la suite ? }

\noindent\emph{Jean-Philippe Genet}: Je sais plus ce que c'est devenu, de toute façon cela dépendait des économistes, de l'UFR 2. Les économistes ont dû remettre la main sur les locaux. Tout ce qui était interdisciplinarité, centré sur l'usage informatique a disparu avec la micro-informatique. Vous avez cité \enquote{ le médiéviste et l'ordinateur }, ça marchait très très bien. Du jour où il y a eu la micro-informatique, le déclin a été continu. On avait une association qui s'appelait \textit{International Association for History and Computing}  basée en Angleterre. J'en ai été le premier président, cela marchait très bien, on a fait d'importants colloques. Mais passé 1995, tout ça a disparu. Il y avait la branche française dont s'occupait \textbf{André Sitzberg} qui avait un bulletin publié par le LISH, ça a également disparu. Tout ça est fini. La seule chose qui surnage de cette époque, c'est parce qu'on s'était refusé à faire de l'informatique, c'était la revue Histoire et Mesure. Une revue qui était au CNRS, mais qui est retenue aujourd'hui par le CRH de l'école des hautes études.

\paragraph*{Sébastien Rey : Donc si je résume bien, de 69 à 85 vous étiez actif au centre de calcul.}

\noindent\emph{Jean-Philippe Genet}: C'est ça, disons 1970-71, je ne sais plus exactement quand cela a commencé. On peut même dire jusqu'à 90, il me semble, mais ma chronologie est très floue.

\paragraph*{Sébastien Rey : vous connaissez des travaux de recherches qui étudient cette période du centre de calcul à Paris 1 ? }  

\noindent\emph{Jean-Philippe Genet}:Un étudiant de Toulouse, \textbf{Castex} , mais il n'a pas fini sa thèse. 

\paragraph*{Sébastien Rey : Avez-vous participé au « bulletin des messaches » de la FMSH ?}

\noindent\emph{Jean-Philippe Genet}: On l'a effectivement lu, mais c'était vraiment du giron de la maison des sciences de l'homme, boulevard Raspail. On était un peu exclu de ce type de publication. A la maison des sciences de l'homme, le coeur de l'utilisation plus que les historiens, c'était surtout les sociologues. Parmi les historiens qui ont beaucoup fréquenté le LISH, il y avait  \textbf{André Sitzberg}, \textbf{Barbet}, \textbf{Laurent Ladury}. Mais voilà, par exemple, \textbf{Laurent Ladury} c'est pas lui qui a réalisé la partie informatique de son travail, c'est \textbf{Sitzberg} qui a principalement travaillé pour lui. Il y avait donc des gens qui travaillaient pour les autres. Par exemple c'est \textbf{Jules Roméro}, le premier enseignant en histoire et informatique ici, qui a fait toutes les thèses de Paris 4. Par conséquent, il n'a pas pu faire la sienne. C'est assez injuste. \textbf{Hainsworth} lui a été viré, voilà ce qui arrive finalement aux gens qui ont compté.

\textbf{Philippe Cibois} je l'ai surtout connu quand il était au LISH. Je le voyais pour ses logiciels à lui. J'utilise d'ailleurs toujours son logiciel tri-deux, c'est le seul auquel je fais vraiment complètement confiance.


\paragraph*{Sébastien Rey : Au niveau de la pratique des centres de calculs, y a-t-il encore un usage du calcul intensif chez les historiens ? }

\noindent\emph{Jean-Philippe Genet}: Non, je ne vois pas trop pour quels usages ... \\

\textit{... pause de quelques secondes ...}
\\
\noindent\emph{Jean-Philippe Genet}: \textbf{Denis Pechanski} a lancé un très gros programme franco-américain qui est appuyé sur le mémorial de Caen, et puis sur les projets muséaux qui entourent le 11 septembre aux Etats-Unis. Dans leurs projets, il y en avait un plus axé sur la visite du musée et les moyens de mesurer l'attention. Là il y avait du très gros calcul. En ce moment de mon côté, on fait un répertoire des membres des écoles et de l'université parisienne au moyen âge, c'est gros parce que ce sont des fiches textes, on en a 16000 en stock, 8000 accessibles en ligne, vous pouvez regarder. Pour certains de ces personnages, cela représente plus de 100 pages à imprimer : Albert Legrand, Thomas Daquin, etc. C'est énorme au niveau des données, mais au niveau des calculs c'est finalement assez banal,  des tris croisés, des analyses factorielles. 

Il y a quelques projets qui utilisent les GIS et mobilisent des moyens de calculs, mais ce n'est pas non plus colossal. On a un gros projet sur le tracé des rues, et l'emplacement des maisons parisiennes. On travaille avec un laboratoire de La Rochelle. En partant de tracés du 19e siècle, on remonte le temps de façon régressive à partir de tout ce que l'on peut savoir sur la voirie, et cela jusqu'à ce que l'on n’ait plus aucun tracé.  On a éventuellement des relevés archéologiques, et surtout on a des registres de tailles pour les maisons. Donc on essaie de remonter dans le temps pour arriver à Paris au 13e, 12e siècle. Le projet s'appelle ALPAGE. C'est \textbf{Hélène Noizet} qui le pilote. Ils ont publié un ouvrage important là-dessus. 

Il y a également les analyses lexicales parce que l'on arrive à rentrer de très très gros textes. Maintenant c'est plus rare de travailler sur un million de mots. Bon cela demande certes du calcul, mais ce n'est pas comparable avec la météo. 
Quand je revois toute cette période là, dans les années 1985, où on travaillait sur gros systèmes, franchement par rapport aux Anglais on a été bien meilleurs. On travaillait au niveau des Américains. Avec le virage de la micro-informatique on s'est retrouvé dix coups derrière, c'était fini. Et incapable de réagir. Par exemple dans les grandes enquêtes du CRH comme la statistique de la France en 1830, un gros projet fait en coopération avec l'université de Chicago. Les Français ont eu des crédits considérables pour faire çà, ils les ont à peine dépensés, parce qu'ils sont restés à réfléchir à comment ils allaient faire. Les américains, ils ont rentré leurs données, ils ont fait des trucs pas terribles, mais peu à peu ils ont aussi fait des trucs très bien. Et surtout, les Américains ont gardé les données, alors que les Français ont dépensé plein d'argent pour les saisir, ils n'en ont rien fait, et en plus ils ont perdu les données. Tout ça a donc disparu. 

Ce qui reste la grande réussite à mes yeux pour les historiens, vraiment le truc le plus extraordinaire que l'on a fait c'est le catasto florentin de 1427. Le catasto ce sont les déclarations fiscales des gens, après on en a fait un plan. En 1427 les Florentins étaient embêtés, ils avaient perdu la guerre, étaient en faillite, alors ils ont fait un grand effort fiscal. Pour être sûrs que tout le monde contribue, ils ont fait en sorte d'avoir des déclarations recoupées de multiples fois, et publiques. Faire quelque chose de plus honnête que ça est difficile. On a donc quelques 90000 déclarations individuelles, qui ont ensuite était regroupées dans des registres, puis ensuite dans des synthèses, et cela tenu à jour pendant 50 ou 60 ans. C'est l'historienne \textbf{Christianne Klapisch} accompagnée de \textbf{David Herlihy} qui se sont lancés dans cette analyse afin d'avoir une photographie d'une région médiévale d'à peu près 400000 habitants, la Toscane.

\paragraph*{Sébastien Rey : Et quand a été faite cette étude ?}

\noindent\emph{Jean-Philippe Genet}: L’enquête a été faite dans les années 1975-1980. C'est le CRH qui pilote, donc cela a été fait au CIRCE et au LISH. \textbf{Christianne Klapisch} l'historienne a fait une très bonne analyse, très bien faite. Lui, il a fait ça sur le plan informatique et statistique, et il a sorti des cartes avec l'aide de \textbf{Jacques Bertin}. Il a cosigné avec \textbf{Klapisch}. \textbf{David Herlihy} c'était un matheux, il a été engagé pour travailler comme ingénieur à l'école des hautes études, et ça l'a tellement intéressé par la suite qu'il a passé une thèse d'histoire. C'est peu fréquent. C'était un type super sympa et intelligent, et très souvent quand les gens cherchaient des données ou des enquêtes c'était dans son grenier dans sa maison de campagne. Personne n'avait songé à archiver, à classer tout ça. A ce moment-là, il y a eu un temps de latence d'une dizaine d'années, car on n'est pas passé directement des gros systèmes à la micro-informatique. Et donc très souvent des données ont été perdues, par exemple l'enquête statistique générale de la France, on est allé chercher à Chicago ce qu'on avait perdu à l'école des hautes études. 

De toute façon ce travail n’est pas valorisé dans la profession. Il ne l'est absolument pas. Peut-être cela va changer, je ne sais pas. Moi j'ai soutenu ma thèse très tard, parce que j'avais toutes ces bases de données à faire. Il y a une base de données, pas le lexical je l'ai laissé tomber, tant pis je l'ai pas mis, mais tout ce que j'avais fait sur la prosopographie j'ai dit \enquote{ben voilà j'ai cette base de données}, et on m'a répondu \enquote{mais vous n'y pensez pas !} Donc j'ai imprimé le contenu de ma base de données, et j'ai soutenu sur une thèse de plus de 8000 pages. Personne n'a été ouvrir un volume depuis évidemment. 

Donc vous pouvez mettre dans votre bibliographie 4 pages que vous avez faites dans les annales sur un compte rendu de mauvais bouquin, ça compte... mais vous faites une base de données qui vous a pris 20 ans ça, ça ne compte pas. En plus, les gens qui sont importants dans l'Histoire en France, ils n'ont jamais fait d'informatique. Puisque ce sont les meilleurs historiens de toute façon, et qu’eux n'en ont pas fait, comment voulez-vous que cela compte ? 

Les conditions sociales de production, je ne veux pas faire du Bourdieu de bas étage, mais c'est fondamental. Aujourd'hui c'est un peu remonté, il y a quelques normaliens qui daignent s'intéresser au quantitatif, mais bon... Ils ne prennent pas trop de risque, et surtout ils ne se tachent pas les doigts à faire de la base de données, ils cogitent. Et puis ils attendent que cela \enquote{ tombe de l'arbre }, les archives commencent à être numérisées, alors pourquoi apprendre à programmer du XML puisqu'on va vous envoyer directement les choses sur votre écran. On n’avance pas beaucoup. Je pense que cette avance qu'ont prit les américains et les anglais, ils vont la conserver encore longtemps.

\section{Echanges avec la géographe et professeure Colette Cauvin}
\label{sec:entretient_cauvin}

\noindent\textbf{- Echange réalisé le : } 5 et 20 mai 2015 \\
\noindent\textbf{- Validation faite le :} 29 juin 2015.

\paragraph*{Sébastien Rey : Dans votre équipe des débuts, avec Sylvie Rimbert, Henri Reymond, Michel Pruvot, j'ai cru comprendre que vous étiez tous programmeurs. Pouvez-vous m'en dire un peu plus sur la formation et les activités des membres de votre équipe ?}

\noindent\emph{Colette Cauvin}: L’équipe a commencé en 68-69 par des lectures et des présentations d’exemples en géographie théorique et quantitative à partir d’articles et d’ouvrages anglo-saxons, avec \textbf{Etienne Dalmasso} (MC puis Professeur), \textbf{Sylvie Rimbert} (MA, puis Chercheur et DR CNRS), \textbf{Monique Schaub} (ingénieur de recherches CNRS) et moi-même (MA à cette époque). À la rentrée 70 \textbf{Michel Pruvot} (MA), revenu du Canada (Université de Sherbrooke), s’est joint à nous, apportant des connaissances en statistiques, lui-même débutant en programmation. En 1971-1972, \textbf{Gérard Schaub} (Ingénieur-technicien CNRS) a complété l’équipe. Et \textbf{Henri Reymond} est rentré du Canada (en remplacement d’\textbf{Étienne Dalmasso} nommé à Paris), professeur dès 1974, apportant toute la dimension théorique qui nous manquait. Mais l’équipe vraiment active en statistiques, informatique a été composée de \textbf{Sylvie}, \textbf{Michel} et moi, et ensuite \textbf{Henri} (pour les stats, l’analyse spatiale, la modélisation et la théorie). Pour \textbf{Sylvie} et moi-même avec la dimension cartographie en plus. 

Dire que nous étions programmeurs serait nous parer d’une qualité que nous ne méritions pas réellement. Le seul qui l’est vraiment devenu à peu près est \textbf{Michel} qui s’est formé tout seul. Cependant, nous avons eu des formations en ce domaine :
\begin{itemize}
 \item En 1971, au stage d’Aix en Provence, nous avons eu une petite séance expliquant ce qu’était un ordinateur.
 \item Au 4e trimestre 1971, à Strasbourg (au centre de calcul, je crois ; sinon en relation avec ce centre), nous avons eu accès à un cours de Fortran. J’ai encore toutes mes notes.
 \item En septembre 1972, lors du stage de maths pour géographes à Paris, nous avons eu un cours de programmation Fortran. Là aussi, je dispose encore de toutes mes notes.
\end{itemize}

Nous avons eu, par la suite, des formations organisées par le CNRS (et également par l’université mais de manière plus limitée) pour le FORTRAN, puis plus tard pour Uniras, et à partir des années 82 (il ne me semble pas avant) pour des logiciels comme Spad, ADDAD, SAS...

Nous programmions de petites choses que nous allions perforer au Centre de Calcul (bien que nous ayons eu une perforatrice à l’Institut). Les perfos, cependant, étaient surtout utilisées pour « taper » nos données. 

En dehors de ces formations, nous avons eu l’aide d’une informaticienne, \textbf{Anne Engelmann}, dépendante du laboratoire de physique de Tricart, le CEREG. Comme, à cette époque, les personnes faisant appel à ses compétences en physique étaient peu nombreuses, nous avons beaucoup travaillé avec elle. Bien que rattachée au départ au laboratoire de physique et disponible de par son statut pour différentes personnes de l’équipe de physique, je crois que très vite elle a beaucoup fonctionné avec les géographes d’humaine ainsi qu’avec le CCS. Lors de la fermeture du centre, elle a été rattachée à d’autres laboratoires du CNRS/ULP aux activités de sciences « dures ». Elle est décédée il y a 2 ou 3 ans ; d’où mon impossibilité d’avoir d’autres précisions.

En ce qui concerne nos capacités en programmation. Seul \textbf{Michel Pruvot} a toujours continué en Fortran même sur les Mac. Il a créé des programmes spécifiques essentiellement pour l’ACP, les classifications, l’analyse de comparaison factorielle Amahavara (je ne suis pas sûre de l’orthographe), l’analyse de variance. \textbf{Michel} s’est occupé essentiellement des étudiants sur la micro, et des statistiques sur le Centre. Ses programmes servaient aussi à ses recherches et ponctuellement à celles du groupe proche en recherche. 

\textbf{Sylvie} faisait de petits programmes en basic à partir des années 80. Elle était entièrement tournée vers la recherche, plus particulièrement en cartographie et en télédétection. Ses travaux sur le Centre s’effectuaient essentiellement avec \textbf{Jacky Hirsch} et \textbf{Charles Schneider}. 

\textbf{Jacky Hirsch} a rejoint notre équipe en 1977 mais avec des contacts dès 1976. Il était rattaché au laboratoire comme ingénieur de recherches (d’études d’abord) ; il n’a jamais quitté le laboratoire. À l’origine il travaillait surtout pour le groupe de \textbf{Sylvie Rimbert} et un peu pour le groupe de géographie quantitative (j’ai eu une position un peu à part car j’appartenais aux 2 groupes ; donc j’ai toujours fonctionné avec \textbf{Jacky}), mais il a été, en permanence, largement disponible pour tous. C’est une véritable perle... Il peut programmer, adapter des programmes, s’occuper du réseau, expliquer, enseigner, participer à la recherche sur tous les plans et toujours avec une grande gentillesse.

\textbf{Charles Schneider} était ingénieur d’études, puis ingénieur de recherches et ensuite chercheur. Il est à la retraite depuis déjà plus d’une dizaine d’années. Il ne travaillait qu’en cartographie et télédétection et dépendait de (et travaillait avec) \textbf{Sylvie Rimbert}. 

Personnellement je faisais de petits programmes d’abord en Fortran, ensuite également en basic et très ponctuellement en Prolog. J’ai peu regardé le C et les langages suivants... Je préférais faire les organigrammes des méthodes à programmer pour les spécialistes... mais je pouvais lire les programmes et discuter. 

\paragraph*{Sébastien Rey : Est-ce que vous accédiez régulièrement au centre de calcul de Strasbourg pour écrire vos programmes ? Disposiez-vous de logiciels et de matériels sur place ?}

\noindent\emph{Colette Cauvin}:  Concrètement nous avons commencé à travailler au centre, je crois, au dernier trimestre de 1970 (au plus tard). 

Au Centre de Calcul, nous avions accès à des programmes (Logiciels ? Bibliothèques de programmes ? Je ne suis pas assez compétente pour ne pas faire d’erreur entre ces termes, mais je pourrai avoir des précisions si nécessaire) de statistiques, en particulier le BMDP (à moins qu’au début, ce fût le BMP : là aussi s’il faut des précisions, je pourrai peut-être les avoir). Ce programme est remarquable et nous a permis d’effectuer des régressions, des analyses factorielles et des classifications (cf. publication à l’AGF en 1971).

Le Centre achetait des bibliothèques graphiques, mathématiques, statistiques, etc., en fonction des demandes des grands laboratoires. Nous en étions avisés d’autant plus facilement que \textbf{Jacky Hirsch} notre informaticien avait son bureau au Centre. Ainsi nous avons pu utiliser, à partir des années 80 (grossièrement), en statistiques, des logiciels comme ADDAD, SPAD, SPSS, SAS (un peu plus tard), en calcul de structures pour les déformations cartographiques, Hercule (logiciel du génie Civil), en images de synthèse, Catia (le logiciel de Dassault) sur lequel nous avons eu une formation (je l’ai fait acheter ultérieurement sur station par notre laboratoire, grâce à un financement exceptionnel de l’Université). À notre niveau, nous n’avions pas les codes sources. Peut-être l’informaticien, mais il faudrait que je le lui demande. 

En ce qui concerne la cartographie, essentielle pour tout géographe, surtout à Strasbourg, nous avons eu grâce à des étudiants Canadiens une version Fortran du célèbre SYMAP (H. Fisher, 1964), mis en service par \textbf{Anne Engelmann} vers 1970-1971. En 69 on ne l’avait pas ; en 71 on l’utilisait. \textbf{Sylvie} ne sait plus à quel moment exactement les étudiants canadiens sont venus et l’ont acheté sur leur bourse. Ils nous l’ont laissé après leur départ. Merci à eux ! C’était entre juin 1970 et janvier 1971.

Dès lors toutes nos cartes étaient faites avec ce logiciel dont les sorties, sur imprimante ligne à ligne, étaient bien sûr peu esthétiques. Le SYMVU pour la 3D impliquait des sorties sur traceur (Benson chez nous. J’ai encore les caractéristiques de ce dernier sur papier, datant de 1973) ; il a dû être disponible vers 1973. Ces deux logiciels venaient du \textit{Harvard Laboratory for Computer Graphics and Spatial Analysis} ; j’en ai rédigé des modes d’emploi accessibles à tous, mis en service au centre de calcul pour les chercheurs et surtout les étudiants. D’autres logiciels de cartographie sont apparus par l’intermédiaire de \textbf{Sylvie} (achetés ou non, là je n’ai pas d’informations mais je pourrai peut-être en obtenir) entre 73 et 79 : avec sorties traceur : Azmap (anamorphose unipolaire, Cerny), Gipsy (cercles, Monmonnier), Calform (carte choroplèthe, Harvard) ; pour imprimante : GRID (carroyage, Harvard). Ultérieurement (sans doute à partir de 79-80 mais il faut que je demande), le centre a fait l’acquisition de la bibliothèque de programmes Uniras (bibliothèque d’origine danoise, permettant de nombreuses applications graphiques) conduisant à créer les programmes souhaités (ce que \textbf{Jacky Hirsch}, l’informaticien de notre laboratoire d’humaine a fait avec grande compétence, à partir de 77). Mes dates sont exactes à 1 ou 2 ans près.

Grâce à la jonction de logiciels de statistiques et de cartographie, nous avons fait notre premier contrat avec \textbf{Sylvie} pour l’Agence d’urbanisme en 1971 (un atlas informatisé sur la ville = 703 îlots, énorme pour nous à cette époque) et notre première publication à l’AGF en septembre 1971.

Mais différents logiciels ont été développés pour les géographes du laboratoire par \textbf{Jacky Hirsch} soit à la demande de \textbf{Sylvie Rimbert} (en particulier pour la télédétection), soit en collaboration avec \textbf{Charles Schneider} (cartographie pièzoplèthe et ensuite procédé Irisos, dérivé de l’analyse spectrale), soit avec les autres membres du laboratoire comme \textbf{Henri Reymond}, \textbf{Aziz Serradj}, \textbf{Christiane Weber} et moi-même, ou encore des doctorants. \textbf{Michel} a toujours fait ses propres programmes.

\paragraph*{Sébastien Rey : Aviez-vous un bureau ou une permanence au centre ?}

\noindent\emph{Colette Cauvin}: Nos fiches perforées demeuraient au centre de calcul et, dans les années 70, nous avions grâce à Anne, accès à un bureau au sous-sol où nous pouvions laisser nos affaires. Nous préparions nos données et les instructions de contrôle propres aux analyses, que nous souhaitions, dans une grande salle au sous-sol, et nous montions les entrer dans le lecteur de cartes au rez-de-chaussée. En attendant nos résultats (cela pouvait durer entre 10 mn et 1 heure, et même davantage, selon le nombre de chercheurs présents au centre), nous pouvions préparer d’autres données ou, merveille, faire des parties de ping-pong ! Excellente détente calmante dans certains cas où l’attente se prolongeait pour aboutir à constater une erreur de perforation qui nous faisait recommencer tout le circuit pour une bêtise. 

\paragraph*{Sébastien Rey : Il semble y avoir eu une forte interdisciplinarité dans votre équipe, cela a-t-il perduré une fois le centre fermé ?}

\noindent\emph{Colette Cauvin}: L’interdisciplinarité est surtout due au fait qu’on croisait sans cesse des personnes de différentes disciplines, sciences humaines comme sciences dures (la géographie à Strasbourg était rattachée aux sciences dures, heureusement !) que ce soit quand on attendait nos résultats, ou lors des stages d’apprentissage de nouveaux logiciels, stages qui étaient communs. Les contacts ont perduré par la suite.

\paragraph*{Sébastien Rey : Pouvez-vous me dire un peu quels étaient les différents équipements que vous avez successivement utilisés au Centre de Calcul de Strasbourg-Cronenbourg et dans votre laboratoire ? }

\noindent\emph{Colette Cauvin}: Le Centre de Calcul a été créé sur le campus du CNRS en 1967-1968, le campus ayant été inauguré en 1960. Il était justifié par les besoins des laboratoires de physique, etc., et devait servir aux chercheurs du CNRS ainsi qu’aux universités. Nous y avons eu accès tant comme équipe CNRS (ERA 214 au départ, puis URA 915) que comme enseignants-chercheurs, bien sûr moyennant finances.

Je crois que tout au début (1967-1968), il y aurait eu un IBM, mais je n’en sais pas plus.

En 1971 (peut-être 1 an avant), nous avons eu un Univac 1108 (Système Exec II) dont j’ai encore le descriptif d’emploi mis à disposition par le Centre. J’ai ainsi le schéma de sa configuration et sa présentation.

En 1983, nous avions encore un Univac mais l’Univac 1110, sur lequel le logiciel Cartel a été élaboré par \textbf{Jacky Hirsch} et \textbf{Charles Schneider} (donc dans notre laboratoire) ; il s’agissait d’un logiciel de traitement et de visualisation des données de télédétection (j’ai encore une version du manuel d’utilisation de 1983).

Je sais qu’ensuite, en 1985 au plus tard, nous avions un IBM 3060 ou 3090 (nous avons dû avoir les 2 successivement, mais là je n’ai rien pour une date plus précise) et qu’on travaillait avec des consoles IBM 3179. En effet, je devais apprendre à utiliser les perforatrices, etc. aux étudiants du DEA à la rentrée 85 en vue du cours de cartographie. Quinze jours avant j’ai appris qu’on avait désormais des consoles et qu’on travaillait directement dessus. Heureusement, je les avais déjà pratiquées aux USA et j’ai pu rebâtir complètement mon cours et les fiches pédagogiques (que j’ai encore).

À côté nous disposions d’imprimantes ligne à ligne (plus tard, des imprimantes laser), un traceur électrostatique Benson (avec des numéros successifs : 9215 en 1983), une console graphique Tektronix 4015, un traceur à plume Benson 1220.

Progressivement, nous avons disposé à l’université Louis Pasteur de terminaux qui nous permettaient de nous connecter sur le centre de calcul (distant d’environ 8,5 km, soit 20-30 minutes de trajet) ; le centre nous renvoyait les listings 2 à 3 fois par jour. Surtout, ces terminaux nous permettaient de préparer nos données sans nous déplacer au Centre.

Précisons qu’à côté de l’offre du Centre de Calcul, nous avions accès dès 71 à une programma 101 à l’École d’ingénieur, alors dénommée ENSAIS ; à une table à digitaliser sur fiches perforées (avec un tableau \enquote{ électrique } pour déterminer les colonnes à perforer), à l’EOST (physique du globe), les deux étant localisés sur le campus universitaire dans le quartier de l’Esplanade.

Vers 1975 nous avons eu également une HP au laboratoire qui offrait la possibilité de faire de petits programmes de statistiques (ainsi le \textit{path analysis}).

Dès 1982-1983, nous avons disposé d'un premier PC (un Sirius si je me souviens bien), puis d’un AMSTRAD, et en 1985 d’un Mac +, et ensuite de toute la gamme des Mac avec ponctuellement des PC jusque 90-94. Mais comme je crois que cette période ne vous intéresse pas, je n’en dis rien de plus. Cependant, je dois avoir encore un inventaire du matériel. 

\paragraph*{Sébastien Rey : J'imagine que l'accès à tout cet équipement du CCSC n'était pas gratuit ?}

\noindent\emph{Colette Cauvin}: L’accès au Centre de Calcul lui-même était gratuit, mais il fallait des crédits pour faire tourner le programme et obtenir les sorties sur un support papier ou autre. Ces crédits, variables selon les années, provenaient essentiellement du laboratoire et donc du CNRS ; mais souvent, compte tenu de nos moyens, nous devions faire des \enquote{ concessions }. Entre autres, les classifications, exigeant – à cette époque – beaucoup de temps calcul, étaient limitées. 

Les crédits ne concernaient que le temps de calcul et les sorties sur les différents périphériques. Certains logiciels étaient fort gourmands en temps (Catia par exemple). Les formations étaient rarement payantes ou alors à un coût minime. Pour les étudiants, nous devions obtenir des crédits spécifiques à l’université.

\paragraph*{Sébastien Rey : Utilisiez-vous des terminaux pour vous connecter également à d'autres centres de calculs, comme celui du CIRCE à Orsay ?}

\noindent\emph{Colette Cauvin}: Pour nous, seulement à partir des années 85-90, en particulier avec Montpellier (CNUSC) en raison de nos travaux avec la Maison de la Géographie et le GIP Reclus.

\paragraph*{Sébastien Rey : Que s'est-il passé ensuite quand le centre a fermé ? Le CRI Curri a-t-il pris la suite, vous avez continué à utiliser ces ressources ?}

\noindent\emph{Colette Cauvin}: Le centre, ouvert en 1967-1968, a fermé le 24 décembre 1993 à 16h30, date que nous ne sommes pas près d’oublier car il a fallu adapter, ou même réécrire dans certains cas, tous les programmes pour la station RS 6000. Cette dernière appartenait à notre laboratoire. En effet, nous avons obtenu à différentes périodes, plus particulièrement après la fermeture du centre, des crédits plus ou moins importants destinés à l’achat de matériel et de logiciels : par exemple, en 1998-1999, une station spécifique pour Catia, le logiciel de Dassault utilisé pour certaines anamorphoses, même si ce n’était pas son rôle à l’origine.

Lors du passage au CURRI, pendant un temps, nous avons pu continuer à travailler \enquote{ presque } comme avec le CCS, et ceci à partir des terminaux du laboratoire ou ceux de l’ULP, les sorties nous étant rapportées une fois par jour à l’ULP. Cependant, \textbf{Jacky} avait réadapté les programmes de cartographie et mis en service le BMDP sur la station avec des sorties au laboratoire sur imprimante laser ; par la suite, sur traceur ou sur une imprimante couleur (propriété du labo). Progressivement le laboratoire a acheté les programmes de cartographie, de statistiques et de SIG nécessaires, pour PC en général. Nous avons eu dès 1994 Arc/Info (station), essentiel pour nos recherches, et Idrisi (surtout pour l’enseignement, sur PC).

Le Curri est resté très utile pour avoir accès à des logiciels comme SPAD, Hercule, etc., en fait pour toutes applications lourdes ou les logiciels auxquels nous ne faisions appel qu’occasionnellement. Seule la télédétection a vraiment impliqué l’utilisation du Curri (mais pas nécessairement sur place) avant l’achat de logiciels spécialisés au début des années 2000.

Précisons que le Curri a été installé à l’ULP dans le nouvel ensemble universitaire à Illkirch (banlieue Sud de Strasbourg). Il a disparu en tant que Curri, sans doute en 2008 (Jacky n’était pas totalement sûr de la date. En 2005 il était encore Curri ; en 2010, non). 

Son importance essentielle, actuellement, réside dans sa fonction d’outil de calcul puissant permettant de développer des modèles qui en général peuvent servir à des chercheurs de différentes disciplines (certains membres du laboratoire les utilisent).

\paragraph*{Sébastien Rey : Quelles étaient votre place et celle de vos collègues dans les institutions à ce moment-là ? Cela a-t-il eu un poids au moment des restructurations ?}

\noindent\emph{Colette Cauvin}: Les géographes n’étaient pas directement dans les instances de décision ; ils pouvaient éventuellement donner un avis par l’intermédiaire du comité d’utilisateurs où \textbf{Jacky} les représentait.

Personnellement, j’ai géré les relations et les finances du centre pour les étudiants en cartographie et partiellement en statistiques, jusqu’à la fermeture du centre. Ensuite j’ai eu ces mêmes fonctions avec la ferme CURRI qui a remplacé le centre et l’université.

\paragraph*{Sébastien Rey : Pouvez-vous m'en dire plus sur l'organisation des enseignements à cette période ? Étaient-ils situés sur place à l'institut, ou au contraire fallait-il que les élèves envoient les cartes perforées de leurs exercices au centre de calcul de Strasbourg ?}

\noindent\emph{Colette Cauvin}: À partir des années 73-74 (je crois), en cartographie, j’ai pu organiser des stages à Cronenbourg pour les étudiants sur le SYMAP, puis le Symvu (on avait obtenu des crédits spécifiques pour les étudiants). J’ai également eu des enseignants-chercheurs d’autres universités qui se sont joints à ces formations accélérées. À partir de 76 environ, les cours se sont multipliés (dans une certaine limite quand même) et \textbf{Michel Pruvot} pouvait faire des TD de stats, mais souvent il préparait les sorties des programmes et faisait travailler les résultats obtenus avec ses étudiants. À partir des années 80, la situation a été plus simple pour les stats, les programmes se multipliant sur micro.

À partir de mon retour des USA (1982), les stages de cartographie ont été réguliers et plus nombreux, d’abord sur les mêmes logiciels, ensuite sur les programmes développés essentiellement par \textbf{Jacky} (et quelques étudiants avancés) à partir d’Uniras. 

\textbf{Michel Pruvot} a lui aussi continué, mais assez rapidement, il est passé sur micro. En 1986, j’ai fait des TD d’analyse spatiale avec un AMSTRAD (auparavant, j’amenais mon petit VIC sur lequel j’avais fait de mini-programmes en basic). Le Vic m’appartenait et je l’apportais en cours entre 82 et 85 pour certains TP. Le \enquote{ Amstrad } était localisé dans une salle où les étudiants avaient un accès relativement libre ; certains l’utilisaient pour leurs exercices.

L’analyse spatiale a été programmée sur Amstrad par \textbf{Bruno Guérin} (logiciel Anaspat). Une précision : Carto2D et Anaspat ont été élaborés à partir de mes cours. Anaspat de manière explicite. Pour Carto2D, c’est un de mes étudiants qui a donné mon cours de base à 2 ingénieurs ENSAIS qui ont développé ce programme, mais je ne l’ai su que longtemps après. 

À partir de 87-88 (environ) une salle, équipée de micros Mac, a été dédiée à l’informatique pour les étudiants. Elle a été conçue et gérée par \textbf{Michel Pruvot}. Ce dernier y faisait ses TD de statistiques et les étudiants pouvaient y venir pour effectuer leurs exercices. \textbf{Michel} a vraiment fait le maximum pour développer les statistiques et les logiciels associés pour la pédagogie. Il a écrit, lui-même, plusieurs logiciels sur Mac pour l’ACP et les classifications.

En ce qui concerne les terminaux de l’ULP, personnellement je ne les ai que rarement utilisés pour des cours, mais j’aidais, dans ces salles dédiées, les étudiants pour leurs mémoires en faisant appel à leurs équipements. Je crois que \textbf{Michel} les a utilisés parfois, de même que \textbf{Aziz Serradj} pour la télédétection mais je ne peux garantir ce dernier point.

Les applications de télédétection ont eu lieu à partir des années 83 (ou 82) à Cronenbourg, sous forme de stages spécifiques (chercheurs, thésards, etc.) ou pour étudiants, à l’aide du logiciel Cartel développé au laboratoire par \textbf{Jacky Hirsch}. En pratique, la formation pour la cartographie et la télédétection s’est toujours déroulée au CCS jusqu’à la disparition du centre car \textbf{Jacky Hirsch} et un certain nombre d’étudiants de géographie avancés (comme \textbf{Aziz Serradj} en particulier, qui est enseignant de cartographie maintenant) avaient développé des programmes de cartographie d'utilisation facile. Entre 1980 et 1994, signalons qu’une grosse différence entre le CCS et la micro, que ce soit pour l’enseignement ou la recherche, était la qualité des sorties. Au CCS nous avions des sorties sur imprimantes électrostatiques, jet d’encre, laser et sur traceur. Au laboratoire nous avions seulement, à cette époque, des sorties sur imprimantes ligne à ligne...

À partir de 1990-1994, nous avons acheté des programmes de cartographie pour mac (Carto2D, en effet, MacGrizdo...). En effet, à partir de cette date les logiciels de ces différents domaines se sont multipliés pour PC. Ayant eu à effectuer une étude au sujet de ces logiciels pour le compte du CNRS, j’ai eu l’occasion d’en tester un grand nombre (environ 180 ; cf. publication informelle pour le CNRS) et nous avons donc eu des possibilités de choix importantes, limitant les nécessités de demandes d’équipements ou de logiciels extérieurs. De plus les coûts de matériel baissaient permettant l’achat de périphériques adaptés pour les sorties graphiques.