% -*- root: These.tex -*-

\chapter{Introduction}

\startcontents[chapters]
\Mprintcontents


1 La simulation de modèles au coeur de la construction des connaissances en géographie

Deux grandes parties : 

1. Les modèles de simulation en géographie

1.1 Introduction de la simulation dans les sciences sociales 
	
	1.1.1 Introduction section
	1.1.2 Irruption de l'outil informatique
	1.1.3 Condition d'apparition de la simulation dans les différentes sciences sociales
	1.1.4 Les premiers modèles de simulation en géographie
	1.1.5 Une crise de confiance envers l'outil

1.2 La validation des modèles de simulation
	1.2.1 Définition V&V générique
	1.2.2 Le tournant explication des années 1970
		1.2.2.1 Quelle réalité dans l'application de la démarche explicative néo-positiviste
			* Un état critique du débat épistémologique néo-positiviste dans les années 1960-70
			* Les principaux instigateurs du mouvement en géographie
			* Une étiquette néo-positiviste critiquée et critiquable
			* Un échec et des critiques qui ne doivent pas masquer la réalité des transformations
		1.2.2.2 L'intégration progressive et naturelle du projet systémique
			* Premiers passeurs et premier débats au cœur de cette nouvelle posture nomologique
			* Les premières revendications systémiques
			* Les pionniers de l'opérationalisation systémique

	1.2.3 Les problèmes de la Validation ?
	
		1.2.3.1 L'impact du programme Forresterien dans le débat sur la validation 
			* Urban Dynamics, un révélateur des nouveaux usages pour la construction et la validation des modèles ?
			* La confrontation entre deux approches de la Validation, Objectiviste et Relativiste
			* Le retour à la neutralité de la V\&V


		1.2.3.1 La dimension philosophique
			* Représentativité système réel vs système simulé

L'exploration devient une nécessité pour la compréhension des modèles
	Multiplicité graphe causaux possible
	Multiplicité lié interdisciplinarité

1.3 Une Plateforme Intégré de Modélisation (PIM)
	1.3.1 Introduction

2 Cas d'utilisation de la plateforme

2.4 - Simpop local
2.5 - MicMac 

--

Depuis les années 1990 et l'avénement de la modélisation multi-agent, les géographes abordent la problématique de la validation sous l'angle de cette technologie.

Dès les années 1960-1970 et sous l'effet de travaux menés par des individus opérant aux frontières de l'inter-disciplinarités, apparaissent dans différentes branches des SHS des publications pour pointer le potentiel explicatif des modèles de simulation, véritable machine à expérimenter 

,  vu comme des laboratoire virtuel permettant de dépasser les limitations  ,  pionniers va s'ensuivre une crise va gréver la diffusion de la simulation dans les sciences sociales, 

--
20 juin 16h fin sur la validation 

% Cet historique nous permet de mettre en exergue deux moments important dans l'introduction de la simulation dans les sciences sociales.

A partir d'une relecture historique qui aborde la simulation dans son introduction dans les sciences sociales, il s'agit de montrer en quoi cette problématique est une question dont la complexité ne peut être analysé en tenant compte seulement des évolutions techniques.	

------------------------------------------------------------------------
ON ABANDONNE L'IDEE de la PARTIE INTRODUCTION AVEC CLIC SUR SECTION, FAIRE UNE CONCLUSION SYNTHETIQUE PLUTOT PAR CHAPITRE, et en étant moins extérieur que je ne le suis déjà dans ce chapitre
------------------------------------------------------------------------



Faut que cela tienne debout tout seul, et que ma patte apparaissent beaucoup plus 

Partie visualisation remonte dans le 3) dans la position de recherche
Grille de lecture qui fait écho, renvoie au problème soulevés dans l'état de la question plus haut et pour lequel on 






Construction = Evaluation

La première partie propose de retracer l'introduction de la simulation dans les sciences sociales, et plus spécifiquement en géographie. 




---

Attention aux termes trop elliptiques, se mettre à la place du lecteur surtout au début de la partie 2.

Annoncer la couleur, cette première page développe chacune des idées 


Ne pas livrer la conclusion au début 

Introduction aux sections, donc attention 



\printbibliography[heading=subbibliography]

\stopcontents[chapters]
