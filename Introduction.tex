% -*- root: These.tex -*-

\chapter{Introduction}

\startcontents[chapters]
\Mprintcontents


1 La simulation de modèles au coeur de la construction des connaissances en géographie

1.1 Introduction de la simulation dans les sciences sociales 
	1.1.1 Irruption de l'outil informatique
	1.1.2 Condition d'apparition de la simulation dans les différentes sciences sociales
	1.1.3 Les premiers modèles de simulation en géographie
	1.1.4 Une crise de confiance envers l'outil

1.2 La validation des modèles de simulation
	1.2.1 Actualités
	1.2.2 Problématique complexe
	1.2.3 Le tournant explication des années 1970
	
	1.3.2 L'impact du "programme Forrester" pour la validation

1.3 Une Plateforme Intégré de Modélisation (PIM)
	1.3.1 Introduction

2 Cas d'utilisation de la plateforme

--

Depuis les années 1990 et l'avénement de la modélisation multi-agent, les géographes abordent la problématique de la validation sous l'angle de cette technologie.

Dès les années 1960-1970 et sous l'effet de travaux menés par des individus opérant aux frontières de l'inter-disciplinarités, apparaissent dans différentes branches des SHS des publications pour pointer le potentiel explicatif des modèles de simulation, véritable machine à expérimenter 

,  vu comme des laboratoire virtuel permettant de dépasser les limitations  ,  pionniers va s'ensuivre une crise va gréver la diffusion de la simulation dans les sciences sociales, 

% Cet historique nous permet de mettre en exergue deux moments important dans l'introduction de la simulation dans les sciences sociales.

A partir d'une relecture historique qui aborde la simulation dans son introduction dans les sciences sociales, il s'agit de montrer en quoi cette problématique est une question dont la complexité ne peut être analysé en tenant compte seulement des évolutions techniques.




Construction = Evaluation

La première partie propose de retracer l'introduction de la simulation dans les sciences sociales, et plus spécifiquement en géographie. 








\printbibliography[heading=subbibliography]

\stopcontents[chapters]
